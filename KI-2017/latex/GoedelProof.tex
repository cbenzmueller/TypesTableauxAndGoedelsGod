%
\begin{isabellebody}%
\setisabellecontext{GoedelProof}%
%
%
%
%
%
%
%
\isamarkupsection{G\"odel's Ontological Argument%
}
\isamarkuptrue%
%
\isamarkupsubsection{Part I - God's Existence is Possible%
}
\isamarkuptrue%
%
\begin{isamarkuptext}%
G\"odel's particular version of the argument is a direct descendent of that of Leibniz, which in turn derives
  from one of Descartes. While Leibniz provides some kind of proof for the compatibility of all perfections,
 G\"odel goes on to prove an analogous result: \emph{(T1) `Every positive property is possibly instantiated'},
 which together with \emph{(T2) `God is a positive property'} directly implies the conclusion.
 In order to prove \emph{T1}, G\"odel assumes \emph{(A2) `Any property entailed by a positive property is itself positive'}.
 As we will see, the success of this argumentation depends on how we choose to formalize our notion of entailment:%
\end{isamarkuptext}\isamarkuptrue%
\isacommand{abbreviation}\isamarkupfalse%
\ Entailment{\isacharcolon}{\isacharcolon}{\isachardoublequoteopen}{\isasymup}{\isasymlangle}{\isasymup}{\isasymlangle}{\isasymzero}{\isasymrangle}{\isacharcomma}{\isasymup}{\isasymlangle}{\isasymzero}{\isasymrangle}{\isasymrangle}{\isachardoublequoteclose}\ {\isacharparenleft}\isakeyword{infix}\ {\isachardoublequoteopen}{\isasymRrightarrow}{\isachardoublequoteclose}\ {\isadigit{6}}{\isadigit{0}}{\isacharparenright}\ \isakeyword{where}\isanewline
\ \ {\isachardoublequoteopen}X\ {\isasymRrightarrow}\ Y\ {\isasymequiv}\ \ \isactrlbold {\isasymbox}{\isacharparenleft}\isactrlbold {\isasymforall}\isactrlsup Ez{\isachardot}\ X\ z\ \isactrlbold {\isasymrightarrow}\ Y\ z{\isacharparenright}{\isachardoublequoteclose}\isanewline
\isacommand{lemma}\isamarkupfalse%
\ {\isachardoublequoteopen}{\isasymlfloor}{\isacharparenleft}{\isasymlambda}x\ w{\isachardot}\ x\ {\isasymnoteq}\ x{\isacharparenright}\ {\isasymRrightarrow}\ {\isasymchi}{\isasymrfloor}{\isachardoublequoteclose}%
\ %
%
\isacommand{by}\isamarkupfalse%
\ simp\ %
\isamarkupcmt{an impossible property entails anything%
}
%
%
%
\isanewline
\isacommand{lemma}\isamarkupfalse%
\ {\isachardoublequoteopen}{\isasymlfloor}\isactrlbold {\isasymnot}{\isacharparenleft}{\isasymphi}\ {\isasymRrightarrow}\ {\isasymchi}{\isacharparenright}\ \isactrlbold {\isasymrightarrow}\ \isactrlbold {\isasymdiamond}\isactrlbold {\isasymexists}\isactrlsup E\ {\isasymphi}{\isasymrfloor}{\isachardoublequoteclose}%
\ %
%
\isacommand{by}\isamarkupfalse%
\ auto\ %
\isamarkupcmt{possible instantiation of \isa{{\isasymphi}} implicit%
}
%
%
%
%
\begin{isamarkuptext}%
The definition of property entailment introduced by G\"odel can be criticized on the grounds that it lacks
 some notion of relevance and is therefore exposed to the paradoxes of material implication.
 In particular, when we assert that property A does not entail property B, we implicitly assume that
 A is possibly instantiated. Conversely, an impossible property (like being a round square) entails any property
 (like being a triangle). It is precisely by virtue of these paradoxes that G\"odel manages to prove \emph{T1}.
 \footnote{When proving T1 we need to use the fact that positive properties cannot \emph{entail} negative ones (A2), 
 from which the possible instantiation of positive properties follow. 
 A computer-friendly formalization of Leibniz's Theory of Concepts can be found in the work of \cite{Zalta_Leibniz}
 where the notion of \emph{concept containment} in contrast to ordinary \emph{property entailment} is discussed at some length.}%
\end{isamarkuptext}\isamarkuptrue%
\isacommand{consts}\isamarkupfalse%
\ Positiveness{\isacharcolon}{\isacharcolon}{\isachardoublequoteopen}{\isasymup}{\isasymlangle}{\isasymup}{\isasymlangle}{\isasymzero}{\isasymrangle}{\isasymrangle}{\isachardoublequoteclose}\ {\isacharparenleft}{\isachardoublequoteopen}{\isasymP}{\isachardoublequoteclose}{\isacharparenright}\ %
\isamarkupcmt{positiveness applies to intensional predicates%
}
\isanewline
\isacommand{abbreviation}\isamarkupfalse%
\ Existence{\isacharcolon}{\isacharcolon}{\isachardoublequoteopen}{\isasymup}{\isasymlangle}{\isasymzero}{\isasymrangle}{\isachardoublequoteclose}\ {\isacharparenleft}{\isachardoublequoteopen}E{\isacharbang}{\isachardoublequoteclose}{\isacharparenright}\ %
\isamarkupcmt{object-language existence predicate%
}
\ \isanewline
\ \ \isakeyword{where}\ {\isachardoublequoteopen}E{\isacharbang}\ x\ \ {\isasymequiv}\ {\isasymlambda}w{\isachardot}\ {\isacharparenleft}\isactrlbold {\isasymexists}\isactrlsup Ey{\isachardot}\ y\isactrlbold {\isasymapprox}x{\isacharparenright}\ w{\isachardoublequoteclose}\ \isanewline
\isacommand{abbreviation}\isamarkupfalse%
\ appliesToPositiveProps{\isacharcolon}{\isacharcolon}{\isachardoublequoteopen}{\isasymup}{\isasymlangle}{\isasymup}{\isasymlangle}{\isasymup}{\isasymlangle}{\isasymzero}{\isasymrangle}{\isasymrangle}{\isasymrangle}{\isachardoublequoteclose}\ {\isacharparenleft}{\isachardoublequoteopen}pos{\isachardoublequoteclose}{\isacharparenright}\ \isakeyword{where}\isanewline
\ \ {\isachardoublequoteopen}pos\ Z\ {\isasymequiv}\ \ \isactrlbold {\isasymforall}X{\isachardot}\ Z\ X\ \isactrlbold {\isasymrightarrow}\ {\isasymP}\ X{\isachardoublequoteclose}\ \ \isanewline
\isacommand{abbreviation}\isamarkupfalse%
\ intersectionOf{\isacharcolon}{\isacharcolon}{\isachardoublequoteopen}{\isasymup}{\isasymlangle}{\isasymup}{\isasymlangle}{\isasymzero}{\isasymrangle}{\isacharcomma}{\isasymup}{\isasymlangle}{\isasymup}{\isasymlangle}{\isasymzero}{\isasymrangle}{\isasymrangle}{\isasymrangle}{\isachardoublequoteclose}\ {\isacharparenleft}{\isachardoublequoteopen}intersec{\isachardoublequoteclose}{\isacharparenright}\ \isakeyword{where}\isanewline
\ \ {\isachardoublequoteopen}intersec\ X\ Z\ {\isasymequiv}\ \ \isactrlbold {\isasymbox}{\isacharparenleft}\isactrlbold {\isasymforall}x{\isachardot}{\isacharparenleft}X\ x\ \isactrlbold {\isasymleftrightarrow}\ {\isacharparenleft}\isactrlbold {\isasymforall}Y{\isachardot}\ {\isacharparenleft}Z\ Y{\isacharparenright}\ \isactrlbold {\isasymrightarrow}\ {\isacharparenleft}Y\ x{\isacharparenright}{\isacharparenright}{\isacharparenright}{\isacharparenright}{\isachardoublequoteclose}\ \ \isanewline
\ \ \isanewline
\isacommand{axiomatization}\isamarkupfalse%
\ \isakeyword{where}\isanewline
\ \ A{\isadigit{1}}a{\isacharcolon}{\isachardoublequoteopen}{\isasymlfloor}\isactrlbold {\isasymforall}X{\isachardot}\ {\isasymP}\ {\isacharparenleft}\isactrlbold {\isasymrightharpoondown}X{\isacharparenright}\ \isactrlbold {\isasymrightarrow}\ \isactrlbold {\isasymnot}{\isacharparenleft}{\isasymP}\ X{\isacharparenright}\ {\isasymrfloor}{\isachardoublequoteclose}\ \isakeyword{and}\ \ \ \ \ \ %
\isamarkupcmt{axiom 11.3A%
}
\isanewline
\ \ A{\isadigit{1}}b{\isacharcolon}{\isachardoublequoteopen}{\isasymlfloor}\isactrlbold {\isasymforall}X{\isachardot}\ \isactrlbold {\isasymnot}{\isacharparenleft}{\isasymP}\ X{\isacharparenright}\ \isactrlbold {\isasymrightarrow}\ {\isasymP}\ {\isacharparenleft}\isactrlbold {\isasymrightharpoondown}X{\isacharparenright}{\isasymrfloor}{\isachardoublequoteclose}\ \isakeyword{and}\ \ \ \ \ \ \ %
\isamarkupcmt{axiom 11.3B%
}
\isanewline
\ \ A{\isadigit{2}}{\isacharcolon}\ {\isachardoublequoteopen}{\isasymlfloor}\isactrlbold {\isasymforall}X\ Y{\isachardot}\ {\isacharparenleft}{\isasymP}\ X\ \isactrlbold {\isasymand}\ {\isacharparenleft}X\ {\isasymRrightarrow}\ Y{\isacharparenright}{\isacharparenright}\ \isactrlbold {\isasymrightarrow}\ {\isasymP}\ Y{\isasymrfloor}{\isachardoublequoteclose}\ \isakeyword{and}\ \ \ %
\isamarkupcmt{axiom 11.5%
}
\isanewline
\ \ A{\isadigit{3}}{\isacharcolon}\ {\isachardoublequoteopen}{\isasymlfloor}\isactrlbold {\isasymforall}Z\ X{\isachardot}\ {\isacharparenleft}pos\ Z\ \isactrlbold {\isasymand}\ intersec\ X\ Z{\isacharparenright}\ \isactrlbold {\isasymrightarrow}\ {\isasymP}\ X{\isasymrfloor}{\isachardoublequoteclose}\ %
\isamarkupcmt{axiom 11.10%
}
\isanewline
\isanewline
\isacommand{lemma}\isamarkupfalse%
\ True\ \isacommand{nitpick}\isamarkupfalse%
{\isacharbrackleft}satisfy{\isacharbrackright}%
\ %
%
\isacommand{oops}\isamarkupfalse%
\ \ \ \ %
\isamarkupcmt{model found: axioms are consistent%
}
%
%
%
\isanewline
\isacommand{lemma}\isamarkupfalse%
\ {\isachardoublequoteopen}{\isasymlfloor}D{\isasymrfloor}{\isachardoublequoteclose}%
\ \ %
%
\isacommand{using}\isamarkupfalse%
\ A{\isadigit{1}}a\ A{\isadigit{1}}b\ A{\isadigit{2}}\ \isacommand{by}\isamarkupfalse%
\ blast\ %
\isamarkupcmt{\emph{D} axiom is implicitely assumed%
}
%
%
%
%
\begin{isamarkuptext}%
Positive properties are possibly instantiated.%
\end{isamarkuptext}\isamarkuptrue%
\isacommand{theorem}\isamarkupfalse%
\ T{\isadigit{1}}{\isacharcolon}\ {\isachardoublequoteopen}{\isasymlfloor}\isactrlbold {\isasymforall}X{\isachardot}\ {\isasymP}\ X\ \isactrlbold {\isasymrightarrow}\ \isactrlbold {\isasymdiamond}\isactrlbold {\isasymexists}\isactrlsup E\ X{\isasymrfloor}{\isachardoublequoteclose}%
\ %
%
\isacommand{using}\isamarkupfalse%
\ A{\isadigit{1}}a\ A{\isadigit{2}}\ \isacommand{by}\isamarkupfalse%
\ blast%
%
%
%
\begin{isamarkuptext}%
Being Godlike is defined as having all (and only) positive properties.%
\end{isamarkuptext}\isamarkuptrue%
\isacommand{abbreviation}\isamarkupfalse%
\ God{\isacharcolon}{\isacharcolon}{\isachardoublequoteopen}{\isasymup}{\isasymlangle}{\isasymzero}{\isasymrangle}{\isachardoublequoteclose}\ {\isacharparenleft}{\isachardoublequoteopen}G{\isachardoublequoteclose}{\isacharparenright}\ \isakeyword{where}\ {\isachardoublequoteopen}G\ {\isasymequiv}\ {\isacharparenleft}{\isasymlambda}x{\isachardot}\ \isactrlbold {\isasymforall}Y{\isachardot}\ {\isasymP}\ Y\ \isactrlbold {\isasymrightarrow}\ Y\ x{\isacharparenright}{\isachardoublequoteclose}\isanewline
\isacommand{abbreviation}\isamarkupfalse%
\ God{\isacharunderscore}star{\isacharcolon}{\isacharcolon}{\isachardoublequoteopen}{\isasymup}{\isasymlangle}{\isasymzero}{\isasymrangle}{\isachardoublequoteclose}\ {\isacharparenleft}{\isachardoublequoteopen}G{\isacharasterisk}{\isachardoublequoteclose}{\isacharparenright}\ \isakeyword{where}\ {\isachardoublequoteopen}G{\isacharasterisk}\ {\isasymequiv}\ {\isacharparenleft}{\isasymlambda}x{\isachardot}\ \isactrlbold {\isasymforall}Y{\isachardot}\ {\isasymP}\ Y\ \isactrlbold {\isasymleftrightarrow}\ Y\ x{\isacharparenright}{\isachardoublequoteclose}%
\begin{isamarkuptext}%
Both are equivalent. We can use either one or the other in our proofs.%
\end{isamarkuptext}\isamarkuptrue%
\isacommand{lemma}\isamarkupfalse%
\ GodDefsAreEquivalent{\isacharcolon}\ {\isachardoublequoteopen}{\isasymlfloor}\isactrlbold {\isasymforall}x{\isachardot}\ G\ x\ \isactrlbold {\isasymleftrightarrow}\ G{\isacharasterisk}\ x{\isasymrfloor}{\isachardoublequoteclose}%
\ %
%
\isacommand{using}\isamarkupfalse%
\ A{\isadigit{1}}b\ \isacommand{by}\isamarkupfalse%
\ force%
%
%
%
\begin{isamarkuptext}%
Being Godlike is itself a positive property.\footnote{This theorem can also be axiomatized directly,
as noted by Dana Scott (see \cite{Fitting}, p. 152). We provide here a proof in Isabelle/Isar, a language
specifically tailored for writing proofs that are both computer- and human-readable. Because of space
constraints we can't show the other proofs in this article.}%
\end{isamarkuptext}\isamarkuptrue%
\isacommand{theorem}\isamarkupfalse%
\ T{\isadigit{2}}{\isacharcolon}\ {\isachardoublequoteopen}{\isasymlfloor}{\isasymP}\ G{\isasymrfloor}{\isachardoublequoteclose}%
\ %
%
\isacommand{proof}\isamarkupfalse%
\ {\isacharminus}\isanewline
\isacommand{{\isacharbraceleft}}\isamarkupfalse%
\ \isacommand{fix}\isamarkupfalse%
\ w\isanewline
\ \ \isacommand{have}\isamarkupfalse%
\ {\isadigit{1}}{\isacharcolon}\ {\isachardoublequoteopen}pos\ {\isasymP}\ w{\isachardoublequoteclose}\ \isacommand{by}\isamarkupfalse%
\ simp\isanewline
\ \ \isacommand{have}\isamarkupfalse%
\ {\isadigit{2}}{\isacharcolon}\ {\isachardoublequoteopen}intersec\ G\ {\isasymP}\ w{\isachardoublequoteclose}\ \isacommand{by}\isamarkupfalse%
\ simp\isanewline
\ \ \isacommand{have}\isamarkupfalse%
\ {\isachardoublequoteopen}{\isasymlfloor}\isactrlbold {\isasymforall}Z\ X{\isachardot}\ {\isacharparenleft}pos\ Z\ \isactrlbold {\isasymand}\ intersec\ X\ Z{\isacharparenright}\ \isactrlbold {\isasymrightarrow}\ {\isasymP}\ X{\isasymrfloor}{\isachardoublequoteclose}\ \isacommand{by}\isamarkupfalse%
\ {\isacharparenleft}rule\ A{\isadigit{3}}{\isacharparenright}\isanewline
\ \ \isacommand{hence}\isamarkupfalse%
\ {\isachardoublequoteopen}{\isacharparenleft}\isactrlbold {\isasymforall}Z\ X{\isachardot}\ {\isacharparenleft}pos\ Z\ \isactrlbold {\isasymand}\ intersec\ X\ Z{\isacharparenright}\ \isactrlbold {\isasymrightarrow}\ {\isasymP}\ X{\isacharparenright}\ w{\isachardoublequoteclose}\ \isacommand{by}\isamarkupfalse%
\ {\isacharparenleft}rule\ allE{\isacharparenright}\isanewline
\ \ \isacommand{hence}\isamarkupfalse%
\ {\isachardoublequoteopen}{\isacharparenleft}\isactrlbold {\isasymforall}X{\isachardot}\ {\isacharparenleft}{\isacharparenleft}pos\ {\isasymP}{\isacharparenright}\ \isactrlbold {\isasymand}\ {\isacharparenleft}intersec\ X\ {\isasymP}{\isacharparenright}{\isacharparenright}\ \isactrlbold {\isasymrightarrow}\ {\isasymP}\ X{\isacharparenright}\ w{\isachardoublequoteclose}\ \ \isacommand{by}\isamarkupfalse%
\ {\isacharparenleft}rule\ allE{\isacharparenright}\ \ \ \isanewline
\ \ \isacommand{hence}\isamarkupfalse%
\ {\isachardoublequoteopen}{\isacharparenleft}{\isacharparenleft}{\isacharparenleft}pos\ {\isasymP}{\isacharparenright}\ \isactrlbold {\isasymand}\ {\isacharparenleft}intersec\ G\ {\isasymP}{\isacharparenright}{\isacharparenright}\ \isactrlbold {\isasymrightarrow}\ {\isasymP}\ G{\isacharparenright}\ w{\isachardoublequoteclose}\ \isacommand{by}\isamarkupfalse%
\ {\isacharparenleft}rule\ allE{\isacharparenright}\isanewline
\ \ \isacommand{hence}\isamarkupfalse%
\ {\isadigit{3}}{\isacharcolon}\ {\isachardoublequoteopen}{\isacharparenleft}{\isacharparenleft}pos\ {\isasymP}\ \isactrlbold {\isasymand}\ intersec\ G\ {\isasymP}{\isacharparenright}\ w{\isacharparenright}\ {\isasymlongrightarrow}\ {\isasymP}\ G\ w{\isachardoublequoteclose}\ \isacommand{by}\isamarkupfalse%
\ simp\isanewline
\ \ \isacommand{hence}\isamarkupfalse%
\ {\isadigit{4}}{\isacharcolon}\ {\isachardoublequoteopen}{\isacharparenleft}{\isacharparenleft}pos\ {\isasymP}{\isacharparenright}\ \isactrlbold {\isasymand}\ {\isacharparenleft}intersec\ G\ {\isasymP}{\isacharparenright}{\isacharparenright}\ w{\isachardoublequoteclose}\ \isacommand{using}\isamarkupfalse%
\ {\isadigit{1}}\ {\isadigit{2}}\ \isacommand{by}\isamarkupfalse%
\ simp\isanewline
\ \ \isacommand{from}\isamarkupfalse%
\ {\isadigit{3}}\ {\isadigit{4}}\ \isacommand{have}\isamarkupfalse%
\ {\isachardoublequoteopen}{\isasymP}\ G\ w{\isachardoublequoteclose}\ \isacommand{by}\isamarkupfalse%
\ {\isacharparenleft}rule\ mp{\isacharparenright}\isanewline
\isacommand{{\isacharbraceright}}\isamarkupfalse%
\ \isacommand{thus}\isamarkupfalse%
\ {\isacharquery}thesis\ \isacommand{by}\isamarkupfalse%
\ {\isacharparenleft}rule\ allI{\isacharparenright}\isanewline
\isacommand{qed}\isamarkupfalse%
%
%
%
%
\begin{isamarkuptext}%
Conclusion for the first part: Possibly God exists.%
\end{isamarkuptext}\isamarkuptrue%
\isacommand{theorem}\isamarkupfalse%
\ T{\isadigit{3}}{\isacharcolon}\ {\isachardoublequoteopen}{\isasymlfloor}\isactrlbold {\isasymdiamond}\isactrlbold {\isasymexists}\isactrlsup E\ G{\isasymrfloor}{\isachardoublequoteclose}%
\ \ %
%
\isacommand{using}\isamarkupfalse%
\ T{\isadigit{1}}\ T{\isadigit{2}}\ \isacommand{by}\isamarkupfalse%
\ simp%
%
%
%
\isamarkupsubsection{Part II - God's Existence is Necessary, if Possible%
}
\isamarkuptrue%
%
\begin{isamarkuptext}%
In this part we show that God's necessary existence follows from its possible existence by adding some
 additional (philosophically controversial) assumptions including an \emph{essentialist} premise 
 and the \emph{S5} axioms. Further derived results like monotheism and absence of free will are also discussed.%
\end{isamarkuptext}\isamarkuptrue%
\isacommand{axiomatization}\isamarkupfalse%
\ \isakeyword{where}\ A{\isadigit{4}}a{\isacharcolon}\ {\isachardoublequoteopen}{\isasymlfloor}\isactrlbold {\isasymforall}X{\isachardot}\ {\isasymP}\ X\ \isactrlbold {\isasymrightarrow}\ \isactrlbold {\isasymbox}{\isacharparenleft}{\isasymP}\ X{\isacharparenright}{\isasymrfloor}{\isachardoublequoteclose}%
\begin{isamarkuptext}%
Following lemma was originally assumed by G\"odel as an axiom:%
\end{isamarkuptext}\isamarkuptrue%
\isacommand{lemma}\isamarkupfalse%
\ A{\isadigit{4}}b{\isacharcolon}\ {\isachardoublequoteopen}{\isasymlfloor}\isactrlbold {\isasymforall}X{\isachardot}\ \isactrlbold {\isasymnot}{\isacharparenleft}{\isasymP}\ X{\isacharparenright}\ \isactrlbold {\isasymrightarrow}\ \isactrlbold {\isasymbox}\isactrlbold {\isasymnot}{\isacharparenleft}{\isasymP}\ X{\isacharparenright}{\isasymrfloor}{\isachardoublequoteclose}%
\ %
%
\isacommand{using}\isamarkupfalse%
\ A{\isadigit{1}}a\ A{\isadigit{1}}b\ A{\isadigit{4}}a\ \isacommand{by}\isamarkupfalse%
\ blast%
%
%
\isanewline
\isacommand{lemma}\isamarkupfalse%
\ True\ \isacommand{nitpick}\isamarkupfalse%
{\isacharbrackleft}satisfy{\isacharbrackright}%
\ %
%
\isacommand{oops}\isamarkupfalse%
\ %
\isamarkupcmt{model found: all axioms A1-4 consistent%
}
%
%
%
%
\begin{isamarkuptext}%
Axiom \emph{A4a} and its consequence \emph{A4b} together imply that \isa{{\isasymP}} satisfies Fitting's
`stability conditions' (\cite{Fitting}, p. 124). This means \isa{{\isasymP}} designates rigidly.
Note that this makes for an \emph{essentialist} assumption which may be considered controversial by
some philosophers: every property considered positive in our world (e.g. honesty) is necessarily so.%
\end{isamarkuptext}\isamarkuptrue%
\isacommand{lemma}\isamarkupfalse%
\ {\isachardoublequoteopen}{\isasymlfloor}rigidPred\ {\isasymP}{\isasymrfloor}{\isachardoublequoteclose}%
\ %
%
\isacommand{using}\isamarkupfalse%
\ A{\isadigit{4}}a\ A{\isadigit{4}}b\ \isacommand{by}\isamarkupfalse%
\ blast%
%
%
%
\begin{isamarkuptext}%
G\"odel defines a particular notion of essence. \emph{Y} is an essence of \emph{x} iff \emph{Y}
\emph{entails} every other property \emph{x} posseses.
\footnote{Essence is defined here (and in Fitting's variant) in the version of Scott; G\"odel's original version
 leads to the inconsistency reported in \cite{C55,C60}}%
\end{isamarkuptext}\isamarkuptrue%
\isacommand{abbreviation}\isamarkupfalse%
\ essenceOf{\isacharcolon}{\isacharcolon}{\isachardoublequoteopen}{\isasymup}{\isasymlangle}{\isasymup}{\isasymlangle}{\isasymzero}{\isasymrangle}{\isacharcomma}{\isasymzero}{\isasymrangle}{\isachardoublequoteclose}\ {\isacharparenleft}{\isachardoublequoteopen}{\isasymE}{\isachardoublequoteclose}{\isacharparenright}\ \isakeyword{where}\isanewline
\ \ {\isachardoublequoteopen}{\isasymE}\ Y\ x\ {\isasymequiv}\ {\isacharparenleft}Y\ x{\isacharparenright}\ \isactrlbold {\isasymand}\ {\isacharparenleft}\isactrlbold {\isasymforall}Z{\isachardot}\ Z\ x\ \isactrlbold {\isasymrightarrow}\ Y\ {\isasymRrightarrow}\ Z{\isacharparenright}{\isachardoublequoteclose}\ \ \ \isanewline
\isacommand{abbreviation}\isamarkupfalse%
\ beingIdenticalTo{\isacharcolon}{\isacharcolon}{\isachardoublequoteopen}{\isasymzero}{\isasymRightarrow}{\isasymup}{\isasymlangle}{\isasymzero}{\isasymrangle}{\isachardoublequoteclose}\ {\isacharparenleft}{\isachardoublequoteopen}id{\isachardoublequoteclose}{\isacharparenright}\ \isakeyword{where}\isanewline
\ \ {\isachardoublequoteopen}id\ x\ \ {\isasymequiv}\ {\isacharparenleft}{\isasymlambda}y{\isachardot}\ y\isactrlbold {\isasymapprox}x{\isacharparenright}{\isachardoublequoteclose}\ %
\isamarkupcmt{\emph{id} is here a rigid predicate (following Kripke \cite{kripke1980})%
}
%
\begin{isamarkuptext}%
Being God-like is an essential property:%
\end{isamarkuptext}\isamarkuptrue%
\isacommand{theorem}\isamarkupfalse%
\ GodIsEssential{\isacharcolon}\ {\isachardoublequoteopen}{\isasymlfloor}\isactrlbold {\isasymforall}x{\isachardot}\ G\ x\ \isactrlbold {\isasymrightarrow}\ {\isacharparenleft}{\isasymE}\ G\ x{\isacharparenright}{\isasymrfloor}{\isachardoublequoteclose}%
\ %
%
\isacommand{using}\isamarkupfalse%
\ A{\isadigit{1}}b\ A{\isadigit{4}}a\ \isacommand{by}\isamarkupfalse%
\ metis%
%
%
%
\begin{isamarkuptext}%
Something can only have \emph{one} essence:%
\end{isamarkuptext}\isamarkuptrue%
\isacommand{theorem}\isamarkupfalse%
\ {\isachardoublequoteopen}{\isasymlfloor}\isactrlbold {\isasymforall}X\ Y\ z{\isachardot}\ {\isacharparenleft}{\isasymE}\ X\ z\ \isactrlbold {\isasymand}\ {\isasymE}\ Y\ z{\isacharparenright}\ \isactrlbold {\isasymrightarrow}\ {\isacharparenleft}X\ {\isasymRrightarrow}\ Y{\isacharparenright}{\isasymrfloor}{\isachardoublequoteclose}%
\ %
%
\isacommand{by}\isamarkupfalse%
\ meson%
%
%
%
\begin{isamarkuptext}%
An essential property offers a complete characterization of an individual:%
\end{isamarkuptext}\isamarkuptrue%
\isacommand{theorem}\isamarkupfalse%
\ EssencesCharacterizeCompletely{\isacharcolon}\ {\isachardoublequoteopen}{\isasymlfloor}\isactrlbold {\isasymforall}X\ y{\isachardot}\ {\isasymE}\ X\ y\ \isactrlbold {\isasymrightarrow}\ {\isacharparenleft}X\ {\isasymRrightarrow}\ {\isacharparenleft}id\ y{\isacharparenright}{\isacharparenright}{\isasymrfloor}{\isachardoublequoteclose}\isanewline
%
\ \ %
%
\isacommand{proof}\isamarkupfalse%
\ {\isacharparenleft}rule\ ccontr{\isacharparenright}\ %
\isamarkupcmt{Isar proof by contradiction not shown here%
}
%
%
%
%
\begin{isamarkuptext}%
G\"odel introduces a particular notion of \emph{necessary existence} as the property something has
 provided any essence of it is necessarily instantiated:%
\end{isamarkuptext}\isamarkuptrue%
\isacommand{abbreviation}\isamarkupfalse%
\ necessaryExistencePredicate{\isacharcolon}{\isacharcolon}{\isachardoublequoteopen}{\isasymup}{\isasymlangle}{\isasymzero}{\isasymrangle}{\isachardoublequoteclose}\ {\isacharparenleft}{\isachardoublequoteopen}NE{\isachardoublequoteclose}{\isacharparenright}\ \isanewline
\ \ \isakeyword{where}\ {\isachardoublequoteopen}NE\ x\ \ {\isasymequiv}\ {\isacharparenleft}{\isasymlambda}w{\isachardot}\ {\isacharparenleft}\isactrlbold {\isasymforall}Y{\isachardot}\ \ {\isasymE}\ Y\ x\ \isactrlbold {\isasymrightarrow}\ \isactrlbold {\isasymbox}\isactrlbold {\isasymexists}\isactrlsup E\ Y{\isacharparenright}\ w{\isacharparenright}{\isachardoublequoteclose}\isanewline
\isanewline
\isacommand{axiomatization}\isamarkupfalse%
\ \isakeyword{where}\ A{\isadigit{5}}{\isacharcolon}\ {\isachardoublequoteopen}{\isasymlfloor}{\isasymP}\ NE{\isasymrfloor}{\isachardoublequoteclose}\ %
\isamarkupcmt{necessary existence is a positive property%
}
\isanewline
\isacommand{lemma}\isamarkupfalse%
\ True\ \isacommand{nitpick}\isamarkupfalse%
{\isacharbrackleft}satisfy{\isacharbrackright}%
\ %
%
\isacommand{oops}\isamarkupfalse%
\ %
\isamarkupcmt{model found: so far all axioms consistent%
}
%
%
%
%
\begin{isamarkuptext}%
(Possibilist) existence of God implies its necessary (actualist) existence:%
\end{isamarkuptext}\isamarkuptrue%
\isacommand{theorem}\isamarkupfalse%
\ GodExistenceImpliesNecEx{\isacharcolon}\ {\isachardoublequoteopen}{\isasymlfloor}\isactrlbold {\isasymexists}\ G\ \isactrlbold {\isasymrightarrow}\ \isactrlbold {\isasymbox}\isactrlbold {\isasymexists}\isactrlsup E\ G{\isasymrfloor}{\isachardoublequoteclose}%
\ %
%
\isacommand{proof}\isamarkupfalse%
\ {\isacharminus}\ %
\isamarkupcmt{not shown%
}
%
%
%
%
\begin{isamarkuptext}%
Below we postulate semantic frame conditions for some modal logics. \footnote{Taken together, reflexivity, transitivity and symmetry
 make for an equivalence relation and therefore an \emph{S5} logic (via \emph{Sahlqvist correspondence}).
 They are individually postulated in order to get more detailed information about their relevance in the proofs presented below.}%
\end{isamarkuptext}\isamarkuptrue%
\isacommand{axiomatization}\isamarkupfalse%
\ \isakeyword{where}\isanewline
\ refl{\isacharcolon}\ {\isachardoublequoteopen}reflexive\ aRel{\isachardoublequoteclose}\ \isakeyword{and}\ tran{\isacharcolon}\ {\isachardoublequoteopen}transitive\ aRel{\isachardoublequoteclose}\ \isakeyword{and}\ symm{\isacharcolon}\ {\isachardoublequoteopen}symmetric\ aRel{\isachardoublequoteclose}\isanewline
\ \isanewline
\isacommand{lemma}\isamarkupfalse%
\ True\ \isacommand{nitpick}\isamarkupfalse%
{\isacharbrackleft}satisfy{\isacharbrackright}%
\ %
%
\isacommand{oops}\isamarkupfalse%
\ %
\isamarkupcmt{model found: axioms still consistent%
}
%
%
%
%
%
%
%
%
%
%
%
%
%
%
%
%
\begin{isamarkuptext}%
Possible existence of God implies its necessary (actualist) existence (note that only symmetry and
transitivity are needed as frame conditions):%
\end{isamarkuptext}\isamarkuptrue%
\isacommand{theorem}\isamarkupfalse%
\ T{\isadigit{4}}{\isacharcolon}\ {\isachardoublequoteopen}{\isasymlfloor}\isactrlbold {\isasymdiamond}\isactrlbold {\isasymexists}\ G{\isasymrfloor}\ {\isasymlongrightarrow}\ {\isasymlfloor}\isactrlbold {\isasymbox}\isactrlbold {\isasymexists}\isactrlsup E\ G{\isasymrfloor}{\isachardoublequoteclose}%
\ %
%
\isacommand{proof}\isamarkupfalse%
\ {\isacharminus}\ %
\isamarkupcmt{not shown%
}
%
%
%
%
\begin{isamarkuptext}%
Conclusion: Necessary (actualist) existence of God:%
\end{isamarkuptext}\isamarkuptrue%
\isacommand{theorem}\isamarkupfalse%
\ GodNecExists{\isacharcolon}\ {\isachardoublequoteopen}{\isasymlfloor}\isactrlbold {\isasymbox}\isactrlbold {\isasymexists}\isactrlsup E\ G{\isasymrfloor}{\isachardoublequoteclose}%
\ %
%
\isacommand{using}\isamarkupfalse%
\ T{\isadigit{3}}\ T{\isadigit{4}}\ \isacommand{by}\isamarkupfalse%
\ metis%
%
%
%
\begin{isamarkuptext}%
To prove validity we still need reflexivity for our frame conditions:%
\end{isamarkuptext}\isamarkuptrue%
\isacommand{theorem}\isamarkupfalse%
\ GodExistenceIsValid{\isacharcolon}\ {\isachardoublequoteopen}{\isasymlfloor}\isactrlbold {\isasymexists}\isactrlsup E\ G{\isasymrfloor}{\isachardoublequoteclose}%
\ %
%
\isacommand{using}\isamarkupfalse%
\ GodNecExists\ refl\ \isacommand{by}\isamarkupfalse%
\ auto%
%
%
%
\begin{isamarkuptext}%
Monotheism for non-normal models (using Leibniz equality) follows directly from God having all
and only positive properties, but the proof for normal models is trickier. We need to consider previous results
 (\cite{Fitting}, p. 162):%
\end{isamarkuptext}\isamarkuptrue%
\isacommand{theorem}\isamarkupfalse%
\ Monotheism{\isacharunderscore}LeibnizEq{\isacharcolon}{\isachardoublequoteopen}{\isasymlfloor}\isactrlbold {\isasymforall}x{\isachardot}\ G{\isacharasterisk}\ x\ \isactrlbold {\isasymrightarrow}\ {\isacharparenleft}\isactrlbold {\isasymforall}y{\isachardot}\ G{\isacharasterisk}\ y\ \isactrlbold {\isasymrightarrow}\ x\isactrlbold {\isasymapprox}\isactrlsup Ly{\isacharparenright}{\isasymrfloor}{\isachardoublequoteclose}%
\ %
%
\isacommand{by}\isamarkupfalse%
\ meson%
%
%
\isanewline
\isacommand{theorem}\isamarkupfalse%
\ Monotheism{\isacharunderscore}normal{\isacharcolon}\ {\isachardoublequoteopen}{\isasymlfloor}\isactrlbold {\isasymexists}x{\isachardot}\isactrlbold {\isasymforall}y{\isachardot}\ G\ y\ \isactrlbold {\isasymleftrightarrow}\ x\ \isactrlbold {\isasymapprox}\ y{\isasymrfloor}{\isachardoublequoteclose}%
\ %
%
\isacommand{proof}\isamarkupfalse%
\ {\isacharminus}\ %
\isamarkupcmt{not shown%
}
%
%
%
%
%
%
%
%
%
%
\begin{isamarkuptext}%
Fitting \cite{Fitting} also discusses the objection raised by Sobel \cite{sobel2004logic}, who argues that G\"odel's axiom system
is too strong: it implies that whatever is the case is so necessarily, i.e. the modal system collapses (\isa{{\isasymphi}\ {\isasymlongrightarrow}\ {\isasymbox}{\isasymphi}}).
This has been philosophically interpreted as implying the absence of free will.    
In the context of our S5 axioms, the \emph{modal collapse} becomes valid (\cite{Fitting}, pp. 163-4):%
\end{isamarkuptext}\isamarkuptrue%
\isacommand{theorem}\isamarkupfalse%
\ ModalCollapse{\isacharcolon}\ {\isachardoublequoteopen}{\isasymlfloor}\isactrlbold {\isasymforall}{\isasymPhi}{\isachardot}{\isacharparenleft}{\isasymPhi}\ \isactrlbold {\isasymrightarrow}\ {\isacharparenleft}\isactrlbold {\isasymbox}\ {\isasymPhi}{\isacharparenright}{\isacharparenright}{\isasymrfloor}{\isachardoublequoteclose}%
\ %
%
\isacommand{proof}\isamarkupfalse%
\ {\isacharminus}\ %
\isamarkupcmt{not shown here%
}
%
%
%
%
%
%
%
%
%
%
\end{isabellebody}%
%%% Local Variables:
%%% mode: latex
%%% TeX-master: "root"
%%% End:
