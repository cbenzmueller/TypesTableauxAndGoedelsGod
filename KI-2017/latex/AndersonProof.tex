%
\begin{isabellebody}%
\setisabellecontext{AndersonProof}%
%
%
%
%
%
%
%
\isamarkupsection{Anderson's Variant%
}
\isamarkuptrue%
%
\begin{isamarkuptext}%
\noindent{In this section, we verify Anderson's emendation of G\"odel's argument \cite{anderson90},
 as presented by Fitting (\cite{Fitting}, pp. 169-171). In the previous variants there were no `indifferent' properties,
 either a property or its negation had to be positive. Anderson makes room for `indifferent' properties by
 dropping axiom \emph{A1b} (\isa{{\isasymlfloor}\isactrlbold {\isasymforall}X{\isachardot}\ \isactrlbold {\isasymnot}{\isacharparenleft}{\isasymP}\ X{\isacharparenright}\ \isactrlbold {\isasymrightarrow}\ {\isasymP}\ {\isacharparenleft}{\isasymrightharpoondown}X{\isacharparenright}{\isasymrfloor}}). As a consequence, he changes the following definitions
 to ensure argument's validity.}%
\end{isamarkuptext}\isamarkuptrue%
\ \ \isacommand{abbreviation}\isamarkupfalse%
\ God{\isacharcolon}{\isacharcolon}{\isachardoublequoteopen}{\isasymup}{\isasymlangle}e{\isasymrangle}{\isachardoublequoteclose}\ {\isacharparenleft}{\isachardoublequoteopen}G{\isachardoublequoteclose}{\isacharparenright}\ \isakeyword{where}\ {\isachardoublequoteopen}G\ {\isasymequiv}\ {\isasymlambda}x{\isachardot}\ \isactrlbold {\isasymforall}Y{\isachardot}\ {\isacharparenleft}{\isasymP}\ Y{\isacharparenright}\ \isactrlbold {\isasymleftrightarrow}\ \isactrlbold {\isasymbox}{\isacharparenleft}Y\ x{\isacharparenright}{\isachardoublequoteclose}\isanewline
\ \ \isacommand{abbreviation}\isamarkupfalse%
\ Essence{\isacharcolon}{\isacharcolon}{\isachardoublequoteopen}{\isasymup}{\isasymlangle}{\isasymup}{\isasymlangle}e{\isasymrangle}{\isacharcomma}e{\isasymrangle}{\isachardoublequoteclose}\ {\isacharparenleft}{\isachardoublequoteopen}{\isasymE}{\isachardoublequoteclose}{\isacharparenright}\ \isakeyword{where}\ {\isachardoublequoteopen}{\isasymE}\ Y\ x\ {\isasymequiv}\ {\isacharparenleft}\isactrlbold {\isasymforall}Z{\isachardot}\ \isactrlbold {\isasymbox}{\isacharparenleft}Z\ x{\isacharparenright}\ \isactrlbold {\isasymleftrightarrow}\ Y\ {\isasymRrightarrow}\ Z{\isacharparenright}{\isachardoublequoteclose}%
\begin{isamarkuptext}%
\noindent{There is now the requirement that a Godlike being must have positive properties \emph{necessarily}.
  For the definition of essence, Scott's addition \cite{ScottNotes}, that the essence of an object 
  actually applies to the object, is dropped. A necessity operator has been introduced instead.\footnote{G\"odel's
  original axioms (without Scott's addition) are proven inconsistent in \cite{C55}.}}%
\end{isamarkuptext}\isamarkuptrue%
%
\begin{isamarkuptext}%
\noindent{The rest of the argument is essentially similar to G\"odel's (also in \emph{S5} logic).}%
\end{isamarkuptext}\isamarkuptrue%
\ \ \isacommand{theorem}\isamarkupfalse%
\ T{\isadigit{1}}{\isacharcolon}\ {\isachardoublequoteopen}{\isasymlfloor}\isactrlbold {\isasymforall}X{\isachardot}\ {\isasymP}\ X\ \isactrlbold {\isasymrightarrow}\ \isactrlbold {\isasymdiamond}\isactrlbold {\isasymexists}\isactrlsup A\ X{\isasymrfloor}{\isachardoublequoteclose}%
\ %
%
\isacommand{using}\isamarkupfalse%
\ A{\isadigit{1}}a\ A{\isadigit{2}}\ \isacommand{by}\isamarkupfalse%
\ blast%
%
%
\isanewline
\ \ \isacommand{theorem}\isamarkupfalse%
\ T{\isadigit{3}}{\isacharcolon}\ {\isachardoublequoteopen}{\isasymlfloor}\isactrlbold {\isasymdiamond}\isactrlbold {\isasymexists}\isactrlsup A\ G{\isasymrfloor}{\isachardoublequoteclose}%
\ %
%
\isacommand{using}\isamarkupfalse%
\ T{\isadigit{1}}\ T{\isadigit{2}}\ \isacommand{by}\isamarkupfalse%
\ simp%
%
%
%
%
%
%
%
%
%
%
%
%
%
%
%
%
%
%
%
%
%
\begin{isamarkuptext}%
\noindent{If g is Godlike, the property of being Godlike is its essence.\footnote{This theorem's proof
  could be completely automatized for G\"odel's and Fitting's variants.
  For Anderson's version however, we had to reproduce in Isabelle/HOL the original natural-language proof 
  given by Anderson (see \cite{anderson90}, Theorem 2*, p. 296)}}%
\end{isamarkuptext}\isamarkuptrue%
\ \ \isacommand{theorem}\isamarkupfalse%
\ GodIsEssential{\isacharcolon}\ {\isachardoublequoteopen}{\isasymlfloor}\isactrlbold {\isasymforall}x{\isachardot}\ G\ x\ \isactrlbold {\isasymrightarrow}\ {\isacharparenleft}{\isasymE}\ G\ x{\isacharparenright}{\isasymrfloor}{\isachardoublequoteclose}%
\ %
%
\isacommand{proof}\isamarkupfalse%
\ {\isacharminus}\ %
\isamarkupcmt{not shown%
}
%
%
%
%
%
%
%
%
%
%
%
%
%
%
%
%
%
%
%
%
%
%
\begin{isamarkuptext}%
\noindent{The necessary existence of God follows from its possible existence.}%
\end{isamarkuptext}\isamarkuptrue%
\ \ \isacommand{theorem}\isamarkupfalse%
\ T{\isadigit{5}}{\isacharcolon}\ {\isachardoublequoteopen}{\isasymlfloor}\isactrlbold {\isasymdiamond}\isactrlbold {\isasymexists}\ G{\isasymrfloor}\ {\isasymlongrightarrow}\ {\isasymlfloor}\isactrlbold {\isasymbox}\isactrlbold {\isasymexists}\isactrlsup A\ G{\isasymrfloor}{\isachardoublequoteclose}%
\ %
%
\isacommand{proof}\isamarkupfalse%
\ {\isacharminus}\ %
\isamarkupcmt{not shown%
}
%
%
%
%
\begin{isamarkuptext}%
\noindent{The conclusion could be proven (with one fewer axiom, though more complex definitions) and
  \emph{Nitpick} is able to find a countermodel for the \emph{modal collapse}.}%
\end{isamarkuptext}\isamarkuptrue%
\ \ \isacommand{lemma}\isamarkupfalse%
\ GodExistsNecessarily{\isacharcolon}\ {\isachardoublequoteopen}{\isasymlfloor}\isactrlbold {\isasymbox}\isactrlbold {\isasymexists}\isactrlsup A\ G{\isasymrfloor}{\isachardoublequoteclose}%
\ %
%
\isacommand{using}\isamarkupfalse%
\ T{\isadigit{3}}\ T{\isadigit{5}}\ \isacommand{by}\isamarkupfalse%
\ metis%
%
%
\isanewline
\ \ \isacommand{lemma}\isamarkupfalse%
\ ModalCollapse{\isacharcolon}\ {\isachardoublequoteopen}{\isasymlfloor}\isactrlbold {\isasymforall}{\isasymPhi}{\isachardot}\ {\isasymPhi}\ \isactrlbold {\isasymrightarrow}\ \isactrlbold {\isasymbox}{\isasymPhi}{\isasymrfloor}{\isachardoublequoteclose}\ \isacommand{nitpick}\isamarkupfalse%
%
\ %
%
\isacommand{oops}\isamarkupfalse%
\ %
\isamarkupcmt{countersatisfiable%
}
%
%
%
%
\isamarkupsection{Conclusion%
}
\isamarkuptrue%
%
\begin{isamarkuptext}%
\noindent{We presented a shallow semantic embedding in Isabelle/HOL for an intensional higher-order modal logic
(a successor of Montague/Gallin intensional logics) and employed this logic to formalize and verify
three different variants of the ontological argument:
the first one by G\"odel himself (resp. Scott), the second one by Fitting and the last one by Anderson.}%
\end{isamarkuptext}\isamarkuptrue%
%
\begin{isamarkuptext}%
\noindent{By employing our embedding of IHOML in Isabelle/HOL, we could not only verify Fitting's results,
but also guarantee consistency of axioms. Moreover, for many theorems we could prove stronger versions
and find better countermodels (i.e. with smaller cardinality) than the ones presented by Fitting.
Another interesting aspect was the possibility to explore the implications of alternative formalizations
of axioms and theorems which shed light on interesting philosophical issues concerning entailment, essentialism and free will.}%
\end{isamarkuptext}\isamarkuptrue%
%
\begin{isamarkuptext}%
\noindent{Latest developments in automated theorem proving, in combination with the embedding approach,
allow us to engage in much better experimentation during the formalization and assessment of arguments than ever before.
The potential reduction (of several orders of magnitude)
in the time needed for proving or disproving theorems (compared to pen-and-paper proofs), results in almost real-time
feedback about the suitability of our speculations. The practical benefits of computer-supported argumentation go beyond
mere quantitative aspects (easier, faster and more reliable proofs). The advantages are also qualitative,
since a significantly different approach to argumentation is fostered: We can now work iteratively (by trial-and-error)
on an argument by making gradual adjustments to its definitions, axioms and theorems. This allows us to continuously
expose and revise the assumptions we indirectly commit ourselves to every time we opt for some particular formalization.
}%
\end{isamarkuptext}\isamarkuptrue%
%
%
%
%
%
%
%
\end{isabellebody}%
%%% Local Variables:
%%% mode: latex
%%% TeX-master: "root"
%%% End:
