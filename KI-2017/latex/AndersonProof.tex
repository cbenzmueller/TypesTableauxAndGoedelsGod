%
\begin{isabellebody}%
\setisabellecontext{AndersonProof}%
%
%
%
%
%
%
%
\isamarkupsection{Anderson's Alternative%
}
\isamarkuptrue%
%
\begin{isamarkuptext}%
In this final section, we verify Anderson's emendation of G\"odel's argument, as it is presented
 by Fitting in \cite{Fitting}, pp. 169-171).%
\end{isamarkuptext}\isamarkuptrue%
\isacommand{abbreviation}\isamarkupfalse%
\ Entailment{\isacharcolon}{\isacharcolon}{\isachardoublequoteopen}{\isasymup}{\isasymlangle}{\isasymup}{\isasymlangle}{\isasymzero}{\isasymrangle}{\isacharcomma}{\isasymup}{\isasymlangle}{\isasymzero}{\isasymrangle}{\isasymrangle}{\isachardoublequoteclose}\ {\isacharparenleft}\isakeyword{infix}\ {\isachardoublequoteopen}{\isasymRrightarrow}{\isachardoublequoteclose}\ {\isadigit{6}}{\isadigit{0}}{\isacharparenright}\ \isakeyword{where}\ %
\isamarkupcmt{def changed%
}
\isanewline
\ \ {\isachardoublequoteopen}X\ {\isasymRrightarrow}\ Y\ {\isasymequiv}\ \ \isactrlbold {\isasymbox}{\isacharparenleft}\isactrlbold {\isasymforall}\isactrlsup Ez{\isachardot}\ X\ z\ \isactrlbold {\isasymrightarrow}\ Y\ z{\isacharparenright}{\isachardoublequoteclose}\ \isanewline
\isacommand{consts}\isamarkupfalse%
\ Positiveness{\isacharcolon}{\isacharcolon}{\isachardoublequoteopen}{\isasymup}{\isasymlangle}{\isasymup}{\isasymlangle}{\isasymzero}{\isasymrangle}{\isasymrangle}{\isachardoublequoteclose}\ {\isacharparenleft}{\isachardoublequoteopen}{\isasymP}{\isachardoublequoteclose}{\isacharparenright}\isanewline
\isacommand{abbreviation}\isamarkupfalse%
\ Existence{\isacharcolon}{\isacharcolon}{\isachardoublequoteopen}{\isasymup}{\isasymlangle}{\isasymzero}{\isasymrangle}{\isachardoublequoteclose}\ {\isacharparenleft}{\isachardoublequoteopen}E{\isacharbang}{\isachardoublequoteclose}{\isacharparenright}\ \isakeyword{where}\ {\isachardoublequoteopen}E{\isacharbang}\ x\ {\isasymequiv}\ {\isasymlambda}w{\isachardot}\ {\isacharparenleft}\isactrlbold {\isasymexists}\isactrlsup Ey{\isachardot}\ y\isactrlbold {\isasymapprox}x{\isacharparenright}\ w{\isachardoublequoteclose}\isanewline
\isacommand{abbreviation}\isamarkupfalse%
\ God{\isacharcolon}{\isacharcolon}{\isachardoublequoteopen}{\isasymup}{\isasymlangle}{\isasymzero}{\isasymrangle}{\isachardoublequoteclose}\ {\isacharparenleft}{\isachardoublequoteopen}G\isactrlsup A{\isachardoublequoteclose}{\isacharparenright}\ \isakeyword{where}\ {\isachardoublequoteopen}G\isactrlsup A\ {\isasymequiv}\ {\isasymlambda}x{\isachardot}\ \isactrlbold {\isasymforall}Y{\isachardot}\ {\isacharparenleft}{\isasymP}\ Y{\isacharparenright}\ \isactrlbold {\isasymleftrightarrow}\ \isactrlbold {\isasymbox}{\isacharparenleft}Y\ x{\isacharparenright}{\isachardoublequoteclose}\ %
\isamarkupcmt{def changed%
}
\isanewline
\isacommand{abbreviation}\isamarkupfalse%
\ essenceOf{\isacharcolon}{\isacharcolon}{\isachardoublequoteopen}{\isasymup}{\isasymlangle}{\isasymup}{\isasymlangle}{\isasymzero}{\isasymrangle}{\isacharcomma}{\isasymzero}{\isasymrangle}{\isachardoublequoteclose}\ {\isacharparenleft}{\isachardoublequoteopen}{\isasymE}\isactrlsup A{\isachardoublequoteclose}{\isacharparenright}\ \isakeyword{where}\ %
\isamarkupcmt{def changed%
}
\isanewline
\ \ {\isachardoublequoteopen}{\isasymE}\isactrlsup A\ Y\ x\ {\isasymequiv}\ {\isacharparenleft}\isactrlbold {\isasymforall}Z{\isachardot}\ \isactrlbold {\isasymbox}{\isacharparenleft}Z\ x{\isacharparenright}\ \isactrlbold {\isasymleftrightarrow}\ Y\ {\isasymRrightarrow}\ Z{\isacharparenright}{\isachardoublequoteclose}\isanewline
\isacommand{abbreviation}\isamarkupfalse%
\ necessaryExistencePred{\isacharcolon}{\isacharcolon}{\isachardoublequoteopen}{\isasymup}{\isasymlangle}{\isasymzero}{\isasymrangle}{\isachardoublequoteclose}\ {\isacharparenleft}{\isachardoublequoteopen}NE\isactrlsup A{\isachardoublequoteclose}{\isacharparenright}\ %
\isamarkupcmt{def changed%
}
\isanewline
\ \ \isakeyword{where}\ {\isachardoublequoteopen}NE\isactrlsup A\ x\ \ {\isasymequiv}\ {\isacharparenleft}{\isasymlambda}w{\isachardot}\ {\isacharparenleft}\isactrlbold {\isasymforall}Y{\isachardot}\ \ {\isasymE}\isactrlsup A\ Y\ x\ \isactrlbold {\isasymrightarrow}\ \isactrlbold {\isasymbox}\isactrlbold {\isasymexists}\isactrlsup E\ Y{\isacharparenright}\ w{\isacharparenright}{\isachardoublequoteclose}\isanewline
\ \ \isanewline
\isacommand{axiomatization}\isamarkupfalse%
\ \isakeyword{where}\isanewline
\ \ A{\isadigit{1}}a{\isacharcolon}{\isachardoublequoteopen}{\isasymlfloor}\isactrlbold {\isasymforall}X{\isachardot}\ {\isasymP}\ {\isacharparenleft}\isactrlbold {\isasymrightharpoondown}X{\isacharparenright}\ \isactrlbold {\isasymrightarrow}\ \isactrlbold {\isasymnot}{\isacharparenleft}{\isasymP}\ X{\isacharparenright}\ {\isasymrfloor}{\isachardoublequoteclose}\ \isakeyword{and}\ \ \ \ \ \ \ \ \ \ \isanewline
\ \ A{\isadigit{2}}{\isacharcolon}\ {\isachardoublequoteopen}{\isasymlfloor}\isactrlbold {\isasymforall}X\ Y{\isachardot}\ {\isacharparenleft}{\isasymP}\ X\ \isactrlbold {\isasymand}\ {\isacharparenleft}X\ {\isasymRrightarrow}\ Y{\isacharparenright}{\isacharparenright}\ \isactrlbold {\isasymrightarrow}\ {\isasymP}\ Y{\isasymrfloor}{\isachardoublequoteclose}\ \isakeyword{and}\ \ \ \isanewline
\ \ T{\isadigit{2}}{\isacharcolon}\ {\isachardoublequoteopen}{\isasymlfloor}{\isasymP}\ G\isactrlsup A{\isasymrfloor}{\isachardoublequoteclose}\ \ \ \ \ \ \ \ \isakeyword{and}\isanewline
\ \ A{\isadigit{4}}a{\isacharcolon}\ {\isachardoublequoteopen}{\isasymlfloor}\isactrlbold {\isasymforall}X{\isachardot}\ {\isasymP}\ X\ \isactrlbold {\isasymrightarrow}\ \isactrlbold {\isasymbox}{\isacharparenleft}{\isasymP}\ X{\isacharparenright}{\isasymrfloor}{\isachardoublequoteclose}\ \ \isakeyword{and}\isanewline
\ \ A{\isadigit{5}}{\isacharcolon}\ {\isachardoublequoteopen}{\isasymlfloor}{\isasymP}\ NE\isactrlsup A{\isasymrfloor}{\isachardoublequoteclose}\ \isanewline
\ \ \isanewline
\isacommand{theorem}\isamarkupfalse%
\ T{\isadigit{1}}{\isacharcolon}\ {\isachardoublequoteopen}{\isasymlfloor}\isactrlbold {\isasymforall}X{\isachardot}\ {\isasymP}\ X\ \isactrlbold {\isasymrightarrow}\ \isactrlbold {\isasymdiamond}\isactrlbold {\isasymexists}\isactrlsup E\ X{\isasymrfloor}{\isachardoublequoteclose}%
\ %
%
\isacommand{using}\isamarkupfalse%
\ A{\isadigit{1}}a\ A{\isadigit{2}}\ \isacommand{by}\isamarkupfalse%
\ blast%
%
%
\isanewline
\isacommand{theorem}\isamarkupfalse%
\ T{\isadigit{3}}{\isacharcolon}\ {\isachardoublequoteopen}{\isasymlfloor}\isactrlbold {\isasymdiamond}\isactrlbold {\isasymexists}\isactrlsup E\ G\isactrlsup A{\isasymrfloor}{\isachardoublequoteclose}%
\ %
%
\isacommand{using}\isamarkupfalse%
\ T{\isadigit{1}}\ T{\isadigit{2}}\ \isacommand{by}\isamarkupfalse%
\ simp%
%
%
\isanewline
\ \ \ \ \ \ \isanewline
\isacommand{axiomatization}\isamarkupfalse%
\ \isakeyword{where}\ %
\isamarkupcmt{We again postulate our \emph{S5} axioms:%
}
\isanewline
\ refl{\isacharcolon}\ {\isachardoublequoteopen}reflexive\ aRel{\isachardoublequoteclose}\ \isakeyword{and}\ tran{\isacharcolon}\ {\isachardoublequoteopen}transitive\ aRel{\isachardoublequoteclose}\ \isakeyword{and}\ symm{\isacharcolon}\ {\isachardoublequoteopen}symmetric\ aRel{\isachardoublequoteclose}\isanewline
\isanewline
\isacommand{lemma}\isamarkupfalse%
\ A{\isadigit{4}}b{\isacharcolon}\ {\isachardoublequoteopen}{\isasymlfloor}\isactrlbold {\isasymforall}X{\isachardot}\ \isactrlbold {\isasymnot}{\isacharparenleft}{\isasymP}\ X{\isacharparenright}\ \isactrlbold {\isasymrightarrow}\ \isactrlbold {\isasymbox}\isactrlbold {\isasymnot}{\isacharparenleft}{\isasymP}\ X{\isacharparenright}{\isasymrfloor}{\isachardoublequoteclose}%
\ %
%
\isacommand{using}\isamarkupfalse%
\ A{\isadigit{4}}a\ symm\ \isacommand{by}\isamarkupfalse%
\ auto%
%
%
\isanewline
\isacommand{lemma}\isamarkupfalse%
\ {\isachardoublequoteopen}{\isasymlfloor}rigidPred\ {\isasymP}{\isasymrfloor}{\isachardoublequoteclose}%
\ %
%
\isacommand{using}\isamarkupfalse%
\ A{\isadigit{4}}a\ A{\isadigit{4}}b\ \isacommand{by}\isamarkupfalse%
\ blast\ %
\isamarkupcmt{\isa{{\isasymP}} is rigid%
}
%
%
%
\isanewline
\isacommand{lemma}\isamarkupfalse%
\ True\ \isacommand{nitpick}\isamarkupfalse%
{\isacharbrackleft}satisfy{\isacharbrackright}%
\ %
%
\isacommand{oops}\isamarkupfalse%
\ %
\isamarkupcmt{model found: so far all axioms consistent%
}
%
%
%
%
\begin{isamarkuptext}%
If g is God-like, the property of being God-like is its essence.
\footnote{As shown before, this theorem's proof could be completely automatized for G\"odel's and Fitting's variants.
For Anderson's version however, we had to provide Isabelle with some help based on the corresponding natural-language proof 
given by Anderson (see \cite{anderson90:_some_emend_of_goedel_ontol_proof} Theorem 2*, p. 296)}%
\end{isamarkuptext}\isamarkuptrue%
\isacommand{theorem}\isamarkupfalse%
\ GodIsEssential{\isacharcolon}\ {\isachardoublequoteopen}{\isasymlfloor}\isactrlbold {\isasymforall}x{\isachardot}\ G\isactrlsup A\ x\ \isactrlbold {\isasymrightarrow}\ {\isacharparenleft}{\isasymE}\isactrlsup A\ G\isactrlsup A\ x{\isacharparenright}{\isasymrfloor}{\isachardoublequoteclose}%
\ %
%
\isacommand{proof}\isamarkupfalse%
\ {\isacharminus}\ %
\isamarkupcmt{not shown%
}
%
%
%
\isacommand{theorem}\isamarkupfalse%
\ GodExistenceImpliesNecExistence{\isacharcolon}\ {\isachardoublequoteopen}{\isasymlfloor}\isactrlbold {\isasymexists}\ G\isactrlsup A\ \isactrlbold {\isasymrightarrow}\ \ \isactrlbold {\isasymbox}\isactrlbold {\isasymexists}\isactrlsup E\ G\isactrlsup A{\isasymrfloor}{\isachardoublequoteclose}%
\ %
%
\isacommand{proof}\isamarkupfalse%
\ {\isacharminus}%
%
%
%
%
%
%
%
%
%
%
%
%
%
%
\isacommand{theorem}\isamarkupfalse%
\ T{\isadigit{4}}{\isacharcolon}\ {\isachardoublequoteopen}{\isasymlfloor}\isactrlbold {\isasymdiamond}\isactrlbold {\isasymexists}\ G\isactrlsup A{\isasymrfloor}\ {\isasymlongrightarrow}\ {\isasymlfloor}\isactrlbold {\isasymbox}\isactrlbold {\isasymexists}\isactrlsup E\ G\isactrlsup A{\isasymrfloor}{\isachardoublequoteclose}%
\ %
%
\isacommand{proof}\isamarkupfalse%
\ {\isacharminus}\ %
\isamarkupcmt{not shown%
}
%
%
%
\isacommand{lemma}\isamarkupfalse%
\ GodNecExists{\isacharcolon}\ {\isachardoublequoteopen}{\isasymlfloor}\isactrlbold {\isasymbox}\isactrlbold {\isasymexists}\isactrlsup E\ G\isactrlsup A{\isasymrfloor}{\isachardoublequoteclose}%
\ %
%
\isacommand{using}\isamarkupfalse%
\ T{\isadigit{3}}\ T{\isadigit{4}}\ \isacommand{by}\isamarkupfalse%
\ metis\ %
\isamarkupcmt{argument's conclusion%
}
%
%
%
\ \isanewline
\isacommand{lemma}\isamarkupfalse%
\ ModalCollapse{\isacharcolon}\ {\isachardoublequoteopen}{\isasymlfloor}\isactrlbold {\isasymforall}{\isasymPhi}{\isachardot}{\isacharparenleft}{\isasymPhi}\ \isactrlbold {\isasymrightarrow}\ {\isacharparenleft}\isactrlbold {\isasymbox}\ {\isasymPhi}{\isacharparenright}{\isacharparenright}{\isasymrfloor}{\isachardoublequoteclose}\ \isacommand{nitpick}\isamarkupfalse%
%
\ %
%
\isacommand{oops}\isamarkupfalse%
\ %
\isamarkupcmt{countersatisfiable%
}
%
%
%
%
\isamarkupsection{Conclusion%
}
\isamarkuptrue%
%
\begin{isamarkuptext}%
We presented a shallow semantical embedding in Isabelle/HOL for an intensional higher-order modal logic
(a successor of Montague/Gallin intensional logics) as introduced by M. Fitting in his textbook \emph{Types, Tableaus and 
G\"odel's God} \cite{Fitting}. 
We employed this logic to formalize and verify all results relevant to the subsequent discussion of three different
variants of the ontological argument: the first one by G\"odel himself (respectively, Scott), the second 
one by Fitting and the last one by Anderson.%
\end{isamarkuptext}\isamarkuptrue%
%
\begin{isamarkuptext}%
By employing an interactive theorem-prover like Isabelle, we were not only able to verify Fitting's results,
but also to guarantee consistency. We could prove even stronger versions
of many of the theorems and find better countermodels (i.e. with smaller cardinality) than the ones presented in his book.
Another interesting aspect was the possibility to explore the implications of alternative formalizations
for definitions and theorems which shed light on interesting philosophical issues concerning entailment,
essentialism and free will, which are currently the subject of some follow-up analysis.%
\end{isamarkuptext}\isamarkuptrue%
%
\begin{isamarkuptext}%
The latest developments in \emph{automated theorem proving} allow us to engage in much more experimentation
during the formalization and assessment of arguments than ever before. The potential reduction (of several orders of magnitude)
in the time needed for proving or disproving theorems (compared to pen-and-paper proofs), results in almost real-time
feedback about the suitability of our speculations. The practical benefits of computer-supported argumentation go beyond
mere quantitative (easier, faster and more reliable proofs). The advantages are also qualitative, since it fosters a
different approach to argumentation: We can now work iteratively (by `trial-and-error') on an argument
by making gradual adjustments to its definitions, axioms and theorems. This allows us to continuously expose and revise 
the assumptions we indirectly commit ourselves everytime we opt for some particular formalization.%
\end{isamarkuptext}\isamarkuptrue%
%
%
%
%
%
%
%
\end{isabellebody}%
%%% Local Variables:
%%% mode: latex
%%% TeX-master: "root"
%%% End:
