% This is LLNCS.DEM the demonstration file of
% the LaTeX macro package from Springer-Verlag
% for Lecture Notes in Computer Science,
% version 2.4 for LaTeX2e as of 16. April 2010
%
\documentclass{llncs}
%
\usepackage{makeidx}  % allows for indexgeneration
\usepackage{amssymb}  % packages for math symbols
\usepackage{isabelle,isabellesym} % packages for Isabelle-specific math symbols

\isabellestyle{it} % theory text in math-similar italics

%
\begin{document}
%
\frontmatter          % for the preliminaries
%
\pagestyle{headings}  % switches on printing of running heads

%
\mainmatter              % start of the contributions
%
\title{Automating Emendations of \\
	the Ontological Argument in\\
	Intensional Higher-Order Modal Logic}
%
%\titlerunning{??}  % abbreviated title (for running head) also used for the TOC unless \toctitle is used
\author{David Fuenmayor\inst{1} \and Christoph Benzm\"uller\inst{2,1}}
%
%\authorrunning{??} % abbreviated author list (for running head)

%%%% list of authors for the TOC (use if author list has to be modified)
\tocauthor{David Fuenmayor, Christoph Benzm\"uller}
%
\institute{Freie Universit\"at Berlin, Germany
%\email{davfuenmayor@gmail.com}
\and
University of Luxembourg, Luxembourg
%\email{c.benzmueller@gmail.com},\\ WWW home page:
%\texttt{http://christoph-benzmueller.de}
}

\maketitle              % typeset the title of the contribution

%\begin{abstract}
%	A computer-formalization in Isabelle/HOL of several variants of G\"odel's ontological argument is presented (as discussed in M. Fitting's textbook \emph{Types, Tableaus and G\"odel's God}). Fitting's work	introduces an intensional higher-order modal logic (by drawing on Montague/Gallin approach), which we shallowly embed here in classical higher-order logic (Isabelle/HOL). We then	utilize the embedded logic for the
%	formalization of the ontological argument. In particular, Fitting's and Anderson's variants are verified and their claims confirmed. These variants aim to avoid the modal collapse, which has been criticized as an undesirable side-effect of Kurt G\"odel's (and Dana Scott's) versions of the ontological argument.
%	\keywords{Automated Theorem Proving. Computational Metaphysics. Isabelle. Modal Logic.
%	Intensional Logic. Ontological Argument}
%\end{abstract}

%
\begin{isabellebody}%
\setisabellecontext{IHOML}%
%
\isadelimtheory
%
\endisadelimtheory
%
\isatagtheory
%
\endisatagtheory
{\isafoldtheory}%
%
\isadelimtheory
%
\endisadelimtheory
%
\isamarkupsection{Introduction%
}
\isamarkuptrue%
%
\begin{isamarkuptext}%
We present a study on Computational Metaphysics: a computer-formalisation and verification
of Fitting's variant of the ontological argument (for the existence of God) as presented in
his textbook \emph{Types, Tableaus and G\"odel's God} \cite{Fitting}. Fitting's argument 
is an emendation of Kurt G\"odel's modern variant \cite{GoedelNotes} (resp. Dana Scott's 
variant \cite{ScottNotes}) of the ontological argument.%
\end{isamarkuptext}\isamarkuptrue%
%
\begin{isamarkuptext}%
The motivation is to avoid the \emph{modal collapse} \cite{Sobel,sobel2004logic}, which has been criticised
as an undesirable side-effect of the axioms of G\"odel resp. Scott. The modal collapse essentially  
states that  there are no contingent truths and that everything is determined.
Several authors (e.g. \cite{anderson90:_some_emend_of_goedel_ontol_proof,AndersonGettings,Hajek2002,bjordal99}) 
have proposed emendations of the argument with the aim of maintaining the essential result 
(the necessary existence of God) while at the same time avoiding the modal collapse. 
Related work  has formalised several of these variants on the computer and verified or falsified them. For example,
G\"odel's axioms \cite{GoedelNotes} have been shown inconsistent \cite{C55,C60}
while Scott's version has been verified \cite{ECAI}. Further experiments, contributing amongst others
to the clarification of a related debate between H\'ajek and Anderson, are presented and discussed in
\cite{J23}. The enabling technique in all of these experiments has been
shallow semantical embeddings of (extensional) higher-order modal logics in classical higher-order
logic (see \cite{J23,R59} and the references therein).%
\end{isamarkuptext}\isamarkuptrue%
%
\begin{isamarkuptext}%
Fitting's emendation also intends to avoid the modal collapse. However, in contrast to the above variants, Fitting's
solution is based on the use of an intensional as opposed to an extensional higher-order modal logic.
For our work this imposed the additional challenge to provide a shallow embedding of this more advanced
logic. The experiments presented below confirm that Fitting's argument as presented in his textbook \cite{Fitting}
is valid and that it avoids the modal collapse as intended.%
\end{isamarkuptext}\isamarkuptrue%
%
\begin{isamarkuptext}%
The work presented here originates from the \emph{Computational Metaphysics} lecture course  
held at FU Berlin in Summer 2016 \cite{C65}. \pagebreak%
\end{isamarkuptext}\isamarkuptrue%
%
\isamarkupsection{Embedding of Intensional Higher-Order Modal Logic%
}
\isamarkuptrue%
%
\begin{isamarkuptext}%
The object logic being embedded, intensional higher-order modal logic (IHOML), is a modification of the intentional logic developed by Montague
and Gallin \cite{Gallin75}. IHOML is introduced by Fitting in the second part of his textbook \cite{Fitting}
in order to formalise his emendation of G\"odel's ontological argument. We offer here a shallow embedding
of this logic in Isabelle/HOL, which has been inspired by previous work on the semantical embedding of
multimodal logics with quantification \cite{J23}. We expand this approach to allow for actualist quantifiers,
intensional types and their related operations.%
\end{isamarkuptext}\isamarkuptrue%
%
\isamarkupsubsection{Type Declarations%
}
\isamarkuptrue%
%
\begin{isamarkuptext}%
Since IHOML and Isabelle/HOL are both typed languages, we introduce a type-mapping between them.
We follow as closely as possible the syntax given by Fitting (see p. 86). According to this syntax,
if \isa{{\isasymtau}} is an extensional type, \isa{{\isasymup}{\isasymtau}} is the corresponding intensional type. For instance,
a set of (red) objects has the extensional type \isa{{\isasymlangle}{\isasymzero}{\isasymrangle}}, whereas the concept `red' has intensional type \isa{{\isasymup}{\isasymlangle}{\isasymzero}{\isasymrangle}}.
In what follows, terms having extensional (intensional) types will be called extensional (intensional) terms.%
\end{isamarkuptext}\isamarkuptrue%
\ \ \isacommand{typedecl}\isamarkupfalse%
\ i\ \ \ \ \ \ \ \ \ \ \ \ \ \ \ \ \ \ \ \ %
\isamarkupcmt{type for possible worlds%
}
\isanewline
\ \ \isacommand{type{\isacharunderscore}synonym}\isamarkupfalse%
\ io\ {\isacharequal}\ {\isachardoublequoteopen}{\isacharparenleft}i{\isasymRightarrow}bool{\isacharparenright}{\isachardoublequoteclose}\ %
\isamarkupcmt{formulas with world-dependent truth-value%
}
\isanewline
\ \ \isacommand{typedecl}\isamarkupfalse%
\ e\ \ {\isacharparenleft}{\isachardoublequoteopen}{\isasymzero}{\isachardoublequoteclose}{\isacharparenright}\ \ \ \ \ \ \ \ \ \ \ \ \ %
\isamarkupcmt{individuals%
}
%
\begin{isamarkuptext}%
Aliases for common unary predicate types:%
\end{isamarkuptext}\isamarkuptrue%
\ \ \isacommand{type{\isacharunderscore}synonym}\isamarkupfalse%
\ ie\ {\isacharequal}\ \ \ \ \ {\isachardoublequoteopen}{\isacharparenleft}i{\isasymRightarrow}{\isasymzero}{\isacharparenright}{\isachardoublequoteclose}\ \ \ \ \ \ \ \ \ \ \ \ \ {\isacharparenleft}{\isachardoublequoteopen}{\isasymup}{\isasymzero}{\isachardoublequoteclose}{\isacharparenright}\isanewline
\ \ \isacommand{type{\isacharunderscore}synonym}\isamarkupfalse%
\ se\ {\isacharequal}\ \ \ \ \ {\isachardoublequoteopen}{\isacharparenleft}{\isasymzero}{\isasymRightarrow}bool{\isacharparenright}{\isachardoublequoteclose}\ \ \ \ \ \ \ \ \ \ {\isacharparenleft}{\isachardoublequoteopen}{\isasymlangle}{\isasymzero}{\isasymrangle}{\isachardoublequoteclose}{\isacharparenright}\isanewline
\ \ \isacommand{type{\isacharunderscore}synonym}\isamarkupfalse%
\ ise\ {\isacharequal}\ \ \ \ {\isachardoublequoteopen}{\isacharparenleft}{\isasymzero}{\isasymRightarrow}io{\isacharparenright}{\isachardoublequoteclose}\ \ \ \ \ \ \ \ \ \ \ {\isacharparenleft}{\isachardoublequoteopen}{\isasymup}{\isasymlangle}{\isasymzero}{\isasymrangle}{\isachardoublequoteclose}{\isacharparenright}\isanewline
\ \ \isacommand{type{\isacharunderscore}synonym}\isamarkupfalse%
\ sie\ {\isacharequal}\ \ \ \ {\isachardoublequoteopen}{\isacharparenleft}{\isasymup}{\isasymzero}{\isasymRightarrow}bool{\isacharparenright}{\isachardoublequoteclose}\ \ \ \ \ \ \ \ {\isacharparenleft}{\isachardoublequoteopen}{\isasymlangle}{\isasymup}{\isasymzero}{\isasymrangle}{\isachardoublequoteclose}{\isacharparenright}\isanewline
\ \ \isacommand{type{\isacharunderscore}synonym}\isamarkupfalse%
\ isie\ {\isacharequal}\ \ \ {\isachardoublequoteopen}{\isacharparenleft}{\isasymup}{\isasymzero}{\isasymRightarrow}io{\isacharparenright}{\isachardoublequoteclose}\ \ \ \ \ \ \ \ \ {\isacharparenleft}{\isachardoublequoteopen}{\isasymup}{\isasymlangle}{\isasymup}{\isasymzero}{\isasymrangle}{\isachardoublequoteclose}{\isacharparenright}\ \ \isanewline
\ \ \isacommand{type{\isacharunderscore}synonym}\isamarkupfalse%
\ sise\ {\isacharequal}\ \ \ {\isachardoublequoteopen}{\isacharparenleft}{\isasymup}{\isasymlangle}{\isasymzero}{\isasymrangle}{\isasymRightarrow}bool{\isacharparenright}{\isachardoublequoteclose}\ \ \ \ \ {\isacharparenleft}{\isachardoublequoteopen}{\isasymlangle}{\isasymup}{\isasymlangle}{\isasymzero}{\isasymrangle}{\isasymrangle}{\isachardoublequoteclose}{\isacharparenright}\isanewline
\ \ \isacommand{type{\isacharunderscore}synonym}\isamarkupfalse%
\ isise\ {\isacharequal}\ \ {\isachardoublequoteopen}{\isacharparenleft}{\isasymup}{\isasymlangle}{\isasymzero}{\isasymrangle}{\isasymRightarrow}io{\isacharparenright}{\isachardoublequoteclose}\ \ \ \ \ \ {\isacharparenleft}{\isachardoublequoteopen}{\isasymup}{\isasymlangle}{\isasymup}{\isasymlangle}{\isasymzero}{\isasymrangle}{\isasymrangle}{\isachardoublequoteclose}{\isacharparenright}\isanewline
\ \ \isacommand{type{\isacharunderscore}synonym}\isamarkupfalse%
\ sisise{\isacharequal}\ \ {\isachardoublequoteopen}{\isacharparenleft}{\isasymup}{\isasymlangle}{\isasymup}{\isasymlangle}{\isasymzero}{\isasymrangle}{\isasymrangle}{\isasymRightarrow}bool{\isacharparenright}{\isachardoublequoteclose}\ {\isacharparenleft}{\isachardoublequoteopen}{\isasymlangle}{\isasymup}{\isasymlangle}{\isasymup}{\isasymlangle}{\isasymzero}{\isasymrangle}{\isasymrangle}{\isasymrangle}{\isachardoublequoteclose}{\isacharparenright}\isanewline
\ \ \isacommand{type{\isacharunderscore}synonym}\isamarkupfalse%
\ isisise{\isacharequal}\ {\isachardoublequoteopen}{\isacharparenleft}{\isasymup}{\isasymlangle}{\isasymup}{\isasymlangle}{\isasymzero}{\isasymrangle}{\isasymrangle}{\isasymRightarrow}io{\isacharparenright}{\isachardoublequoteclose}\ \ {\isacharparenleft}{\isachardoublequoteopen}{\isasymup}{\isasymlangle}{\isasymup}{\isasymlangle}{\isasymup}{\isasymlangle}{\isasymzero}{\isasymrangle}{\isasymrangle}{\isasymrangle}{\isachardoublequoteclose}{\isacharparenright}\isanewline
\ \ \isacommand{type{\isacharunderscore}synonym}\isamarkupfalse%
\ sse\ {\isacharequal}\ \ \ \ {\isachardoublequoteopen}{\isasymlangle}{\isasymzero}{\isasymrangle}{\isasymRightarrow}bool{\isachardoublequoteclose}\ \ \ \ \ \ \ \ \ {\isacharparenleft}{\isachardoublequoteopen}{\isasymlangle}{\isasymlangle}{\isasymzero}{\isasymrangle}{\isasymrangle}{\isachardoublequoteclose}{\isacharparenright}\isanewline
\ \ \isacommand{type{\isacharunderscore}synonym}\isamarkupfalse%
\ isse\ {\isacharequal}\ \ \ {\isachardoublequoteopen}{\isasymlangle}{\isasymzero}{\isasymrangle}{\isasymRightarrow}io{\isachardoublequoteclose}\ \ \ \ \ \ \ \ \ \ {\isacharparenleft}{\isachardoublequoteopen}{\isasymup}{\isasymlangle}{\isasymlangle}{\isasymzero}{\isasymrangle}{\isasymrangle}{\isachardoublequoteclose}{\isacharparenright}%
\begin{isamarkuptext}%
Aliases for common binary relation types:%
\end{isamarkuptext}\isamarkuptrue%
\ \ \isacommand{type{\isacharunderscore}synonym}\isamarkupfalse%
\ see\ {\isacharequal}\ \ \ \ \ \ \ \ {\isachardoublequoteopen}{\isacharparenleft}{\isasymzero}{\isasymRightarrow}{\isasymzero}{\isasymRightarrow}bool{\isacharparenright}{\isachardoublequoteclose}\ \ \ \ \ \ \ \ \ \ {\isacharparenleft}{\isachardoublequoteopen}{\isasymlangle}{\isasymzero}{\isacharcomma}{\isasymzero}{\isasymrangle}{\isachardoublequoteclose}{\isacharparenright}\isanewline
\ \ \isacommand{type{\isacharunderscore}synonym}\isamarkupfalse%
\ isee\ {\isacharequal}\ \ \ \ \ \ \ {\isachardoublequoteopen}{\isacharparenleft}{\isasymzero}{\isasymRightarrow}{\isasymzero}{\isasymRightarrow}io{\isacharparenright}{\isachardoublequoteclose}\ \ \ \ \ \ \ \ \ \ \ {\isacharparenleft}{\isachardoublequoteopen}{\isasymup}{\isasymlangle}{\isasymzero}{\isacharcomma}{\isasymzero}{\isasymrangle}{\isachardoublequoteclose}{\isacharparenright}\isanewline
\ \ \isacommand{type{\isacharunderscore}synonym}\isamarkupfalse%
\ sieie\ {\isacharequal}\ \ \ \ \ \ {\isachardoublequoteopen}{\isacharparenleft}{\isasymup}{\isasymzero}{\isasymRightarrow}{\isasymup}{\isasymzero}{\isasymRightarrow}bool{\isacharparenright}{\isachardoublequoteclose}\ \ \ \ \ \ \ {\isacharparenleft}{\isachardoublequoteopen}{\isasymlangle}{\isasymup}{\isasymzero}{\isacharcomma}{\isasymup}{\isasymzero}{\isasymrangle}{\isachardoublequoteclose}{\isacharparenright}\isanewline
\ \ \isacommand{type{\isacharunderscore}synonym}\isamarkupfalse%
\ isieie\ {\isacharequal}\ \ \ \ \ {\isachardoublequoteopen}{\isacharparenleft}{\isasymup}{\isasymzero}{\isasymRightarrow}{\isasymup}{\isasymzero}{\isasymRightarrow}io{\isacharparenright}{\isachardoublequoteclose}\ \ \ \ \ \ \ \ {\isacharparenleft}{\isachardoublequoteopen}{\isasymup}{\isasymlangle}{\isasymup}{\isasymzero}{\isacharcomma}{\isasymup}{\isasymzero}{\isasymrangle}{\isachardoublequoteclose}{\isacharparenright}\isanewline
\ \ \isacommand{type{\isacharunderscore}synonym}\isamarkupfalse%
\ ssese\ {\isacharequal}\ \ \ \ \ \ {\isachardoublequoteopen}{\isacharparenleft}{\isasymlangle}{\isasymzero}{\isasymrangle}{\isasymRightarrow}{\isasymlangle}{\isasymzero}{\isasymrangle}{\isasymRightarrow}bool{\isacharparenright}{\isachardoublequoteclose}\ \ \ \ \ {\isacharparenleft}{\isachardoublequoteopen}{\isasymlangle}{\isasymlangle}{\isasymzero}{\isasymrangle}{\isacharcomma}{\isasymlangle}{\isasymzero}{\isasymrangle}{\isasymrangle}{\isachardoublequoteclose}{\isacharparenright}\isanewline
\ \ \isacommand{type{\isacharunderscore}synonym}\isamarkupfalse%
\ issese\ {\isacharequal}\ \ \ \ \ {\isachardoublequoteopen}{\isacharparenleft}{\isasymlangle}{\isasymzero}{\isasymrangle}{\isasymRightarrow}{\isasymlangle}{\isasymzero}{\isasymrangle}{\isasymRightarrow}io{\isacharparenright}{\isachardoublequoteclose}\ \ \ \ \ \ {\isacharparenleft}{\isachardoublequoteopen}{\isasymup}{\isasymlangle}{\isasymlangle}{\isasymzero}{\isasymrangle}{\isacharcomma}{\isasymlangle}{\isasymzero}{\isasymrangle}{\isasymrangle}{\isachardoublequoteclose}{\isacharparenright}\isanewline
\ \ \isacommand{type{\isacharunderscore}synonym}\isamarkupfalse%
\ ssee\ {\isacharequal}\ \ \ \ \ \ \ {\isachardoublequoteopen}{\isacharparenleft}{\isasymlangle}{\isasymzero}{\isasymrangle}{\isasymRightarrow}{\isasymzero}{\isasymRightarrow}bool{\isacharparenright}{\isachardoublequoteclose}\ \ \ \ \ \ \ {\isacharparenleft}{\isachardoublequoteopen}{\isasymlangle}{\isasymlangle}{\isasymzero}{\isasymrangle}{\isacharcomma}{\isasymzero}{\isasymrangle}{\isachardoublequoteclose}{\isacharparenright}\isanewline
\ \ \isacommand{type{\isacharunderscore}synonym}\isamarkupfalse%
\ issee\ {\isacharequal}\ \ \ \ \ \ {\isachardoublequoteopen}{\isacharparenleft}{\isasymlangle}{\isasymzero}{\isasymrangle}{\isasymRightarrow}{\isasymzero}{\isasymRightarrow}io{\isacharparenright}{\isachardoublequoteclose}\ \ \ \ \ \ \ \ {\isacharparenleft}{\isachardoublequoteopen}{\isasymup}{\isasymlangle}{\isasymlangle}{\isasymzero}{\isasymrangle}{\isacharcomma}{\isasymzero}{\isasymrangle}{\isachardoublequoteclose}{\isacharparenright}\isanewline
\ \ \isacommand{type{\isacharunderscore}synonym}\isamarkupfalse%
\ isisee\ {\isacharequal}\ \ \ \ \ {\isachardoublequoteopen}{\isacharparenleft}{\isasymup}{\isasymlangle}{\isasymzero}{\isasymrangle}{\isasymRightarrow}{\isasymzero}{\isasymRightarrow}io{\isacharparenright}{\isachardoublequoteclose}\ \ \ \ \ \ {\isacharparenleft}{\isachardoublequoteopen}{\isasymup}{\isasymlangle}{\isasymup}{\isasymlangle}{\isasymzero}{\isasymrangle}{\isacharcomma}{\isasymzero}{\isasymrangle}{\isachardoublequoteclose}{\isacharparenright}\isanewline
\ \ \isacommand{type{\isacharunderscore}synonym}\isamarkupfalse%
\ isiseise\ {\isacharequal}\ \ \ {\isachardoublequoteopen}{\isacharparenleft}{\isasymup}{\isasymlangle}{\isasymzero}{\isasymrangle}{\isasymRightarrow}{\isasymup}{\isasymlangle}{\isasymzero}{\isasymrangle}{\isasymRightarrow}io{\isacharparenright}{\isachardoublequoteclose}\ \ \ \ {\isacharparenleft}{\isachardoublequoteopen}{\isasymup}{\isasymlangle}{\isasymup}{\isasymlangle}{\isasymzero}{\isasymrangle}{\isacharcomma}{\isasymup}{\isasymlangle}{\isasymzero}{\isasymrangle}{\isasymrangle}{\isachardoublequoteclose}{\isacharparenright}\isanewline
\ \ \isacommand{type{\isacharunderscore}synonym}\isamarkupfalse%
\ isiseisise{\isacharequal}\ \ {\isachardoublequoteopen}{\isacharparenleft}{\isasymup}{\isasymlangle}{\isasymzero}{\isasymrangle}{\isasymRightarrow}{\isasymup}{\isasymlangle}{\isasymup}{\isasymlangle}{\isasymzero}{\isasymrangle}{\isasymrangle}{\isasymRightarrow}io{\isacharparenright}{\isachardoublequoteclose}\ {\isacharparenleft}{\isachardoublequoteopen}{\isasymup}{\isasymlangle}{\isasymup}{\isasymlangle}{\isasymzero}{\isasymrangle}{\isacharcomma}{\isasymup}{\isasymlangle}{\isasymup}{\isasymlangle}{\isasymzero}{\isasymrangle}{\isasymrangle}{\isasymrangle}{\isachardoublequoteclose}{\isacharparenright}%
\isamarkupsubsection{Definitions%
}
\isamarkuptrue%
%
\isamarkupsubsubsection{Logical Operators as Truth-Sets%
}
\isamarkuptrue%
\ \ \isacommand{abbreviation}\isamarkupfalse%
\ mnot\ \ \ {\isacharcolon}{\isacharcolon}\ {\isachardoublequoteopen}io{\isasymRightarrow}io{\isachardoublequoteclose}\ {\isacharparenleft}{\isachardoublequoteopen}\isactrlbold {\isasymnot}{\isacharunderscore}{\isachardoublequoteclose}{\isacharbrackleft}{\isadigit{5}}{\isadigit{2}}{\isacharbrackright}{\isadigit{5}}{\isadigit{3}}{\isacharparenright}\isanewline
\ \ \ \ \isakeyword{where}\ {\isachardoublequoteopen}\isactrlbold {\isasymnot}{\isasymphi}\ {\isasymequiv}\ {\isasymlambda}w{\isachardot}\ {\isasymnot}{\isacharparenleft}{\isasymphi}\ w{\isacharparenright}{\isachardoublequoteclose}\isanewline
\ \ \isacommand{abbreviation}\isamarkupfalse%
\ negpred\ {\isacharcolon}{\isacharcolon}\ {\isachardoublequoteopen}{\isasymlangle}{\isasymzero}{\isasymrangle}{\isasymRightarrow}{\isasymlangle}{\isasymzero}{\isasymrangle}{\isachardoublequoteclose}\ {\isacharparenleft}{\isachardoublequoteopen}{\isasymrightharpoondown}{\isacharunderscore}{\isachardoublequoteclose}{\isacharbrackleft}{\isadigit{5}}{\isadigit{2}}{\isacharbrackright}{\isadigit{5}}{\isadigit{3}}{\isacharparenright}\ \isanewline
\ \ \ \ \isakeyword{where}\ {\isachardoublequoteopen}{\isasymrightharpoondown}{\isasymPhi}\ {\isasymequiv}\ {\isasymlambda}x{\isachardot}\ {\isasymnot}{\isacharparenleft}{\isasymPhi}\ x{\isacharparenright}{\isachardoublequoteclose}\ \isanewline
\ \ \isacommand{abbreviation}\isamarkupfalse%
\ mnegpred\ {\isacharcolon}{\isacharcolon}\ {\isachardoublequoteopen}{\isasymup}{\isasymlangle}{\isasymzero}{\isasymrangle}{\isasymRightarrow}{\isasymup}{\isasymlangle}{\isasymzero}{\isasymrangle}{\isachardoublequoteclose}\ {\isacharparenleft}{\isachardoublequoteopen}\isactrlbold {\isasymrightharpoondown}{\isacharunderscore}{\isachardoublequoteclose}{\isacharbrackleft}{\isadigit{5}}{\isadigit{2}}{\isacharbrackright}{\isadigit{5}}{\isadigit{3}}{\isacharparenright}\ \isanewline
\ \ \ \ \isakeyword{where}\ {\isachardoublequoteopen}\isactrlbold {\isasymrightharpoondown}{\isasymPhi}\ {\isasymequiv}\ {\isasymlambda}x{\isachardot}{\isasymlambda}w{\isachardot}\ {\isasymnot}{\isacharparenleft}{\isasymPhi}\ x\ w{\isacharparenright}{\isachardoublequoteclose}\isanewline
\ \ \isacommand{abbreviation}\isamarkupfalse%
\ mand\ \ \ {\isacharcolon}{\isacharcolon}\ {\isachardoublequoteopen}io{\isasymRightarrow}io{\isasymRightarrow}io{\isachardoublequoteclose}\ {\isacharparenleft}\isakeyword{infixr}{\isachardoublequoteopen}\isactrlbold {\isasymand}{\isachardoublequoteclose}{\isadigit{5}}{\isadigit{1}}{\isacharparenright}\isanewline
\ \ \ \ \isakeyword{where}\ {\isachardoublequoteopen}{\isasymphi}\isactrlbold {\isasymand}{\isasympsi}\ {\isasymequiv}\ {\isasymlambda}w{\isachardot}\ {\isacharparenleft}{\isasymphi}\ w{\isacharparenright}{\isasymand}{\isacharparenleft}{\isasympsi}\ w{\isacharparenright}{\isachardoublequoteclose}\ \ \ \isanewline
\ \ \isacommand{abbreviation}\isamarkupfalse%
\ mor\ \ \ \ {\isacharcolon}{\isacharcolon}\ {\isachardoublequoteopen}io{\isasymRightarrow}io{\isasymRightarrow}io{\isachardoublequoteclose}\ {\isacharparenleft}\isakeyword{infixr}{\isachardoublequoteopen}\isactrlbold {\isasymor}{\isachardoublequoteclose}{\isadigit{5}}{\isadigit{0}}{\isacharparenright}\isanewline
\ \ \ \ \isakeyword{where}\ {\isachardoublequoteopen}{\isasymphi}\isactrlbold {\isasymor}{\isasympsi}\ {\isasymequiv}\ {\isasymlambda}w{\isachardot}\ {\isacharparenleft}{\isasymphi}\ w{\isacharparenright}{\isasymor}{\isacharparenleft}{\isasympsi}\ w{\isacharparenright}{\isachardoublequoteclose}\isanewline
\ \ \isacommand{abbreviation}\isamarkupfalse%
\ mimp\ \ \ {\isacharcolon}{\isacharcolon}\ {\isachardoublequoteopen}io{\isasymRightarrow}io{\isasymRightarrow}io{\isachardoublequoteclose}\ {\isacharparenleft}\isakeyword{infixr}{\isachardoublequoteopen}\isactrlbold {\isasymrightarrow}{\isachardoublequoteclose}{\isadigit{4}}{\isadigit{9}}{\isacharparenright}\ \isanewline
\ \ \ \ \isakeyword{where}\ {\isachardoublequoteopen}{\isasymphi}\isactrlbold {\isasymrightarrow}{\isasympsi}\ {\isasymequiv}\ {\isasymlambda}w{\isachardot}\ {\isacharparenleft}{\isasymphi}\ w{\isacharparenright}{\isasymlongrightarrow}{\isacharparenleft}{\isasympsi}\ w{\isacharparenright}{\isachardoublequoteclose}\ \ \isanewline
\ \ \isacommand{abbreviation}\isamarkupfalse%
\ mequ\ \ \ {\isacharcolon}{\isacharcolon}\ {\isachardoublequoteopen}io{\isasymRightarrow}io{\isasymRightarrow}io{\isachardoublequoteclose}\ {\isacharparenleft}\isakeyword{infixr}{\isachardoublequoteopen}\isactrlbold {\isasymleftrightarrow}{\isachardoublequoteclose}{\isadigit{4}}{\isadigit{8}}{\isacharparenright}\isanewline
\ \ \ \ \isakeyword{where}\ {\isachardoublequoteopen}{\isasymphi}\isactrlbold {\isasymleftrightarrow}{\isasympsi}\ {\isasymequiv}\ {\isasymlambda}w{\isachardot}\ {\isacharparenleft}{\isasymphi}\ w{\isacharparenright}{\isasymlongleftrightarrow}{\isacharparenleft}{\isasympsi}\ w{\isacharparenright}{\isachardoublequoteclose}\isanewline
\ \ \isacommand{abbreviation}\isamarkupfalse%
\ xor{\isacharcolon}{\isacharcolon}\ {\isachardoublequoteopen}bool{\isasymRightarrow}bool{\isasymRightarrow}bool{\isachardoublequoteclose}\ {\isacharparenleft}\isakeyword{infixr}{\isachardoublequoteopen}{\isasymoplus}{\isachardoublequoteclose}{\isadigit{5}}{\isadigit{0}}{\isacharparenright}\isanewline
\ \ \ \ \isakeyword{where}\ {\isachardoublequoteopen}{\isasymphi}{\isasymoplus}{\isasympsi}\ {\isasymequiv}\ \ {\isacharparenleft}{\isasymphi}{\isasymor}{\isasympsi}{\isacharparenright}\ {\isasymand}\ {\isasymnot}{\isacharparenleft}{\isasymphi}{\isasymand}{\isasympsi}{\isacharparenright}{\isachardoublequoteclose}\ \isanewline
\ \ \isacommand{abbreviation}\isamarkupfalse%
\ mxor\ \ \ {\isacharcolon}{\isacharcolon}\ {\isachardoublequoteopen}io{\isasymRightarrow}io{\isasymRightarrow}io{\isachardoublequoteclose}\ {\isacharparenleft}\isakeyword{infixr}{\isachardoublequoteopen}\isactrlbold {\isasymoplus}{\isachardoublequoteclose}{\isadigit{5}}{\isadigit{0}}{\isacharparenright}\isanewline
\ \ \ \ \isakeyword{where}\ {\isachardoublequoteopen}{\isasymphi}\isactrlbold {\isasymoplus}{\isasympsi}\ {\isasymequiv}\ {\isasymlambda}w{\isachardot}\ {\isacharparenleft}{\isasymphi}\ w{\isacharparenright}{\isasymoplus}{\isacharparenleft}{\isasympsi}\ w{\isacharparenright}{\isachardoublequoteclose}%
\isamarkupsubsubsection{Possibilist Quantification%
}
\isamarkuptrue%
\ \ \isacommand{abbreviation}\isamarkupfalse%
\ mforall\ \ \ {\isacharcolon}{\isacharcolon}\ {\isachardoublequoteopen}{\isacharparenleft}{\isacharprime}t{\isasymRightarrow}io{\isacharparenright}{\isasymRightarrow}io{\isachardoublequoteclose}\ {\isacharparenleft}{\isachardoublequoteopen}\isactrlbold {\isasymforall}{\isachardoublequoteclose}{\isacharparenright}\ \ \ \ \ \ \isanewline
\ \ \ \ \isakeyword{where}\ {\isachardoublequoteopen}\isactrlbold {\isasymforall}{\isasymPhi}\ {\isasymequiv}\ {\isasymlambda}w{\isachardot}{\isasymforall}x{\isachardot}\ {\isacharparenleft}{\isasymPhi}\ x\ w{\isacharparenright}{\isachardoublequoteclose}\isanewline
\ \ \isacommand{abbreviation}\isamarkupfalse%
\ mexists\ \ \ {\isacharcolon}{\isacharcolon}\ {\isachardoublequoteopen}{\isacharparenleft}{\isacharprime}t{\isasymRightarrow}io{\isacharparenright}{\isasymRightarrow}io{\isachardoublequoteclose}\ {\isacharparenleft}{\isachardoublequoteopen}\isactrlbold {\isasymexists}{\isachardoublequoteclose}{\isacharparenright}\ \isanewline
\ \ \ \ \isakeyword{where}\ {\isachardoublequoteopen}\isactrlbold {\isasymexists}{\isasymPhi}\ {\isasymequiv}\ {\isasymlambda}w{\isachardot}{\isasymexists}x{\isachardot}\ {\isacharparenleft}{\isasymPhi}\ x\ w{\isacharparenright}{\isachardoublequoteclose}\isanewline
\ \ \ \ \isanewline
\ \ \isacommand{abbreviation}\isamarkupfalse%
\ mforallB\ \ {\isacharcolon}{\isacharcolon}\ {\isachardoublequoteopen}{\isacharparenleft}{\isacharprime}t{\isasymRightarrow}io{\isacharparenright}{\isasymRightarrow}io{\isachardoublequoteclose}\ {\isacharparenleft}\isakeyword{binder}{\isachardoublequoteopen}\isactrlbold {\isasymforall}{\isachardoublequoteclose}{\isacharbrackleft}{\isadigit{8}}{\isacharbrackright}{\isadigit{9}}{\isacharparenright}\ %
\isamarkupcmt{Binder notation%
}
\isanewline
\ \ \ \ \isakeyword{where}\ {\isachardoublequoteopen}\isactrlbold {\isasymforall}x{\isachardot}\ {\isasymphi}{\isacharparenleft}x{\isacharparenright}\ {\isasymequiv}\ \isactrlbold {\isasymforall}{\isasymphi}{\isachardoublequoteclose}\ \ \isanewline
\ \ \isacommand{abbreviation}\isamarkupfalse%
\ mexistsB\ \ {\isacharcolon}{\isacharcolon}\ {\isachardoublequoteopen}{\isacharparenleft}{\isacharprime}t{\isasymRightarrow}io{\isacharparenright}{\isasymRightarrow}io{\isachardoublequoteclose}\ {\isacharparenleft}\isakeyword{binder}{\isachardoublequoteopen}\isactrlbold {\isasymexists}{\isachardoublequoteclose}{\isacharbrackleft}{\isadigit{8}}{\isacharbrackright}{\isadigit{9}}{\isacharparenright}\isanewline
\ \ \ \ \isakeyword{where}\ {\isachardoublequoteopen}\isactrlbold {\isasymexists}x{\isachardot}\ {\isasymphi}{\isacharparenleft}x{\isacharparenright}\ {\isasymequiv}\ \isactrlbold {\isasymexists}{\isasymphi}{\isachardoublequoteclose}%
\isamarkupsubsubsection{Actualist Quantification%
}
\isamarkuptrue%
%
\begin{isamarkuptext}%
The following predicate is used to model actualist quantifiers by restricting the domain of quantification at every possible world.
This standard technique has been referred to as \emph{existence relativization} (\cite{fitting98}, p. 106),
highlighting the fact that this predicate can be seen as a kind of meta-logical `existence predicate' telling us
which individuals \emph{actually} exist at a given world. This meta-logical concept does not appear in our object language.%
\end{isamarkuptext}\isamarkuptrue%
\ \ \isacommand{consts}\isamarkupfalse%
\ Exists{\isacharcolon}{\isacharcolon}{\isachardoublequoteopen}{\isasymup}{\isasymlangle}{\isasymzero}{\isasymrangle}{\isachardoublequoteclose}\ {\isacharparenleft}{\isachardoublequoteopen}existsAt{\isachardoublequoteclose}{\isacharparenright}\ \ \isanewline
\isanewline
\ \ \isacommand{abbreviation}\isamarkupfalse%
\ mforallAct\ \ \ {\isacharcolon}{\isacharcolon}\ {\isachardoublequoteopen}{\isasymup}{\isasymlangle}{\isasymup}{\isasymlangle}{\isasymzero}{\isasymrangle}{\isasymrangle}{\isachardoublequoteclose}\ {\isacharparenleft}{\isachardoublequoteopen}\isactrlbold {\isasymforall}\isactrlsup E{\isachardoublequoteclose}{\isacharparenright}\ \ \ \ \isanewline
\ \ \ \ \isakeyword{where}\ {\isachardoublequoteopen}\isactrlbold {\isasymforall}\isactrlsup E{\isasymPhi}\ {\isasymequiv}\ {\isasymlambda}w{\isachardot}{\isasymforall}x{\isachardot}\ {\isacharparenleft}existsAt\ x\ w{\isacharparenright}{\isasymlongrightarrow}{\isacharparenleft}{\isasymPhi}\ x\ w{\isacharparenright}{\isachardoublequoteclose}\isanewline
\ \ \isacommand{abbreviation}\isamarkupfalse%
\ mexistsAct\ \ \ {\isacharcolon}{\isacharcolon}\ {\isachardoublequoteopen}{\isasymup}{\isasymlangle}{\isasymup}{\isasymlangle}{\isasymzero}{\isasymrangle}{\isasymrangle}{\isachardoublequoteclose}\ {\isacharparenleft}{\isachardoublequoteopen}\isactrlbold {\isasymexists}\isactrlsup E{\isachardoublequoteclose}{\isacharparenright}\ \isanewline
\ \ \ \ \isakeyword{where}\ {\isachardoublequoteopen}\isactrlbold {\isasymexists}\isactrlsup E{\isasymPhi}\ {\isasymequiv}\ {\isasymlambda}w{\isachardot}{\isasymexists}x{\isachardot}\ {\isacharparenleft}existsAt\ x\ w{\isacharparenright}\ {\isasymand}\ {\isacharparenleft}{\isasymPhi}\ x\ w{\isacharparenright}{\isachardoublequoteclose}\isanewline
\isanewline
\ \ \isacommand{abbreviation}\isamarkupfalse%
\ mforallActB\ \ {\isacharcolon}{\isacharcolon}\ {\isachardoublequoteopen}{\isasymup}{\isasymlangle}{\isasymup}{\isasymlangle}{\isasymzero}{\isasymrangle}{\isasymrangle}{\isachardoublequoteclose}\ {\isacharparenleft}\isakeyword{binder}{\isachardoublequoteopen}\isactrlbold {\isasymforall}\isactrlsup E{\isachardoublequoteclose}{\isacharbrackleft}{\isadigit{8}}{\isacharbrackright}{\isadigit{9}}{\isacharparenright}\ %
\isamarkupcmt{binder notation%
}
\isanewline
\ \ \ \ \isakeyword{where}\ {\isachardoublequoteopen}\isactrlbold {\isasymforall}\isactrlsup Ex{\isachardot}\ {\isasymphi}{\isacharparenleft}x{\isacharparenright}\ {\isasymequiv}\ \isactrlbold {\isasymforall}\isactrlsup E{\isasymphi}{\isachardoublequoteclose}\ \ \ \ \ \isanewline
\ \ \isacommand{abbreviation}\isamarkupfalse%
\ mexistsActB\ \ {\isacharcolon}{\isacharcolon}\ {\isachardoublequoteopen}{\isasymup}{\isasymlangle}{\isasymup}{\isasymlangle}{\isasymzero}{\isasymrangle}{\isasymrangle}{\isachardoublequoteclose}\ {\isacharparenleft}\isakeyword{binder}{\isachardoublequoteopen}\isactrlbold {\isasymexists}\isactrlsup E{\isachardoublequoteclose}{\isacharbrackleft}{\isadigit{8}}{\isacharbrackright}{\isadigit{9}}{\isacharparenright}\isanewline
\ \ \ \ \isakeyword{where}\ {\isachardoublequoteopen}\isactrlbold {\isasymexists}\isactrlsup Ex{\isachardot}\ {\isasymphi}{\isacharparenleft}x{\isacharparenright}\ {\isasymequiv}\ \isactrlbold {\isasymexists}\isactrlsup E{\isasymphi}{\isachardoublequoteclose}%
\isamarkupsubsubsection{Modal Operators%
}
\isamarkuptrue%
\ \ \ \isacommand{consts}\isamarkupfalse%
\ aRel{\isacharcolon}{\isacharcolon}{\isachardoublequoteopen}i{\isasymRightarrow}i{\isasymRightarrow}bool{\isachardoublequoteclose}\ {\isacharparenleft}\isakeyword{infixr}\ {\isachardoublequoteopen}r{\isachardoublequoteclose}\ {\isadigit{7}}{\isadigit{0}}{\isacharparenright}\ \ %
\isamarkupcmt{accessibility relation \emph{r}%
}
\isanewline
\ \ \isanewline
\ \ \isacommand{abbreviation}\isamarkupfalse%
\ mbox\ \ \ {\isacharcolon}{\isacharcolon}\ {\isachardoublequoteopen}io{\isasymRightarrow}io{\isachardoublequoteclose}\ {\isacharparenleft}{\isachardoublequoteopen}\isactrlbold {\isasymbox}{\isacharunderscore}{\isachardoublequoteclose}{\isacharbrackleft}{\isadigit{5}}{\isadigit{2}}{\isacharbrackright}{\isadigit{5}}{\isadigit{3}}{\isacharparenright}\isanewline
\ \ \ \ \isakeyword{where}\ {\isachardoublequoteopen}\isactrlbold {\isasymbox}{\isasymphi}\ {\isasymequiv}\ {\isasymlambda}w{\isachardot}{\isasymforall}v{\isachardot}\ {\isacharparenleft}w\ r\ v{\isacharparenright}{\isasymlongrightarrow}{\isacharparenleft}{\isasymphi}\ v{\isacharparenright}{\isachardoublequoteclose}\isanewline
\ \ \isacommand{abbreviation}\isamarkupfalse%
\ mdia\ \ \ {\isacharcolon}{\isacharcolon}\ {\isachardoublequoteopen}io{\isasymRightarrow}io{\isachardoublequoteclose}\ {\isacharparenleft}{\isachardoublequoteopen}\isactrlbold {\isasymdiamond}{\isacharunderscore}{\isachardoublequoteclose}{\isacharbrackleft}{\isadigit{5}}{\isadigit{2}}{\isacharbrackright}{\isadigit{5}}{\isadigit{3}}{\isacharparenright}\isanewline
\ \ \ \ \isakeyword{where}\ {\isachardoublequoteopen}\isactrlbold {\isasymdiamond}{\isasymphi}\ {\isasymequiv}\ {\isasymlambda}w{\isachardot}{\isasymexists}v{\isachardot}\ {\isacharparenleft}w\ r\ v{\isacharparenright}{\isasymand}{\isacharparenleft}{\isasymphi}\ v{\isacharparenright}{\isachardoublequoteclose}%
\isamarkupsubsubsection{\emph{Extension-of} Operator%
}
\isamarkuptrue%
%
\begin{isamarkuptext}%
According to Fitting's semantics (\cite{Fitting}, pp. 92-4) \isa{{\isasymdown}} is an unary operator applying only to 
 intensional terms. A term of the form \isa{{\isasymdown}{\isasymalpha}} designates the extension of the intensional object designated by 
 \isa{{\isasymalpha}}, at some \emph{given} world. For instance, suppose we take possible worlds as persons,
 we can therefore think of the concept `red' as a function that maps each person to the set of objects that person
 classifies as red (its extension). We can further state, the intensional term \emph{r} of type \isa{{\isasymup}{\isasymlangle}{\isasymzero}{\isasymrangle}} designates the concept `red'.
 As can be seen, intensional terms in IHOML designate functions on possible worlds and they always do it \emph{rigidly}. 
 We will sometimes refer to an intensional object explicitly as `rigid', implying that its (rigidly) designated function has
 the same extension in all possible worlds.%
\end{isamarkuptext}\isamarkuptrue%
%
\begin{isamarkuptext}%
Terms of the form \isa{{\isasymdown}{\isasymalpha}} are called \emph{relativized} (extensional) terms; they are always derived
from intensional terms and their type is \emph{extensional} (in the color example \isa{{\isasymdown}r} would be of type \isa{{\isasymlangle}{\isasymzero}{\isasymrangle}}).
Relativized terms may vary their denotation from world to world of a model, because the extension of an intensional term can change
from world to world, i.e. they are non-rigid.%
\end{isamarkuptext}\isamarkuptrue%
%
\begin{isamarkuptext}%
To recap: an intensional term denotes the same function in all worlds (i.e. it's rigid), whereas a relativized term
denotes a (possibly) different extension (an object or a set) at every world (i.e. it's non-rigid). To find out
the denotation of a relativized term, a world must be given. Relativized terms are the \emph{only} non-rigid terms.
\bigbreak%
\end{isamarkuptext}\isamarkuptrue%
%
\begin{isamarkuptext}%
For our Isabelle/HOL embedding, we had to follow a slightly different approach; we model \isa{{\isasymdown}}
as a predicate applying to formulas of the form \isa{{\isasymPhi}{\isacharparenleft}{\isasymdown}{\isasymalpha}\isactrlsub {\isadigit{1}}{\isacharcomma}{\isasymdots}{\isasymalpha}\isactrlsub n{\isacharparenright}} (for our treatment
we only need to consider cases involving one or two arguments, the first one being a relativized term).
For instance, the formula \isa{Q{\isacharparenleft}{\isasymdown}a\isactrlsub {\isadigit{1}}{\isacharparenright}\isactrlsup w} (evaluated at world \emph{w}) is modelled as \isa{{\isasymdownharpoonleft}{\isacharparenleft}Q{\isacharcomma}a\isactrlsub {\isadigit{1}}{\isacharparenright}\isactrlsup w}
(or \isa{{\isacharparenleft}Q\ {\isasymdownharpoonleft}\ a\isactrlsub {\isadigit{1}}{\isacharparenright}\isactrlsup w} using infix notation), which gets further translated into \isa{Q{\isacharparenleft}a\isactrlsub {\isadigit{1}}{\isacharparenleft}w{\isacharparenright}{\isacharparenright}\isactrlsup w}.

Depending on the particular types involved, we have to define \isa{{\isasymdown}} differently to ensure type correctness
(see \emph{a-d} below). Nevertheless, the essence of the \emph{Extension-of} operator remains the same:
a term \isa{{\isasymalpha}} preceded by \isa{{\isasymdown}} behaves as a non-rigid term, whose denotation at a given possible world corresponds
to the extension of the original intensional term \isa{{\isasymalpha}} at that world.%
\end{isamarkuptext}\isamarkuptrue%
%
\begin{isamarkuptext}%
(\emph{a}) Predicate \isa{{\isasymphi}} takes as argument a relativized term derived from an (intensional) individual of type \isa{{\isasymup}{\isasymzero}}:%
\end{isamarkuptext}\isamarkuptrue%
\isacommand{abbreviation}\isamarkupfalse%
\ extIndivArg{\isacharcolon}{\isacharcolon}{\isachardoublequoteopen}{\isasymup}{\isasymlangle}{\isasymzero}{\isasymrangle}{\isasymRightarrow}{\isasymup}{\isasymzero}{\isasymRightarrow}io{\isachardoublequoteclose}\ {\isacharparenleft}\isakeyword{infix}\ {\isachardoublequoteopen}{\isasymdownharpoonleft}{\isachardoublequoteclose}\ {\isadigit{6}}{\isadigit{0}}{\isacharparenright}\ \ \ \ \ \ \ \ \ \ \ \ \ \ \ \ \ \ \ \ \ \ \ \ \ \ \ \isanewline
\ \ \isakeyword{where}\ {\isachardoublequoteopen}{\isasymphi}\ {\isasymdownharpoonleft}c\ {\isasymequiv}\ {\isasymlambda}w{\isachardot}\ {\isasymphi}\ {\isacharparenleft}c\ w{\isacharparenright}\ w{\isachardoublequoteclose}%
\begin{isamarkuptext}%
(\emph{b}) A variant of (\emph{a}) for terms derived from predicates (types of form \isa{{\isasymup}{\isasymlangle}t{\isasymrangle}}):%
\end{isamarkuptext}\isamarkuptrue%
\isacommand{abbreviation}\isamarkupfalse%
\ extPredArg{\isacharcolon}{\isacharcolon}{\isachardoublequoteopen}{\isacharparenleft}{\isacharparenleft}{\isacharprime}t{\isasymRightarrow}bool{\isacharparenright}{\isasymRightarrow}io{\isacharparenright}{\isasymRightarrow}{\isacharparenleft}{\isacharprime}t{\isasymRightarrow}io{\isacharparenright}{\isasymRightarrow}io{\isachardoublequoteclose}\ {\isacharparenleft}\isakeyword{infix}\ {\isachardoublequoteopen}{\isasymdown}{\isachardoublequoteclose}\ {\isadigit{6}}{\isadigit{0}}{\isacharparenright}\isanewline
\ \ \isakeyword{where}\ {\isachardoublequoteopen}{\isasymphi}\ {\isasymdown}P\ {\isasymequiv}\ {\isasymlambda}w{\isachardot}\ {\isasymphi}\ {\isacharparenleft}{\isasymlambda}x{\isachardot}\ P\ x\ w{\isacharparenright}\ w{\isachardoublequoteclose}%
\begin{isamarkuptext}%
(\emph{c}) A variant of (\emph{b}) with a second argument (the first one being relativized):%
\end{isamarkuptext}\isamarkuptrue%
\isacommand{abbreviation}\isamarkupfalse%
\ extPredArg{\isadigit{1}}{\isacharcolon}{\isacharcolon}{\isachardoublequoteopen}{\isacharparenleft}{\isacharparenleft}{\isacharprime}t{\isasymRightarrow}bool{\isacharparenright}{\isasymRightarrow}{\isacharprime}b{\isasymRightarrow}io{\isacharparenright}{\isasymRightarrow}{\isacharparenleft}{\isacharprime}t{\isasymRightarrow}io{\isacharparenright}{\isasymRightarrow}{\isacharprime}b{\isasymRightarrow}io{\isachardoublequoteclose}\ {\isacharparenleft}\isakeyword{infix}\ {\isachardoublequoteopen}{\isasymdown}\isactrlsub {\isadigit{1}}{\isachardoublequoteclose}\ {\isadigit{6}}{\isadigit{0}}{\isacharparenright}\isanewline
\ \ \isakeyword{where}\ {\isachardoublequoteopen}{\isasymphi}\ {\isasymdown}\isactrlsub {\isadigit{1}}P\ {\isasymequiv}\ {\isasymlambda}z{\isachardot}\ {\isasymlambda}w{\isachardot}\ {\isasymphi}\ {\isacharparenleft}{\isasymlambda}x{\isachardot}\ P\ x\ w{\isacharparenright}\ z\ w{\isachardoublequoteclose}%
\begin{isamarkuptext}%
In what follows, the `\isa{{\isasymlparr}{\isacharunderscore}{\isasymrparr}}' parentheses are an operator used to convert extensional objects into `rigid' intensional ones:%
\end{isamarkuptext}\isamarkuptrue%
\isacommand{abbreviation}\isamarkupfalse%
\ trivialConversion{\isacharcolon}{\isacharcolon}{\isachardoublequoteopen}bool{\isasymRightarrow}io{\isachardoublequoteclose}\ {\isacharparenleft}{\isachardoublequoteopen}{\isasymlparr}{\isacharunderscore}{\isasymrparr}{\isachardoublequoteclose}{\isacharparenright}\ \isakeyword{where}\ {\isachardoublequoteopen}{\isasymlparr}{\isasymphi}{\isasymrparr}\ {\isasymequiv}\ {\isacharparenleft}{\isasymlambda}w{\isachardot}\ {\isasymphi}{\isacharparenright}{\isachardoublequoteclose}%
\begin{isamarkuptext}%
(\emph{d}) A variant of (\emph{b}) where \isa{{\isasymphi}} takes `rigid' intensional terms as argument:%
\end{isamarkuptext}\isamarkuptrue%
\isacommand{abbreviation}\isamarkupfalse%
\ mextPredArg{\isacharcolon}{\isacharcolon}{\isachardoublequoteopen}{\isacharparenleft}{\isacharparenleft}{\isacharprime}t{\isasymRightarrow}io{\isacharparenright}{\isasymRightarrow}io{\isacharparenright}{\isasymRightarrow}{\isacharparenleft}{\isacharprime}t{\isasymRightarrow}io{\isacharparenright}{\isasymRightarrow}io{\isachardoublequoteclose}\ {\isacharparenleft}\isakeyword{infix}\ {\isachardoublequoteopen}\isactrlbold {\isasymdown}{\isachardoublequoteclose}\ {\isadigit{6}}{\isadigit{0}}{\isacharparenright}\isanewline
\ \ \isakeyword{where}\ {\isachardoublequoteopen}{\isasymphi}\ \isactrlbold {\isasymdown}P\ {\isasymequiv}\ {\isasymlambda}w{\isachardot}\ {\isasymphi}\ {\isacharparenleft}{\isasymlambda}x{\isachardot}\ {\isasymlparr}P\ x\ w{\isasymrparr}{\isacharparenright}\ w{\isachardoublequoteclose}%
\isamarkupsubsubsection{Equality%
}
\isamarkuptrue%
\ \ \isacommand{abbreviation}\isamarkupfalse%
\ meq\ \ \ \ {\isacharcolon}{\isacharcolon}\ {\isachardoublequoteopen}{\isacharprime}t{\isasymRightarrow}{\isacharprime}t{\isasymRightarrow}io{\isachardoublequoteclose}\ {\isacharparenleft}\isakeyword{infix}{\isachardoublequoteopen}\isactrlbold {\isasymapprox}{\isachardoublequoteclose}{\isadigit{6}}{\isadigit{0}}{\isacharparenright}\ %
\isamarkupcmt{normal equality (for all types)%
}
\isanewline
\ \ \ \ \isakeyword{where}\ {\isachardoublequoteopen}x\ \isactrlbold {\isasymapprox}\ y\ {\isasymequiv}\ {\isasymlambda}w{\isachardot}\ x\ {\isacharequal}\ y{\isachardoublequoteclose}\isanewline
\ \ \isacommand{abbreviation}\isamarkupfalse%
\ meqC\ \ \ {\isacharcolon}{\isacharcolon}\ {\isachardoublequoteopen}{\isasymup}{\isasymlangle}{\isasymup}{\isasymzero}{\isacharcomma}{\isasymup}{\isasymzero}{\isasymrangle}{\isachardoublequoteclose}\ {\isacharparenleft}\isakeyword{infixr}{\isachardoublequoteopen}\isactrlbold {\isasymapprox}\isactrlsup C{\isachardoublequoteclose}{\isadigit{5}}{\isadigit{2}}{\isacharparenright}\ %
\isamarkupcmt{eq. for individual concepts%
}
\isanewline
\ \ \ \ \isakeyword{where}\ {\isachardoublequoteopen}x\ \isactrlbold {\isasymapprox}\isactrlsup C\ y\ {\isasymequiv}\ {\isasymlambda}w{\isachardot}\ {\isasymforall}v{\isachardot}\ {\isacharparenleft}x\ v{\isacharparenright}\ {\isacharequal}\ {\isacharparenleft}y\ v{\isacharparenright}{\isachardoublequoteclose}\isanewline
\ \ \isacommand{abbreviation}\isamarkupfalse%
\ meqL\ \ \ {\isacharcolon}{\isacharcolon}\ {\isachardoublequoteopen}{\isasymup}{\isasymlangle}{\isasymzero}{\isacharcomma}{\isasymzero}{\isasymrangle}{\isachardoublequoteclose}\ {\isacharparenleft}\isakeyword{infixr}{\isachardoublequoteopen}\isactrlbold {\isasymapprox}\isactrlsup L{\isachardoublequoteclose}{\isadigit{5}}{\isadigit{2}}{\isacharparenright}\ %
\isamarkupcmt{Leibniz eq. for individuals%
}
\isanewline
\ \ \ \ \isakeyword{where}\ {\isachardoublequoteopen}x\ \isactrlbold {\isasymapprox}\isactrlsup L\ y\ {\isasymequiv}\ \isactrlbold {\isasymforall}{\isasymphi}{\isachardot}\ {\isasymphi}{\isacharparenleft}x{\isacharparenright}\isactrlbold {\isasymrightarrow}{\isasymphi}{\isacharparenleft}y{\isacharparenright}{\isachardoublequoteclose}%
\isamarkupsubsubsection{Meta-logical Predicates%
}
\isamarkuptrue%
\ \isacommand{abbreviation}\isamarkupfalse%
\ valid\ {\isacharcolon}{\isacharcolon}\ {\isachardoublequoteopen}io{\isasymRightarrow}bool{\isachardoublequoteclose}\ {\isacharparenleft}{\isachardoublequoteopen}{\isasymlfloor}{\isacharunderscore}{\isasymrfloor}{\isachardoublequoteclose}\ {\isacharbrackleft}{\isadigit{8}}{\isacharbrackright}{\isacharparenright}\ \isakeyword{where}\ {\isachardoublequoteopen}{\isasymlfloor}{\isasympsi}{\isasymrfloor}\ {\isasymequiv}\ \ {\isasymforall}w{\isachardot}{\isacharparenleft}{\isasympsi}\ w{\isacharparenright}{\isachardoublequoteclose}\isanewline
\ \isacommand{abbreviation}\isamarkupfalse%
\ satisfiable\ {\isacharcolon}{\isacharcolon}\ {\isachardoublequoteopen}io{\isasymRightarrow}bool{\isachardoublequoteclose}\ {\isacharparenleft}{\isachardoublequoteopen}{\isasymlfloor}{\isacharunderscore}{\isasymrfloor}\isactrlsup s\isactrlsup a\isactrlsup t{\isachardoublequoteclose}\ {\isacharbrackleft}{\isadigit{8}}{\isacharbrackright}{\isacharparenright}\ \isakeyword{where}\ {\isachardoublequoteopen}{\isasymlfloor}{\isasympsi}{\isasymrfloor}\isactrlsup s\isactrlsup a\isactrlsup t\ {\isasymequiv}\ {\isasymexists}w{\isachardot}{\isacharparenleft}{\isasympsi}\ w{\isacharparenright}{\isachardoublequoteclose}\isanewline
\ \isacommand{abbreviation}\isamarkupfalse%
\ countersat\ {\isacharcolon}{\isacharcolon}\ {\isachardoublequoteopen}io{\isasymRightarrow}bool{\isachardoublequoteclose}\ {\isacharparenleft}{\isachardoublequoteopen}{\isasymlfloor}{\isacharunderscore}{\isasymrfloor}\isactrlsup c\isactrlsup s\isactrlsup a\isactrlsup t{\isachardoublequoteclose}\ {\isacharbrackleft}{\isadigit{8}}{\isacharbrackright}{\isacharparenright}\ \isakeyword{where}\ {\isachardoublequoteopen}{\isasymlfloor}{\isasympsi}{\isasymrfloor}\isactrlsup c\isactrlsup s\isactrlsup a\isactrlsup t\ {\isasymequiv}\ \ {\isasymexists}w{\isachardot}{\isasymnot}{\isacharparenleft}{\isasympsi}\ w{\isacharparenright}{\isachardoublequoteclose}\isanewline
\ \isacommand{abbreviation}\isamarkupfalse%
\ invalid\ {\isacharcolon}{\isacharcolon}\ {\isachardoublequoteopen}io{\isasymRightarrow}bool{\isachardoublequoteclose}\ {\isacharparenleft}{\isachardoublequoteopen}{\isasymlfloor}{\isacharunderscore}{\isasymrfloor}\isactrlsup i\isactrlsup n\isactrlsup v{\isachardoublequoteclose}\ {\isacharbrackleft}{\isadigit{8}}{\isacharbrackright}{\isacharparenright}\ \isakeyword{where}\ {\isachardoublequoteopen}{\isasymlfloor}{\isasympsi}{\isasymrfloor}\isactrlsup i\isactrlsup n\isactrlsup v\ {\isasymequiv}\ {\isasymforall}w{\isachardot}{\isasymnot}{\isacharparenleft}{\isasympsi}\ w{\isacharparenright}{\isachardoublequoteclose}%
\isamarkupsubsection{Verifying the Embedding%
}
\isamarkuptrue%
%
\begin{isamarkuptext}%
The above definitions introduce modal logic \emph{K} with possibilist and actualist quantifiers,
as evidenced by the following tests:%
\end{isamarkuptext}\isamarkuptrue%
%
\begin{isamarkuptext}%
Verifying \emph{K} Principle and Necessitation:%
\end{isamarkuptext}\isamarkuptrue%
\ \isacommand{lemma}\isamarkupfalse%
\ K{\isacharcolon}\ {\isachardoublequoteopen}{\isasymlfloor}{\isacharparenleft}\isactrlbold {\isasymbox}{\isacharparenleft}{\isasymphi}\ \isactrlbold {\isasymrightarrow}\ {\isasympsi}{\isacharparenright}{\isacharparenright}\ \isactrlbold {\isasymrightarrow}\ {\isacharparenleft}\isactrlbold {\isasymbox}{\isasymphi}\ \isactrlbold {\isasymrightarrow}\ \isactrlbold {\isasymbox}{\isasympsi}{\isacharparenright}{\isasymrfloor}{\isachardoublequoteclose}%
\isadelimproof
\ %
\endisadelimproof
%
\isatagproof
\isacommand{by}\isamarkupfalse%
\ simp\ \ \ \ %
\isamarkupcmt{\emph{K} schema%
}
%
\endisatagproof
{\isafoldproof}%
%
\isadelimproof
%
\endisadelimproof
\isanewline
\ \isacommand{lemma}\isamarkupfalse%
\ NEC{\isacharcolon}\ {\isachardoublequoteopen}{\isasymlfloor}{\isasymphi}{\isasymrfloor}\ {\isasymLongrightarrow}\ {\isasymlfloor}\isactrlbold {\isasymbox}{\isasymphi}{\isasymrfloor}{\isachardoublequoteclose}%
\isadelimproof
\ %
\endisadelimproof
%
\isatagproof
\isacommand{by}\isamarkupfalse%
\ simp\ \ \ \ %
\isamarkupcmt{necessitation%
}
%
\endisatagproof
{\isafoldproof}%
%
\isadelimproof
%
\endisadelimproof
%
\begin{isamarkuptext}%
Local consequence implies global consequence (we will use this lemma often):%
\end{isamarkuptext}\isamarkuptrue%
\ \isacommand{lemma}\isamarkupfalse%
\ localImpGlobalCons{\isacharcolon}\ {\isachardoublequoteopen}{\isasymlfloor}{\isasymphi}\ \isactrlbold {\isasymrightarrow}\ {\isasymxi}{\isasymrfloor}\ {\isasymLongrightarrow}\ {\isasymlfloor}{\isasymphi}{\isasymrfloor}\ {\isasymlongrightarrow}\ {\isasymlfloor}{\isasymxi}{\isasymrfloor}{\isachardoublequoteclose}%
\isadelimproof
\ %
\endisadelimproof
%
\isatagproof
\isacommand{by}\isamarkupfalse%
\ simp%
\endisatagproof
{\isafoldproof}%
%
\isadelimproof
%
\endisadelimproof
%
\begin{isamarkuptext}%
But global consequence does not imply local consequence:%
\end{isamarkuptext}\isamarkuptrue%
\ \isacommand{lemma}\isamarkupfalse%
\ {\isachardoublequoteopen}{\isasymlfloor}{\isasymphi}{\isasymrfloor}\ {\isasymlongrightarrow}\ {\isasymlfloor}{\isasymxi}{\isasymrfloor}\ {\isasymLongrightarrow}\ {\isasymlfloor}{\isasymphi}\ \isactrlbold {\isasymrightarrow}\ {\isasymxi}{\isasymrfloor}{\isachardoublequoteclose}\ \isacommand{nitpick}\isamarkupfalse%
%
\isadelimproof
\ %
\endisadelimproof
%
\isatagproof
\isacommand{oops}\isamarkupfalse%
\ %
\isamarkupcmt{countersatisfiable%
}
%
\endisatagproof
{\isafoldproof}%
%
\isadelimproof
%
\endisadelimproof
%
\begin{isamarkuptext}%
Barcan and Converse Barcan Formulas are satisfied for standard (possibilist) quantifiers:%
\end{isamarkuptext}\isamarkuptrue%
\ \isacommand{lemma}\isamarkupfalse%
\ {\isachardoublequoteopen}{\isasymlfloor}{\isacharparenleft}\isactrlbold {\isasymforall}x{\isachardot}\isactrlbold {\isasymbox}{\isacharparenleft}{\isasymphi}\ x{\isacharparenright}{\isacharparenright}\ \isactrlbold {\isasymrightarrow}\ \isactrlbold {\isasymbox}{\isacharparenleft}\isactrlbold {\isasymforall}x{\isachardot}{\isacharparenleft}{\isasymphi}\ x{\isacharparenright}{\isacharparenright}{\isasymrfloor}{\isachardoublequoteclose}%
\isadelimproof
\ %
\endisadelimproof
%
\isatagproof
\isacommand{by}\isamarkupfalse%
\ simp%
\endisatagproof
{\isafoldproof}%
%
\isadelimproof
%
\endisadelimproof
\isanewline
\ \isacommand{lemma}\isamarkupfalse%
\ {\isachardoublequoteopen}{\isasymlfloor}\isactrlbold {\isasymbox}{\isacharparenleft}\isactrlbold {\isasymforall}x{\isachardot}{\isacharparenleft}{\isasymphi}\ x{\isacharparenright}{\isacharparenright}\ \isactrlbold {\isasymrightarrow}\ {\isacharparenleft}\isactrlbold {\isasymforall}x{\isachardot}\isactrlbold {\isasymbox}{\isacharparenleft}{\isasymphi}\ x{\isacharparenright}{\isacharparenright}{\isasymrfloor}{\isachardoublequoteclose}%
\isadelimproof
\ %
\endisadelimproof
%
\isatagproof
\isacommand{by}\isamarkupfalse%
\ simp%
\endisatagproof
{\isafoldproof}%
%
\isadelimproof
%
\endisadelimproof
%
\begin{isamarkuptext}%
(Converse) Barcan Formulas not satisfied for actualist quantifiers:%
\end{isamarkuptext}\isamarkuptrue%
\ \isacommand{lemma}\isamarkupfalse%
\ {\isachardoublequoteopen}{\isasymlfloor}{\isacharparenleft}\isactrlbold {\isasymforall}\isactrlsup Ex{\isachardot}\isactrlbold {\isasymbox}{\isacharparenleft}{\isasymphi}\ x{\isacharparenright}{\isacharparenright}\ \isactrlbold {\isasymrightarrow}\ \isactrlbold {\isasymbox}{\isacharparenleft}\isactrlbold {\isasymforall}\isactrlsup Ex{\isachardot}{\isacharparenleft}{\isasymphi}\ x{\isacharparenright}{\isacharparenright}{\isasymrfloor}{\isachardoublequoteclose}\ \isacommand{nitpick}\isamarkupfalse%
%
\isadelimproof
\ %
\endisadelimproof
%
\isatagproof
\isacommand{oops}\isamarkupfalse%
\ %
\isamarkupcmt{countersatisfiable%
}
%
\endisatagproof
{\isafoldproof}%
%
\isadelimproof
%
\endisadelimproof
\isanewline
\ \isacommand{lemma}\isamarkupfalse%
\ {\isachardoublequoteopen}{\isasymlfloor}\isactrlbold {\isasymbox}{\isacharparenleft}\isactrlbold {\isasymforall}\isactrlsup Ex{\isachardot}{\isacharparenleft}{\isasymphi}\ x{\isacharparenright}{\isacharparenright}\ \isactrlbold {\isasymrightarrow}\ {\isacharparenleft}\isactrlbold {\isasymforall}\isactrlsup Ex{\isachardot}\isactrlbold {\isasymbox}{\isacharparenleft}{\isasymphi}\ x{\isacharparenright}{\isacharparenright}{\isasymrfloor}{\isachardoublequoteclose}\ \isacommand{nitpick}\isamarkupfalse%
%
\isadelimproof
\ %
\endisadelimproof
%
\isatagproof
\isacommand{oops}\isamarkupfalse%
\ %
\isamarkupcmt{countersatisfiable%
}
%
\endisatagproof
{\isafoldproof}%
%
\isadelimproof
%
\endisadelimproof
%
\begin{isamarkuptext}%
Above we have made use of (counter-)model finder \emph{Nitpick} \cite{Nitpick} for the first time.  
For all the conjectured lemmas above, \emph{Nitpick} has found a countermodel, i.e. a model satisfying all 
the axioms which falsifies the given formula. This means, the formulas are not valid.%
\end{isamarkuptext}\isamarkuptrue%
%
\begin{isamarkuptext}%
Well known relations between meta-logical notions:%
\end{isamarkuptext}\isamarkuptrue%
\ \isacommand{lemma}\isamarkupfalse%
\ \ {\isachardoublequoteopen}{\isasymlfloor}{\isasymphi}{\isasymrfloor}\ {\isasymlongleftrightarrow}\ {\isasymnot}{\isasymlfloor}{\isasymphi}{\isasymrfloor}\isactrlsup c\isactrlsup s\isactrlsup a\isactrlsup t{\isachardoublequoteclose}%
\isadelimproof
\ %
\endisadelimproof
%
\isatagproof
\isacommand{by}\isamarkupfalse%
\ simp%
\endisatagproof
{\isafoldproof}%
%
\isadelimproof
%
\endisadelimproof
\isanewline
\ \isacommand{lemma}\isamarkupfalse%
\ \ {\isachardoublequoteopen}{\isasymlfloor}{\isasymphi}{\isasymrfloor}\isactrlsup s\isactrlsup a\isactrlsup t\ {\isasymlongleftrightarrow}\ {\isasymnot}{\isasymlfloor}{\isasymphi}{\isasymrfloor}\isactrlsup i\isactrlsup n\isactrlsup v\ {\isachardoublequoteclose}%
\isadelimproof
\ %
\endisadelimproof
%
\isatagproof
\isacommand{by}\isamarkupfalse%
\ simp%
\endisatagproof
{\isafoldproof}%
%
\isadelimproof
%
\endisadelimproof
%
\begin{isamarkuptext}%
Contingent truth does not allow for necessitation:%
\end{isamarkuptext}\isamarkuptrue%
\ \isacommand{lemma}\isamarkupfalse%
\ {\isachardoublequoteopen}{\isasymlfloor}\isactrlbold {\isasymdiamond}{\isasymphi}{\isasymrfloor}\ \ {\isasymlongrightarrow}\ {\isasymlfloor}\isactrlbold {\isasymbox}{\isasymphi}{\isasymrfloor}{\isachardoublequoteclose}\ \isacommand{nitpick}\isamarkupfalse%
%
\isadelimproof
\ %
\endisadelimproof
%
\isatagproof
\isacommand{oops}\isamarkupfalse%
\ \ \ \ \ \ \ \ \ \ \ \ %
\isamarkupcmt{countersatisfiable%
}
%
\endisatagproof
{\isafoldproof}%
%
\isadelimproof
%
\endisadelimproof
\isanewline
\ \isacommand{lemma}\isamarkupfalse%
\ {\isachardoublequoteopen}{\isasymlfloor}\isactrlbold {\isasymbox}{\isasymphi}{\isasymrfloor}\isactrlsup s\isactrlsup a\isactrlsup t\ {\isasymlongrightarrow}\ {\isasymlfloor}\isactrlbold {\isasymbox}{\isasymphi}{\isasymrfloor}{\isachardoublequoteclose}\ \isacommand{nitpick}\isamarkupfalse%
%
\isadelimproof
\ %
\endisadelimproof
%
\isatagproof
\isacommand{oops}\isamarkupfalse%
\ \ \ \ \ \ \ \ \ \ \ %
\isamarkupcmt{countersatisfiable%
}
%
\endisatagproof
{\isafoldproof}%
%
\isadelimproof
%
\endisadelimproof
%
\begin{isamarkuptext}%
\emph{Modal collapse} is countersatisfiable:%
\end{isamarkuptext}\isamarkuptrue%
\ \isacommand{lemma}\isamarkupfalse%
\ {\isachardoublequoteopen}{\isasymlfloor}{\isasymphi}\ \isactrlbold {\isasymrightarrow}\ \isactrlbold {\isasymbox}{\isasymphi}{\isasymrfloor}{\isachardoublequoteclose}\ \isacommand{nitpick}\isamarkupfalse%
%
\isadelimproof
\ %
\endisadelimproof
%
\isatagproof
\isacommand{oops}\isamarkupfalse%
\ \ \ \ \ \ \ \ \ \ \ \ \ \ \ \ \ \ %
\isamarkupcmt{countersatisfiable%
}
%
\endisatagproof
{\isafoldproof}%
%
\isadelimproof
%
\endisadelimproof
%
\begin{isamarkuptext}%
\pagebreak%
\end{isamarkuptext}\isamarkuptrue%
%
\isamarkupsubsection{Useful Definitions for Axiomatization of Further Logics%
}
\isamarkuptrue%
%
\begin{isamarkuptext}%
The best known normal logics (\emph{K4, K5, KB, K45, KB5, D, D4, D5, D45, ...}) can be obtained by
 combinations of the following axioms:%
\end{isamarkuptext}\isamarkuptrue%
\ \ \isacommand{abbreviation}\isamarkupfalse%
\ M\ \isanewline
\ \ \ \ \isakeyword{where}\ {\isachardoublequoteopen}M\ {\isasymequiv}\ \isactrlbold {\isasymforall}{\isasymphi}{\isachardot}\ \isactrlbold {\isasymbox}{\isasymphi}\ \isactrlbold {\isasymrightarrow}\ {\isasymphi}{\isachardoublequoteclose}\isanewline
\ \ \isacommand{abbreviation}\isamarkupfalse%
\ B\ \isanewline
\ \ \ \ \isakeyword{where}\ {\isachardoublequoteopen}B\ {\isasymequiv}\ \isactrlbold {\isasymforall}{\isasymphi}{\isachardot}\ {\isasymphi}\ \isactrlbold {\isasymrightarrow}\ \ \isactrlbold {\isasymbox}\isactrlbold {\isasymdiamond}{\isasymphi}{\isachardoublequoteclose}\isanewline
\ \ \isacommand{abbreviation}\isamarkupfalse%
\ D\ \isanewline
\ \ \ \ \isakeyword{where}\ {\isachardoublequoteopen}D\ {\isasymequiv}\ \isactrlbold {\isasymforall}{\isasymphi}{\isachardot}\ \isactrlbold {\isasymbox}{\isasymphi}\ \isactrlbold {\isasymrightarrow}\ \isactrlbold {\isasymdiamond}{\isasymphi}{\isachardoublequoteclose}\isanewline
\ \ \isacommand{abbreviation}\isamarkupfalse%
\ IV\ \isanewline
\ \ \ \ \isakeyword{where}\ {\isachardoublequoteopen}IV\ {\isasymequiv}\ \isactrlbold {\isasymforall}{\isasymphi}{\isachardot}\ \isactrlbold {\isasymbox}{\isasymphi}\ \isactrlbold {\isasymrightarrow}\ \ \isactrlbold {\isasymbox}\isactrlbold {\isasymbox}{\isasymphi}{\isachardoublequoteclose}\isanewline
\ \ \isacommand{abbreviation}\isamarkupfalse%
\ V\ \isanewline
\ \ \ \ \isakeyword{where}\ {\isachardoublequoteopen}V\ {\isasymequiv}\ \isactrlbold {\isasymforall}{\isasymphi}{\isachardot}\ \isactrlbold {\isasymdiamond}{\isasymphi}\ \isactrlbold {\isasymrightarrow}\ \isactrlbold {\isasymbox}\isactrlbold {\isasymdiamond}{\isasymphi}{\isachardoublequoteclose}%
\begin{isamarkuptext}%
Instead of postulating (combinations of) the above  axioms we instead make use of 
  the well-known \emph{Sahlqvist correspondence}, which links axioms to constraints on a model's accessibility
  relation (e.g. reflexive, symmetric, etc.; the definitions of which are not shown here). We show
  that  reflexivity, symmetry, seriality, transitivity and euclideanness imply
  axioms $M, B, D, IV, V$ respectively.%
\end{isamarkuptext}\isamarkuptrue%
\ \ \isacommand{lemma}\isamarkupfalse%
\ {\isachardoublequoteopen}reflexive\ aRel\ \ {\isasymLongrightarrow}\ \ {\isasymlfloor}M{\isasymrfloor}{\isachardoublequoteclose}%
\isadelimproof
\ %
\endisadelimproof
%
\isatagproof
\isacommand{by}\isamarkupfalse%
\ blast\ %
\isamarkupcmt{aka T%
}
%
\endisatagproof
{\isafoldproof}%
%
\isadelimproof
%
\endisadelimproof
\isanewline
\ \ \isacommand{lemma}\isamarkupfalse%
\ {\isachardoublequoteopen}symmetric\ aRel\ {\isasymLongrightarrow}\ {\isasymlfloor}B{\isasymrfloor}{\isachardoublequoteclose}%
\isadelimproof
\ %
\endisadelimproof
%
\isatagproof
\isacommand{by}\isamarkupfalse%
\ blast%
\endisatagproof
{\isafoldproof}%
%
\isadelimproof
%
\endisadelimproof
\isanewline
\ \ \isacommand{lemma}\isamarkupfalse%
\ {\isachardoublequoteopen}serial\ aRel\ \ {\isasymLongrightarrow}\ {\isasymlfloor}D{\isasymrfloor}{\isachardoublequoteclose}%
\isadelimproof
\ %
\endisadelimproof
%
\isatagproof
\isacommand{by}\isamarkupfalse%
\ blast%
\endisatagproof
{\isafoldproof}%
%
\isadelimproof
%
\endisadelimproof
\ \ \ \ \ \ \ \ \ \isanewline
\ \ \isacommand{lemma}\isamarkupfalse%
\ {\isachardoublequoteopen}transitive\ aRel\ \ {\isasymLongrightarrow}\ {\isasymlfloor}IV{\isasymrfloor}{\isachardoublequoteclose}%
\isadelimproof
\ %
\endisadelimproof
%
\isatagproof
\isacommand{by}\isamarkupfalse%
\ blast%
\endisatagproof
{\isafoldproof}%
%
\isadelimproof
%
\endisadelimproof
\ \ \ \isanewline
\ \ \isacommand{lemma}\isamarkupfalse%
\ {\isachardoublequoteopen}euclidean\ aRel\ {\isasymLongrightarrow}\ {\isasymlfloor}V{\isasymrfloor}{\isachardoublequoteclose}%
\isadelimproof
\ %
\endisadelimproof
%
\isatagproof
\isacommand{by}\isamarkupfalse%
\ blast%
\endisatagproof
{\isafoldproof}%
%
\isadelimproof
%
\endisadelimproof
\ \ \ \ \ \ \ \ \ \isanewline
\ \ \isacommand{lemma}\isamarkupfalse%
\ {\isachardoublequoteopen}preorder\ aRel\ {\isasymLongrightarrow}\ \ {\isasymlfloor}M{\isasymrfloor}\ {\isasymand}\ {\isasymlfloor}IV{\isasymrfloor}{\isachardoublequoteclose}%
\isadelimproof
\ %
\endisadelimproof
%
\isatagproof
\isacommand{by}\isamarkupfalse%
\ blast\ %
\isamarkupcmt{S4: reflexive + transitive%
}
%
\endisatagproof
{\isafoldproof}%
%
\isadelimproof
%
\endisadelimproof
\isanewline
\ \ \isacommand{lemma}\isamarkupfalse%
\ {\isachardoublequoteopen}equivalence\ aRel\ \ {\isasymLongrightarrow}\ \ {\isasymlfloor}M{\isasymrfloor}\ {\isasymand}\ {\isasymlfloor}V{\isasymrfloor}{\isachardoublequoteclose}%
\isadelimproof
\ %
\endisadelimproof
%
\isatagproof
\isacommand{by}\isamarkupfalse%
\ blast\ %
\isamarkupcmt{S5: preorder + symmetric%
}
%
\endisatagproof
{\isafoldproof}%
%
\isadelimproof
%
\endisadelimproof
\isanewline
\ \ \isacommand{lemma}\isamarkupfalse%
\ {\isachardoublequoteopen}reflexive\ aRel\ {\isasymand}\ euclidean\ aRel\ \ {\isasymLongrightarrow}\ \ {\isasymlfloor}M{\isasymrfloor}\ {\isasymand}\ {\isasymlfloor}V{\isasymrfloor}{\isachardoublequoteclose}%
\isadelimproof
\ %
\endisadelimproof
%
\isatagproof
\isacommand{by}\isamarkupfalse%
\ blast\ %
\isamarkupcmt{S5%
}
%
\endisatagproof
{\isafoldproof}%
%
\isadelimproof
%
\endisadelimproof
%
\begin{isamarkuptext}%
Using these definitions, we can derive axioms for the most common modal logics (see also \cite{C47}). 
  Thereby we are free to use either the semantic constraints or the related \emph{Sahlqvist} axioms. Here we provide 
  both versions. In what follows we use the semantic constraints (for improved performance).
  \pagebreak%
\end{isamarkuptext}\isamarkuptrue%
%
\isadelimtheory
%
\endisadelimtheory
%
\isatagtheory
%
\endisatagtheory
{\isafoldtheory}%
%
\isadelimtheory
%
\endisadelimtheory
%
\end{isabellebody}%
%%% Local Variables:
%%% mode: latex
%%% TeX-master: "root"
%%% End:

%
\begin{isabellebody}%
\setisabellecontext{GoedelProof}%
%
%
%
%
%
%
%
\isamarkupsection{G\"odel's Ontological Argument%
}
\isamarkuptrue%
%
\isamarkupsubsection{Part I - God's Existence is Possible%
}
\isamarkuptrue%
%
\begin{isamarkuptext}%
\noindent{G\"odel's particular version of the argument is a direct descendant of that of Leibniz, which in turn derives
  from one of Descartes. His argument relies on proving \emph{(T1) `Positive properties are possibly instantiated'},
 which together with \emph{(T2) `God is a positive property'} directly implies the conclusion.
 In order to prove \emph{T1}, G\"odel assumes \emph{(A2) `Any property entailed by a positive property is positive'}.
 As we will see, the success of this argumentation depends on how we formalize our notion of entailment.}%
\end{isamarkuptext}\isamarkuptrue%
\ \ \isacommand{abbreviation}\isamarkupfalse%
\ Entails{\isacharcolon}{\isacharcolon}{\isachardoublequoteopen}{\isasymup}{\isasymlangle}{\isasymup}{\isasymlangle}e{\isasymrangle}{\isacharcomma}{\isasymup}{\isasymlangle}e{\isasymrangle}{\isasymrangle}{\isachardoublequoteclose}\ {\isacharparenleft}\isakeyword{infix}{\isachardoublequoteopen}{\isasymRrightarrow}{\isachardoublequoteclose}{\isacharparenright}\ \isakeyword{where}\ {\isachardoublequoteopen}X{\isasymRrightarrow}Y\ {\isasymequiv}\ \isactrlbold {\isasymbox}{\isacharparenleft}\isactrlbold {\isasymforall}\isactrlsup Az{\isachardot}\ X\ z\ \isactrlbold {\isasymrightarrow}\ Y\ z{\isacharparenright}{\isachardoublequoteclose}\isanewline
\ \ \isacommand{lemma}\isamarkupfalse%
\ {\isachardoublequoteopen}{\isasymlfloor}{\isacharparenleft}{\isasymlambda}x\ w{\isachardot}\ x\ {\isasymnoteq}\ x{\isacharparenright}\ {\isasymRrightarrow}\ {\isasymchi}{\isasymrfloor}{\isachardoublequoteclose}%
\ %
%
\isacommand{by}\isamarkupfalse%
\ simp\ %
\isamarkupcmt{an impossible property entails anything%
}
%
%
%
\isanewline
\ \ \isacommand{lemma}\isamarkupfalse%
\ {\isachardoublequoteopen}{\isasymlfloor}\isactrlbold {\isasymnot}{\isacharparenleft}{\isasymphi}\ {\isasymRrightarrow}\ {\isasymchi}{\isacharparenright}\ \isactrlbold {\isasymrightarrow}\ \isactrlbold {\isasymdiamond}\isactrlbold {\isasymexists}\isactrlsup A\ {\isasymphi}{\isasymrfloor}{\isachardoublequoteclose}%
\ %
%
\isacommand{by}\isamarkupfalse%
\ auto\ %
\isamarkupcmt{possible instantiation of \isa{{\isasymphi}} implicit%
}
%
%
%
%
\begin{isamarkuptext}%
\noindent{The definition of property entailment introduced by G\"odel can be criticized on the grounds that it lacks
 some notion of relevance and is therefore exposed to the paradoxes of material implication.
 In particular, when we assert that property \emph{A} does not entail property \emph{B}, we implicitly assume that
 \emph{A} is possibly instantiated. Conversely, an impossible property (like being a round square) entails any property
 (like being a triangle). It is precisely by virtue of these paradoxes that G\"odel manages to prove \emph{T1}.\footnote{To
 prove T1, the fact is used that positive properties cannot \emph{entail} negative ones (A2), 
 from which the possible instantiation of positive properties follows.
 A computer-formalization of Leibniz's theory of concepts can be found in \cite{Zalta15},
 where the notion of \emph{concept containment} in contrast to ordinary \emph{property entailment} is discussed.}}%
\end{isamarkuptext}\isamarkuptrue%
\ \ \isacommand{consts}\isamarkupfalse%
\ Positiveness{\isacharcolon}{\isacharcolon}{\isachardoublequoteopen}{\isasymup}{\isasymlangle}{\isasymup}{\isasymlangle}e{\isasymrangle}{\isasymrangle}{\isachardoublequoteclose}\ {\isacharparenleft}{\isachardoublequoteopen}{\isasymP}{\isachardoublequoteclose}{\isacharparenright}\ %
\isamarkupcmt{positiveness applies to intensional predicates%
}
\isanewline
\ \ \isacommand{abbreviation}\isamarkupfalse%
\ Existence{\isacharcolon}{\isacharcolon}{\isachardoublequoteopen}{\isasymup}{\isasymlangle}e{\isasymrangle}{\isachardoublequoteclose}\ {\isacharparenleft}{\isachardoublequoteopen}E{\isacharbang}{\isachardoublequoteclose}{\isacharparenright}\ %
\isamarkupcmt{object-language existence predicate%
}
\ \isanewline
\ \ \ \ \isakeyword{where}\ {\isachardoublequoteopen}E{\isacharbang}\ x\ \ {\isasymequiv}\ {\isasymlambda}w{\isachardot}\ {\isacharparenleft}\isactrlbold {\isasymexists}\isactrlsup Ay{\isachardot}\ y\isactrlbold {\isasymapprox}x{\isacharparenright}\ w{\isachardoublequoteclose}%
\begin{isamarkuptext}%
\noindent{G\"odel's axioms for the first part essentially say that (A1) either a property or its negation must be positive,
  (A2) positive properties are closed under entailment and (A3) also closed under conjunction.}%
\end{isamarkuptext}\isamarkuptrue%
\ \ \isacommand{abbreviation}\isamarkupfalse%
\ appliesToPositiveProps{\isacharcolon}{\isacharcolon}{\isachardoublequoteopen}{\isasymup}{\isasymlangle}{\isasymup}{\isasymlangle}{\isasymup}{\isasymlangle}e{\isasymrangle}{\isasymrangle}{\isasymrangle}{\isachardoublequoteclose}\ {\isacharparenleft}{\isachardoublequoteopen}pos{\isachardoublequoteclose}{\isacharparenright}\ \isakeyword{where}\isanewline
\ \ \ \ {\isachardoublequoteopen}pos\ Z\ {\isasymequiv}\ \ \isactrlbold {\isasymforall}X{\isachardot}\ Z\ X\ \isactrlbold {\isasymrightarrow}\ {\isasymP}\ X{\isachardoublequoteclose}\ \ \isanewline
\ \ \isacommand{abbreviation}\isamarkupfalse%
\ intersectionOf{\isacharcolon}{\isacharcolon}{\isachardoublequoteopen}{\isasymup}{\isasymlangle}{\isasymup}{\isasymlangle}e{\isasymrangle}{\isacharcomma}{\isasymup}{\isasymlangle}{\isasymup}{\isasymlangle}e{\isasymrangle}{\isasymrangle}{\isasymrangle}{\isachardoublequoteclose}\ {\isacharparenleft}{\isachardoublequoteopen}intersec{\isachardoublequoteclose}{\isacharparenright}\ \isakeyword{where}\isanewline
\ \ \ \ {\isachardoublequoteopen}intersec\ X\ Z\ {\isasymequiv}\ \ \isactrlbold {\isasymbox}{\isacharparenleft}\isactrlbold {\isasymforall}x{\isachardot}{\isacharparenleft}X\ x\ \isactrlbold {\isasymleftrightarrow}\ {\isacharparenleft}\isactrlbold {\isasymforall}Y{\isachardot}\ {\isacharparenleft}Z\ Y{\isacharparenright}\ \isactrlbold {\isasymrightarrow}\ {\isacharparenleft}Y\ x{\isacharparenright}{\isacharparenright}{\isacharparenright}{\isacharparenright}{\isachardoublequoteclose}\ \ \isanewline
\ \ \isacommand{axiomatization}\isamarkupfalse%
\ \isakeyword{where}\isanewline
\ \ \ A{\isadigit{1}}a{\isacharcolon}\ {\isachardoublequoteopen}{\isasymlfloor}\isactrlbold {\isasymforall}X{\isachardot}\ {\isasymP}\ {\isacharparenleft}\isactrlbold {\isasymrightharpoondown}X{\isacharparenright}\ \isactrlbold {\isasymrightarrow}\ \isactrlbold {\isasymnot}{\isacharparenleft}{\isasymP}\ X{\isacharparenright}\ {\isasymrfloor}{\isachardoublequoteclose}\ \isakeyword{and}\isanewline
\ \ \ A{\isadigit{1}}b{\isacharcolon}\ {\isachardoublequoteopen}{\isasymlfloor}\isactrlbold {\isasymforall}X{\isachardot}\ \isactrlbold {\isasymnot}{\isacharparenleft}{\isasymP}\ X{\isacharparenright}\ \isactrlbold {\isasymrightarrow}\ {\isasymP}\ {\isacharparenleft}\isactrlbold {\isasymrightharpoondown}X{\isacharparenright}{\isasymrfloor}{\isachardoublequoteclose}\ \isakeyword{and}\ \isanewline
\ \ \ A{\isadigit{2}}{\isacharcolon}\ \ {\isachardoublequoteopen}{\isasymlfloor}\isactrlbold {\isasymforall}X\ Y{\isachardot}{\isacharparenleft}{\isasymP}\ X\ \isactrlbold {\isasymand}\ {\isacharparenleft}X\ {\isasymRrightarrow}\ Y{\isacharparenright}{\isacharparenright}\ \isactrlbold {\isasymrightarrow}\ {\isasymP}\ Y{\isasymrfloor}{\isachardoublequoteclose}\ \isakeyword{and}\isanewline
\ \ \ A{\isadigit{3}}{\isacharcolon}\ \ {\isachardoublequoteopen}{\isasymlfloor}\isactrlbold {\isasymforall}Z\ X{\isachardot}\ {\isacharparenleft}pos\ Z\ \isactrlbold {\isasymand}\ intersec\ X\ Z{\isacharparenright}\ \isactrlbold {\isasymrightarrow}\ {\isasymP}\ X{\isasymrfloor}{\isachardoublequoteclose}\isanewline
\ \ \isanewline
\ \ \isacommand{lemma}\isamarkupfalse%
\ True\ \isacommand{nitpick}\isamarkupfalse%
{\isacharbrackleft}satisfy{\isacharbrackright}%
\ %
%
\isacommand{oops}\isamarkupfalse%
\ \ \ \ %
\isamarkupcmt{model found: axioms are consistent%
}
%
%
%
\isanewline
\ \ \isacommand{lemma}\isamarkupfalse%
\ {\isachardoublequoteopen}{\isasymlfloor}D{\isasymrfloor}{\isachardoublequoteclose}%
\ \ %
%
\isacommand{using}\isamarkupfalse%
\ A{\isadigit{1}}a\ A{\isadigit{1}}b\ A{\isadigit{2}}\ \isacommand{by}\isamarkupfalse%
\ blast\ %
\isamarkupcmt{\emph{D} axiom is implicitely assumed%
}
%
%
%
%
\begin{isamarkuptext}%
\noindent{Positive properties are possibly instantiated.}%
\end{isamarkuptext}\isamarkuptrue%
\ \ \isacommand{theorem}\isamarkupfalse%
\ T{\isadigit{1}}{\isacharcolon}\ {\isachardoublequoteopen}{\isasymlfloor}\isactrlbold {\isasymforall}X{\isachardot}\ {\isasymP}\ X\ \isactrlbold {\isasymrightarrow}\ \isactrlbold {\isasymdiamond}\isactrlbold {\isasymexists}\isactrlsup A\ X{\isasymrfloor}{\isachardoublequoteclose}%
\ %
%
\isacommand{using}\isamarkupfalse%
\ A{\isadigit{1}}a\ A{\isadigit{2}}\ \isacommand{by}\isamarkupfalse%
\ blast%
%
%
%
\begin{isamarkuptext}%
\noindent{Being Godlike is defined as having all (and only) positive properties.}%
\end{isamarkuptext}\isamarkuptrue%
\ \ \isacommand{abbreviation}\isamarkupfalse%
\ God{\isacharcolon}{\isacharcolon}{\isachardoublequoteopen}{\isasymup}{\isasymlangle}e{\isasymrangle}{\isachardoublequoteclose}\ {\isacharparenleft}{\isachardoublequoteopen}G{\isachardoublequoteclose}{\isacharparenright}\ \isakeyword{where}\ {\isachardoublequoteopen}G\ {\isasymequiv}\ {\isacharparenleft}{\isasymlambda}x{\isachardot}\ \isactrlbold {\isasymforall}Y{\isachardot}\ {\isasymP}\ Y\ \isactrlbold {\isasymrightarrow}\ Y\ x{\isacharparenright}{\isachardoublequoteclose}\isanewline
\ \ \isacommand{abbreviation}\isamarkupfalse%
\ God{\isacharunderscore}star{\isacharcolon}{\isacharcolon}{\isachardoublequoteopen}{\isasymup}{\isasymlangle}e{\isasymrangle}{\isachardoublequoteclose}\ {\isacharparenleft}{\isachardoublequoteopen}G{\isacharasterisk}{\isachardoublequoteclose}{\isacharparenright}\ \isakeyword{where}\ {\isachardoublequoteopen}G{\isacharasterisk}\ {\isasymequiv}\ {\isacharparenleft}{\isasymlambda}x{\isachardot}\ \isactrlbold {\isasymforall}Y{\isachardot}\ {\isasymP}\ Y\ \isactrlbold {\isasymleftrightarrow}\ Y\ x{\isacharparenright}{\isachardoublequoteclose}\isanewline
\ \ \isacommand{lemma}\isamarkupfalse%
\ GodDefsAreEquivalent{\isacharcolon}\ {\isachardoublequoteopen}{\isasymlfloor}\isactrlbold {\isasymforall}x{\isachardot}\ G\ x\ \isactrlbold {\isasymleftrightarrow}\ G{\isacharasterisk}\ x{\isasymrfloor}{\isachardoublequoteclose}%
\ %
%
\isacommand{using}\isamarkupfalse%
\ A{\isadigit{1}}b\ \isacommand{by}\isamarkupfalse%
\ force%
%
%
%
\begin{isamarkuptext}%
\noindent{While Leibniz provides an informal proof for the compatibility of all perfections,
   G\"odel postulates this as \emph{A3} (the conjunction of \emph{any} collection of positive properties is positive),
  which is a third-order axiom. As shown below, the only use of \emph{A3} is to prove that
  being Godlike is positive (\emph{T2}). Dana Scott, apparently noting this, proposed taking it directly as an axiom
  (see \cite{Fitting}, p. 152).\footnote{We provide a proof in Isabelle/Isar, a language specifically
  tailored for writing proofs that are both computer- and human-readable.
  We refer the reader to \cite{J35} for other proofs not shown in this article.}}%
\end{isamarkuptext}\isamarkuptrue%
\ \ \isacommand{theorem}\isamarkupfalse%
\ T{\isadigit{2}}{\isacharcolon}\ {\isachardoublequoteopen}{\isasymlfloor}{\isasymP}\ G{\isasymrfloor}{\isachardoublequoteclose}%
\ %
%
\isacommand{proof}\isamarkupfalse%
\ {\isacharminus}\isanewline
\ \ \isacommand{{\isacharbraceleft}}\isamarkupfalse%
\ \isacommand{fix}\isamarkupfalse%
\ w\ \isanewline
\ \ \ \ \isacommand{have}\isamarkupfalse%
\ {\isadigit{1}}{\isacharcolon}\ {\isachardoublequoteopen}{\isacharparenleft}{\isacharparenleft}pos\ {\isasymP}{\isacharparenright}\ \isactrlbold {\isasymand}\ {\isacharparenleft}intersec\ G\ {\isasymP}{\isacharparenright}{\isacharparenright}\ w{\isachardoublequoteclose}\ \isacommand{by}\isamarkupfalse%
\ simp\isanewline
\ \ \ \ \isacommand{have}\isamarkupfalse%
\ {\isachardoublequoteopen}{\isacharparenleft}\isactrlbold {\isasymforall}Z\ X{\isachardot}\ {\isacharparenleft}pos\ Z\ \isactrlbold {\isasymand}\ intersec\ X\ Z{\isacharparenright}\ \isactrlbold {\isasymrightarrow}\ {\isasymP}\ X{\isacharparenright}\ w{\isachardoublequoteclose}\ \isacommand{using}\isamarkupfalse%
\ A{\isadigit{3}}\ \isacommand{by}\isamarkupfalse%
\ {\isacharparenleft}rule\ allE{\isacharparenright}\isanewline
\ \ \ \ \isacommand{hence}\isamarkupfalse%
\ {\isachardoublequoteopen}{\isacharparenleft}{\isacharparenleft}{\isacharparenleft}pos\ {\isasymP}{\isacharparenright}\ \isactrlbold {\isasymand}\ {\isacharparenleft}intersec\ G\ {\isasymP}{\isacharparenright}{\isacharparenright}\ \isactrlbold {\isasymrightarrow}\ {\isasymP}\ G{\isacharparenright}\ w{\isachardoublequoteclose}\ \isacommand{using}\isamarkupfalse%
\ allE\ \isacommand{by}\isamarkupfalse%
\ {\isacharparenleft}rule\ allE{\isacharparenright}\isanewline
\ \ \ \ \isacommand{hence}\isamarkupfalse%
\ {\isachardoublequoteopen}{\isacharparenleft}{\isacharparenleft}pos\ {\isasymP}\ \isactrlbold {\isasymand}\ intersec\ G\ {\isasymP}{\isacharparenright}\ w{\isacharparenright}\ {\isasymlongrightarrow}\ {\isasymP}\ G\ w{\isachardoublequoteclose}\ \ \isacommand{by}\isamarkupfalse%
\ simp\isanewline
\ \ \ \ \isacommand{hence}\isamarkupfalse%
\ {\isachardoublequoteopen}{\isasymP}\ G\ w{\isachardoublequoteclose}\ \isacommand{using}\isamarkupfalse%
\ {\isadigit{1}}\ \isacommand{by}\isamarkupfalse%
\ {\isacharparenleft}rule\ mp{\isacharparenright}\isanewline
\ \ \isacommand{{\isacharbraceright}}\isamarkupfalse%
\ \isacommand{thus}\isamarkupfalse%
\ {\isacharquery}thesis\ \isacommand{by}\isamarkupfalse%
\ {\isacharparenleft}rule\ allI{\isacharparenright}\isanewline
\ \ \isacommand{qed}\isamarkupfalse%
%
%
%
%
\begin{isamarkuptext}%
\noindent{Conclusion for the first part: Possibly God exists.}%
\end{isamarkuptext}\isamarkuptrue%
\ \ \isacommand{theorem}\isamarkupfalse%
\ T{\isadigit{3}}{\isacharcolon}\ {\isachardoublequoteopen}{\isasymlfloor}\isactrlbold {\isasymdiamond}\isactrlbold {\isasymexists}\isactrlsup A\ G{\isasymrfloor}{\isachardoublequoteclose}%
\ \ %
%
\isacommand{using}\isamarkupfalse%
\ T{\isadigit{1}}\ T{\isadigit{2}}\ \isacommand{by}\isamarkupfalse%
\ simp%
%
%
%
\isamarkupsubsection{Part II - God's Existence is Necessary, if Possible%
}
\isamarkuptrue%
%
\begin{isamarkuptext}%
\noindent{We show here that some additional (philosophically controversial) assumptions are needed to prove
the argument's conclusion, including an \emph{essentialist} premise and the \emph{S5} axioms.
(G\"odel's resp. Scott's original version works in \emph{extensional} HOML already for modal logic \emph{B} \cite{ECAI,C55}).
Further derived results like monotheism and absence of free will are also discussed.}%
\end{isamarkuptext}\isamarkuptrue%
\ \ \isacommand{axiomatization}\isamarkupfalse%
\ \isakeyword{where}\ A{\isadigit{4}}a{\isacharcolon}\ {\isachardoublequoteopen}{\isasymlfloor}\isactrlbold {\isasymforall}X{\isachardot}\ {\isasymP}\ X\ \isactrlbold {\isasymrightarrow}\ \isactrlbold {\isasymbox}{\isacharparenleft}{\isasymP}\ X{\isacharparenright}{\isasymrfloor}{\isachardoublequoteclose}%
\begin{isamarkuptext}%
\noindent{\emph{A4b} was originally assumed by G\"odel as an axiom. We can now prove it.}%
\end{isamarkuptext}\isamarkuptrue%
\ \ \isacommand{lemma}\isamarkupfalse%
\ A{\isadigit{4}}b{\isacharcolon}\ {\isachardoublequoteopen}{\isasymlfloor}\isactrlbold {\isasymforall}X{\isachardot}\ \isactrlbold {\isasymnot}{\isacharparenleft}{\isasymP}\ X{\isacharparenright}\ \isactrlbold {\isasymrightarrow}\ \isactrlbold {\isasymbox}\isactrlbold {\isasymnot}{\isacharparenleft}{\isasymP}\ X{\isacharparenright}{\isasymrfloor}{\isachardoublequoteclose}%
\ %
%
\isacommand{using}\isamarkupfalse%
\ A{\isadigit{1}}a\ A{\isadigit{1}}b\ A{\isadigit{4}}a\ \isacommand{by}\isamarkupfalse%
\ blast%
%
%
\isanewline
\ \ \isacommand{lemma}\isamarkupfalse%
\ True\ \isacommand{nitpick}\isamarkupfalse%
{\isacharbrackleft}satisfy{\isacharbrackright}%
\ %
%
\isacommand{oops}\isamarkupfalse%
\ %
\isamarkupcmt{model found: all axioms A1-4 consistent%
}
%
%
%
%
\begin{isamarkuptext}%
\noindent{Axiom \emph{A4a} and its consequence \emph{A4b} together imply that \isa{{\isasymP}} satisfies Fitting's
  \emph{stability conditions} (\cite{Fitting}, p. 124). This means \isa{{\isasymP}} designates rigidly.
  Note that this makes for an \emph{essentialist} assumption which may be considered controversial by
  some philosophers: every property considered positive in our world (e.g. honesty) is necessarily so.}%
\end{isamarkuptext}\isamarkuptrue%
\ \ \isacommand{lemma}\isamarkupfalse%
\ {\isachardoublequoteopen}{\isasymlfloor}rigid\ {\isasymP}{\isasymrfloor}{\isachardoublequoteclose}%
\ %
%
\isacommand{using}\isamarkupfalse%
\ A{\isadigit{4}}a\ A{\isadigit{4}}b\ \isacommand{by}\isamarkupfalse%
\ blast%
%
%
%
\begin{isamarkuptext}%
\noindent{G\"odel defines a particular notion of essence. \emph{Y} is an essence of \emph{x} iff \emph{Y}
  \emph{entails} every other property \emph{x} possesses.\footnote{Essence is defined here (and in Fitting's variant)
  in the version of Scott; G\"odel's original version leads to the inconsistency reported in \cite{C55,C60}.}}%
\end{isamarkuptext}\isamarkuptrue%
\ \ \isacommand{abbreviation}\isamarkupfalse%
\ Essence{\isacharcolon}{\isacharcolon}{\isachardoublequoteopen}{\isasymup}{\isasymlangle}{\isasymup}{\isasymlangle}e{\isasymrangle}{\isacharcomma}e{\isasymrangle}{\isachardoublequoteclose}\ {\isacharparenleft}{\isachardoublequoteopen}{\isasymE}{\isachardoublequoteclose}{\isacharparenright}\ \isakeyword{where}\ {\isachardoublequoteopen}{\isasymE}\ Y\ x\ {\isasymequiv}\ Y\ x\ \isactrlbold {\isasymand}\ {\isacharparenleft}\isactrlbold {\isasymforall}Z{\isachardot}\ Z\ x\ \isactrlbold {\isasymrightarrow}\ Y{\isasymRrightarrow}Z{\isacharparenright}{\isachardoublequoteclose}\ \ \ \isanewline
\ \ \isacommand{abbreviation}\isamarkupfalse%
\ beingIdenticalTo{\isacharcolon}{\isacharcolon}{\isachardoublequoteopen}e{\isasymRightarrow}{\isasymup}{\isasymlangle}e{\isasymrangle}{\isachardoublequoteclose}\ {\isacharparenleft}{\isachardoublequoteopen}id{\isachardoublequoteclose}{\isacharparenright}\ \isakeyword{where}\isanewline
\ \ \ \ {\isachardoublequoteopen}id\ x\ \ {\isasymequiv}\ {\isacharparenleft}{\isasymlambda}y{\isachardot}\ y\isactrlbold {\isasymapprox}x{\isacharparenright}{\isachardoublequoteclose}\ %
\isamarkupcmt{\emph{id} is here a rigid predicate%
}
%
\begin{isamarkuptext}%
\noindent{Being Godlike is an essential property.}%
\end{isamarkuptext}\isamarkuptrue%
\ \ \isacommand{lemma}\isamarkupfalse%
\ GodIsEssential{\isacharcolon}\ {\isachardoublequoteopen}{\isasymlfloor}\isactrlbold {\isasymforall}x{\isachardot}\ G\ x\ \isactrlbold {\isasymrightarrow}\ {\isacharparenleft}{\isasymE}\ G\ x{\isacharparenright}{\isasymrfloor}{\isachardoublequoteclose}%
\ %
%
\isacommand{using}\isamarkupfalse%
\ A{\isadigit{1}}b\ A{\isadigit{4}}a\ \isacommand{by}\isamarkupfalse%
\ metis%
%
%
%
\begin{isamarkuptext}%
\noindent{Something can have only \emph{one} essence.}%
\end{isamarkuptext}\isamarkuptrue%
\ \ \isacommand{lemma}\isamarkupfalse%
\ {\isachardoublequoteopen}{\isasymlfloor}\isactrlbold {\isasymforall}X\ Y\ z{\isachardot}\ {\isacharparenleft}{\isasymE}\ X\ z\ \isactrlbold {\isasymand}\ {\isasymE}\ Y\ z{\isacharparenright}\ \isactrlbold {\isasymrightarrow}\ {\isacharparenleft}X\ {\isasymRrightarrow}\ Y{\isacharparenright}{\isasymrfloor}{\isachardoublequoteclose}%
\ %
%
\isacommand{by}\isamarkupfalse%
\ meson%
%
%
%
\begin{isamarkuptext}%
\noindent{An essential property offers a complete characterization of an individual.}%
\end{isamarkuptext}\isamarkuptrue%
\ \ \isacommand{lemma}\isamarkupfalse%
\ EssencesCharacterizeCompletely{\isacharcolon}\ {\isachardoublequoteopen}{\isasymlfloor}\isactrlbold {\isasymforall}X\ y{\isachardot}\ {\isasymE}\ X\ y\ \isactrlbold {\isasymrightarrow}\ {\isacharparenleft}X\ {\isasymRrightarrow}\ {\isacharparenleft}id\ y{\isacharparenright}{\isacharparenright}{\isasymrfloor}{\isachardoublequoteclose}\isanewline
%
\ \ \ \ %
%
\isacommand{proof}\isamarkupfalse%
\ {\isacharparenleft}rule\ ccontr{\isacharparenright}\ %
\isamarkupcmt{Isar proof by contradiction not shown here%
}
%
%
%
%
\begin{isamarkuptext}%
\noindent{G\"odel introduces a particular notion of \emph{necessary existence} as the property something has,
   provided any essence of it is necessarily instantiated.}%
\end{isamarkuptext}\isamarkuptrue%
\ \ \isacommand{abbreviation}\isamarkupfalse%
\ necessaryExistencePredicate{\isacharcolon}{\isacharcolon}{\isachardoublequoteopen}{\isasymup}{\isasymlangle}e{\isasymrangle}{\isachardoublequoteclose}\ {\isacharparenleft}{\isachardoublequoteopen}NE{\isachardoublequoteclose}{\isacharparenright}\ \isanewline
\ \ \ \ \isakeyword{where}\ {\isachardoublequoteopen}NE\ x\ \ {\isasymequiv}\ {\isacharparenleft}{\isasymlambda}w{\isachardot}\ {\isacharparenleft}\isactrlbold {\isasymforall}Y{\isachardot}\ \ {\isasymE}\ Y\ x\ \isactrlbold {\isasymrightarrow}\ \isactrlbold {\isasymbox}\isactrlbold {\isasymexists}\isactrlsup A\ Y{\isacharparenright}\ w{\isacharparenright}{\isachardoublequoteclose}\isanewline
\ \ \isanewline
\ \ \isacommand{axiomatization}\isamarkupfalse%
\ \isakeyword{where}\ A{\isadigit{5}}{\isacharcolon}\ {\isachardoublequoteopen}{\isasymlfloor}{\isasymP}\ NE{\isasymrfloor}{\isachardoublequoteclose}\ %
\isamarkupcmt{necessary existence is a positive property%
}
\isanewline
\ \ \isacommand{lemma}\isamarkupfalse%
\ True\ \isacommand{nitpick}\isamarkupfalse%
{\isacharbrackleft}satisfy{\isacharbrackright}%
\ %
%
\isacommand{oops}\isamarkupfalse%
\ %
\isamarkupcmt{model found: so far all axioms consistent%
}
%
%
%
%
\begin{isamarkuptext}%
\noindent{(Possibilist) existence of God implies its necessary (actualist) existence.}%
\end{isamarkuptext}\isamarkuptrue%
\ \ \isacommand{theorem}\isamarkupfalse%
\ T{\isadigit{4}}{\isacharcolon}\ {\isachardoublequoteopen}{\isasymlfloor}\isactrlbold {\isasymexists}\ G\ \isactrlbold {\isasymrightarrow}\ \isactrlbold {\isasymbox}\isactrlbold {\isasymexists}\isactrlsup A\ G{\isasymrfloor}{\isachardoublequoteclose}%
\ %
%
\isacommand{proof}\isamarkupfalse%
\ {\isacharminus}\ %
\isamarkupcmt{not shown%
}
%
%
%
%
\begin{isamarkuptext}%
\noindent{We postulate the \emph{S5} axioms (via \emph{Sahlqvist correspondence}) separately,
  in order to get more detailed information about their relevance in the proofs below.}%
\end{isamarkuptext}\isamarkuptrue%
\ \ \isacommand{axiomatization}\isamarkupfalse%
\ \isakeyword{where}\isanewline
\ \ \ ax{\isacharunderscore}T{\isacharcolon}\ {\isachardoublequoteopen}reflexive\ aRel{\isachardoublequoteclose}\ \isakeyword{and}\ ax{\isacharunderscore}B{\isacharcolon}\ {\isachardoublequoteopen}symmetric\ aRel{\isachardoublequoteclose}\ \isakeyword{and}\ ax{\isacharunderscore}IV{\isacharcolon}\ {\isachardoublequoteopen}transitive\ aRel{\isachardoublequoteclose}\ \isanewline
\ \ \ \isanewline
\ \ \isacommand{lemma}\isamarkupfalse%
\ True\ \isacommand{nitpick}\isamarkupfalse%
{\isacharbrackleft}satisfy{\isacharbrackright}%
\ %
%
\isacommand{oops}\isamarkupfalse%
\ %
\isamarkupcmt{model found: axioms still consistent%
}
%
%
%
%
%
%
%
%
%
%
%
%
%
%
%
%
\begin{isamarkuptext}%
\noindent{Possible existence of God implies its necessary (actualist) existence (note that we only rely on axioms \emph{B} and \emph{IV}).}%
\end{isamarkuptext}\isamarkuptrue%
\ \ \isacommand{theorem}\isamarkupfalse%
\ T{\isadigit{5}}{\isacharcolon}\ {\isachardoublequoteopen}{\isasymlfloor}\isactrlbold {\isasymdiamond}\isactrlbold {\isasymexists}\ G{\isasymrfloor}\ {\isasymlongrightarrow}\ {\isasymlfloor}\isactrlbold {\isasymbox}\isactrlbold {\isasymexists}\isactrlsup A\ G{\isasymrfloor}{\isachardoublequoteclose}%
\ %
%
\isacommand{proof}\isamarkupfalse%
\ {\isacharminus}\ %
\isamarkupcmt{not shown%
}
%
%
%
\ \ \ \ \isanewline
\ \ \isacommand{theorem}\isamarkupfalse%
\ GodExistsNecessarily{\isacharcolon}\ {\isachardoublequoteopen}{\isasymlfloor}\isactrlbold {\isasymbox}\isactrlbold {\isasymexists}\isactrlsup A\ G{\isasymrfloor}{\isachardoublequoteclose}%
\ %
%
\isacommand{using}\isamarkupfalse%
\ T{\isadigit{3}}\ T{\isadigit{5}}\ \isacommand{by}\isamarkupfalse%
\ metis%
%
%
\isanewline
\ \ \isacommand{lemma}\isamarkupfalse%
\ GodExistenceIsValid{\isacharcolon}\ {\isachardoublequoteopen}{\isasymlfloor}\isactrlbold {\isasymexists}\isactrlsup A\ G{\isasymrfloor}{\isachardoublequoteclose}%
\ %
%
\isacommand{using}\isamarkupfalse%
\ GodExistsNecessarily\ ax{\isacharunderscore}T\ \isacommand{by}\isamarkupfalse%
\ auto%
%
%
%
\begin{isamarkuptext}%
\noindent{Monotheism for non-normal models (using Leibniz equality) follows directly from God having all
  and only positive properties, but the proof for normal models is trickier. We need to consider previous results
 (\cite{Fitting}, p. 162).}%
\end{isamarkuptext}\isamarkuptrue%
\ \ \isacommand{lemma}\isamarkupfalse%
\ Monotheism{\isacharunderscore}LeibnizEq{\isacharcolon}{\isachardoublequoteopen}{\isasymlfloor}\isactrlbold {\isasymforall}x{\isachardot}\ G{\isacharasterisk}\ x\ \isactrlbold {\isasymrightarrow}\ {\isacharparenleft}\isactrlbold {\isasymforall}y{\isachardot}\ G{\isacharasterisk}\ y\ \isactrlbold {\isasymrightarrow}\ x\isactrlbold {\isasymapprox}\isactrlsup Ly{\isacharparenright}{\isasymrfloor}{\isachardoublequoteclose}%
\ %
%
\isacommand{by}\isamarkupfalse%
\ meson%
%
%
\isanewline
\ \ \isacommand{lemma}\isamarkupfalse%
\ Monotheism{\isacharunderscore}normal{\isacharcolon}\ {\isachardoublequoteopen}{\isasymlfloor}\isactrlbold {\isasymexists}x{\isachardot}\isactrlbold {\isasymforall}y{\isachardot}\ G\ y\ \isactrlbold {\isasymleftrightarrow}\ x\ \isactrlbold {\isasymapprox}\ y{\isasymrfloor}{\isachardoublequoteclose}%
\ %
%
\isacommand{proof}\isamarkupfalse%
\ {\isacharminus}\ %
\isamarkupcmt{not shown%
}
%
%
%
%
\begin{isamarkuptext}%
\noindent{Fitting \cite{Fitting} also discusses the objection raised by Sobel \cite{sobel2004logic}, 
  who argues that G\"odel's axiom system is too strong since it implies that whatever is the case is so necessarily: the modal system collapses.
  In the context of our S5 axioms, we can formalize Sobel's argument and prove \emph{modal collapse} valid (\cite{Fitting}, pp. 163-4).}%
\end{isamarkuptext}\isamarkuptrue%
\ \ \isacommand{lemma}\isamarkupfalse%
\ useful{\isacharcolon}\ {\isachardoublequoteopen}{\isacharparenleft}{\isasymforall}x{\isachardot}\ {\isasymphi}\ x\ {\isasymlongrightarrow}\ {\isasympsi}{\isacharparenright}\ {\isasymLongrightarrow}\ {\isacharparenleft}{\isacharparenleft}{\isasymexists}x{\isachardot}\ {\isasymphi}\ x{\isacharparenright}\ {\isasymlongrightarrow}\ {\isasympsi}{\isacharparenright}{\isachardoublequoteclose}%
\ %
%
\isacommand{by}\isamarkupfalse%
\ simp%
%
%
\isanewline
\ \ \isacommand{lemma}\isamarkupfalse%
\ ModalCollapse{\isacharcolon}\ {\isachardoublequoteopen}{\isasymlfloor}\isactrlbold {\isasymforall}{\isasymPhi}{\isachardot}\ {\isasymPhi}\ \isactrlbold {\isasymrightarrow}\ \isactrlbold {\isasymbox}{\isasymPhi}{\isasymrfloor}{\isachardoublequoteclose}%
\ %
%
\isacommand{proof}\isamarkupfalse%
\ {\isacharminus}\isanewline
\ \ \ \isacommand{{\isacharbraceleft}}\isamarkupfalse%
\ \isacommand{fix}\isamarkupfalse%
\ w\isanewline
\ \ \ \ \ \isacommand{{\isacharbraceleft}}\isamarkupfalse%
\ \isacommand{fix}\isamarkupfalse%
\ Q\isanewline
\ \ \ \ \ \ \isacommand{have}\isamarkupfalse%
\ {\isachardoublequoteopen}{\isacharparenleft}\isactrlbold {\isasymforall}x{\isachardot}\ G\ x\ \isactrlbold {\isasymrightarrow}\ {\isacharparenleft}{\isasymE}\ G\ x{\isacharparenright}{\isacharparenright}\ w{\isachardoublequoteclose}\ \isacommand{using}\isamarkupfalse%
\ GodIsEssential\ \isacommand{by}\isamarkupfalse%
\ {\isacharparenleft}rule\ allE{\isacharparenright}\isanewline
\ \ \ \ \ \ \isacommand{hence}\isamarkupfalse%
\ {\isachardoublequoteopen}{\isasymforall}x{\isachardot}\ G\ x\ w\ {\isasymlongrightarrow}\ {\isacharparenleft}Q\ \isactrlbold {\isasymrightarrow}\ \isactrlbold {\isasymbox}{\isacharparenleft}\isactrlbold {\isasymforall}\isactrlsup Az{\isachardot}\ G\ z\ \isactrlbold {\isasymrightarrow}\ Q{\isacharparenright}{\isacharparenright}\ w{\isachardoublequoteclose}\ \isacommand{by}\isamarkupfalse%
\ force\isanewline
\ \ \ \ \ \ \isacommand{hence}\isamarkupfalse%
\ {\isadigit{1}}{\isacharcolon}\ {\isachardoublequoteopen}{\isacharparenleft}{\isasymexists}x{\isachardot}\ G\ x\ w{\isacharparenright}\ {\isasymlongrightarrow}\ {\isacharparenleft}{\isacharparenleft}Q\ \isactrlbold {\isasymrightarrow}\ \isactrlbold {\isasymbox}{\isacharparenleft}\isactrlbold {\isasymforall}\isactrlsup Az{\isachardot}\ G\ z\ \isactrlbold {\isasymrightarrow}\ Q{\isacharparenright}{\isacharparenright}\ w{\isacharparenright}{\isachardoublequoteclose}\ \isacommand{by}\isamarkupfalse%
\ {\isacharparenleft}rule\ useful{\isacharparenright}\isanewline
\ \ \ \ \ \ \isacommand{have}\isamarkupfalse%
\ {\isachardoublequoteopen}{\isasymexists}x{\isachardot}\ G\ x\ w{\isachardoublequoteclose}\ \isacommand{using}\isamarkupfalse%
\ GodExistenceIsValid\ \isacommand{by}\isamarkupfalse%
\ auto\isanewline
\ \ \ \ \ \ \isacommand{from}\isamarkupfalse%
\ {\isadigit{1}}\ this\ \isacommand{have}\isamarkupfalse%
\ {\isachardoublequoteopen}{\isacharparenleft}Q\ \isactrlbold {\isasymrightarrow}\ \isactrlbold {\isasymbox}{\isacharparenleft}\isactrlbold {\isasymforall}\isactrlsup Az{\isachardot}\ G\ z\ \isactrlbold {\isasymrightarrow}\ Q{\isacharparenright}{\isacharparenright}\ w{\isachardoublequoteclose}\ \isacommand{by}\isamarkupfalse%
\ {\isacharparenleft}rule\ mp{\isacharparenright}\isanewline
\ \ \ \ \ \ \isacommand{hence}\isamarkupfalse%
\ {\isachardoublequoteopen}{\isacharparenleft}Q\ \isactrlbold {\isasymrightarrow}\ \isactrlbold {\isasymbox}{\isacharparenleft}{\isacharparenleft}\isactrlbold {\isasymexists}\isactrlsup Az{\isachardot}\ G\ z{\isacharparenright}\ \isactrlbold {\isasymrightarrow}\ Q{\isacharparenright}{\isacharparenright}\ w{\isachardoublequoteclose}\ \isacommand{using}\isamarkupfalse%
\ useful\ \isacommand{by}\isamarkupfalse%
\ blast\isanewline
\ \ \ \ \ \ \isacommand{hence}\isamarkupfalse%
\ {\isachardoublequoteopen}{\isacharparenleft}Q\ \isactrlbold {\isasymrightarrow}\ {\isacharparenleft}\isactrlbold {\isasymbox}{\isacharparenleft}\isactrlbold {\isasymexists}\isactrlsup Az{\isachardot}\ G\ z{\isacharparenright}\ \isactrlbold {\isasymrightarrow}\ \isactrlbold {\isasymbox}Q{\isacharparenright}{\isacharparenright}\ w{\isachardoublequoteclose}\ \isacommand{by}\isamarkupfalse%
\ simp\isanewline
\ \ \ \ \ \ \isacommand{hence}\isamarkupfalse%
\ {\isachardoublequoteopen}{\isacharparenleft}Q\ \isactrlbold {\isasymrightarrow}\ \isactrlbold {\isasymbox}Q{\isacharparenright}\ w{\isachardoublequoteclose}\ \isacommand{using}\isamarkupfalse%
\ GodExistsNecessarily\ \isacommand{by}\isamarkupfalse%
\ simp\isanewline
\ \ \ \ \ \isacommand{{\isacharbraceright}}\isamarkupfalse%
\ \isacommand{hence}\isamarkupfalse%
\ {\isachardoublequoteopen}{\isacharparenleft}\isactrlbold {\isasymforall}{\isasymPhi}{\isachardot}\ {\isasymPhi}\ \isactrlbold {\isasymrightarrow}\ \isactrlbold {\isasymbox}\ {\isasymPhi}{\isacharparenright}\ w{\isachardoublequoteclose}\ \isacommand{by}\isamarkupfalse%
\ {\isacharparenleft}rule\ allI{\isacharparenright}\isanewline
\ \ \ \ \isacommand{{\isacharbraceright}}\isamarkupfalse%
\ \isacommand{thus}\isamarkupfalse%
\ {\isacharquery}thesis\ \isacommand{by}\isamarkupfalse%
\ {\isacharparenleft}rule\ allI{\isacharparenright}\isanewline
\ \ \isacommand{qed}\isamarkupfalse%
%
%
%
%
%
%
%
%
%
%
\end{isabellebody}%
%%% Local Variables:
%%% mode: latex
%%% TeX-master: "root"
%%% End:

%
\begin{isabellebody}%
\setisabellecontext{FittingProof}%
%
%
%
%
%
%
%
\isamarkupsection{Fitting's Variant%
}
\isamarkuptrue%
%
\begin{isamarkuptext}%
In this section we consider Fitting's solution to the objections raised in his discussion of G\"odel's Argument pp. 164-9, 
especially the problem of \emph{modal collapse}, which has been metaphysically interpreted as implying a rejection of free will.
Since we are generally commited to the existence of free will (in a pre-theoretical sense), such a result is
philosophically unappealing and rather seen as a problem in the argument's formalization.%
\end{isamarkuptext}\isamarkuptrue%
%
\begin{isamarkuptext}%
Remark: The `\isa{{\isasymlparr}{\isacharunderscore}{\isasymrparr}}' parentheses are used to convert an extensional object into its `rigid'
intensional counterpart (e.g. \isa{{\isasymlparr}{\isasymphi}{\isasymrparr}\ {\isasymequiv}\ {\isasymlambda}w{\isachardot}\ {\isasymphi}}).%
\end{isamarkuptext}\isamarkuptrue%
\isacommand{abbreviation}\isamarkupfalse%
\ Entailment{\isacharcolon}{\isacharcolon}{\isachardoublequoteopen}{\isasymup}{\isasymlangle}{\isasymlangle}{\isasymzero}{\isasymrangle}{\isacharcomma}{\isasymlangle}{\isasymzero}{\isasymrangle}{\isasymrangle}{\isachardoublequoteclose}\ {\isacharparenleft}\isakeyword{infix}{\isachardoublequoteopen}{\isasymRrightarrow}{\isachardoublequoteclose}{\isadigit{6}}{\isadigit{0}}{\isacharparenright}\isanewline
\ \ \isakeyword{where}\ {\isachardoublequoteopen}X\ {\isasymRrightarrow}\ Y\ {\isasymequiv}\ \isactrlbold {\isasymbox}{\isacharparenleft}\isactrlbold {\isasymforall}\isactrlsup Ez{\isachardot}\ {\isasymlparr}X\ z{\isasymrparr}\ \isactrlbold {\isasymrightarrow}\ {\isasymlparr}Y\ z{\isasymrparr}{\isacharparenright}{\isachardoublequoteclose}\ \ \isanewline
\isacommand{consts}\isamarkupfalse%
\ Positiveness{\isacharcolon}{\isacharcolon}{\isachardoublequoteopen}{\isasymup}{\isasymlangle}{\isasymlangle}{\isasymzero}{\isasymrangle}{\isasymrangle}{\isachardoublequoteclose}\ {\isacharparenleft}{\isachardoublequoteopen}{\isasymP}{\isachardoublequoteclose}{\isacharparenright}\isanewline
\isacommand{abbreviation}\isamarkupfalse%
\ Existence{\isacharcolon}{\isacharcolon}{\isachardoublequoteopen}{\isasymup}{\isasymlangle}{\isasymzero}{\isasymrangle}{\isachardoublequoteclose}\ {\isacharparenleft}{\isachardoublequoteopen}E{\isacharbang}{\isachardoublequoteclose}{\isacharparenright}\ \isakeyword{where}\ {\isachardoublequoteopen}E{\isacharbang}\ x\ {\isasymequiv}\ {\isasymlambda}w{\isachardot}\ {\isacharparenleft}\isactrlbold {\isasymexists}\isactrlsup Ey{\isachardot}\ y\isactrlbold {\isasymapprox}x{\isacharparenright}\ w{\isachardoublequoteclose}\isanewline
\isacommand{abbreviation}\isamarkupfalse%
\ God{\isacharcolon}{\isacharcolon}{\isachardoublequoteopen}{\isasymup}{\isasymlangle}{\isasymzero}{\isasymrangle}{\isachardoublequoteclose}\ {\isacharparenleft}{\isachardoublequoteopen}G{\isachardoublequoteclose}{\isacharparenright}\ \isakeyword{where}\ {\isachardoublequoteopen}G\ {\isasymequiv}\ {\isacharparenleft}{\isasymlambda}x{\isachardot}\ \isactrlbold {\isasymforall}Y{\isachardot}\ {\isasymP}\ Y\ \isactrlbold {\isasymrightarrow}\ {\isasymlparr}Y\ x{\isasymrparr}{\isacharparenright}{\isachardoublequoteclose}%
\isamarkupsubsection{Part I - God's Existence is Possible%
}
\isamarkuptrue%
\isacommand{axiomatization}\isamarkupfalse%
\ \isakeyword{where}\isanewline
\ \ A{\isadigit{1}}a{\isacharcolon}{\isachardoublequoteopen}{\isasymlfloor}\isactrlbold {\isasymforall}X{\isachardot}\ {\isasymP}\ {\isacharparenleft}{\isasymrightharpoondown}X{\isacharparenright}\ \isactrlbold {\isasymrightarrow}\ \isactrlbold {\isasymnot}{\isacharparenleft}{\isasymP}\ X{\isacharparenright}\ {\isasymrfloor}{\isachardoublequoteclose}\ \isakeyword{and}\ \ \ \ \ \ \ \ %
\isamarkupcmt{axiom 11.3A%
}
\isanewline
\ \ A{\isadigit{1}}b{\isacharcolon}{\isachardoublequoteopen}{\isasymlfloor}\isactrlbold {\isasymforall}X{\isachardot}\ \isactrlbold {\isasymnot}{\isacharparenleft}{\isasymP}\ X{\isacharparenright}\ \isactrlbold {\isasymrightarrow}\ {\isasymP}\ {\isacharparenleft}{\isasymrightharpoondown}X{\isacharparenright}{\isasymrfloor}{\isachardoublequoteclose}\ \isakeyword{and}\ \ \ \ \ \ \ \ \ %
\isamarkupcmt{axiom 11.3B%
}
\isanewline
\ \ A{\isadigit{2}}{\isacharcolon}\ {\isachardoublequoteopen}{\isasymlfloor}\isactrlbold {\isasymforall}X\ Y{\isachardot}\ {\isacharparenleft}{\isasymP}\ X\ \isactrlbold {\isasymand}\ {\isacharparenleft}X\ {\isasymRrightarrow}\ Y{\isacharparenright}{\isacharparenright}\ \isactrlbold {\isasymrightarrow}\ {\isasymP}\ Y{\isasymrfloor}{\isachardoublequoteclose}\ \isakeyword{and}\ \ %
\isamarkupcmt{axiom 11.5%
}
\isanewline
\ \ T{\isadigit{2}}{\isacharcolon}\ {\isachardoublequoteopen}{\isasymlfloor}{\isasymP}\ {\isasymdown}G{\isasymrfloor}{\isachardoublequoteclose}\ \ \ \ \ \ \ \ \ \ \ \ \ \ \ \ \ \ \ \ \ \ \ \ \ \ \ \ \ \ \ %
\isamarkupcmt{proposition 11.16 (modified)%
}
\isanewline
\isacommand{lemma}\isamarkupfalse%
\ True\ \isacommand{nitpick}\isamarkupfalse%
{\isacharbrackleft}satisfy{\isacharbrackright}%
\ %
%
\isacommand{oops}\isamarkupfalse%
\ %
\isamarkupcmt{model found: axioms are consistent%
}
%
%
%
%
\begin{isamarkuptext}%
\emph{T1} Positive properties are possibly instantiated%
\end{isamarkuptext}\isamarkuptrue%
\isacommand{theorem}\isamarkupfalse%
\ T{\isadigit{1}}{\isacharcolon}\ {\isachardoublequoteopen}{\isasymlfloor}\isactrlbold {\isasymforall}X{\isacharcolon}{\isacharcolon}{\isasymlangle}{\isasymzero}{\isasymrangle}{\isachardot}\ {\isasymP}\ X\ \isactrlbold {\isasymrightarrow}\ \isactrlbold {\isasymdiamond}{\isacharparenleft}\isactrlbold {\isasymexists}\isactrlsup Ez{\isachardot}\ {\isasymlparr}X\ z{\isasymrparr}{\isacharparenright}{\isasymrfloor}{\isachardoublequoteclose}%
\ %
%
\isacommand{using}\isamarkupfalse%
\ A{\isadigit{1}}a\ A{\isadigit{2}}\ \isacommand{by}\isamarkupfalse%
\ blast%
%
%
%
\begin{isamarkuptext}%
\emph{T3} (God exists possibly) can be formalized in two different ways, using a \emph{de re} or a \emph{de dicto} reading.%
\end{isamarkuptext}\isamarkuptrue%
\isacommand{theorem}\isamarkupfalse%
\ T{\isadigit{3}}{\isacharunderscore}deRe{\isacharcolon}\ {\isachardoublequoteopen}{\isasymlfloor}{\isacharparenleft}{\isasymlambda}X{\isachardot}\ \isactrlbold {\isasymdiamond}\isactrlbold {\isasymexists}\isactrlsup E\ X{\isacharparenright}\ \isactrlbold {\isasymdown}G{\isasymrfloor}{\isachardoublequoteclose}%
\ %
%
\isacommand{using}\isamarkupfalse%
\ T{\isadigit{1}}\ T{\isadigit{2}}\ \isacommand{by}\isamarkupfalse%
\ simp%
%
%
\ \isanewline
\isacommand{theorem}\isamarkupfalse%
\ T{\isadigit{3}}{\isacharunderscore}deDicto{\isacharcolon}\ {\isachardoublequoteopen}{\isasymlfloor}\isactrlbold {\isasymdiamond}\isactrlbold {\isasymexists}\isactrlsup E\ \isactrlbold {\isasymdown}G{\isasymrfloor}{\isachardoublequoteclose}\ \isacommand{nitpick}\isamarkupfalse%
%
\ %
%
\isacommand{oops}\isamarkupfalse%
\ %
\isamarkupcmt{countersatisfiable: not used%
}
%
%
%
%
\isamarkupsubsection{Part II - God's Existence is Necessary if Possible%
}
\isamarkuptrue%
\isacommand{axiomatization}\isamarkupfalse%
\ \isakeyword{where}\isanewline
\ \ \ \ \ \ A{\isadigit{4}}a{\isacharcolon}\ {\isachardoublequoteopen}{\isasymlfloor}\isactrlbold {\isasymforall}X{\isachardot}\ {\isasymP}\ X\ \isactrlbold {\isasymrightarrow}\ \isactrlbold {\isasymbox}{\isacharparenleft}{\isasymP}\ X{\isacharparenright}{\isasymrfloor}{\isachardoublequoteclose}\ \ \ \ \ \ %
\isamarkupcmt{axiom 11.11%
}
\isanewline
\isacommand{lemma}\isamarkupfalse%
\ A{\isadigit{4}}b{\isacharcolon}\ {\isachardoublequoteopen}{\isasymlfloor}\isactrlbold {\isasymforall}X{\isachardot}\ \isactrlbold {\isasymnot}{\isacharparenleft}{\isasymP}\ X{\isacharparenright}\ \isactrlbold {\isasymrightarrow}\ \isactrlbold {\isasymbox}\isactrlbold {\isasymnot}{\isacharparenleft}{\isasymP}\ X{\isacharparenright}{\isasymrfloor}{\isachardoublequoteclose}%
\ %
%
\isacommand{using}\isamarkupfalse%
\ A{\isadigit{1}}a\ A{\isadigit{1}}b\ A{\isadigit{4}}a\ \isacommand{by}\isamarkupfalse%
\ blast%
%
%
\isanewline
\ \ \ \ \isanewline
\isacommand{lemma}\isamarkupfalse%
\ True\ \isacommand{nitpick}\isamarkupfalse%
{\isacharbrackleft}satisfy{\isacharbrackright}%
\ %
%
\isacommand{oops}\isamarkupfalse%
\ %
\isamarkupcmt{model found: so far all axioms consistent%
}
%
%
%
\isanewline
\isacommand{lemma}\isamarkupfalse%
\ {\isachardoublequoteopen}{\isasymlfloor}rigidPred\ {\isasymP}{\isasymrfloor}{\isachardoublequoteclose}%
\ %
%
\isacommand{using}\isamarkupfalse%
\ A{\isadigit{4}}a\ A{\isadigit{4}}b\ \isacommand{by}\isamarkupfalse%
\ blast\ %
\isamarkupcmt{\isa{{\isasymP}} designates rigidly%
}
%
%
%
\isanewline
\ \ \ \ \isanewline
\isacommand{abbreviation}\isamarkupfalse%
\ essenceOf{\isacharcolon}{\isacharcolon}{\isachardoublequoteopen}{\isasymup}{\isasymlangle}{\isasymlangle}{\isasymzero}{\isasymrangle}{\isacharcomma}{\isasymzero}{\isasymrangle}{\isachardoublequoteclose}\ {\isacharparenleft}{\isachardoublequoteopen}{\isasymE}{\isachardoublequoteclose}{\isacharparenright}\ \isakeyword{where}\isanewline
\ \ {\isachardoublequoteopen}{\isasymE}\ Y\ x\ {\isasymequiv}\ {\isasymlparr}Y\ x{\isasymrparr}\ \isactrlbold {\isasymand}\ {\isacharparenleft}\isactrlbold {\isasymforall}Z{\isacharcolon}{\isacharcolon}{\isasymlangle}{\isasymzero}{\isasymrangle}{\isachardot}\ {\isasymlparr}Z\ x{\isasymrparr}\ \isactrlbold {\isasymrightarrow}\ Y\ {\isasymRrightarrow}\ Z{\isacharparenright}{\isachardoublequoteclose}\isanewline
\isacommand{theorem}\isamarkupfalse%
\ GodIsEssential{\isacharcolon}\ {\isachardoublequoteopen}{\isasymlfloor}\isactrlbold {\isasymforall}x{\isachardot}\ G\ x\ \isactrlbold {\isasymrightarrow}\ {\isacharparenleft}{\isacharparenleft}{\isasymE}\ {\isasymdown}\isactrlsub {\isadigit{1}}G{\isacharparenright}\ x{\isacharparenright}{\isasymrfloor}{\isachardoublequoteclose}%
\ %
%
\isacommand{using}\isamarkupfalse%
\ A{\isadigit{1}}b\ \isacommand{by}\isamarkupfalse%
\ metis%
%
%
\isanewline
\ \ \ \ \isanewline
\isacommand{abbreviation}\isamarkupfalse%
\ necessaryExistencePredicate\ {\isacharcolon}{\isacharcolon}\ {\isachardoublequoteopen}{\isasymup}{\isasymlangle}{\isasymzero}{\isasymrangle}{\isachardoublequoteclose}\ {\isacharparenleft}{\isachardoublequoteopen}NE{\isachardoublequoteclose}{\isacharparenright}\ \isakeyword{where}\isanewline
\ \ {\isachardoublequoteopen}NE\ x\ \ {\isasymequiv}\ {\isasymlambda}w{\isachardot}\ {\isacharparenleft}\isactrlbold {\isasymforall}Y{\isachardot}\ \ {\isasymE}\ Y\ x\ \isactrlbold {\isasymrightarrow}\ \isactrlbold {\isasymbox}{\isacharparenleft}\isactrlbold {\isasymexists}\isactrlsup Ez{\isachardot}\ {\isasymlparr}Y\ z{\isasymrparr}{\isacharparenright}{\isacharparenright}\ w{\isachardoublequoteclose}\isanewline
\ \ \isanewline
\isacommand{axiomatization}\isamarkupfalse%
\ \isakeyword{where}\ A{\isadigit{5}}{\isacharcolon}\ {\isachardoublequoteopen}{\isasymlfloor}{\isasymP}\ {\isasymdown}NE{\isasymrfloor}{\isachardoublequoteclose}\ \ \ \ \isanewline
\isacommand{lemma}\isamarkupfalse%
\ True\ \isacommand{nitpick}\isamarkupfalse%
{\isacharbrackleft}satisfy{\isacharbrackright}%
\ %
%
\isacommand{oops}\isamarkupfalse%
\ %
\isamarkupcmt{model found: so far all axioms consistent%
}
%
%
%
%
\begin{isamarkuptext}%
Theorem 11.26 (Informal Proposition 7) - (possibilist) existence of God implies necessary (actualist) existence.
This theorem can be formalized in two ways. Both of them are proven valid:%
\end{isamarkuptext}\isamarkuptrue%
\isacommand{theorem}\isamarkupfalse%
\ GodExImpNecEx{\isacharunderscore}v{\isadigit{1}}{\isacharcolon}\ {\isachardoublequoteopen}{\isasymlfloor}\isactrlbold {\isasymexists}\ \isactrlbold {\isasymdown}G\ \isactrlbold {\isasymrightarrow}\ \ \isactrlbold {\isasymbox}\isactrlbold {\isasymexists}\isactrlsup E\ \isactrlbold {\isasymdown}G{\isasymrfloor}{\isachardoublequoteclose}%
\ %
%
\isacommand{proof}\isamarkupfalse%
\ {\isacharminus}\ %
\isamarkupcmt{not shown here%
}
%
%
%
\ \ \isanewline
\isacommand{theorem}\isamarkupfalse%
\ GodExImpNecEx{\isacharunderscore}v{\isadigit{2}}{\isacharcolon}\ {\isachardoublequoteopen}{\isasymlfloor}\isactrlbold {\isasymexists}\ \isactrlbold {\isasymdown}G\ \isactrlbold {\isasymrightarrow}\ {\isacharparenleft}{\isacharparenleft}{\isasymlambda}X{\isachardot}\ \isactrlbold {\isasymbox}\isactrlbold {\isasymexists}\isactrlsup E\ X{\isacharparenright}\ \isactrlbold {\isasymdown}G{\isacharparenright}{\isasymrfloor}{\isachardoublequoteclose}\isanewline
%
\ \ %
%
\isacommand{using}\isamarkupfalse%
\ A{\isadigit{4}}a\ GodExImpNecEx{\isacharunderscore}v{\isadigit{1}}\ \isacommand{by}\isamarkupfalse%
\ metis\ %
\isamarkupcmt{can be proven by automated tools%
}
%
%
%
%
\begin{isamarkuptext}%
In contrast to G\"odel's argument (as presented by Fitting), the following theorems can be proven in logic \emph{K}
 (the \emph{S5} axioms are no longer needed):%
\end{isamarkuptext}\isamarkuptrue%
\isacommand{theorem}\isamarkupfalse%
\ possExImpNecEx{\isacharunderscore}v{\isadigit{1}}{\isacharcolon}\ {\isachardoublequoteopen}{\isasymlfloor}\isactrlbold {\isasymdiamond}\isactrlbold {\isasymexists}\ \isactrlbold {\isasymdown}G\ \isactrlbold {\isasymrightarrow}\ \isactrlbold {\isasymbox}\isactrlbold {\isasymexists}\isactrlsup E\ \isactrlbold {\isasymdown}G{\isasymrfloor}{\isachardoublequoteclose}\isanewline
%
\ \ %
%
\isacommand{using}\isamarkupfalse%
\ GodExImpNecEx{\isacharunderscore}v{\isadigit{1}}\ T{\isadigit{3}}{\isacharunderscore}deRe\ \isacommand{by}\isamarkupfalse%
\ metis%
%
\isanewline
%
\isacommand{theorem}\isamarkupfalse%
\ possExImpNecEx{\isacharunderscore}v{\isadigit{2}}{\isacharcolon}\ {\isachardoublequoteopen}{\isasymlfloor}{\isacharparenleft}{\isasymlambda}X{\isachardot}\isactrlbold {\isasymdiamond}\isactrlbold {\isasymexists}\isactrlsup E\ X{\isacharparenright}\ \isactrlbold {\isasymdown}G\ \isactrlbold {\isasymrightarrow}\ {\isacharparenleft}{\isasymlambda}X{\isachardot}\ \isactrlbold {\isasymbox}\isactrlbold {\isasymexists}\isactrlsup E\ X{\isacharparenright}\ \isactrlbold {\isasymdown}G{\isasymrfloor}{\isachardoublequoteclose}\isanewline
%
\ \ %
%
\isacommand{using}\isamarkupfalse%
\ GodExImpNecEx{\isacharunderscore}v{\isadigit{2}}\ \isacommand{by}\isamarkupfalse%
\ blast%
%
\isanewline
%
\isanewline
\isacommand{lemma}\isamarkupfalse%
\ T{\isadigit{4}}{\isacharunderscore}v{\isadigit{1}}{\isacharcolon}{\isachardoublequoteopen}{\isasymlfloor}\isactrlbold {\isasymdiamond}\isactrlbold {\isasymexists}\ \isactrlbold {\isasymdown}G{\isasymrfloor}\ {\isasymlongrightarrow}\ {\isasymlfloor}\isactrlbold {\isasymbox}\isactrlbold {\isasymexists}\isactrlsup E\ \isactrlbold {\isasymdown}G{\isasymrfloor}{\isachardoublequoteclose}%
\ %
%
\isacommand{using}\isamarkupfalse%
\ possExImpNecEx{\isacharunderscore}v{\isadigit{1}}\ \isacommand{by}\isamarkupfalse%
\ simp%
%
%
\isanewline
\isacommand{lemma}\isamarkupfalse%
\ T{\isadigit{4}}{\isacharunderscore}v{\isadigit{2}}{\isacharcolon}{\isachardoublequoteopen}{\isasymlfloor}{\isacharparenleft}{\isasymlambda}X{\isachardot}\ \isactrlbold {\isasymdiamond}\isactrlbold {\isasymexists}\isactrlsup E\ X{\isacharparenright}\ \isactrlbold {\isasymdown}G{\isasymrfloor}{\isasymlongrightarrow}{\isasymlfloor}{\isacharparenleft}{\isasymlambda}X{\isachardot}\ \isactrlbold {\isasymbox}\isactrlbold {\isasymexists}\isactrlsup E\ X{\isacharparenright}\ \isactrlbold {\isasymdown}G{\isasymrfloor}{\isachardoublequoteclose}%
\ %
%
\isacommand{using}\isamarkupfalse%
\ possExImpNecEx{\isacharunderscore}v{\isadigit{2}}\ \isacommand{by}\isamarkupfalse%
\ simp%
%
%
%
\isamarkupsubsection{Conclusion (\emph{De Re} and \emph{De Dicto} Reading)%
}
\isamarkuptrue%
%
\begin{isamarkuptext}%
Version I - Necessary Existence of God (\emph{de dicto}):%
\end{isamarkuptext}\isamarkuptrue%
\isacommand{lemma}\isamarkupfalse%
\ GodNecExists{\isacharunderscore}v{\isadigit{1}}{\isacharcolon}\ {\isachardoublequoteopen}{\isasymlfloor}\isactrlbold {\isasymbox}\isactrlbold {\isasymexists}\isactrlsup E\ \isactrlbold {\isasymdown}G{\isasymrfloor}{\isachardoublequoteclose}\isanewline
%
\ \ %
%
\isacommand{using}\isamarkupfalse%
\ GodExImpNecEx{\isacharunderscore}v{\isadigit{1}}\ T{\isadigit{3}}{\isacharunderscore}deRe\ \isacommand{by}\isamarkupfalse%
\ fastforce\ %
\isamarkupcmt{corollary 11.28%
}
%
%
\isanewline
%
\isacommand{lemma}\isamarkupfalse%
\ {\isachardoublequoteopen}{\isasymlfloor}\isactrlbold {\isasymbox}{\isacharparenleft}{\isasymlambda}X{\isachardot}\ \isactrlbold {\isasymexists}\isactrlsup E\ X{\isacharparenright}\ \isactrlbold {\isasymdown}G{\isasymrfloor}{\isachardoublequoteclose}\isanewline
%
\ \ %
%
\isacommand{using}\isamarkupfalse%
\ GodNecExists{\isacharunderscore}v{\isadigit{1}}\ \isacommand{by}\isamarkupfalse%
\ simp\ %
\isamarkupcmt{\emph{de dicto} shown here explicitly%
}
%
%
%
%
\begin{isamarkuptext}%
Version II - Necessary Existence of God (\emph{de re})%
\end{isamarkuptext}\isamarkuptrue%
\isacommand{lemma}\isamarkupfalse%
\ GodNecExists{\isacharunderscore}v{\isadigit{2}}{\isacharcolon}\ {\isachardoublequoteopen}{\isasymlfloor}{\isacharparenleft}{\isasymlambda}X{\isachardot}\ \isactrlbold {\isasymbox}\isactrlbold {\isasymexists}\isactrlsup E\ X{\isacharparenright}\ \isactrlbold {\isasymdown}G{\isasymrfloor}{\isachardoublequoteclose}\isanewline
%
\ \ %
%
\isacommand{using}\isamarkupfalse%
\ T{\isadigit{3}}{\isacharunderscore}deRe\ T{\isadigit{4}}{\isacharunderscore}v{\isadigit{2}}\ \isacommand{by}\isamarkupfalse%
\ blast%
%
%
%
\isamarkupsubsection{Modal Collapse%
}
\isamarkuptrue%
%
\begin{isamarkuptext}%
Modal collapse is countersatisfiable even in \emph{S5}. Note that countermodels with a cardinality of one 
for the domain of individuals are found by \emph{Nitpick} (the countermodel shown in the book has cardinality of two).%
\end{isamarkuptext}\isamarkuptrue%
\isacommand{axiomatization}\isamarkupfalse%
\ \isakeyword{where}\ S{\isadigit{5}}{\isacharcolon}\ {\isachardoublequoteopen}equivalence\ aRel{\isachardoublequoteclose}\ %
\isamarkupcmt{\emph{S5} axioms assumed%
}
\isanewline
\isacommand{lemma}\isamarkupfalse%
\ {\isachardoublequoteopen}{\isasymlfloor}\isactrlbold {\isasymforall}{\isasymPhi}{\isachardot}{\isacharparenleft}{\isasymPhi}\ \isactrlbold {\isasymrightarrow}\ {\isacharparenleft}\isactrlbold {\isasymbox}\ {\isasymPhi}{\isacharparenright}{\isacharparenright}{\isasymrfloor}{\isachardoublequoteclose}\ \isacommand{nitpick}\isamarkupfalse%
{\isacharbrackleft}card\ {\isacharprime}t{\isacharequal}{\isadigit{1}}{\isacharcomma}\ card\ i{\isacharequal}{\isadigit{2}}{\isacharbrackright}%
\ %
%
\isacommand{oops}\isamarkupfalse%
\ %
\isamarkupcmt{countermodel%
}
%
%
%
%
%
%
%
%
%
%
\end{isabellebody}%
%%% Local Variables:
%%% mode: latex
%%% TeX-master: "root"
%%% End:

%
\begin{isabellebody}%
\setisabellecontext{AndersonProof}%
%
%
%
%
%
%
%
\isamarkupsection{Anderson's Variant%
}
\isamarkuptrue%
%
\begin{isamarkuptext}%
\noindent{In this section, we verify Anderson's emendation of G\"odel's argument \cite{anderson90},
 as presented by Fitting (\cite{Fitting}, pp. 169-171). In the previous variants there were no `indifferent' properties,
 either a property or its negation had to be positive. Anderson makes room for `indifferent' properties by
 dropping axiom \emph{A1b} (\isa{{\isasymlfloor}\isactrlbold {\isasymforall}X{\isachardot}\ \isactrlbold {\isasymnot}{\isacharparenleft}{\isasymP}\ X{\isacharparenright}\ \isactrlbold {\isasymrightarrow}\ {\isasymP}\ {\isacharparenleft}{\isasymrightharpoondown}X{\isacharparenright}{\isasymrfloor}}). As a consequence, he changes the following definitions
 to ensure argument's validity.}%
\end{isamarkuptext}\isamarkuptrue%
\ \ \isacommand{abbreviation}\isamarkupfalse%
\ God{\isacharcolon}{\isacharcolon}{\isachardoublequoteopen}{\isasymup}{\isasymlangle}e{\isasymrangle}{\isachardoublequoteclose}\ {\isacharparenleft}{\isachardoublequoteopen}G{\isachardoublequoteclose}{\isacharparenright}\ \isakeyword{where}\ {\isachardoublequoteopen}G\ {\isasymequiv}\ {\isasymlambda}x{\isachardot}\ \isactrlbold {\isasymforall}Y{\isachardot}\ {\isacharparenleft}{\isasymP}\ Y{\isacharparenright}\ \isactrlbold {\isasymleftrightarrow}\ \isactrlbold {\isasymbox}{\isacharparenleft}Y\ x{\isacharparenright}{\isachardoublequoteclose}\isanewline
\ \ \isacommand{abbreviation}\isamarkupfalse%
\ Essence{\isacharcolon}{\isacharcolon}{\isachardoublequoteopen}{\isasymup}{\isasymlangle}{\isasymup}{\isasymlangle}e{\isasymrangle}{\isacharcomma}e{\isasymrangle}{\isachardoublequoteclose}\ {\isacharparenleft}{\isachardoublequoteopen}{\isasymE}{\isachardoublequoteclose}{\isacharparenright}\ \isakeyword{where}\ {\isachardoublequoteopen}{\isasymE}\ Y\ x\ {\isasymequiv}\ {\isacharparenleft}\isactrlbold {\isasymforall}Z{\isachardot}\ \isactrlbold {\isasymbox}{\isacharparenleft}Z\ x{\isacharparenright}\ \isactrlbold {\isasymleftrightarrow}\ Y\ {\isasymRrightarrow}\ Z{\isacharparenright}{\isachardoublequoteclose}%
\begin{isamarkuptext}%
\noindent{There is now the requirement that a Godlike being must have positive properties \emph{necessarily}.
  For the definition of essence, Scott's addition \cite{ScottNotes}, that the essence of an object 
  actually applies to the object, is dropped. A necessity operator has been introduced instead.\footnote{G\"odel's
  original axioms (without Scott's addition) are proven inconsistent in \cite{C55}.}}%
\end{isamarkuptext}\isamarkuptrue%
%
\begin{isamarkuptext}%
\noindent{The rest of the argument is essentially similar to G\"odel's (also in \emph{S5} logic).}%
\end{isamarkuptext}\isamarkuptrue%
\ \ \isacommand{theorem}\isamarkupfalse%
\ T{\isadigit{1}}{\isacharcolon}\ {\isachardoublequoteopen}{\isasymlfloor}\isactrlbold {\isasymforall}X{\isachardot}\ {\isasymP}\ X\ \isactrlbold {\isasymrightarrow}\ \isactrlbold {\isasymdiamond}\isactrlbold {\isasymexists}\isactrlsup A\ X{\isasymrfloor}{\isachardoublequoteclose}%
\ %
%
\isacommand{using}\isamarkupfalse%
\ A{\isadigit{1}}a\ A{\isadigit{2}}\ \isacommand{by}\isamarkupfalse%
\ blast%
%
%
\isanewline
\ \ \isacommand{theorem}\isamarkupfalse%
\ T{\isadigit{3}}{\isacharcolon}\ {\isachardoublequoteopen}{\isasymlfloor}\isactrlbold {\isasymdiamond}\isactrlbold {\isasymexists}\isactrlsup A\ G{\isasymrfloor}{\isachardoublequoteclose}%
\ %
%
\isacommand{using}\isamarkupfalse%
\ T{\isadigit{1}}\ T{\isadigit{2}}\ \isacommand{by}\isamarkupfalse%
\ simp%
%
%
%
%
%
%
%
%
%
%
%
%
%
%
%
%
%
%
%
%
%
\begin{isamarkuptext}%
\noindent{If g is Godlike, the property of being Godlike is its essence.\footnote{This theorem's proof
  could be completely automatized for G\"odel's and Fitting's variants.
  For Anderson's version however, we had to reproduce in Isabelle/HOL the original natural-language proof 
  given by Anderson (see \cite{anderson90}, Theorem 2*, p. 296)}}%
\end{isamarkuptext}\isamarkuptrue%
\ \ \isacommand{theorem}\isamarkupfalse%
\ GodIsEssential{\isacharcolon}\ {\isachardoublequoteopen}{\isasymlfloor}\isactrlbold {\isasymforall}x{\isachardot}\ G\ x\ \isactrlbold {\isasymrightarrow}\ {\isacharparenleft}{\isasymE}\ G\ x{\isacharparenright}{\isasymrfloor}{\isachardoublequoteclose}%
\ %
%
\isacommand{proof}\isamarkupfalse%
\ {\isacharminus}\ %
\isamarkupcmt{not shown%
}
%
%
%
%
%
%
%
%
%
%
%
%
%
%
%
%
%
%
%
%
%
%
\begin{isamarkuptext}%
\noindent{The necessary existence of God follows from its possible existence.}%
\end{isamarkuptext}\isamarkuptrue%
\ \ \isacommand{theorem}\isamarkupfalse%
\ T{\isadigit{5}}{\isacharcolon}\ {\isachardoublequoteopen}{\isasymlfloor}\isactrlbold {\isasymdiamond}\isactrlbold {\isasymexists}\ G{\isasymrfloor}\ {\isasymlongrightarrow}\ {\isasymlfloor}\isactrlbold {\isasymbox}\isactrlbold {\isasymexists}\isactrlsup A\ G{\isasymrfloor}{\isachardoublequoteclose}%
\ %
%
\isacommand{proof}\isamarkupfalse%
\ {\isacharminus}\ %
\isamarkupcmt{not shown%
}
%
%
%
%
\begin{isamarkuptext}%
\noindent{The conclusion could be proven (with one fewer axiom, though more complex definitions) and
  \emph{Nitpick} is able to find a countermodel for the \emph{modal collapse}.}%
\end{isamarkuptext}\isamarkuptrue%
\ \ \isacommand{lemma}\isamarkupfalse%
\ GodExistsNecessarily{\isacharcolon}\ {\isachardoublequoteopen}{\isasymlfloor}\isactrlbold {\isasymbox}\isactrlbold {\isasymexists}\isactrlsup A\ G{\isasymrfloor}{\isachardoublequoteclose}%
\ %
%
\isacommand{using}\isamarkupfalse%
\ T{\isadigit{3}}\ T{\isadigit{5}}\ \isacommand{by}\isamarkupfalse%
\ metis%
%
%
\isanewline
\ \ \isacommand{lemma}\isamarkupfalse%
\ ModalCollapse{\isacharcolon}\ {\isachardoublequoteopen}{\isasymlfloor}\isactrlbold {\isasymforall}{\isasymPhi}{\isachardot}\ {\isasymPhi}\ \isactrlbold {\isasymrightarrow}\ \isactrlbold {\isasymbox}{\isasymPhi}{\isasymrfloor}{\isachardoublequoteclose}\ \isacommand{nitpick}\isamarkupfalse%
%
\ %
%
\isacommand{oops}\isamarkupfalse%
\ %
\isamarkupcmt{countersatisfiable%
}
%
%
%
%
\isamarkupsection{Conclusion%
}
\isamarkuptrue%
%
\begin{isamarkuptext}%
\noindent{We presented a shallow semantic embedding in Isabelle/HOL for an intensional higher-order modal logic
(a successor of Montague/Gallin intensional logics) and employed this logic to formalize and verify
three different variants of the ontological argument:
the first one by G\"odel himself (resp. Scott), the second one by Fitting and the last one by Anderson.}%
\end{isamarkuptext}\isamarkuptrue%
%
\begin{isamarkuptext}%
\noindent{By employing our embedding of IHOML in Isabelle/HOL, we could not only verify Fitting's results,
but also guarantee consistency of axioms. Moreover, for many theorems we could prove stronger versions
and find better countermodels (i.e. with smaller cardinality) than the ones presented by Fitting.
Another interesting aspect was the possibility to explore the implications of alternative formalizations
of axioms and theorems which shed light on interesting philosophical issues concerning entailment, essentialism and free will.}%
\end{isamarkuptext}\isamarkuptrue%
%
\begin{isamarkuptext}%
\noindent{Latest developments in automated theorem proving, in combination with the embedding approach,
allow us to engage in much better experimentation during the formalization and assessment of arguments than ever before.
The potential reduction (of several orders of magnitude)
in the time needed for proving or disproving theorems (compared to pen-and-paper proofs), results in almost real-time
feedback about the suitability of our speculations. The practical benefits of computer-supported argumentation go beyond
mere quantitative aspects (easier, faster and more reliable proofs). The advantages are also qualitative,
since a significantly different approach to argumentation is fostered: We can now work iteratively (by trial-and-error)
on an argument by making gradual adjustments to its definitions, axioms and theorems. This allows us to continuously
expose and revise the assumptions we indirectly commit ourselves to every time we opt for some particular formalization.
}%
\end{isamarkuptext}\isamarkuptrue%
%
%
%
%
%
%
%
\end{isabellebody}%
%%% Local Variables:
%%% mode: latex
%%% TeX-master: "root"
%%% End:


%
% ---- Bibliography ----
\bibliographystyle{abbrv}
\bibliography{FittingBook}


%\clearpage
%\addtocmark[2]{Author Index} % additional numbered TOC entry
%\renewcommand{\indexname}{Author Index}
%\printindex
%\clearpage
%\addtocmark[2]{Subject Index} % additional numbered TOC entry
%\markboth{Subject Index}{Subject Index}
%\renewcommand{\indexname}{Subject Index}
%\input{subjidx.ind}
\end{document}
