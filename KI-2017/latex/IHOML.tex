%
\begin{isabellebody}%
\setisabellecontext{IHOML}%
%
%
%
%
%
%
%
\begin{isamarkuptext}%
\begin{abstract}
  A shallow semantical embedding of an intensional higher-order modal logic (IHOML) in Isabelle/HOL is presented.
  IHOML draws on Montague/Gallin intensional logics and has been introduced by M. Fitting in his textbook
  \emph{Types, Tableaus and G\"odel's God} in order to discuss his emendation of G\"odel's ontological argument
  (for the existence of God). We subsequently	utilize the embedded logic for the computer-formalization and evaluation
  of these arguments. In particular, Fitting's and Anderson's variants are verified and their claims confirmed.
  These variants aim to avoid the modal collapse, which has been criticized as
  an undesirable side-effect of Kurt G\"odel's (and Dana Scott's) versions of the ontological argument.
\\ \\
  \textbf{Keywords:} Automated Theorem Proving. Computational Metaphysics. Isabelle. Modal Logic.
	Intensional Logic. Ontological Argument. Semantic Analysis
\end{abstract}%
\end{isamarkuptext}\isamarkuptrue%
%
\isamarkupsection{Introduction%
}
\isamarkuptrue%
%
\begin{isamarkuptext}%
This work is divided in two parts. In the first one we present a shallow semantical embedding of
an \emph{intensional} higher-order modal logic (IHOML) in Isabelle/HOL,
which has been introduced by Fitting in his textbook \emph{Types, Tableaus and G\"odel's God} \cite{Fitting}, in order
to formalize his emendation of G\"odel's ontological argument. IHOML is a modification of the
intentional logic originally developed by Montague and later expanded by Gallin \cite{Gallin75} by building upon Church's
type theory and Kripke's possible-world semantics. Our approach has been inspired by previous work on the semantical embedding of
multimodal logics with quantification \cite{J23}, which we expand here to allow for actualist quantification,
intensional terms and their related operations.
From an AI perspective we contribute a highly flexible `implementation' of an automated reasoning infrastructure for IHOML.
Such an intensional logic has not been automated before and it is highly relevant e.g. for the deep semantical analysis of
natural language rational arguments. In this sense, our work contributes to the objective
of the new DFG Schwerpunktprogramm RATIO (SPP 1999).%
\end{isamarkuptext}\isamarkuptrue%
%
\begin{isamarkuptext}%
For the second part, we present an exemplary, non-trivial application of this reasoning infrastructure,
a study on Computational Metaphysics: the computer-formalization and critical assessment
of G\"odel's \cite{GoedelNotes} (resp. Dana Scott's \cite{ScottNotes}) modern variant of the ontological argument.
Several authors (e.g. \cite{anderson90:_some_emend_of_goedel_ontol_proof,AndersonGettings,bjordal99,Hajek2002,Fitting}) 
have proposed emendations of this argument with the aim of retaining its essential result 
(the necessary existence of God) while at the same time avoiding the \emph{modal collapse} \cite{Sobel,sobel2004logic},
which has been criticized as an undesirable side-effect of the axioms of G\"odel (resp. Scott). The modal collapse essentially  
states that there are no contingent truths and that everything is determined.
Related work has formalized several of these variants on the computer and verified or falsified them. For example,
G\"odel's axiom's system has been shown inconsistent \cite{C55,C60},
while Scott's version has been verified \cite{ECAI}. Further experiments, contributing amongst others
to the clarification of a related debate between H\'ajek and Anderson, are presented and discussed in
\cite{J23}. The enabling technique in all of these experiments has been
shallow semantical embeddings of (extensional) higher-order modal logics in classical higher-order
logic (see \cite{J23,R59} and the references therein).%
\end{isamarkuptext}\isamarkuptrue%
%
\begin{isamarkuptext}%
In our work, we additionally discuss two emendations of G\"odel's argument (by Fitting \cite{Fitting} and
Anderson \cite{anderson90:_some_emend_of_goedel_ontol_proof}).
In contrast to all variants mentioned above, Fitting's solution is based on the use of an intensional as opposed
to an extensional higher-order modal logic. For our work this imposed the additional challenge to provide
a shallow embedding of this more advanced logic (IHOML). The experiments presented below confirm that Fitting's argument
as presented in his textbook \cite{Fitting} is valid and that it avoids the modal collapse as intended.
The work presented here originates from the \emph{Computational Metaphysics} lecture course held at
FU Berlin in Summer 2016 \cite{C65}.%
\end{isamarkuptext}\isamarkuptrue%
%
\isamarkupsection{Embedding of Intensional Higher-Order Modal Logic%
}
\isamarkuptrue%
%
\isamarkupsubsection{Type Declarations%
}
\isamarkuptrue%
%
\begin{isamarkuptext}%
Since IHOML and Isabelle/HOL are both typed languages, we introduce a type-mapping between them.
We follow as closely as possible the syntax given by Fitting (see p. 86). According to this syntax,
if \isa{{\isasymtau}} is an extensional type, \isa{{\isasymup}{\isasymtau}} is the corresponding intensional type. For instance,
a set of (red) objects has the extensional type \isa{{\isasymlangle}{\isasymzero}{\isasymrangle}}, whereas the concept `red' has intensional type \isa{{\isasymup}{\isasymlangle}{\isasymzero}{\isasymrangle}}.%
\end{isamarkuptext}\isamarkuptrue%
\isacommand{typedecl}\isamarkupfalse%
\ i\ \ \ \ \ \ \ \ \ \ \ \ \ \ \ \ \ \ \ \ %
\isamarkupcmt{type for possible worlds%
}
\isanewline
\isacommand{type{\isacharunderscore}synonym}\isamarkupfalse%
\ io\ {\isacharequal}\ {\isachardoublequoteopen}{\isacharparenleft}i{\isasymRightarrow}bool{\isacharparenright}{\isachardoublequoteclose}\ %
\isamarkupcmt{formulas with world-dependent truth-value%
}
\isanewline
\isacommand{typedecl}\isamarkupfalse%
\ e\ \ {\isacharparenleft}{\isachardoublequoteopen}{\isasymzero}{\isachardoublequoteclose}{\isacharparenright}\ \ \ \ \ \ \ \ \ \ \ \ \ %
\isamarkupcmt{individual objects%
}
%
\begin{isamarkuptext}%
Aliases for common complex types (predicates and relations):%
\end{isamarkuptext}\isamarkuptrue%
\isacommand{type{\isacharunderscore}synonym}\isamarkupfalse%
\ ie{\isacharequal}{\isachardoublequoteopen}{\isacharparenleft}i{\isasymRightarrow}{\isasymzero}{\isacharparenright}{\isachardoublequoteclose}\ {\isacharparenleft}{\isachardoublequoteopen}{\isasymup}{\isasymzero}{\isachardoublequoteclose}{\isacharparenright}\ %
\isamarkupcmt{individual concepts map worlds to objects%
}
\isanewline
\isacommand{type{\isacharunderscore}synonym}\isamarkupfalse%
\ se{\isacharequal}{\isachardoublequoteopen}{\isacharparenleft}{\isasymzero}{\isasymRightarrow}bool{\isacharparenright}{\isachardoublequoteclose}\ {\isacharparenleft}{\isachardoublequoteopen}{\isasymlangle}{\isasymzero}{\isasymrangle}{\isachardoublequoteclose}{\isacharparenright}\ %
\isamarkupcmt{(extensional) sets%
}
\isanewline
\isacommand{type{\isacharunderscore}synonym}\isamarkupfalse%
\ ise{\isacharequal}{\isachardoublequoteopen}{\isacharparenleft}{\isasymzero}{\isasymRightarrow}io{\isacharparenright}{\isachardoublequoteclose}\ {\isacharparenleft}{\isachardoublequoteopen}{\isasymup}{\isasymlangle}{\isasymzero}{\isasymrangle}{\isachardoublequoteclose}{\isacharparenright}\ %
\isamarkupcmt{intensional (predicate) concepts%
}
\ \isanewline
\isacommand{type{\isacharunderscore}synonym}\isamarkupfalse%
\ sise{\isacharequal}{\isachardoublequoteopen}{\isacharparenleft}{\isasymup}{\isasymlangle}{\isasymzero}{\isasymrangle}{\isasymRightarrow}bool{\isacharparenright}{\isachardoublequoteclose}\ {\isacharparenleft}{\isachardoublequoteopen}{\isasymlangle}{\isasymup}{\isasymlangle}{\isasymzero}{\isasymrangle}{\isasymrangle}{\isachardoublequoteclose}{\isacharparenright}\ %
\isamarkupcmt{sets of concepts%
}
\ \isanewline
\isacommand{type{\isacharunderscore}synonym}\isamarkupfalse%
\ isise{\isacharequal}{\isachardoublequoteopen}{\isacharparenleft}{\isasymup}{\isasymlangle}{\isasymzero}{\isasymrangle}{\isasymRightarrow}io{\isacharparenright}{\isachardoublequoteclose}\ {\isacharparenleft}{\isachardoublequoteopen}{\isasymup}{\isasymlangle}{\isasymup}{\isasymlangle}{\isasymzero}{\isasymrangle}{\isasymrangle}{\isachardoublequoteclose}{\isacharparenright}\ %
\isamarkupcmt{2nd-order intensional concepts%
}
\ \ \isanewline
\isacommand{type{\isacharunderscore}synonym}\isamarkupfalse%
\ see{\isacharequal}{\isachardoublequoteopen}{\isacharparenleft}{\isasymzero}{\isasymRightarrow}{\isasymzero}{\isasymRightarrow}bool{\isacharparenright}{\isachardoublequoteclose}\ {\isacharparenleft}{\isachardoublequoteopen}{\isasymlangle}{\isasymzero}{\isacharcomma}{\isasymzero}{\isasymrangle}{\isachardoublequoteclose}{\isacharparenright}\ %
\isamarkupcmt{(extensional) relations%
}
\isanewline
\isacommand{type{\isacharunderscore}synonym}\isamarkupfalse%
\ isee{\isacharequal}{\isachardoublequoteopen}{\isacharparenleft}{\isasymzero}{\isasymRightarrow}{\isasymzero}{\isasymRightarrow}io{\isacharparenright}{\isachardoublequoteclose}\ {\isacharparenleft}{\isachardoublequoteopen}{\isasymup}{\isasymlangle}{\isasymzero}{\isacharcomma}{\isasymzero}{\isasymrangle}{\isachardoublequoteclose}{\isacharparenright}\ %
\isamarkupcmt{intensional relational concepts%
}
\isanewline
\isacommand{type{\isacharunderscore}synonym}\isamarkupfalse%
\ isisee{\isacharequal}{\isachardoublequoteopen}{\isacharparenleft}{\isasymup}{\isasymlangle}{\isasymzero}{\isasymrangle}{\isasymRightarrow}{\isasymzero}{\isasymRightarrow}io{\isacharparenright}{\isachardoublequoteclose}\ {\isacharparenleft}{\isachardoublequoteopen}{\isasymup}{\isasymlangle}{\isasymup}{\isasymlangle}{\isasymzero}{\isasymrangle}{\isacharcomma}{\isasymzero}{\isasymrangle}{\isachardoublequoteclose}{\isacharparenright}\ %
\isamarkupcmt{2nd-order intensional relation%
}
%
\isamarkupsubsection{Logical Constants as Truth-Sets%
}
\isamarkuptrue%
%
\begin{isamarkuptext}%
We embed each modal operator as the set of worlds satisfying the corresponding HOL formula.%
\end{isamarkuptext}\isamarkuptrue%
\isacommand{abbreviation}\isamarkupfalse%
\ mnot{\isacharcolon}{\isacharcolon}{\isachardoublequoteopen}io{\isasymRightarrow}io{\isachardoublequoteclose}\ {\isacharparenleft}{\isachardoublequoteopen}\isactrlbold {\isasymnot}{\isacharunderscore}{\isachardoublequoteclose}{\isacharbrackleft}{\isadigit{5}}{\isadigit{2}}{\isacharbrackright}{\isadigit{5}}{\isadigit{3}}{\isacharparenright}\ \isakeyword{where}\ {\isachardoublequoteopen}\isactrlbold {\isasymnot}{\isasymphi}\ {\isasymequiv}\ {\isasymlambda}w{\isachardot}\ {\isasymnot}{\isacharparenleft}{\isasymphi}\ w{\isacharparenright}{\isachardoublequoteclose}\isanewline
\isacommand{abbreviation}\isamarkupfalse%
\ mand{\isacharcolon}{\isacharcolon}{\isachardoublequoteopen}io{\isasymRightarrow}io{\isasymRightarrow}io{\isachardoublequoteclose}\ {\isacharparenleft}\isakeyword{infixr}{\isachardoublequoteopen}\isactrlbold {\isasymand}{\isachardoublequoteclose}{\isadigit{5}}{\isadigit{1}}{\isacharparenright}\ \isakeyword{where}\ {\isachardoublequoteopen}{\isasymphi}\isactrlbold {\isasymand}{\isasympsi}\ {\isasymequiv}\ {\isasymlambda}w{\isachardot}\ {\isacharparenleft}{\isasymphi}\ w{\isacharparenright}{\isasymand}{\isacharparenleft}{\isasympsi}\ w{\isacharparenright}{\isachardoublequoteclose}\isanewline
\isacommand{abbreviation}\isamarkupfalse%
\ mor{\isacharcolon}{\isacharcolon}{\isachardoublequoteopen}io{\isasymRightarrow}io{\isasymRightarrow}io{\isachardoublequoteclose}\ {\isacharparenleft}\isakeyword{infixr}{\isachardoublequoteopen}\isactrlbold {\isasymor}{\isachardoublequoteclose}{\isadigit{5}}{\isadigit{0}}{\isacharparenright}\ \isakeyword{where}\ {\isachardoublequoteopen}{\isasymphi}\isactrlbold {\isasymor}{\isasympsi}\ {\isasymequiv}\ {\isasymlambda}w{\isachardot}\ {\isacharparenleft}{\isasymphi}\ w{\isacharparenright}{\isasymor}{\isacharparenleft}{\isasympsi}\ w{\isacharparenright}{\isachardoublequoteclose}\isanewline
\isacommand{abbreviation}\isamarkupfalse%
\ mimp{\isacharcolon}{\isacharcolon}{\isachardoublequoteopen}io{\isasymRightarrow}io{\isasymRightarrow}io{\isachardoublequoteclose}\ {\isacharparenleft}\isakeyword{infix}{\isachardoublequoteopen}\isactrlbold {\isasymrightarrow}{\isachardoublequoteclose}{\isadigit{4}}{\isadigit{9}}{\isacharparenright}\ \isakeyword{where}\ {\isachardoublequoteopen}{\isasymphi}\isactrlbold {\isasymrightarrow}{\isasympsi}\ {\isasymequiv}\ {\isasymlambda}w{\isachardot}{\isacharparenleft}{\isasymphi}\ w{\isacharparenright}{\isasymlongrightarrow}{\isacharparenleft}{\isasympsi}\ w{\isacharparenright}{\isachardoublequoteclose}%
\begin{isamarkuptext}%
\emph{Possibilist} and \emph{actualist} quantifiers are embedded as follows.
\footnote{Possibilist and actualist quantification can be seen as the semantic counterparts of the concepts
of possibilism and actualism in the metaphysics of modality. They relate to natural-language expressions such as
`there is', `exists', `is actual', etc.}%
\end{isamarkuptext}\isamarkuptrue%
\ \ \isacommand{abbreviation}\isamarkupfalse%
\ mforall{\isacharcolon}{\isacharcolon}{\isachardoublequoteopen}{\isacharparenleft}{\isacharprime}t{\isasymRightarrow}io{\isacharparenright}{\isasymRightarrow}io{\isachardoublequoteclose}\ {\isacharparenleft}{\isachardoublequoteopen}\isactrlbold {\isasymforall}{\isachardoublequoteclose}{\isacharparenright}\ \isakeyword{where}\ {\isachardoublequoteopen}\isactrlbold {\isasymforall}{\isasymPhi}\ {\isasymequiv}\ {\isasymlambda}w{\isachardot}{\isasymforall}x{\isachardot}\ {\isacharparenleft}{\isasymPhi}\ x\ w{\isacharparenright}{\isachardoublequoteclose}\isanewline
\ \ \isacommand{abbreviation}\isamarkupfalse%
\ mexists{\isacharcolon}{\isacharcolon}{\isachardoublequoteopen}{\isacharparenleft}{\isacharprime}t{\isasymRightarrow}io{\isacharparenright}{\isasymRightarrow}io{\isachardoublequoteclose}\ {\isacharparenleft}{\isachardoublequoteopen}\isactrlbold {\isasymexists}{\isachardoublequoteclose}{\isacharparenright}\ \isakeyword{where}\ {\isachardoublequoteopen}\isactrlbold {\isasymexists}{\isasymPhi}\ {\isasymequiv}\ {\isasymlambda}w{\isachardot}{\isasymexists}x{\isachardot}\ {\isacharparenleft}{\isasymPhi}\ x\ w{\isacharparenright}{\isachardoublequoteclose}%
\begin{isamarkuptext}%
The \emph{existsAt} predicate is used to embed \emph{actualist} quantifiers by restricting the domain of quantification
at every possible world. This standard technique has been referred to as \emph{existence relativization} (\cite{fitting98}, p. 106),
highlighting the fact that this predicate can be seen as a kind of meta-logical `existence predicate' telling us
which individuals \emph{actually} exist at a given world. This meta-logical concept does not appear in our object language.%
\end{isamarkuptext}\isamarkuptrue%
\ \ \isacommand{consts}\isamarkupfalse%
\ ExistsAt{\isacharcolon}{\isacharcolon}{\isachardoublequoteopen}{\isasymup}{\isasymlangle}{\isasymzero}{\isasymrangle}{\isachardoublequoteclose}\ {\isacharparenleft}\isakeyword{infix}\ {\isachardoublequoteopen}existsAt{\isachardoublequoteclose}\ {\isadigit{7}}{\isadigit{0}}{\isacharparenright}\ \ \isanewline
\isanewline
\ \ \isacommand{abbreviation}\isamarkupfalse%
\ mforallAct{\isacharcolon}{\isacharcolon}{\isachardoublequoteopen}{\isasymup}{\isasymlangle}{\isasymup}{\isasymlangle}{\isasymzero}{\isasymrangle}{\isasymrangle}{\isachardoublequoteclose}\ {\isacharparenleft}{\isachardoublequoteopen}\isactrlbold {\isasymforall}\isactrlsup E{\isachardoublequoteclose}{\isacharparenright}\ %
\isamarkupcmt{actualist variants use superscript%
}
\isanewline
\ \ \ \ \isakeyword{where}\ {\isachardoublequoteopen}\isactrlbold {\isasymforall}\isactrlsup E{\isasymPhi}\ {\isasymequiv}\ {\isasymlambda}w{\isachardot}{\isasymforall}x{\isachardot}\ {\isacharparenleft}x\ existsAt\ w{\isacharparenright}{\isasymlongrightarrow}{\isacharparenleft}{\isasymPhi}\ x\ w{\isacharparenright}{\isachardoublequoteclose}\isanewline
\ \ \isacommand{abbreviation}\isamarkupfalse%
\ mexistsAct{\isacharcolon}{\isacharcolon}{\isachardoublequoteopen}{\isasymup}{\isasymlangle}{\isasymup}{\isasymlangle}{\isasymzero}{\isasymrangle}{\isasymrangle}{\isachardoublequoteclose}\ {\isacharparenleft}{\isachardoublequoteopen}\isactrlbold {\isasymexists}\isactrlsup E{\isachardoublequoteclose}{\isacharparenright}\isanewline
\ \ \ \ \isakeyword{where}\ {\isachardoublequoteopen}\isactrlbold {\isasymexists}\isactrlsup E{\isasymPhi}\ {\isasymequiv}\ {\isasymlambda}w{\isachardot}{\isasymexists}x{\isachardot}\ {\isacharparenleft}x\ existsAt\ w{\isacharparenright}\ {\isasymand}\ {\isacharparenleft}{\isasymPhi}\ x\ w{\isacharparenright}{\isachardoublequoteclose}%
\begin{isamarkuptext}%
Model's \emph{accessibility relation} and modal operators \isa{{\isasymbox}} and \isa{{\isasymdiamond}}.%
\end{isamarkuptext}\isamarkuptrue%
\ \ \isacommand{consts}\isamarkupfalse%
\ aRel{\isacharcolon}{\isacharcolon}{\isachardoublequoteopen}i{\isasymRightarrow}i{\isasymRightarrow}bool{\isachardoublequoteclose}\ {\isacharparenleft}\isakeyword{infixr}\ {\isachardoublequoteopen}r{\isachardoublequoteclose}\ {\isadigit{7}}{\isadigit{0}}{\isacharparenright}\isanewline
\ \ \isacommand{abbreviation}\isamarkupfalse%
\ mbox\ {\isacharcolon}{\isacharcolon}\ {\isachardoublequoteopen}io{\isasymRightarrow}io{\isachardoublequoteclose}\ {\isacharparenleft}{\isachardoublequoteopen}\isactrlbold {\isasymbox}{\isacharunderscore}{\isachardoublequoteclose}{\isacharbrackleft}{\isadigit{5}}{\isadigit{2}}{\isacharbrackright}{\isadigit{5}}{\isadigit{3}}{\isacharparenright}\ \isakeyword{where}\ {\isachardoublequoteopen}\isactrlbold {\isasymbox}{\isasymphi}\ {\isasymequiv}\ {\isasymlambda}w{\isachardot}{\isasymforall}v{\isachardot}\ {\isacharparenleft}w\ r\ v{\isacharparenright}{\isasymlongrightarrow}{\isacharparenleft}{\isasymphi}\ v{\isacharparenright}{\isachardoublequoteclose}\isanewline
\ \ \isacommand{abbreviation}\isamarkupfalse%
\ mdia\ {\isacharcolon}{\isacharcolon}\ {\isachardoublequoteopen}io{\isasymRightarrow}io{\isachardoublequoteclose}\ {\isacharparenleft}{\isachardoublequoteopen}\isactrlbold {\isasymdiamond}{\isacharunderscore}{\isachardoublequoteclose}{\isacharbrackleft}{\isadigit{5}}{\isadigit{2}}{\isacharbrackright}{\isadigit{5}}{\isadigit{3}}{\isacharparenright}\ \isakeyword{where}\ {\isachardoublequoteopen}\isactrlbold {\isasymdiamond}{\isasymphi}\ {\isasymequiv}\ {\isasymlambda}w{\isachardot}{\isasymexists}v{\isachardot}\ {\isacharparenleft}w\ r\ v{\isacharparenright}{\isasymand}{\isacharparenleft}{\isasymphi}\ v{\isacharparenright}{\isachardoublequoteclose}\isanewline
\ \ \isanewline
\isacommand{abbreviation}\isamarkupfalse%
\ meq{\isacharcolon}{\isacharcolon}\ {\isachardoublequoteopen}{\isacharprime}t{\isasymRightarrow}{\isacharprime}t{\isasymRightarrow}io{\isachardoublequoteclose}\ {\isacharparenleft}\isakeyword{infix}{\isachardoublequoteopen}\isactrlbold {\isasymapprox}{\isachardoublequoteclose}{\isadigit{6}}{\isadigit{0}}{\isacharparenright}\ %
\isamarkupcmt{normal equality (for all types)%
}
\isanewline
\ \ \isakeyword{where}\ {\isachardoublequoteopen}x\ \isactrlbold {\isasymapprox}\ y\ {\isasymequiv}\ {\isasymlambda}w{\isachardot}\ x\ {\isacharequal}\ y{\isachardoublequoteclose}\isanewline
\isacommand{abbreviation}\isamarkupfalse%
\ meqC{\isacharcolon}{\isacharcolon}\ {\isachardoublequoteopen}{\isasymup}{\isasymlangle}{\isasymup}{\isasymzero}{\isacharcomma}{\isasymup}{\isasymzero}{\isasymrangle}{\isachardoublequoteclose}\ {\isacharparenleft}\isakeyword{infixr}{\isachardoublequoteopen}\isactrlbold {\isasymapprox}\isactrlsup C{\isachardoublequoteclose}{\isadigit{5}}{\isadigit{2}}{\isacharparenright}\ %
\isamarkupcmt{equality for individual concepts%
}
\isanewline
\ \ \isakeyword{where}\ {\isachardoublequoteopen}x\ \isactrlbold {\isasymapprox}\isactrlsup C\ y\ {\isasymequiv}\ {\isasymlambda}w{\isachardot}\ {\isasymforall}v{\isachardot}\ {\isacharparenleft}x\ v{\isacharparenright}\ {\isacharequal}\ {\isacharparenleft}y\ v{\isacharparenright}{\isachardoublequoteclose}\isanewline
\isacommand{abbreviation}\isamarkupfalse%
\ meqL{\isacharcolon}{\isacharcolon}\ {\isachardoublequoteopen}{\isasymup}{\isasymlangle}{\isasymzero}{\isacharcomma}{\isasymzero}{\isasymrangle}{\isachardoublequoteclose}\ {\isacharparenleft}\isakeyword{infixr}{\isachardoublequoteopen}\isactrlbold {\isasymapprox}\isactrlsup L{\isachardoublequoteclose}{\isadigit{5}}{\isadigit{2}}{\isacharparenright}\ %
\isamarkupcmt{Leibniz equality for individuals%
}
\isanewline
\ \ \isakeyword{where}\ {\isachardoublequoteopen}x\ \isactrlbold {\isasymapprox}\isactrlsup L\ y\ {\isasymequiv}\ \isactrlbold {\isasymforall}{\isasymphi}{\isachardot}\ {\isasymphi}{\isacharparenleft}x{\isacharparenright}\isactrlbold {\isasymrightarrow}{\isasymphi}{\isacharparenleft}y{\isacharparenright}{\isachardoublequoteclose}%
\isamarkupsubsection{\emph{Extension-of} Operator%
}
\isamarkuptrue%
%
\begin{isamarkuptext}%
According to Fitting's semantics (\cite{Fitting}, pp. 92-4) \isa{{\isasymdown}} is an unary operator applying only to 
 intensional terms. A term of the form \isa{{\isasymdown}{\isasymalpha}} designates the extension of the intensional object designated by 
 \isa{{\isasymalpha}}, at some \emph{given} world. For instance, suppose we take possible worlds as persons,
 we can therefore think of the concept `red' as a function that maps each person to the set of objects that person
 classifies as red (its extension). We can further state, the intensional term \emph{r} of type \isa{{\isasymup}{\isasymlangle}{\isasymzero}{\isasymrangle}} designates the concept `red'.
 As can be seen, intensional terms in IHOML designate functions on possible worlds and they always do it \emph{rigidly}. 
 We will sometimes refer to an intensional object explicitly as `rigid', implying that its (rigidly) designated function has
 the same extension in all possible worlds. \footnote{The notion of rigidity was introduced by Kripke in \cite{kripke1980},
 where he discusses its interesting philosophical ramifications at some length.}%
\end{isamarkuptext}\isamarkuptrue%
%
\begin{isamarkuptext}%
Terms of the form \isa{{\isasymdown}{\isasymalpha}} are called \emph{relativized} (extensional) terms; they are always derived
from intensional terms and their type is \emph{extensional} (in the color example \isa{{\isasymdown}r} would be of type \isa{{\isasymlangle}{\isasymzero}{\isasymrangle}}).
Relativized terms may vary their denotation from world to world of a model, because the extension of an intensional term can change
from world to world, i.e. they are non-rigid.%
\end{isamarkuptext}\isamarkuptrue%
%
\begin{isamarkuptext}%
For our Isabelle/HOL embedding, we had to follow a slightly different approach; we model \isa{{\isasymdown}}
as a predicate applying to formulas of the form \isa{{\isasymPhi}{\isacharparenleft}{\isasymdown}{\isasymalpha}\isactrlsub {\isadigit{1}}{\isacharcomma}{\isasymdots}{\isasymalpha}\isactrlsub n{\isacharparenright}} (for our treatment
we only need to consider cases involving one or two arguments, the first one being a relativized term).
For instance, the formula \isa{Q{\isacharparenleft}{\isasymdown}a\isactrlsub {\isadigit{1}}{\isacharparenright}\isactrlsup w} (evaluated at world \emph{w}) is modelled as \isa{{\isasymdownharpoonleft}{\isacharparenleft}Q{\isacharcomma}a\isactrlsub {\isadigit{1}}{\isacharparenright}\isactrlsup w}
(or \isa{{\isacharparenleft}Q\ {\isasymdownharpoonleft}\ a\isactrlsub {\isadigit{1}}{\isacharparenright}\isactrlsup w} using infix notation), which gets further translated into \isa{Q{\isacharparenleft}a\isactrlsub {\isadigit{1}}{\isacharparenleft}w{\isacharparenright}{\isacharparenright}\isactrlsup w}.%
\end{isamarkuptext}\isamarkuptrue%
%
\begin{isamarkuptext}%
(\emph{a}) Predicate \isa{{\isasymphi}} takes as argument a relativized term derived from an (intensional) individual of type \isa{{\isasymup}{\isasymzero}}.%
\end{isamarkuptext}\isamarkuptrue%
\isacommand{abbreviation}\isamarkupfalse%
\ extIndArg{\isacharcolon}{\isacharcolon}{\isachardoublequoteopen}{\isasymup}{\isasymlangle}{\isasymzero}{\isasymrangle}{\isasymRightarrow}{\isasymup}{\isasymzero}{\isasymRightarrow}io{\isachardoublequoteclose}\ {\isacharparenleft}\isakeyword{infix}\ {\isachardoublequoteopen}{\isasymdownharpoonleft}{\isachardoublequoteclose}{\isadigit{6}}{\isadigit{0}}{\isacharparenright}\ \isakeyword{where}\ {\isachardoublequoteopen}{\isasymphi}\ {\isasymdownharpoonleft}c\ {\isasymequiv}\ {\isasymlambda}w{\isachardot}\ {\isasymphi}\ {\isacharparenleft}c\ w{\isacharparenright}\ w{\isachardoublequoteclose}%
\begin{isamarkuptext}%
(\emph{b}) A variant of (\emph{a}) for terms derived from predicates (types of form \isa{{\isasymup}{\isasymlangle}t{\isasymrangle}}).%
\end{isamarkuptext}\isamarkuptrue%
\isacommand{abbreviation}\isamarkupfalse%
\ extPredArg{\isacharcolon}{\isacharcolon}{\isachardoublequoteopen}{\isacharparenleft}{\isacharparenleft}{\isacharprime}t{\isasymRightarrow}bool{\isacharparenright}{\isasymRightarrow}io{\isacharparenright}{\isasymRightarrow}{\isacharparenleft}{\isacharprime}t{\isasymRightarrow}io{\isacharparenright}{\isasymRightarrow}io{\isachardoublequoteclose}\ {\isacharparenleft}\isakeyword{infix}\ {\isachardoublequoteopen}{\isasymdown}{\isachardoublequoteclose}\ {\isadigit{6}}{\isadigit{0}}{\isacharparenright}\isanewline
\ \ \isakeyword{where}\ {\isachardoublequoteopen}{\isasymphi}\ {\isasymdown}P\ {\isasymequiv}\ {\isasymlambda}w{\isachardot}\ {\isasymphi}\ {\isacharparenleft}{\isasymlambda}x{\isachardot}\ P\ x\ w{\isacharparenright}\ w{\isachardoublequoteclose}%
\isamarkupsubsection{Verifying the Embedding%
}
\isamarkuptrue%
%
\begin{isamarkuptext}%
The above definitions introduce modal logic \emph{K} with possibilist and actualist quantifiers,
as evidenced by following tests:\footnote{In our computer-formalization and assessment of Fitting's textbook \cite{J35},
we provide further evidence that our embedded logic works as intended by formalizing and verifying
the book's theorems and examples. We refer the reader to this work for further details.}%
\end{isamarkuptext}\isamarkuptrue%
\ \isacommand{abbreviation}\isamarkupfalse%
\ valid{\isacharcolon}{\isacharcolon}{\isachardoublequoteopen}io{\isasymRightarrow}bool{\isachardoublequoteclose}\ {\isacharparenleft}{\isachardoublequoteopen}{\isasymlfloor}{\isacharunderscore}{\isasymrfloor}{\isachardoublequoteclose}{\isacharparenright}\ \isakeyword{where}\ {\isachardoublequoteopen}{\isasymlfloor}{\isasympsi}{\isasymrfloor}\ {\isasymequiv}\ \ {\isasymforall}w{\isachardot}{\isacharparenleft}{\isasympsi}\ w{\isacharparenright}{\isachardoublequoteclose}\ %
\isamarkupcmt{modal validity%
}
%
\begin{isamarkuptext}%
Verifying \emph{K} principle and the \emph{necessitation} rule.
 \footnote{We prove here our first theorem with Isabelle, as indicated by the keyword `by' followed by
 the name of the method used for the proof. In this case Isabelle's simplifier (term rewriting) sufficed.
 Other proof methods used in this work are: \emph{blast} (tableaus), \emph{meson} (model elimination),
 \emph{metis} (ordered resolution and paramodulation), \emph{auto} (classical reasoning and term rewriting)
 and \emph{force} (exhaustive search trying different tools).}%
\end{isamarkuptext}\isamarkuptrue%
\ \isacommand{lemma}\isamarkupfalse%
\ K{\isacharcolon}\ {\isachardoublequoteopen}{\isasymlfloor}{\isacharparenleft}\isactrlbold {\isasymbox}{\isacharparenleft}{\isasymphi}\ \isactrlbold {\isasymrightarrow}\ {\isasympsi}{\isacharparenright}{\isacharparenright}\ \isactrlbold {\isasymrightarrow}\ {\isacharparenleft}\isactrlbold {\isasymbox}{\isasymphi}\ \isactrlbold {\isasymrightarrow}\ \isactrlbold {\isasymbox}{\isasympsi}{\isacharparenright}{\isasymrfloor}{\isachardoublequoteclose}%
\ %
%
\isacommand{by}\isamarkupfalse%
\ simp\ \ \ \ %
\isamarkupcmt{\emph{K} schema%
}
%
%
%
\isanewline
\ \isacommand{lemma}\isamarkupfalse%
\ NEC{\isacharcolon}\ {\isachardoublequoteopen}{\isasymlfloor}{\isasymphi}{\isasymrfloor}\ {\isasymLongrightarrow}\ {\isasymlfloor}\isactrlbold {\isasymbox}{\isasymphi}{\isasymrfloor}{\isachardoublequoteclose}%
\ %
%
\isacommand{by}\isamarkupfalse%
\ simp\ \ \ \ %
\isamarkupcmt{necessitation%
}
%
%
%
%
\begin{isamarkuptext}%
Local consequence implies global consequence (not the other way round).
  \footnote{We utilize here (counter-)model finder \emph{Nitpick} \cite{Nitpick} for the first time.  
  For the conjectured lemma, \emph{Nitpick} finds a countermodel, i.e. a model satisfying all 
  the axioms which falsifies the given formula, which means it is not valid, as indicated by the `oops' keyword.}%
\end{isamarkuptext}\isamarkuptrue%
\ \isacommand{lemma}\isamarkupfalse%
\ localImpGlobalCons{\isacharcolon}\ {\isachardoublequoteopen}{\isasymlfloor}{\isasymphi}\ \isactrlbold {\isasymrightarrow}\ {\isasymxi}{\isasymrfloor}\ {\isasymLongrightarrow}\ {\isasymlfloor}{\isasymphi}{\isasymrfloor}\ {\isasymlongrightarrow}\ {\isasymlfloor}{\isasymxi}{\isasymrfloor}{\isachardoublequoteclose}%
\ %
%
\isacommand{by}\isamarkupfalse%
\ simp%
%
%
\isanewline
\ \isacommand{lemma}\isamarkupfalse%
\ {\isachardoublequoteopen}{\isasymlfloor}{\isasymphi}{\isasymrfloor}\ {\isasymlongrightarrow}\ {\isasymlfloor}{\isasymxi}{\isasymrfloor}\ {\isasymLongrightarrow}\ {\isasymlfloor}{\isasymphi}\ \isactrlbold {\isasymrightarrow}\ {\isasymxi}{\isasymrfloor}{\isachardoublequoteclose}\ \isacommand{nitpick}\isamarkupfalse%
%
\ %
%
\isacommand{oops}\isamarkupfalse%
\ %
\isamarkupcmt{countersatisfiable%
}
%
%
%
%
\begin{isamarkuptext}%
(Converse-)Barcan formulas are satisfied for possibilist, but not for actualist, quantification.%
\end{isamarkuptext}\isamarkuptrue%
\ \isacommand{lemma}\isamarkupfalse%
\ {\isachardoublequoteopen}{\isasymlfloor}{\isacharparenleft}\isactrlbold {\isasymforall}x{\isachardot}\isactrlbold {\isasymbox}{\isacharparenleft}{\isasymphi}\ x{\isacharparenright}{\isacharparenright}\ \isactrlbold {\isasymrightarrow}\ \isactrlbold {\isasymbox}{\isacharparenleft}\isactrlbold {\isasymforall}x{\isachardot}{\isacharparenleft}{\isasymphi}\ x{\isacharparenright}{\isacharparenright}{\isasymrfloor}{\isachardoublequoteclose}%
\ %
%
\isacommand{by}\isamarkupfalse%
\ simp%
%
%
\isanewline
\ \isacommand{lemma}\isamarkupfalse%
\ {\isachardoublequoteopen}{\isasymlfloor}\isactrlbold {\isasymbox}{\isacharparenleft}\isactrlbold {\isasymforall}x{\isachardot}{\isacharparenleft}{\isasymphi}\ x{\isacharparenright}{\isacharparenright}\ \isactrlbold {\isasymrightarrow}\ {\isacharparenleft}\isactrlbold {\isasymforall}x{\isachardot}\isactrlbold {\isasymbox}{\isacharparenleft}{\isasymphi}\ x{\isacharparenright}{\isacharparenright}{\isasymrfloor}{\isachardoublequoteclose}%
\ %
%
\isacommand{by}\isamarkupfalse%
\ simp%
%
%
\isanewline
\ \isacommand{lemma}\isamarkupfalse%
\ {\isachardoublequoteopen}{\isasymlfloor}{\isacharparenleft}\isactrlbold {\isasymforall}\isactrlsup Ex{\isachardot}\isactrlbold {\isasymbox}{\isacharparenleft}{\isasymphi}\ x{\isacharparenright}{\isacharparenright}\ \isactrlbold {\isasymrightarrow}\ \isactrlbold {\isasymbox}{\isacharparenleft}\isactrlbold {\isasymforall}\isactrlsup Ex{\isachardot}{\isacharparenleft}{\isasymphi}\ x{\isacharparenright}{\isacharparenright}{\isasymrfloor}{\isachardoublequoteclose}\ \isacommand{nitpick}\isamarkupfalse%
%
\ %
%
\isacommand{oops}\isamarkupfalse%
\ %
\isamarkupcmt{countersatisfiable%
}
%
%
%
\isanewline
\ \isacommand{lemma}\isamarkupfalse%
\ {\isachardoublequoteopen}{\isasymlfloor}\isactrlbold {\isasymbox}{\isacharparenleft}\isactrlbold {\isasymforall}\isactrlsup Ex{\isachardot}{\isacharparenleft}{\isasymphi}\ x{\isacharparenright}{\isacharparenright}\ \isactrlbold {\isasymrightarrow}\ {\isacharparenleft}\isactrlbold {\isasymforall}\isactrlsup Ex{\isachardot}\isactrlbold {\isasymbox}{\isacharparenleft}{\isasymphi}\ x{\isacharparenright}{\isacharparenright}{\isasymrfloor}{\isachardoublequoteclose}\ \isacommand{nitpick}\isamarkupfalse%
%
\ %
%
\isacommand{oops}\isamarkupfalse%
\ %
\isamarkupcmt{countersatisfiable%
}
%
%
%
%
\begin{isamarkuptext}%
\isa{{\isasymbeta}{\isasymeta}}-redex is valid for non-relativized (intensional or extensional) terms.%
\end{isamarkuptext}\isamarkuptrue%
\ \isacommand{lemma}\isamarkupfalse%
\ {\isachardoublequoteopen}{\isasymlfloor}{\isacharparenleft}{\isacharparenleft}{\isasymlambda}{\isasymalpha}{\isachardot}\ {\isasymphi}\ {\isasymalpha}{\isacharparenright}\ \ {\isacharparenleft}{\isasymtau}{\isacharcolon}{\isacharcolon}{\isasymup}{\isasymzero}{\isacharparenright}{\isacharparenright}\ \isactrlbold {\isasymleftrightarrow}\ {\isacharparenleft}{\isasymphi}\ \ {\isasymtau}{\isacharparenright}{\isasymrfloor}{\isachardoublequoteclose}%
\ %
%
\isacommand{by}\isamarkupfalse%
\ simp%
%
%
\isanewline
\ \isacommand{lemma}\isamarkupfalse%
\ {\isachardoublequoteopen}{\isasymlfloor}{\isacharparenleft}{\isacharparenleft}{\isasymlambda}{\isasymalpha}{\isachardot}\ {\isasymphi}\ {\isasymalpha}{\isacharparenright}\ \ {\isacharparenleft}{\isasymtau}{\isacharcolon}{\isacharcolon}{\isasymzero}{\isacharparenright}{\isacharparenright}\ \isactrlbold {\isasymleftrightarrow}\ {\isacharparenleft}{\isasymphi}\ \ {\isasymtau}{\isacharparenright}{\isasymrfloor}{\isachardoublequoteclose}%
\ %
%
\isacommand{by}\isamarkupfalse%
\ simp%
%
%
\isanewline
\ \isacommand{lemma}\isamarkupfalse%
\ {\isachardoublequoteopen}{\isasymlfloor}{\isacharparenleft}{\isacharparenleft}{\isasymlambda}{\isasymalpha}{\isachardot}\ \isactrlbold {\isasymbox}{\isasymphi}\ {\isasymalpha}{\isacharparenright}\ {\isacharparenleft}{\isasymtau}{\isacharcolon}{\isacharcolon}{\isasymup}{\isasymzero}{\isacharparenright}{\isacharparenright}\ \isactrlbold {\isasymleftrightarrow}\ {\isacharparenleft}\isactrlbold {\isasymbox}{\isasymphi}\ {\isasymtau}{\isacharparenright}{\isasymrfloor}{\isachardoublequoteclose}%
\ %
%
\isacommand{by}\isamarkupfalse%
\ simp%
%
%
\isanewline
\ \isacommand{lemma}\isamarkupfalse%
\ {\isachardoublequoteopen}{\isasymlfloor}{\isacharparenleft}{\isacharparenleft}{\isasymlambda}{\isasymalpha}{\isachardot}\ \isactrlbold {\isasymbox}{\isasymphi}\ {\isasymalpha}{\isacharparenright}\ {\isacharparenleft}{\isasymtau}{\isacharcolon}{\isacharcolon}{\isasymzero}{\isacharparenright}{\isacharparenright}\ \isactrlbold {\isasymleftrightarrow}\ {\isacharparenleft}\isactrlbold {\isasymbox}{\isasymphi}\ {\isasymtau}{\isacharparenright}{\isasymrfloor}{\isachardoublequoteclose}%
\ %
%
\isacommand{by}\isamarkupfalse%
\ simp%
%
%
%
\begin{isamarkuptext}%
\isa{{\isasymbeta}{\isasymeta}}-redex is valid for relativized terms as long as no modal operators occur inside the predicate abstract.%
\end{isamarkuptext}\isamarkuptrue%
\ \isacommand{lemma}\isamarkupfalse%
\ {\isachardoublequoteopen}{\isasymlfloor}{\isacharparenleft}{\isacharparenleft}{\isasymlambda}{\isasymalpha}{\isachardot}\ {\isasymphi}\ {\isasymalpha}{\isacharparenright}\ {\isasymdownharpoonleft}{\isacharparenleft}{\isasymtau}{\isacharcolon}{\isacharcolon}{\isasymup}{\isasymzero}{\isacharparenright}{\isacharparenright}\ \isactrlbold {\isasymleftrightarrow}\ {\isacharparenleft}{\isasymphi}\ {\isasymdownharpoonleft}{\isasymtau}{\isacharparenright}{\isasymrfloor}{\isachardoublequoteclose}%
\ %
%
\isacommand{by}\isamarkupfalse%
\ simp%
%
%
\isanewline
\ \isacommand{lemma}\isamarkupfalse%
\ {\isachardoublequoteopen}{\isasymlfloor}{\isacharparenleft}{\isacharparenleft}{\isasymlambda}{\isasymalpha}{\isachardot}\ \isactrlbold {\isasymbox}{\isasymphi}\ {\isasymalpha}{\isacharparenright}\ {\isasymdownharpoonleft}{\isacharparenleft}{\isasymtau}{\isacharcolon}{\isacharcolon}{\isasymup}{\isasymzero}{\isacharparenright}{\isacharparenright}\ \isactrlbold {\isasymleftrightarrow}\ {\isacharparenleft}\isactrlbold {\isasymbox}{\isasymphi}\ {\isasymdownharpoonleft}{\isasymtau}{\isacharparenright}{\isasymrfloor}{\isachardoublequoteclose}\ \isacommand{nitpick}\isamarkupfalse%
%
\ %
%
\isacommand{oops}\isamarkupfalse%
\ %
\isamarkupcmt{countersatisfiable%
}
%
%
%
%
\begin{isamarkuptext}%
Modal collapse is countersatisfiable.%
\end{isamarkuptext}\isamarkuptrue%
\isacommand{lemma}\isamarkupfalse%
\ {\isachardoublequoteopen}{\isasymlfloor}{\isasymphi}\ \isactrlbold {\isasymrightarrow}\ \isactrlbold {\isasymbox}{\isasymphi}{\isasymrfloor}{\isachardoublequoteclose}\ \isacommand{nitpick}\isamarkupfalse%
%
\ %
%
\isacommand{oops}\isamarkupfalse%
\ \ \ %
\isamarkupcmt{countersatisfiable%
}
%
%
%
%
\isamarkupsubsection{Stability, Rigid Designation, \emph{De Re} and \emph{De Dicto}%
}
\isamarkuptrue%
%
\begin{isamarkuptext}%
As said before, intensional terms are trivially rigid. The following predicate tests whether an intensional
predicate is `rigid' in the sense of denoting a world-independent function.%
\end{isamarkuptext}\isamarkuptrue%
\isacommand{abbreviation}\isamarkupfalse%
\ rigidPred{\isacharcolon}{\isacharcolon}{\isachardoublequoteopen}{\isacharparenleft}{\isacharprime}t{\isasymRightarrow}io{\isacharparenright}{\isasymRightarrow}io{\isachardoublequoteclose}\ \isakeyword{where}\isanewline
\ \ {\isachardoublequoteopen}rigidPred\ {\isasymtau}\ {\isasymequiv}\ {\isacharparenleft}{\isasymlambda}{\isasymbeta}{\isachardot}\ \isactrlbold {\isasymbox}{\isacharparenleft}{\isacharparenleft}{\isasymlambda}z{\isachardot}\ {\isasymbeta}\ \isactrlbold {\isasymapprox}\ z{\isacharparenright}\ \isactrlbold {\isasymdown}{\isasymtau}{\isacharparenright}{\isacharparenright}\ \isactrlbold {\isasymdown}{\isasymtau}{\isachardoublequoteclose}%
\begin{isamarkuptext}%
Following definitions are called `stability conditions' by Fitting (\cite{Fitting}, p. 124).%
\end{isamarkuptext}\isamarkuptrue%
\isacommand{abbreviation}\isamarkupfalse%
\ stabilityA{\isacharcolon}{\isacharcolon}{\isachardoublequoteopen}{\isacharparenleft}{\isacharprime}t{\isasymRightarrow}io{\isacharparenright}{\isasymRightarrow}io{\isachardoublequoteclose}\ \isakeyword{where}\ {\isachardoublequoteopen}stabilityA\ {\isasymtau}\ {\isasymequiv}\ \isactrlbold {\isasymforall}{\isasymalpha}{\isachardot}\ {\isacharparenleft}{\isasymtau}\ {\isasymalpha}{\isacharparenright}\ \isactrlbold {\isasymrightarrow}\ \isactrlbold {\isasymbox}{\isacharparenleft}{\isasymtau}\ {\isasymalpha}{\isacharparenright}{\isachardoublequoteclose}\isanewline
\isacommand{abbreviation}\isamarkupfalse%
\ stabilityB{\isacharcolon}{\isacharcolon}{\isachardoublequoteopen}{\isacharparenleft}{\isacharprime}t{\isasymRightarrow}io{\isacharparenright}{\isasymRightarrow}io{\isachardoublequoteclose}\ \isakeyword{where}\ {\isachardoublequoteopen}stabilityB\ {\isasymtau}\ {\isasymequiv}\ \isactrlbold {\isasymforall}{\isasymalpha}{\isachardot}\ \isactrlbold {\isasymdiamond}{\isacharparenleft}{\isasymtau}\ {\isasymalpha}{\isacharparenright}\ \isactrlbold {\isasymrightarrow}\ {\isacharparenleft}{\isasymtau}\ {\isasymalpha}{\isacharparenright}{\isachardoublequoteclose}%
\begin{isamarkuptext}%
We prove them equivalent in \emph{S5} logic (using \emph{Sahlqvist correspondence}).%
\end{isamarkuptext}\isamarkuptrue%
\isacommand{lemma}\isamarkupfalse%
\ {\isachardoublequoteopen}equivalence\ aRel\ {\isasymLongrightarrow}\ {\isasymlfloor}stabilityA\ {\isacharparenleft}{\isasymtau}{\isacharcolon}{\isacharcolon}{\isasymup}{\isasymlangle}{\isasymzero}{\isasymrangle}{\isacharparenright}{\isasymrfloor}\ {\isasymlongrightarrow}\ {\isasymlfloor}stabilityB\ {\isasymtau}{\isasymrfloor}{\isachardoublequoteclose}%
\ %
%
\isacommand{by}\isamarkupfalse%
\ blast%
%
%
\ \ \ \ \isanewline
\isacommand{lemma}\isamarkupfalse%
\ {\isachardoublequoteopen}equivalence\ aRel\ {\isasymLongrightarrow}\ {\isasymlfloor}stabilityB\ {\isacharparenleft}{\isasymtau}{\isacharcolon}{\isacharcolon}{\isasymup}{\isasymlangle}{\isasymzero}{\isasymrangle}{\isacharparenright}{\isasymrfloor}\ {\isasymlongrightarrow}\ {\isasymlfloor}stabilityA\ {\isasymtau}{\isasymrfloor}{\isachardoublequoteclose}%
\ %
%
\isacommand{by}\isamarkupfalse%
\ blast%
%
%
%
\begin{isamarkuptext}%
A term is rigid if and only if it satisfies the stability conditions.%
\end{isamarkuptext}\isamarkuptrue%
\isacommand{theorem}\isamarkupfalse%
\ {\isachardoublequoteopen}{\isasymlfloor}rigidPred\ {\isacharparenleft}{\isasymtau}{\isacharcolon}{\isacharcolon}{\isasymup}{\isasymlangle}{\isasymzero}{\isasymrangle}{\isacharparenright}{\isasymrfloor}\ {\isasymlongleftrightarrow}\ {\isasymlfloor}{\isacharparenleft}stabilityA\ {\isasymtau}\ \isactrlbold {\isasymand}\ stabilityB\ {\isasymtau}{\isacharparenright}{\isasymrfloor}{\isachardoublequoteclose}%
\ %
%
\isacommand{by}\isamarkupfalse%
\ meson%
%
%
\ \ \ \isanewline
\isacommand{theorem}\isamarkupfalse%
\ {\isachardoublequoteopen}{\isasymlfloor}rigidPred\ {\isacharparenleft}{\isasymtau}{\isacharcolon}{\isacharcolon}{\isasymup}{\isasymlangle}{\isasymup}{\isasymzero}{\isasymrangle}{\isacharparenright}{\isasymrfloor}\ {\isasymlongleftrightarrow}\ {\isasymlfloor}{\isacharparenleft}stabilityA\ {\isasymtau}\ \isactrlbold {\isasymand}\ stabilityB\ {\isasymtau}{\isacharparenright}{\isasymrfloor}{\isachardoublequoteclose}%
\ %
%
\isacommand{by}\isamarkupfalse%
\ meson%
%
%
%
\begin{isamarkuptext}%
\emph{De re} is equivalent to \emph{de dicto} for non-relativized (i.e. rigid) terms.%
\end{isamarkuptext}\isamarkuptrue%
\isacommand{lemma}\isamarkupfalse%
\ {\isachardoublequoteopen}{\isasymlfloor}\isactrlbold {\isasymforall}{\isasymalpha}{\isachardot}\ {\isacharparenleft}{\isacharparenleft}{\isasymlambda}{\isasymbeta}{\isachardot}\ \isactrlbold {\isasymbox}{\isacharparenleft}{\isasymalpha}\ {\isasymbeta}{\isacharparenright}{\isacharparenright}\ {\isacharparenleft}{\isasymtau}{\isacharcolon}{\isacharcolon}{\isasymlangle}{\isasymzero}{\isasymrangle}{\isacharparenright}{\isacharparenright}\ \ \isactrlbold {\isasymleftrightarrow}\ \isactrlbold {\isasymbox}{\isacharparenleft}{\isacharparenleft}{\isasymlambda}{\isasymbeta}{\isachardot}\ {\isacharparenleft}{\isasymalpha}\ {\isasymbeta}{\isacharparenright}{\isacharparenright}\ {\isasymtau}{\isacharparenright}{\isasymrfloor}{\isachardoublequoteclose}%
\ %
%
\isacommand{by}\isamarkupfalse%
\ simp%
%
%
\isanewline
\isacommand{lemma}\isamarkupfalse%
\ {\isachardoublequoteopen}{\isasymlfloor}\isactrlbold {\isasymforall}{\isasymalpha}{\isachardot}\ {\isacharparenleft}{\isacharparenleft}{\isasymlambda}{\isasymbeta}{\isachardot}\ \isactrlbold {\isasymbox}{\isacharparenleft}{\isasymalpha}\ {\isasymbeta}{\isacharparenright}{\isacharparenright}\ {\isacharparenleft}{\isasymtau}{\isacharcolon}{\isacharcolon}{\isasymup}{\isasymlangle}{\isasymzero}{\isasymrangle}{\isacharparenright}{\isacharparenright}\ \isactrlbold {\isasymleftrightarrow}\ \isactrlbold {\isasymbox}{\isacharparenleft}{\isacharparenleft}{\isasymlambda}{\isasymbeta}{\isachardot}\ {\isacharparenleft}{\isasymalpha}\ {\isasymbeta}{\isacharparenright}{\isacharparenright}\ {\isasymtau}{\isacharparenright}{\isasymrfloor}{\isachardoublequoteclose}%
\ %
%
\isacommand{by}\isamarkupfalse%
\ simp%
%
%
%
\begin{isamarkuptext}%
\emph{De re} is not equivalent to \emph{de dicto} for relativized terms.%
\end{isamarkuptext}\isamarkuptrue%
\isacommand{lemma}\isamarkupfalse%
\ {\isachardoublequoteopen}{\isasymlfloor}\isactrlbold {\isasymforall}{\isasymalpha}{\isachardot}\ {\isacharparenleft}{\isacharparenleft}{\isasymlambda}{\isasymbeta}{\isachardot}\ \isactrlbold {\isasymbox}{\isacharparenleft}{\isasymalpha}\ {\isasymbeta}{\isacharparenright}{\isacharparenright}\ \isactrlbold {\isasymdown}{\isacharparenleft}{\isasymtau}{\isacharcolon}{\isacharcolon}{\isasymup}{\isasymlangle}{\isasymzero}{\isasymrangle}{\isacharparenright}{\isacharparenright}\ \isactrlbold {\isasymleftrightarrow}\ \isactrlbold {\isasymbox}{\isacharparenleft}{\isacharparenleft}{\isasymlambda}{\isasymbeta}{\isachardot}\ {\isacharparenleft}{\isasymalpha}\ {\isasymbeta}{\isacharparenright}{\isacharparenright}\ \isactrlbold {\isasymdown}{\isasymtau}{\isacharparenright}{\isasymrfloor}{\isachardoublequoteclose}\ \isanewline
\ \ \isacommand{nitpick}\isamarkupfalse%
{\isacharbrackleft}card\ {\isacharprime}t{\isacharequal}{\isadigit{1}}{\isacharcomma}\ card\ i{\isacharequal}{\isadigit{2}}{\isacharbrackright}%
\ %
%
\isacommand{oops}\isamarkupfalse%
\ %
\isamarkupcmt{countersatisfiable%
}
%
%
%
%
\isamarkupsubsection{Useful Definitions for Axiomatization of Further Logics%
}
\isamarkuptrue%
%
\begin{isamarkuptext}%
The best known normal logics (\emph{K4, K5, KB, K45, KB5, D, D4, D5, D45, ...}) can be obtained by
 combinations of the following axioms:%
\end{isamarkuptext}\isamarkuptrue%
\ \ \isacommand{abbreviation}\isamarkupfalse%
\ M\ \ \isakeyword{where}\ {\isachardoublequoteopen}M\ {\isasymequiv}\ \isactrlbold {\isasymforall}{\isasymphi}{\isachardot}\ \isactrlbold {\isasymbox}{\isasymphi}\ \isactrlbold {\isasymrightarrow}\ {\isasymphi}{\isachardoublequoteclose}\isanewline
\ \ \isacommand{abbreviation}\isamarkupfalse%
\ B\ \ \isakeyword{where}\ {\isachardoublequoteopen}B\ {\isasymequiv}\ \isactrlbold {\isasymforall}{\isasymphi}{\isachardot}\ {\isasymphi}\ \isactrlbold {\isasymrightarrow}\ \ \isactrlbold {\isasymbox}\isactrlbold {\isasymdiamond}{\isasymphi}{\isachardoublequoteclose}\isanewline
\ \ \isacommand{abbreviation}\isamarkupfalse%
\ D\ \ \isakeyword{where}\ {\isachardoublequoteopen}D\ {\isasymequiv}\ \isactrlbold {\isasymforall}{\isasymphi}{\isachardot}\ \isactrlbold {\isasymbox}{\isasymphi}\ \isactrlbold {\isasymrightarrow}\ \isactrlbold {\isasymdiamond}{\isasymphi}{\isachardoublequoteclose}\isanewline
\ \ \isacommand{abbreviation}\isamarkupfalse%
\ IV\ \isakeyword{where}\ {\isachardoublequoteopen}IV\ {\isasymequiv}\ \isactrlbold {\isasymforall}{\isasymphi}{\isachardot}\ \isactrlbold {\isasymbox}{\isasymphi}\ \isactrlbold {\isasymrightarrow}\ \ \isactrlbold {\isasymbox}\isactrlbold {\isasymbox}{\isasymphi}{\isachardoublequoteclose}\isanewline
\ \ \isacommand{abbreviation}\isamarkupfalse%
\ V\ \ \isakeyword{where}\ {\isachardoublequoteopen}V\ {\isasymequiv}\ \isactrlbold {\isasymforall}{\isasymphi}{\isachardot}\ \isactrlbold {\isasymdiamond}{\isasymphi}\ \isactrlbold {\isasymrightarrow}\ \isactrlbold {\isasymbox}\isactrlbold {\isasymdiamond}{\isasymphi}{\isachardoublequoteclose}%
\begin{isamarkuptext}%
Instead of postulating (combinations of) the above  axioms we instead make use of 
  the well-known \emph{Sahlqvist correspondence}, which links axioms to constraints on a model's accessibility
  relation (e.g. reflexive, symmetric, etc). We show  that  reflexivity, symmetry, seriality, transitivity and euclideanness imply
  axioms $M, B, D, IV, V$ respectively.
  \footnote{Implication can also be proven in the reverse direction (which is not needed for our purposes).
  Using these definitions, we can derive axioms for the most common modal logics (see also \cite{C47}). 
  Thereby we are free to use either the semantic constraints or the related \emph{Sahlqvist} axioms. Here we provide 
  both versions. In what follows we use the semantic constraints (for improved performance).}%
\end{isamarkuptext}\isamarkuptrue%
\ \ \isacommand{lemma}\isamarkupfalse%
\ {\isachardoublequoteopen}reflexive\ aRel\ \ {\isasymLongrightarrow}\ \ {\isasymlfloor}M{\isasymrfloor}{\isachardoublequoteclose}%
\ %
%
\isacommand{by}\isamarkupfalse%
\ blast\ %
\isamarkupcmt{aka T%
}
%
%
%
\isanewline
\ \ \isacommand{lemma}\isamarkupfalse%
\ {\isachardoublequoteopen}symmetric\ aRel\ {\isasymLongrightarrow}\ {\isasymlfloor}B{\isasymrfloor}{\isachardoublequoteclose}%
\ %
%
\isacommand{by}\isamarkupfalse%
\ blast%
%
%
\isanewline
\ \ \isacommand{lemma}\isamarkupfalse%
\ {\isachardoublequoteopen}serial\ aRel\ \ {\isasymLongrightarrow}\ {\isasymlfloor}D{\isasymrfloor}{\isachardoublequoteclose}%
\ %
%
\isacommand{by}\isamarkupfalse%
\ blast%
%
%
\ \ \ \ \ \ \ \ \ \isanewline
\ \ \isacommand{lemma}\isamarkupfalse%
\ {\isachardoublequoteopen}transitive\ aRel\ \ {\isasymLongrightarrow}\ {\isasymlfloor}IV{\isasymrfloor}{\isachardoublequoteclose}%
\ %
%
\isacommand{by}\isamarkupfalse%
\ blast%
%
%
\ \ \ \isanewline
\ \ \isacommand{lemma}\isamarkupfalse%
\ {\isachardoublequoteopen}euclidean\ aRel\ {\isasymLongrightarrow}\ {\isasymlfloor}V{\isasymrfloor}{\isachardoublequoteclose}%
\ %
%
\isacommand{by}\isamarkupfalse%
\ blast%
%
%
\ \ \ \ \ \ \ \ \ \isanewline
\ \ \isacommand{lemma}\isamarkupfalse%
\ {\isachardoublequoteopen}preorder\ aRel\ {\isasymLongrightarrow}\ {\isasymlfloor}M{\isasymrfloor}\ {\isasymand}\ {\isasymlfloor}IV{\isasymrfloor}{\isachardoublequoteclose}%
\ %
%
\isacommand{by}\isamarkupfalse%
\ blast\ %
\isamarkupcmt{S4: reflexive + transitive%
}
%
%
%
\isanewline
\ \ \isacommand{lemma}\isamarkupfalse%
\ {\isachardoublequoteopen}equivalence\ aRel\ {\isasymLongrightarrow}\ {\isasymlfloor}M{\isasymrfloor}\ {\isasymand}\ {\isasymlfloor}V{\isasymrfloor}{\isachardoublequoteclose}%
\ %
%
\isacommand{by}\isamarkupfalse%
\ blast\ %
\isamarkupcmt{S5: preorder + symmetric%
}
%
%
%
%
%
%
%
%
%
%
%
%
%
%
%
%
%
%
%
%
%
%
%
%
%
%
%
%
%
%
%
%
%
%
%
%
%
%
%
%
%
%
%
%
%
%
%
%
%
%
%
%
%
%
%
%
%
%
%
%
%
%
%
%
%
%
%
%
%
%
%
%
%
%
%
%
%
%
%
%
%
%
\end{isabellebody}%
%%% Local Variables:
%%% mode: latex
%%% TeX-master: "root"
%%% End:
