%
\begin{isabellebody}%
\setisabellecontext{FittingProof}%
%
%
%
%
%
%
%
\isamarkupsection{Fitting's Variant%
}
\isamarkuptrue%
%
\begin{isamarkuptext}%
\noindent{In this section we consider Fitting's solution to the objections raised in his discussion of G\"odel's Argument (\cite{Fitting}, pp. 164-9), 
especially the problem of modal collapse, which has been metaphysically interpreted as implying a rejection of free will.
In G\"odel's variant, positiveness and essence were thought of as predicates applying to \emph{intensional} properties and
correspondingly formalized using intensional types for their arguments (\isa{{\isasymup}{\isasymlangle}{\isasymup}{\isasymlangle}e{\isasymrangle}{\isasymrangle}} and \isa{{\isasymup}{\isasymlangle}{\isasymup}{\isasymlangle}e{\isasymrangle}{\isacharcomma}e{\isasymrangle}} respectively).
In this variant, Fitting chooses to reformulate these definitions using \emph{extensional} types (\isa{{\isasymup}{\isasymlangle}{\isasymlangle}e{\isasymrangle}{\isasymrangle}} and \isa{{\isasymup}{\isasymlangle}{\isasymlangle}e{\isasymrangle}{\isacharcomma}e{\isasymrangle}}) instead,
and makes the corresponding adjustments to the rest of the argument (to ensure type correctness).
This has some philosophical repercussions; e.g. while we could say before that honesty (as concept) was a
positive property, now we can only talk of its extension at some world and say of some group of people
that they are honest (necessarily honest, in fact, because \isa{{\isasymP}} has also been proven rigid in this variant).\footnote{
In what follows, the `\isa{{\isasymlparr}{\isacharunderscore}{\isasymrparr}}' parentheses are used to convert an extensional object into its `rigid'
intensional counterpart (e.g. \isa{{\isasymlparr}{\isasymphi}{\isasymrparr}\ {\isasymequiv}\ {\isasymlambda}w{\isachardot}\ {\isasymphi}}).}}%
\end{isamarkuptext}\isamarkuptrue%
\ \ \isacommand{consts}\isamarkupfalse%
\ Positiveness{\isacharcolon}{\isacharcolon}{\isachardoublequoteopen}{\isasymup}{\isasymlangle}{\isasymlangle}e{\isasymrangle}{\isasymrangle}{\isachardoublequoteclose}\ {\isacharparenleft}{\isachardoublequoteopen}{\isasymP}{\isachardoublequoteclose}{\isacharparenright}\isanewline
\ \ \isacommand{abbreviation}\isamarkupfalse%
\ Entails{\isacharcolon}{\isacharcolon}{\isachardoublequoteopen}{\isasymup}{\isasymlangle}{\isasymlangle}e{\isasymrangle}{\isacharcomma}{\isasymlangle}e{\isasymrangle}{\isasymrangle}{\isachardoublequoteclose}\ {\isacharparenleft}\isakeyword{infix}{\isachardoublequoteopen}{\isasymRrightarrow}{\isachardoublequoteclose}{\isacharparenright}\ \isakeyword{where}\ {\isachardoublequoteopen}X{\isasymRrightarrow}Y\ {\isasymequiv}\ \isactrlbold {\isasymbox}{\isacharparenleft}\isactrlbold {\isasymforall}\isactrlsup Az{\isachardot}\ {\isasymlparr}X\ z{\isasymrparr}\isactrlbold {\isasymrightarrow}{\isasymlparr}Y\ z{\isasymrparr}{\isacharparenright}{\isachardoublequoteclose}\isanewline
\ \ \isacommand{abbreviation}\isamarkupfalse%
\ Essence{\isacharcolon}{\isacharcolon}{\isachardoublequoteopen}{\isasymup}{\isasymlangle}{\isasymlangle}e{\isasymrangle}{\isacharcomma}e{\isasymrangle}{\isachardoublequoteclose}\ {\isacharparenleft}{\isachardoublequoteopen}{\isasymE}{\isachardoublequoteclose}{\isacharparenright}\ \isakeyword{where}\ {\isachardoublequoteopen}{\isasymE}\ Y\ x\ {\isasymequiv}\ {\isasymlparr}Y\ x{\isasymrparr}\ \isactrlbold {\isasymand}\ {\isacharparenleft}\isactrlbold {\isasymforall}Z{\isachardot}{\isasymlparr}Z\ x{\isasymrparr}\isactrlbold {\isasymrightarrow}{\isacharparenleft}Y{\isasymRrightarrow}Z{\isacharparenright}{\isacharparenright}{\isachardoublequoteclose}%
%
%
%
%
%
%
%
%
%
%
%
%
%
%
%
%
%
%
\begin{isamarkuptext}%
\noindent{Axioms and theorems remain essentially the same. Particularly (T2) \isa{{\isasymlfloor}{\isasymP}\ {\isasymdown}G{\isasymrfloor}} and (A5) \isa{{\isasymlfloor}{\isasymP}\ {\isasymdown}NE{\isasymrfloor}}
  work with \emph{relativized} extensional terms now.}%
\end{isamarkuptext}\isamarkuptrue%
\ \ \isacommand{theorem}\isamarkupfalse%
\ T{\isadigit{1}}{\isacharcolon}\ {\isachardoublequoteopen}{\isasymlfloor}\isactrlbold {\isasymforall}X{\isacharcolon}{\isacharcolon}{\isasymlangle}e{\isasymrangle}{\isachardot}\ {\isasymP}\ X\ \isactrlbold {\isasymrightarrow}\ \isactrlbold {\isasymdiamond}{\isacharparenleft}\isactrlbold {\isasymexists}\isactrlsup Az{\isachardot}\ {\isasymlparr}X\ z{\isasymrparr}{\isacharparenright}{\isasymrfloor}{\isachardoublequoteclose}%
\ %
%
\isacommand{using}\isamarkupfalse%
\ A{\isadigit{1}}a\ A{\isadigit{2}}\ \isacommand{by}\isamarkupfalse%
\ blast%
%
%
\ \isanewline
\ \ \isacommand{theorem}\isamarkupfalse%
\ T{\isadigit{3}}deRe{\isacharcolon}\ {\isachardoublequoteopen}{\isasymlfloor}{\isacharparenleft}{\isasymlambda}X{\isachardot}\ \isactrlbold {\isasymdiamond}\isactrlbold {\isasymexists}\isactrlsup A\ X{\isacharparenright}\ \isactrlbold {\isasymdown}G{\isasymrfloor}{\isachardoublequoteclose}%
\ %
%
\isacommand{using}\isamarkupfalse%
\ T{\isadigit{1}}\ T{\isadigit{2}}\ \isacommand{by}\isamarkupfalse%
\ simp%
%
%
\isanewline
\ \ \isacommand{lemma}\isamarkupfalse%
\ GodIsEssential{\isacharcolon}\ {\isachardoublequoteopen}{\isasymlfloor}\isactrlbold {\isasymforall}x{\isachardot}\ G\ x\ \isactrlbold {\isasymrightarrow}\ {\isacharparenleft}{\isacharparenleft}{\isasymE}\ {\isasymdown}\isactrlsub {\isadigit{1}}G{\isacharparenright}\ x{\isacharparenright}{\isasymrfloor}{\isachardoublequoteclose}%
\ %
%
\isacommand{using}\isamarkupfalse%
\ A{\isadigit{1}}b\ \isacommand{by}\isamarkupfalse%
\ metis%
%
%
%
\begin{isamarkuptext}%
\noindent{The following theorem could be formalized in two variants\footnote{Fitting's original
treatment in \cite{Fitting} left several details unspecified and
we had to fill in the gaps by choosing appropriate formalization variants (see \cite{J35} for details).}
(drawing on the \emph{de re/de dicto} distinction).
We prove both of them valid and show how the argument splits, culminating in two non-equivalent versions
of the conclusion, both of which are proven valid.}%
\end{isamarkuptext}\isamarkuptrue%
\ \ \isacommand{lemma}\isamarkupfalse%
\ T{\isadigit{4}}v{\isadigit{1}}{\isacharcolon}\ {\isachardoublequoteopen}{\isasymlfloor}\isactrlbold {\isasymexists}\ \isactrlbold {\isasymdown}G\ \isactrlbold {\isasymrightarrow}\ \isactrlbold {\isasymbox}\isactrlbold {\isasymexists}\isactrlsup A\ \isactrlbold {\isasymdown}G{\isasymrfloor}{\isachardoublequoteclose}%
\ %
%
\isacommand{proof}\isamarkupfalse%
\ {\isacharminus}\ %
\isamarkupcmt{not shown%
}
%
%
%
\ \ \ \ \isanewline
\ \ \isacommand{lemma}\isamarkupfalse%
\ T{\isadigit{4}}v{\isadigit{2}}{\isacharcolon}\ {\isachardoublequoteopen}{\isasymlfloor}\isactrlbold {\isasymexists}\ \isactrlbold {\isasymdown}G\ \isactrlbold {\isasymrightarrow}\ {\isacharparenleft}{\isacharparenleft}{\isasymlambda}X{\isachardot}\ \isactrlbold {\isasymbox}\isactrlbold {\isasymexists}\isactrlsup A\ X{\isacharparenright}\ \isactrlbold {\isasymdown}G{\isacharparenright}{\isasymrfloor}{\isachardoublequoteclose}%
\ %
%
\isacommand{using}\isamarkupfalse%
\ A{\isadigit{4}}a\ T{\isadigit{4}}v{\isadigit{1}}\ \isacommand{by}\isamarkupfalse%
\ metis%
%
%
%
\begin{isamarkuptext}%
\noindent{In contrast to G\"odel's version (as presented by Fitting), the following theorems can be proven in logic \emph{K}
   (the \emph{S5} axioms are no longer needed).}%
\end{isamarkuptext}\isamarkuptrue%
\ \ \isacommand{lemma}\isamarkupfalse%
\ T{\isadigit{5}}v{\isadigit{1}}{\isacharcolon}{\isachardoublequoteopen}{\isasymlfloor}\isactrlbold {\isasymdiamond}\isactrlbold {\isasymexists}\ \isactrlbold {\isasymdown}G{\isasymrfloor}{\isasymlongrightarrow}{\isasymlfloor}\isactrlbold {\isasymbox}\isactrlbold {\isasymexists}\isactrlsup A\ \isactrlbold {\isasymdown}G{\isasymrfloor}{\isachardoublequoteclose}%
\ %
%
\isacommand{using}\isamarkupfalse%
\ T{\isadigit{4}}v{\isadigit{1}}\ T{\isadigit{3}}deRe\ \isacommand{by}\isamarkupfalse%
\ metis%
%
%
\isanewline
\ \ \isacommand{lemma}\isamarkupfalse%
\ T{\isadigit{5}}v{\isadigit{2}}{\isacharcolon}{\isachardoublequoteopen}{\isasymlfloor}{\isacharparenleft}{\isasymlambda}X{\isachardot}\ \isactrlbold {\isasymdiamond}\isactrlbold {\isasymexists}\isactrlsup A\ X{\isacharparenright}\ \isactrlbold {\isasymdown}G{\isasymrfloor}\ {\isasymlongrightarrow}\ {\isasymlfloor}{\isacharparenleft}{\isasymlambda}X{\isachardot}\ \isactrlbold {\isasymbox}\isactrlbold {\isasymexists}\isactrlsup A\ X{\isacharparenright}\ \isactrlbold {\isasymdown}G{\isasymrfloor}{\isachardoublequoteclose}%
\ %
%
\isacommand{using}\isamarkupfalse%
\ T{\isadigit{4}}v{\isadigit{2}}\ \isacommand{by}\isamarkupfalse%
\ blast%
%
%
%
\begin{isamarkuptext}%
\noindent{Necessary Existence of God (\emph{de dicto} and \emph{de re} readings).}%
\end{isamarkuptext}\isamarkuptrue%
\ \ \isacommand{lemma}\isamarkupfalse%
\ GodNecExists{\isacharunderscore}deDicto{\isacharcolon}\ {\isachardoublequoteopen}{\isasymlfloor}\isactrlbold {\isasymbox}\isactrlbold {\isasymexists}\isactrlsup A\ \isactrlbold {\isasymdown}G{\isasymrfloor}{\isachardoublequoteclose}%
\ %
%
\isacommand{using}\isamarkupfalse%
\ T{\isadigit{3}}deRe\ T{\isadigit{4}}v{\isadigit{1}}\ \isacommand{by}\isamarkupfalse%
\ blast%
%
%
\isanewline
\ \ \isacommand{lemma}\isamarkupfalse%
\ GodNecExists{\isacharunderscore}deRe{\isacharcolon}\ {\isachardoublequoteopen}{\isasymlfloor}{\isacharparenleft}{\isasymlambda}X{\isachardot}\ \isactrlbold {\isasymbox}\isactrlbold {\isasymexists}\isactrlsup A\ X{\isacharparenright}\ \isactrlbold {\isasymdown}G{\isasymrfloor}{\isachardoublequoteclose}%
\ %
%
\isacommand{using}\isamarkupfalse%
\ T{\isadigit{3}}deRe\ T{\isadigit{5}}v{\isadigit{2}}\ \isacommand{by}\isamarkupfalse%
\ blast%
%
%
%
\begin{isamarkuptext}%
\noindent{Modal collapse is countersatisfiable even in \emph{S5}. Note that countermodels with a cardinality of \emph{one} 
  for the domain of individuals are found by \emph{Nitpick} (the countermodel shown in Fitting's book has cardinality of \emph{two}).}%
\end{isamarkuptext}\isamarkuptrue%
\ \ \isacommand{lemma}\isamarkupfalse%
\ {\isachardoublequoteopen}equivalence\ aRel{\isasymLongrightarrow}{\isasymlfloor}\isactrlbold {\isasymforall}{\isasymPhi}{\isachardot}\ {\isasymPhi}\isactrlbold {\isasymrightarrow}\ \isactrlbold {\isasymbox}{\isasymPhi}{\isasymrfloor}{\isachardoublequoteclose}\ \isacommand{nitpick}\isamarkupfalse%
{\isacharbrackleft}card\ e{\isacharequal}{\isadigit{1}}{\isacharcomma}\ card\ w{\isacharequal}{\isadigit{2}}{\isacharbrackright}%
\ %
%
\isacommand{oops}\isamarkupfalse%
%
%
%
%
%
%
%
%
%
%
\end{isabellebody}%
%%% Local Variables:
%%% mode: latex
%%% TeX-master: "root"
%%% End:
