%
\begin{isabellebody}%
\setisabellecontext{FittingProof}%
%
%
%
%
%
%
%
\isamarkupsection{Fitting's Variant%
}
\isamarkuptrue%
%
\begin{isamarkuptext}%
In this section we consider Fitting's solution to the objections raised in his discussion of G\"odel's Argument (\cite{Fitting}, pp. 164-9), 
especially the problem of modal collapse, which has been metaphysically interpreted as implying a rejection of free will.
In G\"odel variant, positiveness and essence were thought of as predicates applying to \emph{intensional} properties and
correspondingly formalized using intensional types for their arguments (\isa{{\isasymup}{\isasymlangle}{\isasymup}{\isasymlangle}{\isasymzero}{\isasymrangle}{\isasymrangle}} and \isa{{\isasymup}{\isasymlangle}{\isasymup}{\isasymlangle}{\isasymzero}{\isasymrangle}{\isacharcomma}{\isasymzero}{\isasymrangle}} respectively).
In this variant, Fitting chooses to reformulate these definitions using \emph{extensional} types instead (\isa{{\isasymup}{\isasymlangle}{\isasymlangle}{\isasymzero}{\isasymrangle}{\isasymrangle}} and \isa{{\isasymup}{\isasymlangle}{\isasymlangle}{\isasymzero}{\isasymrangle}{\isacharcomma}{\isasymzero}{\isasymrangle}})
and makes the corresponding adjustments to the rest of the argument (to ensure type correctness).
This has some philosophical repercusions, e.g. while we could say before that honesty (as concept) was a
positive property, we can now only talk of its extension at some world and say of some group of people
that they are honest (and in fact, that they are \emph{necessarily} so, since \isa{{\isasymP}} has also be proven rigid in this variant).%
\end{isamarkuptext}\isamarkuptrue%
\isacommand{consts}\isamarkupfalse%
\ Positiveness{\isacharcolon}{\isacharcolon}{\isachardoublequoteopen}{\isasymup}{\isasymlangle}{\isasymlangle}{\isasymzero}{\isasymrangle}{\isasymrangle}{\isachardoublequoteclose}\ {\isacharparenleft}{\isachardoublequoteopen}{\isasymP}{\isachardoublequoteclose}{\isacharparenright}\isanewline
\isacommand{abbreviation}\isamarkupfalse%
\ Entailment{\isacharcolon}{\isacharcolon}{\isachardoublequoteopen}{\isasymup}{\isasymlangle}{\isasymlangle}{\isasymzero}{\isasymrangle}{\isacharcomma}{\isasymlangle}{\isasymzero}{\isasymrangle}{\isasymrangle}{\isachardoublequoteclose}\ {\isacharparenleft}\isakeyword{infix}{\isachardoublequoteopen}{\isasymRrightarrow}{\isachardoublequoteclose}{\isadigit{6}}{\isadigit{0}}{\isacharparenright}\isanewline
\ \ \isakeyword{where}\ {\isachardoublequoteopen}X\ {\isasymRrightarrow}\ Y\ {\isasymequiv}\ \isactrlbold {\isasymbox}{\isacharparenleft}\isactrlbold {\isasymforall}\isactrlsup Ez{\isachardot}\ {\isasymlparr}X\ z{\isasymrparr}\ \isactrlbold {\isasymrightarrow}\ {\isasymlparr}Y\ z{\isasymrparr}{\isacharparenright}{\isachardoublequoteclose}\ \ \isanewline
\isacommand{abbreviation}\isamarkupfalse%
\ essenceOf{\isacharcolon}{\isacharcolon}{\isachardoublequoteopen}{\isasymup}{\isasymlangle}{\isasymlangle}{\isasymzero}{\isasymrangle}{\isacharcomma}{\isasymzero}{\isasymrangle}{\isachardoublequoteclose}\ {\isacharparenleft}{\isachardoublequoteopen}{\isasymE}{\isachardoublequoteclose}{\isacharparenright}\ \isakeyword{where}\isanewline
\ \ \ \ {\isachardoublequoteopen}{\isasymE}\ Y\ x\ {\isasymequiv}\ {\isasymlparr}Y\ x{\isasymrparr}\ \isactrlbold {\isasymand}\ {\isacharparenleft}\isactrlbold {\isasymforall}Z{\isacharcolon}{\isacharcolon}{\isasymlangle}{\isasymzero}{\isasymrangle}{\isachardot}\ {\isasymlparr}Z\ x{\isasymrparr}\ \isactrlbold {\isasymrightarrow}\ Y\ {\isasymRrightarrow}\ Z{\isacharparenright}{\isachardoublequoteclose}%
%
%
%
%
%
%
%
%
%
%
%
%
%
%
%
%
%
%
\begin{isamarkuptext}%
Axioms and theorems remain essentially the same. Particularly (T2) \isa{{\isasymlfloor}{\isasymP}\ {\isasymdown}G{\isasymrfloor}} and (A5) \isa{{\isasymlfloor}{\isasymP}\ {\isasymdown}NE{\isasymrfloor}}
work with \emph{relativized} extensional terms now. Fitting's original treatment in \cite{Fitting} left several
details unspecified and we had to fill in the gaps by choosing appropriate formalization variants (see \cite{J35} for details).%
\end{isamarkuptext}\isamarkuptrue%
\isacommand{theorem}\isamarkupfalse%
\ T{\isadigit{1}}{\isacharcolon}\ {\isachardoublequoteopen}{\isasymlfloor}\isactrlbold {\isasymforall}X{\isacharcolon}{\isacharcolon}{\isasymlangle}{\isasymzero}{\isasymrangle}{\isachardot}\ {\isasymP}\ X\ \isactrlbold {\isasymrightarrow}\ \isactrlbold {\isasymdiamond}{\isacharparenleft}\isactrlbold {\isasymexists}\isactrlsup Ez{\isachardot}\ {\isasymlparr}X\ z{\isasymrparr}{\isacharparenright}{\isasymrfloor}{\isachardoublequoteclose}%
\ %
%
\isacommand{using}\isamarkupfalse%
\ A{\isadigit{1}}a\ A{\isadigit{2}}\ \isacommand{by}\isamarkupfalse%
\ blast%
%
%
\ \isanewline
\isacommand{theorem}\isamarkupfalse%
\ T{\isadigit{3}}{\isacharcolon}\ {\isachardoublequoteopen}{\isasymlfloor}{\isacharparenleft}{\isasymlambda}X{\isachardot}\ \isactrlbold {\isasymdiamond}\isactrlbold {\isasymexists}\isactrlsup E\ X{\isacharparenright}\ \isactrlbold {\isasymdown}G{\isasymrfloor}{\isachardoublequoteclose}%
\ %
%
\isacommand{using}\isamarkupfalse%
\ T{\isadigit{1}}\ T{\isadigit{2}}\ \isacommand{by}\isamarkupfalse%
\ simp\ %
\isamarkupcmt{\emph{de re} variant chosen%
}
%
%
%
\ \isanewline
\isacommand{lemma}\isamarkupfalse%
\ GodIsEssential{\isacharcolon}\ {\isachardoublequoteopen}{\isasymlfloor}\isactrlbold {\isasymforall}x{\isachardot}\ G\ x\ \isactrlbold {\isasymrightarrow}\ {\isacharparenleft}{\isacharparenleft}{\isasymE}\ {\isasymdown}\isactrlsub {\isadigit{1}}G{\isacharparenright}\ x{\isacharparenright}{\isasymrfloor}{\isachardoublequoteclose}%
\ %
%
\isacommand{using}\isamarkupfalse%
\ A{\isadigit{1}}b\ \isacommand{by}\isamarkupfalse%
\ metis%
%
%
%
\begin{isamarkuptext}%
(Possibilist) existence of God implies necessary (actualist) existence.
This theorem could be formalized in two variants (drawing on the \emph{de re/de dicto} distinction).
We prove both of them valid and show how the argument splits in two, culminating in two non-equivalent versions
of the conclusion (one \emph{de re} and the other \emph{de dicto}), both of which are proven valid.%
\end{isamarkuptext}\isamarkuptrue%
\isacommand{lemma}\isamarkupfalse%
\ GodExImpNecEx{\isadigit{1}}{\isacharcolon}\ {\isachardoublequoteopen}{\isasymlfloor}\isactrlbold {\isasymexists}\ \isactrlbold {\isasymdown}G\ \isactrlbold {\isasymrightarrow}\ \isactrlbold {\isasymbox}\isactrlbold {\isasymexists}\isactrlsup E\ \isactrlbold {\isasymdown}G{\isasymrfloor}{\isachardoublequoteclose}%
\ %
%
\isacommand{proof}\isamarkupfalse%
\ {\isacharminus}\ %
\isamarkupcmt{not shown%
}
%
%
%
\ \ \isanewline
\isacommand{lemma}\isamarkupfalse%
\ GodExImpNecEx{\isadigit{2}}{\isacharcolon}\ {\isachardoublequoteopen}{\isasymlfloor}\isactrlbold {\isasymexists}\ \isactrlbold {\isasymdown}G\ \isactrlbold {\isasymrightarrow}\ {\isacharparenleft}{\isacharparenleft}{\isasymlambda}X{\isachardot}\ \isactrlbold {\isasymbox}\isactrlbold {\isasymexists}\isactrlsup E\ X{\isacharparenright}\ \isactrlbold {\isasymdown}G{\isacharparenright}{\isasymrfloor}{\isachardoublequoteclose}\isanewline
%
\ \ %
%
\isacommand{using}\isamarkupfalse%
\ A{\isadigit{4}}a\ GodExImpNecEx{\isadigit{1}}\ \isacommand{by}\isamarkupfalse%
\ metis%
%
%
%
\begin{isamarkuptext}%
In contrast to G\"odel's argument (as presented by Fitting), the following theorems can be proven in logic \emph{K}
 (the \emph{S5} axioms are no longer needed):%
\end{isamarkuptext}\isamarkuptrue%
\isacommand{lemma}\isamarkupfalse%
\ T{\isadigit{4}}v{\isadigit{1}}{\isacharcolon}{\isachardoublequoteopen}{\isasymlfloor}\isactrlbold {\isasymdiamond}\isactrlbold {\isasymexists}\ \isactrlbold {\isasymdown}G{\isasymrfloor}{\isasymlongrightarrow}{\isasymlfloor}\isactrlbold {\isasymbox}\isactrlbold {\isasymexists}\isactrlsup E\ \isactrlbold {\isasymdown}G{\isasymrfloor}{\isachardoublequoteclose}%
\ %
%
\isacommand{using}\isamarkupfalse%
\ GodExImpNecEx{\isadigit{1}}\ T{\isadigit{3}}\ \isacommand{by}\isamarkupfalse%
\ metis%
%
%
\isanewline
\isacommand{lemma}\isamarkupfalse%
\ T{\isadigit{4}}v{\isadigit{2}}{\isacharcolon}{\isachardoublequoteopen}{\isasymlfloor}{\isacharparenleft}{\isasymlambda}X{\isachardot}\ \isactrlbold {\isasymdiamond}\isactrlbold {\isasymexists}\isactrlsup E\ X{\isacharparenright}\ \isactrlbold {\isasymdown}G{\isasymrfloor}\ {\isasymlongrightarrow}\ {\isasymlfloor}{\isacharparenleft}{\isasymlambda}X{\isachardot}\ \isactrlbold {\isasymbox}\isactrlbold {\isasymexists}\isactrlsup E\ X{\isacharparenright}\ \isactrlbold {\isasymdown}G{\isasymrfloor}{\isachardoublequoteclose}\isanewline
%
\ \ %
%
\isacommand{using}\isamarkupfalse%
\ GodExImpNecEx{\isadigit{2}}\ \isacommand{by}\isamarkupfalse%
\ blast%
%
%
%
\begin{isamarkuptext}%
Necessary Existence of God (\emph{de dicto} and \emph{de re} readings).%
\end{isamarkuptext}\isamarkuptrue%
\isacommand{lemma}\isamarkupfalse%
\ GodNecExists{\isacharunderscore}deDicto{\isacharcolon}\ {\isachardoublequoteopen}{\isasymlfloor}\isactrlbold {\isasymbox}\isactrlbold {\isasymexists}\isactrlsup E\ \isactrlbold {\isasymdown}G{\isasymrfloor}{\isachardoublequoteclose}%
\ %
%
\isacommand{using}\isamarkupfalse%
\ GodExImpNecEx{\isadigit{1}}\ T{\isadigit{3}}\ \isacommand{by}\isamarkupfalse%
\ force%
%
%
\isanewline
\isacommand{lemma}\isamarkupfalse%
\ GodNecExists{\isacharunderscore}deRe{\isacharcolon}\ {\isachardoublequoteopen}{\isasymlfloor}{\isacharparenleft}{\isasymlambda}X{\isachardot}\ \isactrlbold {\isasymbox}\isactrlbold {\isasymexists}\isactrlsup E\ X{\isacharparenright}\ \isactrlbold {\isasymdown}G{\isasymrfloor}{\isachardoublequoteclose}%
\ %
%
\isacommand{using}\isamarkupfalse%
\ T{\isadigit{3}}\ T{\isadigit{4}}v{\isadigit{2}}\ \isacommand{by}\isamarkupfalse%
\ blast%
%
%
%
\begin{isamarkuptext}%
Modal collapse is countersatisfiable even in \emph{S5}. Note that countermodels with a cardinality of \emph{one} 
for the domain of individuals are found by \emph{Nitpick} (the countermodel shown in Fitting's book has cardinality of \emph{two}).%
\end{isamarkuptext}\isamarkuptrue%
\isacommand{lemma}\isamarkupfalse%
\ {\isachardoublequoteopen}equivalence\ aRel\ {\isasymLongrightarrow}\ {\isasymlfloor}\isactrlbold {\isasymforall}{\isasymPhi}{\isachardot}{\isacharparenleft}{\isasymPhi}\ \isactrlbold {\isasymrightarrow}\ {\isacharparenleft}\isactrlbold {\isasymbox}\ {\isasymPhi}{\isacharparenright}{\isacharparenright}{\isasymrfloor}{\isachardoublequoteclose}\ \isacommand{nitpick}\isamarkupfalse%
{\isacharbrackleft}card\ {\isacharprime}t{\isacharequal}{\isadigit{1}}{\isacharcomma}\ card\ i{\isacharequal}{\isadigit{2}}{\isacharbrackright}%
\ %
%
\isacommand{oops}\isamarkupfalse%
\ %
%
%
%
%
%
%
%
%
%
\end{isabellebody}%
%%% Local Variables:
%%% mode: latex
%%% TeX-master: "root"
%%% End:
