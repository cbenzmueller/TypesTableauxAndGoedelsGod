%
\begin{isabellebody}%
\setisabellecontext{IHOML}%
%
\isadelimtheory
%
\endisadelimtheory
%
\isatagtheory
%
\endisatagtheory
{\isafoldtheory}%
%
\isadelimtheory
%
\endisadelimtheory
%
\isamarkupsection{Introduction%
}
\isamarkuptrue%
%
\begin{isamarkuptext}%
We present a study on Computational Metaphysics: a computer-formalisation and verification
of Fitting's variant of the ontological argument (for the existence of God) as presented in
his textbook \emph{Types, Tableaus and G\"odel's God} \cite{Fitting}. Fitting's argument 
is an emendation of Kurt G\"odel's modern variant \cite{GoedelNotes} (resp. Dana Scott's 
variant \cite{ScottNotes}) of the ontological argument.%
\end{isamarkuptext}\isamarkuptrue%
%
\begin{isamarkuptext}%
The motivation is to avoid the \emph{modal collapse} \cite{Sobel,sobel2004logic}, which has been criticised
as an undesirable side-effect of the axioms of G\"odel resp. Scott. The modal collapse essentially  
states that  there are no contingent truths and that everything is determined.
Several authors (e.g. \cite{anderson90:_some_emend_of_goedel_ontol_proof,AndersonGettings,Hajek2002,bjordal99}) 
have proposed emendations of the argument with the aim of maintaining the essential result 
(the necessary existence of God) while at the same time avoiding the modal collapse. 
Related work  has formalised several of these variants on the computer and verified or falsified them. For example,
G\"odel's axioms \cite{GoedelNotes} have been shown inconsistent \cite{C55,C60}
while Scott's version has been verified \cite{ECAI}. Further experiments, contributing amongst others
to the clarification of a related debate between H\'ajek and Anderson, are presented and discussed in
\cite{J23}. The enabling technique in all of these experiments has been
shallow semantical embeddings of (extensional) higher-order modal logics in classical higher-order
logic (see \cite{J23,R59} and the references therein).%
\end{isamarkuptext}\isamarkuptrue%
%
\begin{isamarkuptext}%
Fitting's emendation also intends to avoid the modal collapse. However, in contrast to the above variants, Fitting's
solution is based on the use of an intensional as opposed to an extensional higher-order modal logic.
For our work this imposed the additional challenge to provide a shallow embedding of this more advanced
logic. The experiments presented below confirm that Fitting's argument as presented in his textbook \cite{Fitting}
is valid and that it avoids the modal collapse as intended.%
\end{isamarkuptext}\isamarkuptrue%
%
\begin{isamarkuptext}%
The work presented here originates from the \emph{Computational Metaphysics} lecture course  
held at FU Berlin in Summer 2016 \cite{C65}. \pagebreak%
\end{isamarkuptext}\isamarkuptrue%
%
\isamarkupsection{Embedding of Intensional Higher-Order Modal Logic%
}
\isamarkuptrue%
%
\begin{isamarkuptext}%
The object logic being embedded (IHOML) is a modification of the intentional logic developed by Montague
and Gallin (see \cite{Gallin75}). IHOML is introduced by Fitting in the second part of the book \cite{Fitting}
in order to formalise his emendation of G\"odel's ontological argument. We offer here a shallow embedding
of this logic in Isabelle/HOL, which has been inspired by previous work on the semantical embedding of
multimodal logics with quantification \cite{J23}. We expand this approach to allow for actualist quantifiers,
intensional types and their related operations.%
\end{isamarkuptext}\isamarkuptrue%
%
\isamarkupsubsection{Type Declarations%
}
\isamarkuptrue%
%
\begin{isamarkuptext}%
Since IHOML and Isabelle/HOL are both typed languages, we introduce a type-mapping between them
by following as closely as possible the syntax given by Fitting (see p. 86).%
\end{isamarkuptext}\isamarkuptrue%
\ \ \isacommand{typedecl}\isamarkupfalse%
\ i\ \ \ \ \ \ \ \ \ \ \ \ \ \ \ \ \ \ \ \ %
\isamarkupcmt{type for possible worlds%
}
\isanewline
\ \ \isacommand{type{\isacharunderscore}synonym}\isamarkupfalse%
\ io\ {\isacharequal}\ {\isachardoublequoteopen}{\isacharparenleft}i{\isasymRightarrow}bool{\isacharparenright}{\isachardoublequoteclose}\ %
\isamarkupcmt{formulas with world-dependent truth-value%
}
\isanewline
\ \ \isacommand{typedecl}\isamarkupfalse%
\ e\ \ {\isacharparenleft}{\isachardoublequoteopen}{\isasymzero}{\isachardoublequoteclose}{\isacharparenright}\ \ \ \ \ \ \ \ \ \ \ \ \ %
\isamarkupcmt{individuals%
}
%
\begin{isamarkuptext}%
Aliases for common unary predicate types:%
\end{isamarkuptext}\isamarkuptrue%
\ \ \isacommand{type{\isacharunderscore}synonym}\isamarkupfalse%
\ ie\ {\isacharequal}\ \ \ \ \ {\isachardoublequoteopen}{\isacharparenleft}i{\isasymRightarrow}{\isasymzero}{\isacharparenright}{\isachardoublequoteclose}\ \ \ \ \ \ \ \ \ \ \ \ \ {\isacharparenleft}{\isachardoublequoteopen}{\isasymup}{\isasymzero}{\isachardoublequoteclose}{\isacharparenright}\isanewline
\ \ \isacommand{type{\isacharunderscore}synonym}\isamarkupfalse%
\ se\ {\isacharequal}\ \ \ \ \ {\isachardoublequoteopen}{\isacharparenleft}{\isasymzero}{\isasymRightarrow}bool{\isacharparenright}{\isachardoublequoteclose}\ \ \ \ \ \ \ \ \ \ {\isacharparenleft}{\isachardoublequoteopen}{\isasymlangle}{\isasymzero}{\isasymrangle}{\isachardoublequoteclose}{\isacharparenright}\isanewline
\ \ \isacommand{type{\isacharunderscore}synonym}\isamarkupfalse%
\ ise\ {\isacharequal}\ \ \ \ {\isachardoublequoteopen}{\isacharparenleft}{\isasymzero}{\isasymRightarrow}io{\isacharparenright}{\isachardoublequoteclose}\ \ \ \ \ \ \ \ \ \ \ {\isacharparenleft}{\isachardoublequoteopen}{\isasymup}{\isasymlangle}{\isasymzero}{\isasymrangle}{\isachardoublequoteclose}{\isacharparenright}\isanewline
\ \ \isacommand{type{\isacharunderscore}synonym}\isamarkupfalse%
\ sie\ {\isacharequal}\ \ \ \ {\isachardoublequoteopen}{\isacharparenleft}{\isasymup}{\isasymzero}{\isasymRightarrow}bool{\isacharparenright}{\isachardoublequoteclose}\ \ \ \ \ \ \ \ {\isacharparenleft}{\isachardoublequoteopen}{\isasymlangle}{\isasymup}{\isasymzero}{\isasymrangle}{\isachardoublequoteclose}{\isacharparenright}\isanewline
\ \ \isacommand{type{\isacharunderscore}synonym}\isamarkupfalse%
\ isie\ {\isacharequal}\ \ \ {\isachardoublequoteopen}{\isacharparenleft}{\isasymup}{\isasymzero}{\isasymRightarrow}io{\isacharparenright}{\isachardoublequoteclose}\ \ \ \ \ \ \ \ \ {\isacharparenleft}{\isachardoublequoteopen}{\isasymup}{\isasymlangle}{\isasymup}{\isasymzero}{\isasymrangle}{\isachardoublequoteclose}{\isacharparenright}\ \ \isanewline
\ \ \isacommand{type{\isacharunderscore}synonym}\isamarkupfalse%
\ sise\ {\isacharequal}\ \ \ {\isachardoublequoteopen}{\isacharparenleft}{\isasymup}{\isasymlangle}{\isasymzero}{\isasymrangle}{\isasymRightarrow}bool{\isacharparenright}{\isachardoublequoteclose}\ \ \ \ \ {\isacharparenleft}{\isachardoublequoteopen}{\isasymlangle}{\isasymup}{\isasymlangle}{\isasymzero}{\isasymrangle}{\isasymrangle}{\isachardoublequoteclose}{\isacharparenright}\isanewline
\ \ \isacommand{type{\isacharunderscore}synonym}\isamarkupfalse%
\ isise\ {\isacharequal}\ \ {\isachardoublequoteopen}{\isacharparenleft}{\isasymup}{\isasymlangle}{\isasymzero}{\isasymrangle}{\isasymRightarrow}io{\isacharparenright}{\isachardoublequoteclose}\ \ \ \ \ \ {\isacharparenleft}{\isachardoublequoteopen}{\isasymup}{\isasymlangle}{\isasymup}{\isasymlangle}{\isasymzero}{\isasymrangle}{\isasymrangle}{\isachardoublequoteclose}{\isacharparenright}\isanewline
\ \ \isacommand{type{\isacharunderscore}synonym}\isamarkupfalse%
\ sisise{\isacharequal}\ \ {\isachardoublequoteopen}{\isacharparenleft}{\isasymup}{\isasymlangle}{\isasymup}{\isasymlangle}{\isasymzero}{\isasymrangle}{\isasymrangle}{\isasymRightarrow}bool{\isacharparenright}{\isachardoublequoteclose}\ {\isacharparenleft}{\isachardoublequoteopen}{\isasymlangle}{\isasymup}{\isasymlangle}{\isasymup}{\isasymlangle}{\isasymzero}{\isasymrangle}{\isasymrangle}{\isasymrangle}{\isachardoublequoteclose}{\isacharparenright}\isanewline
\ \ \isacommand{type{\isacharunderscore}synonym}\isamarkupfalse%
\ isisise{\isacharequal}\ {\isachardoublequoteopen}{\isacharparenleft}{\isasymup}{\isasymlangle}{\isasymup}{\isasymlangle}{\isasymzero}{\isasymrangle}{\isasymrangle}{\isasymRightarrow}io{\isacharparenright}{\isachardoublequoteclose}\ \ {\isacharparenleft}{\isachardoublequoteopen}{\isasymup}{\isasymlangle}{\isasymup}{\isasymlangle}{\isasymup}{\isasymlangle}{\isasymzero}{\isasymrangle}{\isasymrangle}{\isasymrangle}{\isachardoublequoteclose}{\isacharparenright}\isanewline
\ \ \isacommand{type{\isacharunderscore}synonym}\isamarkupfalse%
\ sse\ {\isacharequal}\ \ \ \ {\isachardoublequoteopen}{\isasymlangle}{\isasymzero}{\isasymrangle}{\isasymRightarrow}bool{\isachardoublequoteclose}\ \ \ \ \ \ \ \ \ {\isacharparenleft}{\isachardoublequoteopen}{\isasymlangle}{\isasymlangle}{\isasymzero}{\isasymrangle}{\isasymrangle}{\isachardoublequoteclose}{\isacharparenright}\isanewline
\ \ \isacommand{type{\isacharunderscore}synonym}\isamarkupfalse%
\ isse\ {\isacharequal}\ \ \ {\isachardoublequoteopen}{\isasymlangle}{\isasymzero}{\isasymrangle}{\isasymRightarrow}io{\isachardoublequoteclose}\ \ \ \ \ \ \ \ \ \ {\isacharparenleft}{\isachardoublequoteopen}{\isasymup}{\isasymlangle}{\isasymlangle}{\isasymzero}{\isasymrangle}{\isasymrangle}{\isachardoublequoteclose}{\isacharparenright}%
\begin{isamarkuptext}%
Aliases for common binary relation types:%
\end{isamarkuptext}\isamarkuptrue%
\ \ \isacommand{type{\isacharunderscore}synonym}\isamarkupfalse%
\ see\ {\isacharequal}\ \ \ \ \ \ \ \ {\isachardoublequoteopen}{\isacharparenleft}{\isasymzero}{\isasymRightarrow}{\isasymzero}{\isasymRightarrow}bool{\isacharparenright}{\isachardoublequoteclose}\ \ \ \ \ \ \ \ \ \ {\isacharparenleft}{\isachardoublequoteopen}{\isasymlangle}{\isasymzero}{\isacharcomma}{\isasymzero}{\isasymrangle}{\isachardoublequoteclose}{\isacharparenright}\isanewline
\ \ \isacommand{type{\isacharunderscore}synonym}\isamarkupfalse%
\ isee\ {\isacharequal}\ \ \ \ \ \ \ {\isachardoublequoteopen}{\isacharparenleft}{\isasymzero}{\isasymRightarrow}{\isasymzero}{\isasymRightarrow}io{\isacharparenright}{\isachardoublequoteclose}\ \ \ \ \ \ \ \ \ \ \ {\isacharparenleft}{\isachardoublequoteopen}{\isasymup}{\isasymlangle}{\isasymzero}{\isacharcomma}{\isasymzero}{\isasymrangle}{\isachardoublequoteclose}{\isacharparenright}\isanewline
\ \ \isacommand{type{\isacharunderscore}synonym}\isamarkupfalse%
\ sieie\ {\isacharequal}\ \ \ \ \ \ {\isachardoublequoteopen}{\isacharparenleft}{\isasymup}{\isasymzero}{\isasymRightarrow}{\isasymup}{\isasymzero}{\isasymRightarrow}bool{\isacharparenright}{\isachardoublequoteclose}\ \ \ \ \ \ \ {\isacharparenleft}{\isachardoublequoteopen}{\isasymlangle}{\isasymup}{\isasymzero}{\isacharcomma}{\isasymup}{\isasymzero}{\isasymrangle}{\isachardoublequoteclose}{\isacharparenright}\isanewline
\ \ \isacommand{type{\isacharunderscore}synonym}\isamarkupfalse%
\ isieie\ {\isacharequal}\ \ \ \ \ {\isachardoublequoteopen}{\isacharparenleft}{\isasymup}{\isasymzero}{\isasymRightarrow}{\isasymup}{\isasymzero}{\isasymRightarrow}io{\isacharparenright}{\isachardoublequoteclose}\ \ \ \ \ \ \ \ {\isacharparenleft}{\isachardoublequoteopen}{\isasymup}{\isasymlangle}{\isasymup}{\isasymzero}{\isacharcomma}{\isasymup}{\isasymzero}{\isasymrangle}{\isachardoublequoteclose}{\isacharparenright}\isanewline
\ \ \isacommand{type{\isacharunderscore}synonym}\isamarkupfalse%
\ ssese\ {\isacharequal}\ \ \ \ \ \ {\isachardoublequoteopen}{\isacharparenleft}{\isasymlangle}{\isasymzero}{\isasymrangle}{\isasymRightarrow}{\isasymlangle}{\isasymzero}{\isasymrangle}{\isasymRightarrow}bool{\isacharparenright}{\isachardoublequoteclose}\ \ \ \ \ {\isacharparenleft}{\isachardoublequoteopen}{\isasymlangle}{\isasymlangle}{\isasymzero}{\isasymrangle}{\isacharcomma}{\isasymlangle}{\isasymzero}{\isasymrangle}{\isasymrangle}{\isachardoublequoteclose}{\isacharparenright}\isanewline
\ \ \isacommand{type{\isacharunderscore}synonym}\isamarkupfalse%
\ issese\ {\isacharequal}\ \ \ \ \ {\isachardoublequoteopen}{\isacharparenleft}{\isasymlangle}{\isasymzero}{\isasymrangle}{\isasymRightarrow}{\isasymlangle}{\isasymzero}{\isasymrangle}{\isasymRightarrow}io{\isacharparenright}{\isachardoublequoteclose}\ \ \ \ \ \ {\isacharparenleft}{\isachardoublequoteopen}{\isasymup}{\isasymlangle}{\isasymlangle}{\isasymzero}{\isasymrangle}{\isacharcomma}{\isasymlangle}{\isasymzero}{\isasymrangle}{\isasymrangle}{\isachardoublequoteclose}{\isacharparenright}\isanewline
\ \ \isacommand{type{\isacharunderscore}synonym}\isamarkupfalse%
\ ssee\ {\isacharequal}\ \ \ \ \ \ \ {\isachardoublequoteopen}{\isacharparenleft}{\isasymlangle}{\isasymzero}{\isasymrangle}{\isasymRightarrow}{\isasymzero}{\isasymRightarrow}bool{\isacharparenright}{\isachardoublequoteclose}\ \ \ \ \ \ \ {\isacharparenleft}{\isachardoublequoteopen}{\isasymlangle}{\isasymlangle}{\isasymzero}{\isasymrangle}{\isacharcomma}{\isasymzero}{\isasymrangle}{\isachardoublequoteclose}{\isacharparenright}\isanewline
\ \ \isacommand{type{\isacharunderscore}synonym}\isamarkupfalse%
\ issee\ {\isacharequal}\ \ \ \ \ \ {\isachardoublequoteopen}{\isacharparenleft}{\isasymlangle}{\isasymzero}{\isasymrangle}{\isasymRightarrow}{\isasymzero}{\isasymRightarrow}io{\isacharparenright}{\isachardoublequoteclose}\ \ \ \ \ \ \ \ {\isacharparenleft}{\isachardoublequoteopen}{\isasymup}{\isasymlangle}{\isasymlangle}{\isasymzero}{\isasymrangle}{\isacharcomma}{\isasymzero}{\isasymrangle}{\isachardoublequoteclose}{\isacharparenright}\isanewline
\ \ \isacommand{type{\isacharunderscore}synonym}\isamarkupfalse%
\ isisee\ {\isacharequal}\ \ \ \ \ {\isachardoublequoteopen}{\isacharparenleft}{\isasymup}{\isasymlangle}{\isasymzero}{\isasymrangle}{\isasymRightarrow}{\isasymzero}{\isasymRightarrow}io{\isacharparenright}{\isachardoublequoteclose}\ \ \ \ \ \ {\isacharparenleft}{\isachardoublequoteopen}{\isasymup}{\isasymlangle}{\isasymup}{\isasymlangle}{\isasymzero}{\isasymrangle}{\isacharcomma}{\isasymzero}{\isasymrangle}{\isachardoublequoteclose}{\isacharparenright}\isanewline
\ \ \isacommand{type{\isacharunderscore}synonym}\isamarkupfalse%
\ isiseise\ {\isacharequal}\ \ \ {\isachardoublequoteopen}{\isacharparenleft}{\isasymup}{\isasymlangle}{\isasymzero}{\isasymrangle}{\isasymRightarrow}{\isasymup}{\isasymlangle}{\isasymzero}{\isasymrangle}{\isasymRightarrow}io{\isacharparenright}{\isachardoublequoteclose}\ \ \ \ {\isacharparenleft}{\isachardoublequoteopen}{\isasymup}{\isasymlangle}{\isasymup}{\isasymlangle}{\isasymzero}{\isasymrangle}{\isacharcomma}{\isasymup}{\isasymlangle}{\isasymzero}{\isasymrangle}{\isasymrangle}{\isachardoublequoteclose}{\isacharparenright}\isanewline
\ \ \isacommand{type{\isacharunderscore}synonym}\isamarkupfalse%
\ isiseisise{\isacharequal}\ \ {\isachardoublequoteopen}{\isacharparenleft}{\isasymup}{\isasymlangle}{\isasymzero}{\isasymrangle}{\isasymRightarrow}{\isasymup}{\isasymlangle}{\isasymup}{\isasymlangle}{\isasymzero}{\isasymrangle}{\isasymrangle}{\isasymRightarrow}io{\isacharparenright}{\isachardoublequoteclose}\ {\isacharparenleft}{\isachardoublequoteopen}{\isasymup}{\isasymlangle}{\isasymup}{\isasymlangle}{\isasymzero}{\isasymrangle}{\isacharcomma}{\isasymup}{\isasymlangle}{\isasymup}{\isasymlangle}{\isasymzero}{\isasymrangle}{\isasymrangle}{\isasymrangle}{\isachardoublequoteclose}{\isacharparenright}%
\isamarkupsubsection{Definitions%
}
\isamarkuptrue%
%
\isamarkupsubsubsection{Logical Operators as Truth-Sets%
}
\isamarkuptrue%
\ \ \isacommand{abbreviation}\isamarkupfalse%
\ mnot\ \ \ {\isacharcolon}{\isacharcolon}\ {\isachardoublequoteopen}io{\isasymRightarrow}io{\isachardoublequoteclose}\ {\isacharparenleft}{\isachardoublequoteopen}\isactrlbold {\isasymnot}{\isacharunderscore}{\isachardoublequoteclose}{\isacharbrackleft}{\isadigit{5}}{\isadigit{2}}{\isacharbrackright}{\isadigit{5}}{\isadigit{3}}{\isacharparenright}\isanewline
\ \ \ \ \isakeyword{where}\ {\isachardoublequoteopen}\isactrlbold {\isasymnot}{\isasymphi}\ {\isasymequiv}\ {\isasymlambda}w{\isachardot}\ {\isasymnot}{\isacharparenleft}{\isasymphi}\ w{\isacharparenright}{\isachardoublequoteclose}\isanewline
\ \ \isacommand{abbreviation}\isamarkupfalse%
\ negpred\ {\isacharcolon}{\isacharcolon}\ {\isachardoublequoteopen}{\isasymlangle}{\isasymzero}{\isasymrangle}{\isasymRightarrow}{\isasymlangle}{\isasymzero}{\isasymrangle}{\isachardoublequoteclose}\ {\isacharparenleft}{\isachardoublequoteopen}{\isasymrightharpoondown}{\isacharunderscore}{\isachardoublequoteclose}{\isacharbrackleft}{\isadigit{5}}{\isadigit{2}}{\isacharbrackright}{\isadigit{5}}{\isadigit{3}}{\isacharparenright}\ \isanewline
\ \ \ \ \isakeyword{where}\ {\isachardoublequoteopen}{\isasymrightharpoondown}{\isasymPhi}\ {\isasymequiv}\ {\isasymlambda}x{\isachardot}\ {\isasymnot}{\isacharparenleft}{\isasymPhi}\ x{\isacharparenright}{\isachardoublequoteclose}\ \isanewline
\ \ \isacommand{abbreviation}\isamarkupfalse%
\ mnegpred\ {\isacharcolon}{\isacharcolon}\ {\isachardoublequoteopen}{\isasymup}{\isasymlangle}{\isasymzero}{\isasymrangle}{\isasymRightarrow}{\isasymup}{\isasymlangle}{\isasymzero}{\isasymrangle}{\isachardoublequoteclose}\ {\isacharparenleft}{\isachardoublequoteopen}\isactrlbold {\isasymrightharpoondown}{\isacharunderscore}{\isachardoublequoteclose}{\isacharbrackleft}{\isadigit{5}}{\isadigit{2}}{\isacharbrackright}{\isadigit{5}}{\isadigit{3}}{\isacharparenright}\ \isanewline
\ \ \ \ \isakeyword{where}\ {\isachardoublequoteopen}\isactrlbold {\isasymrightharpoondown}{\isasymPhi}\ {\isasymequiv}\ {\isasymlambda}x{\isachardot}{\isasymlambda}w{\isachardot}\ {\isasymnot}{\isacharparenleft}{\isasymPhi}\ x\ w{\isacharparenright}{\isachardoublequoteclose}\isanewline
\ \ \isacommand{abbreviation}\isamarkupfalse%
\ mand\ \ \ {\isacharcolon}{\isacharcolon}\ {\isachardoublequoteopen}io{\isasymRightarrow}io{\isasymRightarrow}io{\isachardoublequoteclose}\ {\isacharparenleft}\isakeyword{infixr}{\isachardoublequoteopen}\isactrlbold {\isasymand}{\isachardoublequoteclose}{\isadigit{5}}{\isadigit{1}}{\isacharparenright}\isanewline
\ \ \ \ \isakeyword{where}\ {\isachardoublequoteopen}{\isasymphi}\isactrlbold {\isasymand}{\isasympsi}\ {\isasymequiv}\ {\isasymlambda}w{\isachardot}\ {\isacharparenleft}{\isasymphi}\ w{\isacharparenright}{\isasymand}{\isacharparenleft}{\isasympsi}\ w{\isacharparenright}{\isachardoublequoteclose}\ \ \ \isanewline
\ \ \isacommand{abbreviation}\isamarkupfalse%
\ mor\ \ \ \ {\isacharcolon}{\isacharcolon}\ {\isachardoublequoteopen}io{\isasymRightarrow}io{\isasymRightarrow}io{\isachardoublequoteclose}\ {\isacharparenleft}\isakeyword{infixr}{\isachardoublequoteopen}\isactrlbold {\isasymor}{\isachardoublequoteclose}{\isadigit{5}}{\isadigit{0}}{\isacharparenright}\isanewline
\ \ \ \ \isakeyword{where}\ {\isachardoublequoteopen}{\isasymphi}\isactrlbold {\isasymor}{\isasympsi}\ {\isasymequiv}\ {\isasymlambda}w{\isachardot}\ {\isacharparenleft}{\isasymphi}\ w{\isacharparenright}{\isasymor}{\isacharparenleft}{\isasympsi}\ w{\isacharparenright}{\isachardoublequoteclose}\isanewline
\ \ \isacommand{abbreviation}\isamarkupfalse%
\ mimp\ \ \ {\isacharcolon}{\isacharcolon}\ {\isachardoublequoteopen}io{\isasymRightarrow}io{\isasymRightarrow}io{\isachardoublequoteclose}\ {\isacharparenleft}\isakeyword{infixr}{\isachardoublequoteopen}\isactrlbold {\isasymrightarrow}{\isachardoublequoteclose}{\isadigit{4}}{\isadigit{9}}{\isacharparenright}\ \isanewline
\ \ \ \ \isakeyword{where}\ {\isachardoublequoteopen}{\isasymphi}\isactrlbold {\isasymrightarrow}{\isasympsi}\ {\isasymequiv}\ {\isasymlambda}w{\isachardot}\ {\isacharparenleft}{\isasymphi}\ w{\isacharparenright}{\isasymlongrightarrow}{\isacharparenleft}{\isasympsi}\ w{\isacharparenright}{\isachardoublequoteclose}\ \ \isanewline
\ \ \isacommand{abbreviation}\isamarkupfalse%
\ mequ\ \ \ {\isacharcolon}{\isacharcolon}\ {\isachardoublequoteopen}io{\isasymRightarrow}io{\isasymRightarrow}io{\isachardoublequoteclose}\ {\isacharparenleft}\isakeyword{infixr}{\isachardoublequoteopen}\isactrlbold {\isasymleftrightarrow}{\isachardoublequoteclose}{\isadigit{4}}{\isadigit{8}}{\isacharparenright}\isanewline
\ \ \ \ \isakeyword{where}\ {\isachardoublequoteopen}{\isasymphi}\isactrlbold {\isasymleftrightarrow}{\isasympsi}\ {\isasymequiv}\ {\isasymlambda}w{\isachardot}\ {\isacharparenleft}{\isasymphi}\ w{\isacharparenright}{\isasymlongleftrightarrow}{\isacharparenleft}{\isasympsi}\ w{\isacharparenright}{\isachardoublequoteclose}\isanewline
\ \ \isacommand{abbreviation}\isamarkupfalse%
\ xor{\isacharcolon}{\isacharcolon}\ {\isachardoublequoteopen}bool{\isasymRightarrow}bool{\isasymRightarrow}bool{\isachardoublequoteclose}\ {\isacharparenleft}\isakeyword{infixr}{\isachardoublequoteopen}{\isasymoplus}{\isachardoublequoteclose}{\isadigit{5}}{\isadigit{0}}{\isacharparenright}\isanewline
\ \ \ \ \isakeyword{where}\ {\isachardoublequoteopen}{\isasymphi}{\isasymoplus}{\isasympsi}\ {\isasymequiv}\ \ {\isacharparenleft}{\isasymphi}{\isasymor}{\isasympsi}{\isacharparenright}\ {\isasymand}\ {\isasymnot}{\isacharparenleft}{\isasymphi}{\isasymand}{\isasympsi}{\isacharparenright}{\isachardoublequoteclose}\ \isanewline
\ \ \isacommand{abbreviation}\isamarkupfalse%
\ mxor\ \ \ {\isacharcolon}{\isacharcolon}\ {\isachardoublequoteopen}io{\isasymRightarrow}io{\isasymRightarrow}io{\isachardoublequoteclose}\ {\isacharparenleft}\isakeyword{infixr}{\isachardoublequoteopen}\isactrlbold {\isasymoplus}{\isachardoublequoteclose}{\isadigit{5}}{\isadigit{0}}{\isacharparenright}\isanewline
\ \ \ \ \isakeyword{where}\ {\isachardoublequoteopen}{\isasymphi}\isactrlbold {\isasymoplus}{\isasympsi}\ {\isasymequiv}\ {\isasymlambda}w{\isachardot}\ {\isacharparenleft}{\isasymphi}\ w{\isacharparenright}{\isasymoplus}{\isacharparenleft}{\isasympsi}\ w{\isacharparenright}{\isachardoublequoteclose}%
\isamarkupsubsubsection{Possibilist Quantification%
}
\isamarkuptrue%
\ \ \isacommand{abbreviation}\isamarkupfalse%
\ mforall\ \ \ {\isacharcolon}{\isacharcolon}\ {\isachardoublequoteopen}{\isacharparenleft}{\isacharprime}t{\isasymRightarrow}io{\isacharparenright}{\isasymRightarrow}io{\isachardoublequoteclose}\ {\isacharparenleft}{\isachardoublequoteopen}\isactrlbold {\isasymforall}{\isachardoublequoteclose}{\isacharparenright}\ \ \ \ \ \ \isanewline
\ \ \ \ \isakeyword{where}\ {\isachardoublequoteopen}\isactrlbold {\isasymforall}{\isasymPhi}\ {\isasymequiv}\ {\isasymlambda}w{\isachardot}{\isasymforall}x{\isachardot}\ {\isacharparenleft}{\isasymPhi}\ x\ w{\isacharparenright}{\isachardoublequoteclose}\isanewline
\ \ \isacommand{abbreviation}\isamarkupfalse%
\ mexists\ \ \ {\isacharcolon}{\isacharcolon}\ {\isachardoublequoteopen}{\isacharparenleft}{\isacharprime}t{\isasymRightarrow}io{\isacharparenright}{\isasymRightarrow}io{\isachardoublequoteclose}\ {\isacharparenleft}{\isachardoublequoteopen}\isactrlbold {\isasymexists}{\isachardoublequoteclose}{\isacharparenright}\ \isanewline
\ \ \ \ \isakeyword{where}\ {\isachardoublequoteopen}\isactrlbold {\isasymexists}{\isasymPhi}\ {\isasymequiv}\ {\isasymlambda}w{\isachardot}{\isasymexists}x{\isachardot}\ {\isacharparenleft}{\isasymPhi}\ x\ w{\isacharparenright}{\isachardoublequoteclose}\isanewline
\ \ \ \ \ \ \isanewline
\ \ \isacommand{abbreviation}\isamarkupfalse%
\ mforallB\ \ {\isacharcolon}{\isacharcolon}\ {\isachardoublequoteopen}{\isacharparenleft}{\isacharprime}t{\isasymRightarrow}io{\isacharparenright}{\isasymRightarrow}io{\isachardoublequoteclose}\ {\isacharparenleft}\isakeyword{binder}{\isachardoublequoteopen}\isactrlbold {\isasymforall}{\isachardoublequoteclose}{\isacharbrackleft}{\isadigit{8}}{\isacharbrackright}{\isadigit{9}}{\isacharparenright}\ %
\isamarkupcmt{Binder notation%
}
\isanewline
\ \ \ \ \isakeyword{where}\ {\isachardoublequoteopen}\isactrlbold {\isasymforall}x{\isachardot}\ {\isasymphi}{\isacharparenleft}x{\isacharparenright}\ {\isasymequiv}\ \isactrlbold {\isasymforall}{\isasymphi}{\isachardoublequoteclose}\ \ \isanewline
\ \ \isacommand{abbreviation}\isamarkupfalse%
\ mexistsB\ \ {\isacharcolon}{\isacharcolon}\ {\isachardoublequoteopen}{\isacharparenleft}{\isacharprime}t{\isasymRightarrow}io{\isacharparenright}{\isasymRightarrow}io{\isachardoublequoteclose}\ {\isacharparenleft}\isakeyword{binder}{\isachardoublequoteopen}\isactrlbold {\isasymexists}{\isachardoublequoteclose}{\isacharbrackleft}{\isadigit{8}}{\isacharbrackright}{\isadigit{9}}{\isacharparenright}\isanewline
\ \ \ \ \isakeyword{where}\ {\isachardoublequoteopen}\isactrlbold {\isasymexists}x{\isachardot}\ {\isasymphi}{\isacharparenleft}x{\isacharparenright}\ {\isasymequiv}\ \isactrlbold {\isasymexists}{\isasymphi}{\isachardoublequoteclose}%
\isamarkupsubsubsection{Actualist Quantification%
}
\isamarkuptrue%
%
\begin{isamarkuptext}%
The following predicate is used to model actualist quantifiers by restricting the domain of quantification at every possible world.
This standard technique has been referred to as \emph{existence relativization} (\cite{fitting98} p. 106),
highlighting the fact that this predicate can be seen as a kind of meta-logical `existence predicate' telling us
which individuals \emph{actually} exist at a given world. Note that since this is a meta-logical
concept it will never appear in our object language.%
\end{isamarkuptext}\isamarkuptrue%
\ \ \isacommand{consts}\isamarkupfalse%
\ Exists{\isacharcolon}{\isacharcolon}{\isachardoublequoteopen}{\isasymup}{\isasymlangle}{\isasymzero}{\isasymrangle}{\isachardoublequoteclose}\ {\isacharparenleft}{\isachardoublequoteopen}existsAt{\isachardoublequoteclose}{\isacharparenright}\isanewline
\ \ \isanewline
\isanewline
\ \ \isacommand{abbreviation}\isamarkupfalse%
\ mforallAct\ \ \ {\isacharcolon}{\isacharcolon}\ {\isachardoublequoteopen}{\isasymup}{\isasymlangle}{\isasymup}{\isasymlangle}{\isasymzero}{\isasymrangle}{\isasymrangle}{\isachardoublequoteclose}\ {\isacharparenleft}{\isachardoublequoteopen}\isactrlbold {\isasymforall}\isactrlsup E{\isachardoublequoteclose}{\isacharparenright}\ \ \ \ \ \ \isanewline
\ \ \ \ \isakeyword{where}\ {\isachardoublequoteopen}\isactrlbold {\isasymforall}\isactrlsup E{\isasymPhi}\ {\isasymequiv}\ {\isasymlambda}w{\isachardot}{\isasymforall}x{\isachardot}\ {\isacharparenleft}existsAt\ x\ w{\isacharparenright}{\isasymlongrightarrow}{\isacharparenleft}{\isasymPhi}\ x\ w{\isacharparenright}{\isachardoublequoteclose}\isanewline
\ \ \isacommand{abbreviation}\isamarkupfalse%
\ mexistsAct\ \ \ {\isacharcolon}{\isacharcolon}\ {\isachardoublequoteopen}{\isasymup}{\isasymlangle}{\isasymup}{\isasymlangle}{\isasymzero}{\isasymrangle}{\isasymrangle}{\isachardoublequoteclose}\ {\isacharparenleft}{\isachardoublequoteopen}\isactrlbold {\isasymexists}\isactrlsup E{\isachardoublequoteclose}{\isacharparenright}\ \isanewline
\ \ \ \ \isakeyword{where}\ {\isachardoublequoteopen}\isactrlbold {\isasymexists}\isactrlsup E{\isasymPhi}\ {\isasymequiv}\ {\isasymlambda}w{\isachardot}{\isasymexists}x{\isachardot}\ {\isacharparenleft}existsAt\ x\ w{\isacharparenright}\ {\isasymand}\ {\isacharparenleft}{\isasymPhi}\ x\ w{\isacharparenright}{\isachardoublequoteclose}\isanewline
\isanewline
\ \ \isacommand{abbreviation}\isamarkupfalse%
\ mforallActB\ \ {\isacharcolon}{\isacharcolon}\ {\isachardoublequoteopen}{\isasymup}{\isasymlangle}{\isasymup}{\isasymlangle}{\isasymzero}{\isasymrangle}{\isasymrangle}{\isachardoublequoteclose}\ {\isacharparenleft}\isakeyword{binder}{\isachardoublequoteopen}\isactrlbold {\isasymforall}\isactrlsup E{\isachardoublequoteclose}{\isacharbrackleft}{\isadigit{8}}{\isacharbrackright}{\isadigit{9}}{\isacharparenright}\ %
\isamarkupcmt{binder notation%
}
\isanewline
\ \ \ \ \isakeyword{where}\ {\isachardoublequoteopen}\isactrlbold {\isasymforall}\isactrlsup Ex{\isachardot}\ {\isasymphi}{\isacharparenleft}x{\isacharparenright}\ {\isasymequiv}\ \isactrlbold {\isasymforall}\isactrlsup E{\isasymphi}{\isachardoublequoteclose}\ \ \ \ \ \isanewline
\ \ \isacommand{abbreviation}\isamarkupfalse%
\ mexistsActB\ \ {\isacharcolon}{\isacharcolon}\ {\isachardoublequoteopen}{\isasymup}{\isasymlangle}{\isasymup}{\isasymlangle}{\isasymzero}{\isasymrangle}{\isasymrangle}{\isachardoublequoteclose}\ {\isacharparenleft}\isakeyword{binder}{\isachardoublequoteopen}\isactrlbold {\isasymexists}\isactrlsup E{\isachardoublequoteclose}{\isacharbrackleft}{\isadigit{8}}{\isacharbrackright}{\isadigit{9}}{\isacharparenright}\isanewline
\ \ \ \ \isakeyword{where}\ {\isachardoublequoteopen}\isactrlbold {\isasymexists}\isactrlsup Ex{\isachardot}\ {\isasymphi}{\isacharparenleft}x{\isacharparenright}\ {\isasymequiv}\ \isactrlbold {\isasymexists}\isactrlsup E{\isasymphi}{\isachardoublequoteclose}%
\isamarkupsubsubsection{Modal Operators%
}
\isamarkuptrue%
\ \ \ \isacommand{consts}\isamarkupfalse%
\ aRel{\isacharcolon}{\isacharcolon}{\isachardoublequoteopen}i{\isasymRightarrow}i{\isasymRightarrow}bool{\isachardoublequoteclose}\ {\isacharparenleft}\isakeyword{infixr}\ {\isachardoublequoteopen}r{\isachardoublequoteclose}\ {\isadigit{7}}{\isadigit{0}}{\isacharparenright}\ \ %
\isamarkupcmt{accessibility relation \emph{r}%
}
\isanewline
\ \ \isanewline
\ \ \isacommand{abbreviation}\isamarkupfalse%
\ mbox\ \ \ {\isacharcolon}{\isacharcolon}\ {\isachardoublequoteopen}io{\isasymRightarrow}io{\isachardoublequoteclose}\ {\isacharparenleft}{\isachardoublequoteopen}\isactrlbold {\isasymbox}{\isacharunderscore}{\isachardoublequoteclose}{\isacharbrackleft}{\isadigit{5}}{\isadigit{2}}{\isacharbrackright}{\isadigit{5}}{\isadigit{3}}{\isacharparenright}\isanewline
\ \ \ \ \isakeyword{where}\ {\isachardoublequoteopen}\isactrlbold {\isasymbox}{\isasymphi}\ {\isasymequiv}\ {\isasymlambda}w{\isachardot}{\isasymforall}v{\isachardot}\ {\isacharparenleft}w\ r\ v{\isacharparenright}{\isasymlongrightarrow}{\isacharparenleft}{\isasymphi}\ v{\isacharparenright}{\isachardoublequoteclose}\isanewline
\ \ \isacommand{abbreviation}\isamarkupfalse%
\ mdia\ \ \ {\isacharcolon}{\isacharcolon}\ {\isachardoublequoteopen}io{\isasymRightarrow}io{\isachardoublequoteclose}\ {\isacharparenleft}{\isachardoublequoteopen}\isactrlbold {\isasymdiamond}{\isacharunderscore}{\isachardoublequoteclose}{\isacharbrackleft}{\isadigit{5}}{\isadigit{2}}{\isacharbrackright}{\isadigit{5}}{\isadigit{3}}{\isacharparenright}\isanewline
\ \ \ \ \isakeyword{where}\ {\isachardoublequoteopen}\isactrlbold {\isasymdiamond}{\isasymphi}\ {\isasymequiv}\ {\isasymlambda}w{\isachardot}{\isasymexists}v{\isachardot}\ {\isacharparenleft}w\ r\ v{\isacharparenright}{\isasymand}{\isacharparenleft}{\isasymphi}\ v{\isacharparenright}{\isachardoublequoteclose}%
\isamarkupsubsubsection{\emph{Extension-of} Operator%
}
\isamarkuptrue%
%
\begin{isamarkuptext}%
The IHOML \isa{{\isasymdown}} operator is embedded as a predicate applying to (world-dependent) atomic formulas
 whose first argument is a \emph{relativized term} (i.e. a non-rigid term). This approach can be contrasted with 
 the one taken in Fitting's book, where \isa{{\isasymdown}} is an operator which, applied to a (rigid) intensional term,
 gives us a new (non-rigid) extensional term (\cite{Fitting} p. 93, for more details). Also note that,
 depending on the types involved, we had to define this operator differently (emph{a-d}) to ensure type correctness.
 Nevertheless, in both approaches the essence of the \emph{Extension-of} operator remains the same:
 a term of the form \isa{{\isasymdown}{\isasymphi}} behaves as a non-rigid term, whose denotation at a given possible world corresponds
 to the extension of the original intensional term \isa{{\isasymphi}} at that world.%
\end{isamarkuptext}\isamarkuptrue%
%
\begin{isamarkuptext}%
(\emph{a}) Predicate \isa{{\isasymphi}} takes an (intensional) individual concept as argument:%
\end{isamarkuptext}\isamarkuptrue%
\isacommand{abbreviation}\isamarkupfalse%
\ mextIndiv{\isacharcolon}{\isacharcolon}{\isachardoublequoteopen}{\isasymup}{\isasymlangle}{\isasymzero}{\isasymrangle}{\isasymRightarrow}{\isasymup}{\isasymzero}{\isasymRightarrow}io{\isachardoublequoteclose}\ {\isacharparenleft}\isakeyword{infix}\ {\isachardoublequoteopen}\isactrlbold {\isasymdownharpoonleft}{\isachardoublequoteclose}\ {\isadigit{6}}{\isadigit{0}}{\isacharparenright}\ \ \ \ \ \ \ \ \ \ \ \ \ \ \ \ \ \ \ \ \ \ \ \ \ \ \ \ \ \isanewline
\ \ \isakeyword{where}\ {\isachardoublequoteopen}{\isasymphi}\ \isactrlbold {\isasymdownharpoonleft}c\ {\isasymequiv}\ {\isasymlambda}w{\isachardot}\ {\isasymphi}\ {\isacharparenleft}c\ w{\isacharparenright}\ w{\isachardoublequoteclose}%
\begin{isamarkuptext}%
(\emph{b}) Predicate \isa{{\isasymphi}} takes an intensional predicate as argument:%
\end{isamarkuptext}\isamarkuptrue%
\isacommand{abbreviation}\isamarkupfalse%
\ mextPredArg{\isacharcolon}{\isacharcolon}{\isachardoublequoteopen}{\isacharparenleft}{\isacharparenleft}{\isacharprime}t{\isasymRightarrow}io{\isacharparenright}{\isasymRightarrow}io{\isacharparenright}{\isasymRightarrow}{\isacharparenleft}{\isacharprime}t{\isasymRightarrow}io{\isacharparenright}{\isasymRightarrow}io{\isachardoublequoteclose}\ {\isacharparenleft}\isakeyword{infix}\ {\isachardoublequoteopen}\isactrlbold {\isasymdown}{\isachardoublequoteclose}\ {\isadigit{6}}{\isadigit{0}}{\isacharparenright}\isanewline
\ \ \isakeyword{where}\ {\isachardoublequoteopen}{\isasymphi}\ \isactrlbold {\isasymdown}P\ {\isasymequiv}\ {\isasymlambda}w{\isachardot}\ {\isasymphi}\ {\isacharparenleft}{\isasymlambda}x\ u{\isachardot}\ P\ x\ w{\isacharparenright}\ w{\isachardoublequoteclose}%
\begin{isamarkuptext}%
(\emph{c}) Predicate \isa{{\isasymphi}} takes an extensional predicate as argument:%
\end{isamarkuptext}\isamarkuptrue%
\isacommand{abbreviation}\isamarkupfalse%
\ extPredArg{\isacharcolon}{\isacharcolon}{\isachardoublequoteopen}{\isacharparenleft}{\isacharparenleft}{\isacharprime}t{\isasymRightarrow}bool{\isacharparenright}{\isasymRightarrow}io{\isacharparenright}{\isasymRightarrow}{\isacharparenleft}{\isacharprime}t{\isasymRightarrow}io{\isacharparenright}{\isasymRightarrow}io{\isachardoublequoteclose}\ {\isacharparenleft}\isakeyword{infix}\ {\isachardoublequoteopen}{\isasymdown}{\isachardoublequoteclose}\ {\isadigit{6}}{\isadigit{0}}{\isacharparenright}\isanewline
\ \ \isakeyword{where}\ {\isachardoublequoteopen}{\isasymphi}\ {\isasymdown}P\ {\isasymequiv}\ {\isasymlambda}w{\isachardot}\ {\isasymphi}\ {\isacharparenleft}{\isasymlambda}x{\isachardot}\ P\ x\ w{\isacharparenright}\ w{\isachardoublequoteclose}%
\begin{isamarkuptext}%
(\emph{d}) Predicate \isa{{\isasymphi}} takes an extensional predicate as \emph{first} argument:%
\end{isamarkuptext}\isamarkuptrue%
\isacommand{abbreviation}\isamarkupfalse%
\ extPredArg{\isadigit{1}}{\isacharcolon}{\isacharcolon}{\isachardoublequoteopen}{\isacharparenleft}{\isacharparenleft}{\isacharprime}t{\isasymRightarrow}bool{\isacharparenright}{\isasymRightarrow}{\isacharprime}b{\isasymRightarrow}io{\isacharparenright}{\isasymRightarrow}{\isacharparenleft}{\isacharprime}t{\isasymRightarrow}io{\isacharparenright}{\isasymRightarrow}{\isacharprime}b{\isasymRightarrow}io{\isachardoublequoteclose}\ {\isacharparenleft}\isakeyword{infix}\ {\isachardoublequoteopen}{\isasymdown}\isactrlsub {\isadigit{1}}{\isachardoublequoteclose}\ {\isadigit{6}}{\isadigit{0}}{\isacharparenright}\isanewline
\ \ \isakeyword{where}\ {\isachardoublequoteopen}{\isasymphi}\ {\isasymdown}\isactrlsub {\isadigit{1}}P\ {\isasymequiv}\ {\isasymlambda}z{\isachardot}\ {\isasymlambda}w{\isachardot}\ {\isasymphi}\ {\isacharparenleft}{\isasymlambda}x{\isachardot}\ P\ x\ w{\isacharparenright}\ z\ w{\isachardoublequoteclose}%
\isamarkupsubsubsection{Equality%
}
\isamarkuptrue%
\ \ \isacommand{abbreviation}\isamarkupfalse%
\ meq\ \ \ \ {\isacharcolon}{\isacharcolon}\ {\isachardoublequoteopen}{\isacharprime}t{\isasymRightarrow}{\isacharprime}t{\isasymRightarrow}io{\isachardoublequoteclose}\ {\isacharparenleft}\isakeyword{infix}{\isachardoublequoteopen}\isactrlbold {\isasymapprox}{\isachardoublequoteclose}{\isadigit{6}}{\isadigit{0}}{\isacharparenright}\ %
\isamarkupcmt{normal equality (for all types)%
}
\isanewline
\ \ \ \ \isakeyword{where}\ {\isachardoublequoteopen}x\ \isactrlbold {\isasymapprox}\ y\ {\isasymequiv}\ {\isasymlambda}w{\isachardot}\ x\ {\isacharequal}\ y{\isachardoublequoteclose}\isanewline
\ \ \isacommand{abbreviation}\isamarkupfalse%
\ meqC\ \ \ {\isacharcolon}{\isacharcolon}\ {\isachardoublequoteopen}{\isasymup}{\isasymlangle}{\isasymup}{\isasymzero}{\isacharcomma}{\isasymup}{\isasymzero}{\isasymrangle}{\isachardoublequoteclose}\ {\isacharparenleft}\isakeyword{infixr}{\isachardoublequoteopen}\isactrlbold {\isasymapprox}\isactrlsup C{\isachardoublequoteclose}{\isadigit{5}}{\isadigit{2}}{\isacharparenright}\ %
\isamarkupcmt{eq. for individual concepts%
}
\isanewline
\ \ \ \ \isakeyword{where}\ {\isachardoublequoteopen}x\ \isactrlbold {\isasymapprox}\isactrlsup C\ y\ {\isasymequiv}\ {\isasymlambda}w{\isachardot}\ {\isasymforall}v{\isachardot}\ {\isacharparenleft}x\ v{\isacharparenright}\ {\isacharequal}\ {\isacharparenleft}y\ v{\isacharparenright}{\isachardoublequoteclose}\isanewline
\ \ \isacommand{abbreviation}\isamarkupfalse%
\ meqL\ \ \ {\isacharcolon}{\isacharcolon}\ {\isachardoublequoteopen}{\isasymup}{\isasymlangle}{\isasymzero}{\isacharcomma}{\isasymzero}{\isasymrangle}{\isachardoublequoteclose}\ {\isacharparenleft}\isakeyword{infixr}{\isachardoublequoteopen}\isactrlbold {\isasymapprox}\isactrlsup L{\isachardoublequoteclose}{\isadigit{5}}{\isadigit{2}}{\isacharparenright}\ %
\isamarkupcmt{Leibniz eq. for individuals%
}
\isanewline
\ \ \ \ \isakeyword{where}\ {\isachardoublequoteopen}x\ \isactrlbold {\isasymapprox}\isactrlsup L\ y\ {\isasymequiv}\ \isactrlbold {\isasymforall}{\isasymphi}{\isachardot}\ {\isasymphi}{\isacharparenleft}x{\isacharparenright}\isactrlbold {\isasymrightarrow}{\isasymphi}{\isacharparenleft}y{\isacharparenright}{\isachardoublequoteclose}%
\isamarkupsubsubsection{Meta-logical Predicates%
}
\isamarkuptrue%
\ \isacommand{abbreviation}\isamarkupfalse%
\ valid\ {\isacharcolon}{\isacharcolon}\ {\isachardoublequoteopen}io{\isasymRightarrow}bool{\isachardoublequoteclose}\ {\isacharparenleft}{\isachardoublequoteopen}{\isasymlfloor}{\isacharunderscore}{\isasymrfloor}{\isachardoublequoteclose}\ {\isacharbrackleft}{\isadigit{8}}{\isacharbrackright}{\isacharparenright}\ \isakeyword{where}\ {\isachardoublequoteopen}{\isasymlfloor}{\isasympsi}{\isasymrfloor}\ {\isasymequiv}\ \ {\isasymforall}w{\isachardot}{\isacharparenleft}{\isasympsi}\ w{\isacharparenright}{\isachardoublequoteclose}\isanewline
\ \isacommand{abbreviation}\isamarkupfalse%
\ satisfiable\ {\isacharcolon}{\isacharcolon}\ {\isachardoublequoteopen}io{\isasymRightarrow}bool{\isachardoublequoteclose}\ {\isacharparenleft}{\isachardoublequoteopen}{\isasymlfloor}{\isacharunderscore}{\isasymrfloor}\isactrlsup s\isactrlsup a\isactrlsup t{\isachardoublequoteclose}\ {\isacharbrackleft}{\isadigit{8}}{\isacharbrackright}{\isacharparenright}\ \isakeyword{where}\ {\isachardoublequoteopen}{\isasymlfloor}{\isasympsi}{\isasymrfloor}\isactrlsup s\isactrlsup a\isactrlsup t\ {\isasymequiv}\ {\isasymexists}w{\isachardot}{\isacharparenleft}{\isasympsi}\ w{\isacharparenright}{\isachardoublequoteclose}\isanewline
\ \isacommand{abbreviation}\isamarkupfalse%
\ countersat\ {\isacharcolon}{\isacharcolon}\ {\isachardoublequoteopen}io{\isasymRightarrow}bool{\isachardoublequoteclose}\ {\isacharparenleft}{\isachardoublequoteopen}{\isasymlfloor}{\isacharunderscore}{\isasymrfloor}\isactrlsup c\isactrlsup s\isactrlsup a\isactrlsup t{\isachardoublequoteclose}\ {\isacharbrackleft}{\isadigit{8}}{\isacharbrackright}{\isacharparenright}\ \isakeyword{where}\ {\isachardoublequoteopen}{\isasymlfloor}{\isasympsi}{\isasymrfloor}\isactrlsup c\isactrlsup s\isactrlsup a\isactrlsup t\ {\isasymequiv}\ \ {\isasymexists}w{\isachardot}{\isasymnot}{\isacharparenleft}{\isasympsi}\ w{\isacharparenright}{\isachardoublequoteclose}\isanewline
\ \isacommand{abbreviation}\isamarkupfalse%
\ invalid\ {\isacharcolon}{\isacharcolon}\ {\isachardoublequoteopen}io{\isasymRightarrow}bool{\isachardoublequoteclose}\ {\isacharparenleft}{\isachardoublequoteopen}{\isasymlfloor}{\isacharunderscore}{\isasymrfloor}\isactrlsup i\isactrlsup n\isactrlsup v{\isachardoublequoteclose}\ {\isacharbrackleft}{\isadigit{8}}{\isacharbrackright}{\isacharparenright}\ \isakeyword{where}\ {\isachardoublequoteopen}{\isasymlfloor}{\isasympsi}{\isasymrfloor}\isactrlsup i\isactrlsup n\isactrlsup v\ {\isasymequiv}\ {\isasymforall}w{\isachardot}{\isasymnot}{\isacharparenleft}{\isasympsi}\ w{\isacharparenright}{\isachardoublequoteclose}%
\isamarkupsubsection{Verifying the Embedding%
}
\isamarkuptrue%
%
\begin{isamarkuptext}%
The above definitions introduce modal logic \emph{K} with possibilist and actualist quantifiers,
as evidenced by the following tests:%
\end{isamarkuptext}\isamarkuptrue%
%
\begin{isamarkuptext}%
Verifying \emph{K} Principle and Necessitation:%
\end{isamarkuptext}\isamarkuptrue%
\ \isacommand{lemma}\isamarkupfalse%
\ K{\isacharcolon}\ {\isachardoublequoteopen}{\isasymlfloor}{\isacharparenleft}\isactrlbold {\isasymbox}{\isacharparenleft}{\isasymphi}\ \isactrlbold {\isasymrightarrow}\ {\isasympsi}{\isacharparenright}{\isacharparenright}\ \isactrlbold {\isasymrightarrow}\ {\isacharparenleft}\isactrlbold {\isasymbox}{\isasymphi}\ \isactrlbold {\isasymrightarrow}\ \isactrlbold {\isasymbox}{\isasympsi}{\isacharparenright}{\isasymrfloor}{\isachardoublequoteclose}%
\isadelimproof
\ %
\endisadelimproof
%
\isatagproof
\isacommand{by}\isamarkupfalse%
\ simp\ \ \ \ %
\isamarkupcmt{\emph{K} schema%
}
%
\endisatagproof
{\isafoldproof}%
%
\isadelimproof
%
\endisadelimproof
\isanewline
\ \isacommand{lemma}\isamarkupfalse%
\ NEC{\isacharcolon}\ {\isachardoublequoteopen}{\isasymlfloor}{\isasymphi}{\isasymrfloor}\ {\isasymLongrightarrow}\ {\isasymlfloor}\isactrlbold {\isasymbox}{\isasymphi}{\isasymrfloor}{\isachardoublequoteclose}%
\isadelimproof
\ %
\endisadelimproof
%
\isatagproof
\isacommand{by}\isamarkupfalse%
\ simp\ \ \ \ %
\isamarkupcmt{necessitation%
}
%
\endisatagproof
{\isafoldproof}%
%
\isadelimproof
%
\endisadelimproof
%
\begin{isamarkuptext}%
Local consequence implies global consequence (we will use this lemma often):%
\end{isamarkuptext}\isamarkuptrue%
\ \isacommand{lemma}\isamarkupfalse%
\ localImpGlobalCons{\isacharcolon}\ {\isachardoublequoteopen}{\isasymlfloor}{\isasymphi}\ \isactrlbold {\isasymrightarrow}\ {\isasymxi}{\isasymrfloor}\ {\isasymLongrightarrow}\ {\isasymlfloor}{\isasymphi}{\isasymrfloor}\ {\isasymlongrightarrow}\ {\isasymlfloor}{\isasymxi}{\isasymrfloor}{\isachardoublequoteclose}%
\isadelimproof
\ %
\endisadelimproof
%
\isatagproof
\isacommand{by}\isamarkupfalse%
\ simp%
\endisatagproof
{\isafoldproof}%
%
\isadelimproof
%
\endisadelimproof
%
\begin{isamarkuptext}%
But global consequence does not imply local consequence:%
\end{isamarkuptext}\isamarkuptrue%
\ \isacommand{lemma}\isamarkupfalse%
\ {\isachardoublequoteopen}{\isasymlfloor}{\isasymphi}{\isasymrfloor}\ {\isasymlongrightarrow}\ {\isasymlfloor}{\isasymxi}{\isasymrfloor}\ {\isasymLongrightarrow}\ {\isasymlfloor}{\isasymphi}\ \isactrlbold {\isasymrightarrow}\ {\isasymxi}{\isasymrfloor}{\isachardoublequoteclose}\ \isacommand{nitpick}\isamarkupfalse%
%
\isadelimproof
\ %
\endisadelimproof
%
\isatagproof
\isacommand{oops}\isamarkupfalse%
\ %
\isamarkupcmt{countersatisfiable%
}
%
\endisatagproof
{\isafoldproof}%
%
\isadelimproof
%
\endisadelimproof
%
\begin{isamarkuptext}%
Barcan and Converse Barcan Formulas are satisfied for standard (possibilist) quantifiers:%
\end{isamarkuptext}\isamarkuptrue%
\ \isacommand{lemma}\isamarkupfalse%
\ {\isachardoublequoteopen}{\isasymlfloor}{\isacharparenleft}\isactrlbold {\isasymforall}x{\isachardot}\isactrlbold {\isasymbox}{\isacharparenleft}{\isasymphi}\ x{\isacharparenright}{\isacharparenright}\ \isactrlbold {\isasymrightarrow}\ \isactrlbold {\isasymbox}{\isacharparenleft}\isactrlbold {\isasymforall}x{\isachardot}{\isacharparenleft}{\isasymphi}\ x{\isacharparenright}{\isacharparenright}{\isasymrfloor}{\isachardoublequoteclose}%
\isadelimproof
\ %
\endisadelimproof
%
\isatagproof
\isacommand{by}\isamarkupfalse%
\ simp%
\endisatagproof
{\isafoldproof}%
%
\isadelimproof
%
\endisadelimproof
\isanewline
\ \isacommand{lemma}\isamarkupfalse%
\ {\isachardoublequoteopen}{\isasymlfloor}\isactrlbold {\isasymbox}{\isacharparenleft}\isactrlbold {\isasymforall}x{\isachardot}{\isacharparenleft}{\isasymphi}\ x{\isacharparenright}{\isacharparenright}\ \isactrlbold {\isasymrightarrow}\ {\isacharparenleft}\isactrlbold {\isasymforall}x{\isachardot}\isactrlbold {\isasymbox}{\isacharparenleft}{\isasymphi}\ x{\isacharparenright}{\isacharparenright}{\isasymrfloor}{\isachardoublequoteclose}%
\isadelimproof
\ %
\endisadelimproof
%
\isatagproof
\isacommand{by}\isamarkupfalse%
\ simp%
\endisatagproof
{\isafoldproof}%
%
\isadelimproof
%
\endisadelimproof
%
\begin{isamarkuptext}%
(Converse) Barcan Formulas not satisfied for actualist quantifiers:%
\end{isamarkuptext}\isamarkuptrue%
\ \isacommand{lemma}\isamarkupfalse%
\ {\isachardoublequoteopen}{\isasymlfloor}{\isacharparenleft}\isactrlbold {\isasymforall}\isactrlsup Ex{\isachardot}\isactrlbold {\isasymbox}{\isacharparenleft}{\isasymphi}\ x{\isacharparenright}{\isacharparenright}\ \isactrlbold {\isasymrightarrow}\ \isactrlbold {\isasymbox}{\isacharparenleft}\isactrlbold {\isasymforall}\isactrlsup Ex{\isachardot}{\isacharparenleft}{\isasymphi}\ x{\isacharparenright}{\isacharparenright}{\isasymrfloor}{\isachardoublequoteclose}\ \isacommand{nitpick}\isamarkupfalse%
%
\isadelimproof
\ %
\endisadelimproof
%
\isatagproof
\isacommand{oops}\isamarkupfalse%
\ %
\isamarkupcmt{countersatisfiable%
}
%
\endisatagproof
{\isafoldproof}%
%
\isadelimproof
%
\endisadelimproof
\isanewline
\ \isacommand{lemma}\isamarkupfalse%
\ {\isachardoublequoteopen}{\isasymlfloor}\isactrlbold {\isasymbox}{\isacharparenleft}\isactrlbold {\isasymforall}\isactrlsup Ex{\isachardot}{\isacharparenleft}{\isasymphi}\ x{\isacharparenright}{\isacharparenright}\ \isactrlbold {\isasymrightarrow}\ {\isacharparenleft}\isactrlbold {\isasymforall}\isactrlsup Ex{\isachardot}\isactrlbold {\isasymbox}{\isacharparenleft}{\isasymphi}\ x{\isacharparenright}{\isacharparenright}{\isasymrfloor}{\isachardoublequoteclose}\ \isacommand{nitpick}\isamarkupfalse%
%
\isadelimproof
\ %
\endisadelimproof
%
\isatagproof
\isacommand{oops}\isamarkupfalse%
\ %
\isamarkupcmt{countersatisfiable%
}
%
\endisatagproof
{\isafoldproof}%
%
\isadelimproof
%
\endisadelimproof
%
\begin{isamarkuptext}%
Note that we have just made use of \emph{Nitpick} for the first time here. \emph{Nitpick} is a (counter-)model finder
for Isabelle/HOL. In the lemmas above, \emph{Nitpick} has found a model satisfying all axioms 
while falsifying the given formula. This means, the formula is not valid (i.e. is countersatisfiable).%
\end{isamarkuptext}\isamarkuptrue%
%
\begin{isamarkuptext}%
Well known relations between meta-logical notions:%
\end{isamarkuptext}\isamarkuptrue%
\ \isacommand{lemma}\isamarkupfalse%
\ \ {\isachardoublequoteopen}{\isasymlfloor}{\isasymphi}{\isasymrfloor}\ {\isasymlongleftrightarrow}\ {\isasymnot}{\isasymlfloor}{\isasymphi}{\isasymrfloor}\isactrlsup c\isactrlsup s\isactrlsup a\isactrlsup t{\isachardoublequoteclose}%
\isadelimproof
\ %
\endisadelimproof
%
\isatagproof
\isacommand{by}\isamarkupfalse%
\ simp%
\endisatagproof
{\isafoldproof}%
%
\isadelimproof
%
\endisadelimproof
\isanewline
\ \isacommand{lemma}\isamarkupfalse%
\ \ {\isachardoublequoteopen}{\isasymlfloor}{\isasymphi}{\isasymrfloor}\isactrlsup s\isactrlsup a\isactrlsup t\ {\isasymlongleftrightarrow}\ {\isasymnot}{\isasymlfloor}{\isasymphi}{\isasymrfloor}\isactrlsup i\isactrlsup n\isactrlsup v\ {\isachardoublequoteclose}%
\isadelimproof
\ %
\endisadelimproof
%
\isatagproof
\isacommand{by}\isamarkupfalse%
\ simp%
\endisatagproof
{\isafoldproof}%
%
\isadelimproof
%
\endisadelimproof
%
\begin{isamarkuptext}%
Contingent truth does not allow for necessitation:%
\end{isamarkuptext}\isamarkuptrue%
\ \isacommand{lemma}\isamarkupfalse%
\ {\isachardoublequoteopen}{\isasymlfloor}\isactrlbold {\isasymdiamond}{\isasymphi}{\isasymrfloor}\ \ {\isasymlongrightarrow}\ {\isasymlfloor}\isactrlbold {\isasymbox}{\isasymphi}{\isasymrfloor}{\isachardoublequoteclose}\ \isacommand{nitpick}\isamarkupfalse%
%
\isadelimproof
\ %
\endisadelimproof
%
\isatagproof
\isacommand{oops}\isamarkupfalse%
\ \ \ \ \ \ \ \ \ \ \ \ %
\isamarkupcmt{countersatisfiable%
}
%
\endisatagproof
{\isafoldproof}%
%
\isadelimproof
%
\endisadelimproof
\isanewline
\ \isacommand{lemma}\isamarkupfalse%
\ {\isachardoublequoteopen}{\isasymlfloor}\isactrlbold {\isasymbox}{\isasymphi}{\isasymrfloor}\isactrlsup s\isactrlsup a\isactrlsup t\ {\isasymlongrightarrow}\ {\isasymlfloor}\isactrlbold {\isasymbox}{\isasymphi}{\isasymrfloor}{\isachardoublequoteclose}\ \isacommand{nitpick}\isamarkupfalse%
%
\isadelimproof
\ %
\endisadelimproof
%
\isatagproof
\isacommand{oops}\isamarkupfalse%
\ \ \ \ \ \ \ \ \ \ \ %
\isamarkupcmt{countersatisfiable%
}
%
\endisatagproof
{\isafoldproof}%
%
\isadelimproof
%
\endisadelimproof
%
\begin{isamarkuptext}%
\emph{Modal collapse} is countersatisfiable:%
\end{isamarkuptext}\isamarkuptrue%
\ \isacommand{lemma}\isamarkupfalse%
\ {\isachardoublequoteopen}{\isasymlfloor}{\isasymphi}\ \isactrlbold {\isasymrightarrow}\ \isactrlbold {\isasymbox}{\isasymphi}{\isasymrfloor}{\isachardoublequoteclose}\ \isacommand{nitpick}\isamarkupfalse%
%
\isadelimproof
\ %
\endisadelimproof
%
\isatagproof
\isacommand{oops}\isamarkupfalse%
\ \ \ \ \ \ \ \ \ \ \ \ \ \ \ \ \ \ %
\isamarkupcmt{countersatisfiable%
}
%
\endisatagproof
{\isafoldproof}%
%
\isadelimproof
%
\endisadelimproof
%
\isamarkupsubsection{Useful Definitions for Axiomatization of Further Logics%
}
\isamarkuptrue%
%
\begin{isamarkuptext}%
The best known normal logics (\emph{K4, K5, KB, K45, KB5, D, D4, D5, D45, ...}) can be obtained by
 combinations of the following axioms:%
\end{isamarkuptext}\isamarkuptrue%
\ \ \isacommand{abbreviation}\isamarkupfalse%
\ M\ \isanewline
\ \ \ \ \isakeyword{where}\ {\isachardoublequoteopen}M\ {\isasymequiv}\ \isactrlbold {\isasymforall}{\isasymphi}{\isachardot}\ \isactrlbold {\isasymbox}{\isasymphi}\ \isactrlbold {\isasymrightarrow}\ {\isasymphi}{\isachardoublequoteclose}\isanewline
\ \ \isacommand{abbreviation}\isamarkupfalse%
\ B\ \isanewline
\ \ \ \ \isakeyword{where}\ {\isachardoublequoteopen}B\ {\isasymequiv}\ \isactrlbold {\isasymforall}{\isasymphi}{\isachardot}\ {\isasymphi}\ \isactrlbold {\isasymrightarrow}\ \ \isactrlbold {\isasymbox}\isactrlbold {\isasymdiamond}{\isasymphi}{\isachardoublequoteclose}\isanewline
\ \ \isacommand{abbreviation}\isamarkupfalse%
\ D\ \isanewline
\ \ \ \ \isakeyword{where}\ {\isachardoublequoteopen}D\ {\isasymequiv}\ \isactrlbold {\isasymforall}{\isasymphi}{\isachardot}\ \isactrlbold {\isasymbox}{\isasymphi}\ \isactrlbold {\isasymrightarrow}\ \isactrlbold {\isasymdiamond}{\isasymphi}{\isachardoublequoteclose}\isanewline
\ \ \isacommand{abbreviation}\isamarkupfalse%
\ IV\ \isanewline
\ \ \ \ \isakeyword{where}\ {\isachardoublequoteopen}IV\ {\isasymequiv}\ \isactrlbold {\isasymforall}{\isasymphi}{\isachardot}\ \isactrlbold {\isasymbox}{\isasymphi}\ \isactrlbold {\isasymrightarrow}\ \ \isactrlbold {\isasymbox}\isactrlbold {\isasymbox}{\isasymphi}{\isachardoublequoteclose}\isanewline
\ \ \isacommand{abbreviation}\isamarkupfalse%
\ V\ \isanewline
\ \ \ \ \isakeyword{where}\ {\isachardoublequoteopen}V\ {\isasymequiv}\ \isactrlbold {\isasymforall}{\isasymphi}{\isachardot}\ \isactrlbold {\isasymdiamond}{\isasymphi}\ \isactrlbold {\isasymrightarrow}\ \isactrlbold {\isasymbox}\isactrlbold {\isasymdiamond}{\isasymphi}{\isachardoublequoteclose}%
\begin{isamarkuptext}%
Because the embedding is of a semantic nature, it is more efficient to instead make use of 
  the well-known \emph{Sahlqvist correspondence}, which links axioms to constraints on a model's accessibility
  relation (e.g. reflexive, symmetric, etc. whose definitions are not shown here). We show below that
  axioms $M, B, D, IV, V$ impose reflexivity, symmetry, seriality, transitivity and euclideanness respectively.%
\end{isamarkuptext}\isamarkuptrue%
\ \ \isacommand{lemma}\isamarkupfalse%
\ {\isachardoublequoteopen}reflexive\ aRel\ \ {\isasymLongrightarrow}\ \ {\isasymlfloor}M{\isasymrfloor}{\isachardoublequoteclose}%
\isadelimproof
\ %
\endisadelimproof
%
\isatagproof
\isacommand{by}\isamarkupfalse%
\ blast\ %
\isamarkupcmt{aka T%
}
%
\endisatagproof
{\isafoldproof}%
%
\isadelimproof
%
\endisadelimproof
\isanewline
\ \ \isacommand{lemma}\isamarkupfalse%
\ {\isachardoublequoteopen}symmetric\ aRel\ {\isasymLongrightarrow}\ {\isasymlfloor}B{\isasymrfloor}{\isachardoublequoteclose}%
\isadelimproof
\ %
\endisadelimproof
%
\isatagproof
\isacommand{by}\isamarkupfalse%
\ blast%
\endisatagproof
{\isafoldproof}%
%
\isadelimproof
%
\endisadelimproof
\isanewline
\ \ \isacommand{lemma}\isamarkupfalse%
\ {\isachardoublequoteopen}serial\ aRel\ \ {\isasymLongrightarrow}\ {\isasymlfloor}D{\isasymrfloor}{\isachardoublequoteclose}%
\isadelimproof
\ %
\endisadelimproof
%
\isatagproof
\isacommand{by}\isamarkupfalse%
\ blast%
\endisatagproof
{\isafoldproof}%
%
\isadelimproof
%
\endisadelimproof
\ \ \ \ \ \ \ \ \ \isanewline
\ \ \isacommand{lemma}\isamarkupfalse%
\ {\isachardoublequoteopen}transitive\ aRel\ \ {\isasymLongrightarrow}\ {\isasymlfloor}IV{\isasymrfloor}{\isachardoublequoteclose}%
\isadelimproof
\ %
\endisadelimproof
%
\isatagproof
\isacommand{by}\isamarkupfalse%
\ blast%
\endisatagproof
{\isafoldproof}%
%
\isadelimproof
%
\endisadelimproof
\ \ \ \ \ \ \ \ \ \isanewline
\ \ \isacommand{lemma}\isamarkupfalse%
\ {\isachardoublequoteopen}preorder\ aRel\ {\isasymLongrightarrow}\ \ {\isasymlfloor}M{\isasymrfloor}\ {\isasymand}\ {\isasymlfloor}IV{\isasymrfloor}{\isachardoublequoteclose}%
\isadelimproof
\ %
\endisadelimproof
%
\isatagproof
\isacommand{by}\isamarkupfalse%
\ blast\ %
\isamarkupcmt{S4: reflexive + transitive%
}
%
\endisatagproof
{\isafoldproof}%
%
\isadelimproof
%
\endisadelimproof
\isanewline
\ \ \isacommand{lemma}\isamarkupfalse%
\ {\isachardoublequoteopen}equivalence\ aRel\ \ {\isasymLongrightarrow}\ \ {\isasymlfloor}M{\isasymrfloor}\ {\isasymand}\ {\isasymlfloor}V{\isasymrfloor}{\isachardoublequoteclose}%
\isadelimproof
\ %
\endisadelimproof
%
\isatagproof
\isacommand{by}\isamarkupfalse%
\ blast\ %
\isamarkupcmt{S5: preorder + symmetric%
}
%
\endisatagproof
{\isafoldproof}%
%
\isadelimproof
%
\endisadelimproof
\isanewline
\ \ \isacommand{lemma}\isamarkupfalse%
\ {\isachardoublequoteopen}reflexive\ aRel\ {\isasymand}\ euclidean\ aRel\ \ {\isasymLongrightarrow}\ \ {\isasymlfloor}M{\isasymrfloor}\ {\isasymand}\ {\isasymlfloor}V{\isasymrfloor}{\isachardoublequoteclose}%
\isadelimproof
\ %
\endisadelimproof
%
\isatagproof
\isacommand{by}\isamarkupfalse%
\ blast\ %
\isamarkupcmt{S5%
}
%
\endisatagproof
{\isafoldproof}%
%
\isadelimproof
%
\endisadelimproof
%
\begin{isamarkuptext}%
Using these definitions, we can derive axioms for the most common modal logics. Thereby we 
  are free to use either the semantic constraints or the related \emph{Sahlqvist} axioms. Here we provide 
  both versions. In what follows we use the semantic constraints for improved performance.
  \pagebreak%
\end{isamarkuptext}\isamarkuptrue%
%
\isadelimtheory
%
\endisadelimtheory
%
\isatagtheory
%
\endisatagtheory
{\isafoldtheory}%
%
\isadelimtheory
%
\endisadelimtheory
%
\end{isabellebody}%
%%% Local Variables:
%%% mode: latex
%%% TeX-master: "root"
%%% End:
