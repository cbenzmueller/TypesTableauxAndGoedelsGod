\documentclass[11pt,a4paper]{article}
\usepackage{authblk}
%\usepackage{a4wide}
\usepackage{isabelle,isabellesym}

% further packages required for unusual symbols (see also
% isabellesym.sty), use only when needed

\usepackage{amssymb}
  %for \<leadsto>, \<box>, \<diamond>, \<sqsupset>, \<mho>, \<Join>,
  %\<lhd>, \<lesssim>, \<greatersim>, \<lessapprox>, \<greaterapprox>,
  %\<triangleq>, \<yen>, \<lozenge>

%\usepackage{eurosym}
  %for \<euro>

%\usepackage[only,bigsqcap]{stmaryrd}
  %for \<Sqinter>

%\usepackage{eufrak}
  %for \<AA> ... \<ZZ>, \<aa> ... \<zz> (also included in amssymb)

%\usepackage{textcomp}
  %for \<onequarter>, \<onehalf>, \<threequarters>, \<degree>, \<cent>,
  %\<currency>

% this should be the last package used
\usepackage{pdfsetup}

% urls in roman style, theory text in math-similar italics
\urlstyle{rm}
\isabellestyle{it}

% for uniform font size
%\renewcommand{\isastyle}{\isastyleminor}


\begin{document}

\title{Types, Tableaus and G\"odel's God \\ in Isabelle/HOL}
%\author{David Fuenmayor, Christoph Benzm\"uller}
\author[1]{David Fuenmayor}
\author[2]{Christoph Benzm\"uller}
\affil[1]{Freie Universit\"at Berlin, Germany}
\affil[2]{University of Luxemburg, Luxemburg \& Freie Universit\"at
  Berlin, Germany}

\maketitle

\begin{abstract}
  A computer-formalisation of the most essential parts of Fitting's textbook
  \emph{Types, Tableaus and G\"odel's Go}d in Isabelle/HOL is
  presented. In particular, Fitting's variant of the ontological
  argument is verified and confirmed. This variant avoids the modal
  collapse, which has been criticised as an undesirable side-effect of Kurt G\"odel's (and
  Dana Scott's) versions of the ontological argument. Fitting's work
  is employing an intensional higher-order modal logic, which we
  shallowly embed here in classical higher-order logic. We then
  utilize the embedded logic for the formalisation of Fitting's argument.
\end{abstract}

\tableofcontents

% sane default for proof documents
\parindent 0pt\parskip 0.5ex

% generated text of all theories
\input{Relations.tex}

%
\begin{isabellebody}%
\setisabellecontext{HOML{\isacharunderscore}int}%
%
\isadelimtheory
%
\endisadelimtheory
%
\isatagtheory
%
\endisatagtheory
{\isafoldtheory}%
%
\isadelimtheory
%
\endisadelimtheory
%
\isamarkupsection{Introduction%
}
\isamarkuptrue%
%
\begin{isamarkuptext}%
We present a study in Computational Metaphysics: a computer-formalisation and verification
of Fitting's emendation of the ontological argument (for the existence of God) as presented in
his well-known textbook \emph{Types, Tableaus and G\"odel's God} \cite{Fitting}. Fitting's argument 
is an emendation of Kurt G\"odel's modern variant \cite{GoedelNotes} (resp. Dana Scott's 
variant \cite{ScottNotes}) of the ontological argument.

The motivation is to avoid the modal collapse \cite{Sobel,sobel2004logic}, which has been criticised
as an undesirable side-effect of the axioms of G\"odel resp. Scott. The modal collapse essentially  
states that  there are no contingent truths and that everything is determined.
Several authors (see e.g. \cite{anderson90:_some_emend_of_goedel_ontol_proof,AndersonGettings,Hajek,Bjordal} 
have proposed emendations of the argument with the aim of maintaining the essential result 
(the necessary existence of God) while at the same time avoiding the modal collapse. 
Related work  has 
formalised several of these variants on the computer and verified or falsified them. For example,
G\"odel's axioms \cite{GoedelNotes} have been shown inconsistent \cite{IJCAI,C60}
while Scott's version has been verified \cite{ECAI}. Further experiments, contributing amongst others
to the clarification of a related debate between Hajek and Anderson, are presented and discussed in
\cite{J23}. The enabling technique that has been employed in all of these experiments has been
shallow semantical embeddings of (extensional) higher-order modal logics in classical higher-order
logic (see \cite{J23,R59} and the references therein).

Fitting's emendation also intends to avoid the modal collapse. In contrast to the above emendations, Fitting's
solution is based on the use of an  intensional as opposed to an extensional higher-order modal logic.
For our work this imposed the additional challenge to provide an shallow embedding of this more advanced
logic. The experiments presented below confirm that Fitting's argument as presented in \cite{Fitting}
is valid and that it avoids the modal collapse as intended.

The work presented here originates from the \emph{Computational Metaphysics} lecture course  
held at FU Berlin  in Summer 2016.%
\end{isamarkuptext}\isamarkuptrue%
%
\isamarkupsection{Embedding of Intensional Higher-Order Modal Logic%
}
\isamarkuptrue%
%
\begin{isamarkuptext}%
The following shallow embedding of Intensional Higher-Order Modal Logic (IHOML) in Isabelle/HOL is inspired by the work of \cite{J23}.
We expand this approach to allow for intensional types and actualist quantifiers as employed in Fitting's 
textbook (\cite{Fitting}).%
\end{isamarkuptext}\isamarkuptrue%
%
\isamarkupsubsection{Declarations%
}
\isamarkuptrue%
\ \ \isacommand{typedecl}\isamarkupfalse%
\ i\ \ \ \ \ \ \ \ \ \ \ \ \ \ \ \ \ \ \ \ %
\isamarkupcmt{Type for possible worlds%
}
\isanewline
\ \ \isacommand{type{\isacharunderscore}synonym}\isamarkupfalse%
\ io\ {\isacharequal}\ {\isachardoublequoteopen}{\isacharparenleft}i{\isasymRightarrow}bool{\isacharparenright}{\isachardoublequoteclose}\ %
\isamarkupcmt{Type for formulas whose truth-value is world-dependent%
}
\isanewline
\ \ \isacommand{typedecl}\isamarkupfalse%
\ e\ \ {\isacharparenleft}{\isachardoublequoteopen}{\isasymzero}{\isachardoublequoteclose}{\isacharparenright}\ \ \ \ \ \ \ \ \ \ \ \ \ %
\isamarkupcmt{Type for individuals%
}
%
\begin{isamarkuptext}%
Aliases for common unary predicate types:%
\end{isamarkuptext}\isamarkuptrue%
\ \ \isacommand{type{\isacharunderscore}synonym}\isamarkupfalse%
\ ie\ {\isacharequal}\ \ \ \ \ {\isachardoublequoteopen}{\isacharparenleft}i{\isasymRightarrow}{\isasymzero}{\isacharparenright}{\isachardoublequoteclose}\ \ \ \ \ \ \ \ \ \ \ \ \ {\isacharparenleft}{\isachardoublequoteopen}{\isasymup}{\isasymzero}{\isachardoublequoteclose}{\isacharparenright}\isanewline
\ \ \isacommand{type{\isacharunderscore}synonym}\isamarkupfalse%
\ se\ {\isacharequal}\ \ \ \ \ {\isachardoublequoteopen}{\isacharparenleft}{\isasymzero}{\isasymRightarrow}bool{\isacharparenright}{\isachardoublequoteclose}\ \ \ \ \ \ \ \ \ \ {\isacharparenleft}{\isachardoublequoteopen}{\isasymlangle}{\isasymzero}{\isasymrangle}{\isachardoublequoteclose}{\isacharparenright}\isanewline
\ \ \isacommand{type{\isacharunderscore}synonym}\isamarkupfalse%
\ ise\ {\isacharequal}\ \ \ \ {\isachardoublequoteopen}{\isacharparenleft}{\isasymzero}{\isasymRightarrow}io{\isacharparenright}{\isachardoublequoteclose}\ \ \ \ \ \ \ \ \ \ \ {\isacharparenleft}{\isachardoublequoteopen}{\isasymup}{\isasymlangle}{\isasymzero}{\isasymrangle}{\isachardoublequoteclose}{\isacharparenright}\isanewline
\ \ \isacommand{type{\isacharunderscore}synonym}\isamarkupfalse%
\ sie\ {\isacharequal}\ \ \ \ {\isachardoublequoteopen}{\isacharparenleft}{\isasymup}{\isasymzero}{\isasymRightarrow}bool{\isacharparenright}{\isachardoublequoteclose}\ \ \ \ \ \ \ \ {\isacharparenleft}{\isachardoublequoteopen}{\isasymlangle}{\isasymup}{\isasymzero}{\isasymrangle}{\isachardoublequoteclose}{\isacharparenright}\isanewline
\ \ \isacommand{type{\isacharunderscore}synonym}\isamarkupfalse%
\ isie\ {\isacharequal}\ \ \ {\isachardoublequoteopen}{\isacharparenleft}{\isasymup}{\isasymzero}{\isasymRightarrow}io{\isacharparenright}{\isachardoublequoteclose}\ \ \ \ \ \ \ \ \ {\isacharparenleft}{\isachardoublequoteopen}{\isasymup}{\isasymlangle}{\isasymup}{\isasymzero}{\isasymrangle}{\isachardoublequoteclose}{\isacharparenright}\ \ \isanewline
\ \ \isacommand{type{\isacharunderscore}synonym}\isamarkupfalse%
\ sise\ {\isacharequal}\ \ \ {\isachardoublequoteopen}{\isacharparenleft}{\isasymup}{\isasymlangle}{\isasymzero}{\isasymrangle}{\isasymRightarrow}bool{\isacharparenright}{\isachardoublequoteclose}\ \ \ \ \ {\isacharparenleft}{\isachardoublequoteopen}{\isasymlangle}{\isasymup}{\isasymlangle}{\isasymzero}{\isasymrangle}{\isasymrangle}{\isachardoublequoteclose}{\isacharparenright}\isanewline
\ \ \isacommand{type{\isacharunderscore}synonym}\isamarkupfalse%
\ isise\ {\isacharequal}\ \ {\isachardoublequoteopen}{\isacharparenleft}{\isasymup}{\isasymlangle}{\isasymzero}{\isasymrangle}{\isasymRightarrow}io{\isacharparenright}{\isachardoublequoteclose}\ \ \ \ \ \ {\isacharparenleft}{\isachardoublequoteopen}{\isasymup}{\isasymlangle}{\isasymup}{\isasymlangle}{\isasymzero}{\isasymrangle}{\isasymrangle}{\isachardoublequoteclose}{\isacharparenright}\isanewline
\ \ \isacommand{type{\isacharunderscore}synonym}\isamarkupfalse%
\ sisise{\isacharequal}\ \ {\isachardoublequoteopen}{\isacharparenleft}{\isasymup}{\isasymlangle}{\isasymup}{\isasymlangle}{\isasymzero}{\isasymrangle}{\isasymrangle}{\isasymRightarrow}bool{\isacharparenright}{\isachardoublequoteclose}\ {\isacharparenleft}{\isachardoublequoteopen}{\isasymlangle}{\isasymup}{\isasymlangle}{\isasymup}{\isasymlangle}{\isasymzero}{\isasymrangle}{\isasymrangle}{\isasymrangle}{\isachardoublequoteclose}{\isacharparenright}\isanewline
\ \ \isacommand{type{\isacharunderscore}synonym}\isamarkupfalse%
\ isisise{\isacharequal}\ {\isachardoublequoteopen}{\isacharparenleft}{\isasymup}{\isasymlangle}{\isasymup}{\isasymlangle}{\isasymzero}{\isasymrangle}{\isasymrangle}{\isasymRightarrow}io{\isacharparenright}{\isachardoublequoteclose}\ \ {\isacharparenleft}{\isachardoublequoteopen}{\isasymup}{\isasymlangle}{\isasymup}{\isasymlangle}{\isasymup}{\isasymlangle}{\isasymzero}{\isasymrangle}{\isasymrangle}{\isasymrangle}{\isachardoublequoteclose}{\isacharparenright}\isanewline
\ \ \isacommand{type{\isacharunderscore}synonym}\isamarkupfalse%
\ sse\ {\isacharequal}\ \ \ \ {\isachardoublequoteopen}{\isasymlangle}{\isasymzero}{\isasymrangle}{\isasymRightarrow}bool{\isachardoublequoteclose}\ \ \ \ \ \ \ \ \ {\isacharparenleft}{\isachardoublequoteopen}{\isasymlangle}{\isasymlangle}{\isasymzero}{\isasymrangle}{\isasymrangle}{\isachardoublequoteclose}{\isacharparenright}\isanewline
\ \ \isacommand{type{\isacharunderscore}synonym}\isamarkupfalse%
\ isse\ {\isacharequal}\ \ \ {\isachardoublequoteopen}{\isasymlangle}{\isasymzero}{\isasymrangle}{\isasymRightarrow}io{\isachardoublequoteclose}\ \ \ \ \ \ \ \ \ \ {\isacharparenleft}{\isachardoublequoteopen}{\isasymup}{\isasymlangle}{\isasymlangle}{\isasymzero}{\isasymrangle}{\isasymrangle}{\isachardoublequoteclose}{\isacharparenright}%
\begin{isamarkuptext}%
Aliases for common binary relation types:%
\end{isamarkuptext}\isamarkuptrue%
\ \ \isacommand{type{\isacharunderscore}synonym}\isamarkupfalse%
\ see\ {\isacharequal}\ \ \ \ \ \ \ \ {\isachardoublequoteopen}{\isacharparenleft}{\isasymzero}{\isasymRightarrow}{\isasymzero}{\isasymRightarrow}bool{\isacharparenright}{\isachardoublequoteclose}\ \ \ \ \ \ \ \ \ \ {\isacharparenleft}{\isachardoublequoteopen}{\isasymlangle}{\isasymzero}{\isacharcomma}{\isasymzero}{\isasymrangle}{\isachardoublequoteclose}{\isacharparenright}\isanewline
\ \ \isacommand{type{\isacharunderscore}synonym}\isamarkupfalse%
\ isee\ {\isacharequal}\ \ \ \ \ \ \ {\isachardoublequoteopen}{\isacharparenleft}{\isasymzero}{\isasymRightarrow}{\isasymzero}{\isasymRightarrow}io{\isacharparenright}{\isachardoublequoteclose}\ \ \ \ \ \ \ \ \ \ \ {\isacharparenleft}{\isachardoublequoteopen}{\isasymup}{\isasymlangle}{\isasymzero}{\isacharcomma}{\isasymzero}{\isasymrangle}{\isachardoublequoteclose}{\isacharparenright}\isanewline
\ \ \isacommand{type{\isacharunderscore}synonym}\isamarkupfalse%
\ sieie\ {\isacharequal}\ \ \ \ \ \ {\isachardoublequoteopen}{\isacharparenleft}{\isasymup}{\isasymzero}{\isasymRightarrow}{\isasymup}{\isasymzero}{\isasymRightarrow}bool{\isacharparenright}{\isachardoublequoteclose}\ \ \ \ \ \ \ {\isacharparenleft}{\isachardoublequoteopen}{\isasymlangle}{\isasymup}{\isasymzero}{\isacharcomma}{\isasymup}{\isasymzero}{\isasymrangle}{\isachardoublequoteclose}{\isacharparenright}\isanewline
\ \ \isacommand{type{\isacharunderscore}synonym}\isamarkupfalse%
\ isieie\ {\isacharequal}\ \ \ \ \ {\isachardoublequoteopen}{\isacharparenleft}{\isasymup}{\isasymzero}{\isasymRightarrow}{\isasymup}{\isasymzero}{\isasymRightarrow}io{\isacharparenright}{\isachardoublequoteclose}\ \ \ \ \ \ \ \ {\isacharparenleft}{\isachardoublequoteopen}{\isasymup}{\isasymlangle}{\isasymup}{\isasymzero}{\isacharcomma}{\isasymup}{\isasymzero}{\isasymrangle}{\isachardoublequoteclose}{\isacharparenright}\isanewline
\ \ \isacommand{type{\isacharunderscore}synonym}\isamarkupfalse%
\ ssese\ {\isacharequal}\ \ \ \ \ \ {\isachardoublequoteopen}{\isacharparenleft}{\isasymlangle}{\isasymzero}{\isasymrangle}{\isasymRightarrow}{\isasymlangle}{\isasymzero}{\isasymrangle}{\isasymRightarrow}bool{\isacharparenright}{\isachardoublequoteclose}\ \ \ \ \ {\isacharparenleft}{\isachardoublequoteopen}{\isasymlangle}{\isasymlangle}{\isasymzero}{\isasymrangle}{\isacharcomma}{\isasymlangle}{\isasymzero}{\isasymrangle}{\isasymrangle}{\isachardoublequoteclose}{\isacharparenright}\isanewline
\ \ \isacommand{type{\isacharunderscore}synonym}\isamarkupfalse%
\ issese\ {\isacharequal}\ \ \ \ \ {\isachardoublequoteopen}{\isacharparenleft}{\isasymlangle}{\isasymzero}{\isasymrangle}{\isasymRightarrow}{\isasymlangle}{\isasymzero}{\isasymrangle}{\isasymRightarrow}io{\isacharparenright}{\isachardoublequoteclose}\ \ \ \ \ \ {\isacharparenleft}{\isachardoublequoteopen}{\isasymup}{\isasymlangle}{\isasymlangle}{\isasymzero}{\isasymrangle}{\isacharcomma}{\isasymlangle}{\isasymzero}{\isasymrangle}{\isasymrangle}{\isachardoublequoteclose}{\isacharparenright}\isanewline
\ \ \isacommand{type{\isacharunderscore}synonym}\isamarkupfalse%
\ ssee\ {\isacharequal}\ \ \ \ \ \ \ {\isachardoublequoteopen}{\isacharparenleft}{\isasymlangle}{\isasymzero}{\isasymrangle}{\isasymRightarrow}{\isasymzero}{\isasymRightarrow}bool{\isacharparenright}{\isachardoublequoteclose}\ \ \ \ \ \ \ {\isacharparenleft}{\isachardoublequoteopen}{\isasymlangle}{\isasymlangle}{\isasymzero}{\isasymrangle}{\isacharcomma}{\isasymzero}{\isasymrangle}{\isachardoublequoteclose}{\isacharparenright}\isanewline
\ \ \isacommand{type{\isacharunderscore}synonym}\isamarkupfalse%
\ issee\ {\isacharequal}\ \ \ \ \ \ {\isachardoublequoteopen}{\isacharparenleft}{\isasymlangle}{\isasymzero}{\isasymrangle}{\isasymRightarrow}{\isasymzero}{\isasymRightarrow}io{\isacharparenright}{\isachardoublequoteclose}\ \ \ \ \ \ \ \ {\isacharparenleft}{\isachardoublequoteopen}{\isasymup}{\isasymlangle}{\isasymlangle}{\isasymzero}{\isasymrangle}{\isacharcomma}{\isasymzero}{\isasymrangle}{\isachardoublequoteclose}{\isacharparenright}\isanewline
\ \ \isacommand{type{\isacharunderscore}synonym}\isamarkupfalse%
\ isisee\ {\isacharequal}\ \ \ \ \ {\isachardoublequoteopen}{\isacharparenleft}{\isasymup}{\isasymlangle}{\isasymzero}{\isasymrangle}{\isasymRightarrow}{\isasymzero}{\isasymRightarrow}io{\isacharparenright}{\isachardoublequoteclose}\ \ \ \ \ \ {\isacharparenleft}{\isachardoublequoteopen}{\isasymup}{\isasymlangle}{\isasymup}{\isasymlangle}{\isasymzero}{\isasymrangle}{\isacharcomma}{\isasymzero}{\isasymrangle}{\isachardoublequoteclose}{\isacharparenright}\isanewline
\ \ \isacommand{type{\isacharunderscore}synonym}\isamarkupfalse%
\ isiseise\ {\isacharequal}\ \ \ {\isachardoublequoteopen}{\isacharparenleft}{\isasymup}{\isasymlangle}{\isasymzero}{\isasymrangle}{\isasymRightarrow}{\isasymup}{\isasymlangle}{\isasymzero}{\isasymrangle}{\isasymRightarrow}io{\isacharparenright}{\isachardoublequoteclose}\ \ \ \ {\isacharparenleft}{\isachardoublequoteopen}{\isasymup}{\isasymlangle}{\isasymup}{\isasymlangle}{\isasymzero}{\isasymrangle}{\isacharcomma}{\isasymup}{\isasymlangle}{\isasymzero}{\isasymrangle}{\isasymrangle}{\isachardoublequoteclose}{\isacharparenright}\isanewline
\ \ \isacommand{type{\isacharunderscore}synonym}\isamarkupfalse%
\ isiseisise{\isacharequal}\ \ {\isachardoublequoteopen}{\isacharparenleft}{\isasymup}{\isasymlangle}{\isasymzero}{\isasymrangle}{\isasymRightarrow}{\isasymup}{\isasymlangle}{\isasymup}{\isasymlangle}{\isasymzero}{\isasymrangle}{\isasymrangle}{\isasymRightarrow}io{\isacharparenright}{\isachardoublequoteclose}\ {\isacharparenleft}{\isachardoublequoteopen}{\isasymup}{\isasymlangle}{\isasymup}{\isasymlangle}{\isasymzero}{\isasymrangle}{\isacharcomma}{\isasymup}{\isasymlangle}{\isasymup}{\isasymlangle}{\isasymzero}{\isasymrangle}{\isasymrangle}{\isasymrangle}{\isachardoublequoteclose}{\isacharparenright}\isanewline
\ \ \isanewline
\ \ \isacommand{consts}\isamarkupfalse%
\ aRel{\isacharcolon}{\isacharcolon}{\isachardoublequoteopen}i{\isasymRightarrow}i{\isasymRightarrow}bool{\isachardoublequoteclose}\ {\isacharparenleft}\isakeyword{infixr}\ {\isachardoublequoteopen}r{\isachardoublequoteclose}\ {\isadigit{7}}{\isadigit{0}}{\isacharparenright}\ \ %
\isamarkupcmt{Accessibility relation%
}
%
\isamarkupsubsection{Definition of Logical Operators%
}
\isamarkuptrue%
\ \ \isacommand{abbreviation}\isamarkupfalse%
\ mnot\ \ \ {\isacharcolon}{\isacharcolon}\ {\isachardoublequoteopen}io{\isasymRightarrow}io{\isachardoublequoteclose}\ {\isacharparenleft}{\isachardoublequoteopen}\isactrlbold {\isasymnot}{\isacharunderscore}{\isachardoublequoteclose}{\isacharbrackleft}{\isadigit{5}}{\isadigit{2}}{\isacharbrackright}{\isadigit{5}}{\isadigit{3}}{\isacharparenright}\isanewline
\ \ \ \ \isakeyword{where}\ {\isachardoublequoteopen}\isactrlbold {\isasymnot}{\isasymphi}\ {\isasymequiv}\ {\isasymlambda}w{\isachardot}\ {\isasymnot}{\isacharparenleft}{\isasymphi}\ w{\isacharparenright}{\isachardoublequoteclose}\ \isanewline
\ \ \isacommand{abbreviation}\isamarkupfalse%
\ mand\ \ \ {\isacharcolon}{\isacharcolon}\ {\isachardoublequoteopen}io{\isasymRightarrow}io{\isasymRightarrow}io{\isachardoublequoteclose}\ {\isacharparenleft}\isakeyword{infixr}{\isachardoublequoteopen}\isactrlbold {\isasymand}{\isachardoublequoteclose}{\isadigit{5}}{\isadigit{1}}{\isacharparenright}\isanewline
\ \ \ \ \isakeyword{where}\ {\isachardoublequoteopen}{\isasymphi}\isactrlbold {\isasymand}{\isasympsi}\ {\isasymequiv}\ {\isasymlambda}w{\isachardot}\ {\isacharparenleft}{\isasymphi}\ w{\isacharparenright}{\isasymand}{\isacharparenleft}{\isasympsi}\ w{\isacharparenright}{\isachardoublequoteclose}\ \ \ \isanewline
\ \ \isacommand{abbreviation}\isamarkupfalse%
\ mor\ \ \ \ {\isacharcolon}{\isacharcolon}\ {\isachardoublequoteopen}io{\isasymRightarrow}io{\isasymRightarrow}io{\isachardoublequoteclose}\ {\isacharparenleft}\isakeyword{infixr}{\isachardoublequoteopen}\isactrlbold {\isasymor}{\isachardoublequoteclose}{\isadigit{5}}{\isadigit{0}}{\isacharparenright}\isanewline
\ \ \ \ \isakeyword{where}\ {\isachardoublequoteopen}{\isasymphi}\isactrlbold {\isasymor}{\isasympsi}\ {\isasymequiv}\ {\isasymlambda}w{\isachardot}\ {\isacharparenleft}{\isasymphi}\ w{\isacharparenright}{\isasymor}{\isacharparenleft}{\isasympsi}\ w{\isacharparenright}{\isachardoublequoteclose}\isanewline
\ \ \isacommand{abbreviation}\isamarkupfalse%
\ mimp\ \ \ {\isacharcolon}{\isacharcolon}\ {\isachardoublequoteopen}io{\isasymRightarrow}io{\isasymRightarrow}io{\isachardoublequoteclose}\ {\isacharparenleft}\isakeyword{infixr}{\isachardoublequoteopen}\isactrlbold {\isasymrightarrow}{\isachardoublequoteclose}{\isadigit{4}}{\isadigit{9}}{\isacharparenright}\ \isanewline
\ \ \ \ \isakeyword{where}\ {\isachardoublequoteopen}{\isasymphi}\isactrlbold {\isasymrightarrow}{\isasympsi}\ {\isasymequiv}\ {\isasymlambda}w{\isachardot}\ {\isacharparenleft}{\isasymphi}\ w{\isacharparenright}{\isasymlongrightarrow}{\isacharparenleft}{\isasympsi}\ w{\isacharparenright}{\isachardoublequoteclose}\ \ \isanewline
\ \ \isacommand{abbreviation}\isamarkupfalse%
\ mequ\ \ \ {\isacharcolon}{\isacharcolon}\ {\isachardoublequoteopen}io{\isasymRightarrow}io{\isasymRightarrow}io{\isachardoublequoteclose}\ {\isacharparenleft}\isakeyword{infixr}{\isachardoublequoteopen}\isactrlbold {\isasymleftrightarrow}{\isachardoublequoteclose}{\isadigit{4}}{\isadigit{8}}{\isacharparenright}\isanewline
\ \ \ \ \isakeyword{where}\ {\isachardoublequoteopen}{\isasymphi}\isactrlbold {\isasymleftrightarrow}{\isasympsi}\ {\isasymequiv}\ {\isasymlambda}w{\isachardot}\ {\isacharparenleft}{\isasymphi}\ w{\isacharparenright}{\isasymlongleftrightarrow}{\isacharparenleft}{\isasympsi}\ w{\isacharparenright}{\isachardoublequoteclose}\isanewline
\ \ \isacommand{abbreviation}\isamarkupfalse%
\ xor{\isacharcolon}{\isacharcolon}\ {\isachardoublequoteopen}bool{\isasymRightarrow}bool{\isasymRightarrow}bool{\isachardoublequoteclose}\ {\isacharparenleft}\isakeyword{infixr}{\isachardoublequoteopen}{\isasymoplus}{\isachardoublequoteclose}{\isadigit{5}}{\isadigit{0}}{\isacharparenright}\isanewline
\ \ \ \ \isakeyword{where}\ {\isachardoublequoteopen}{\isasymphi}{\isasymoplus}{\isasympsi}\ {\isasymequiv}\ \ {\isacharparenleft}{\isasymphi}{\isasymor}{\isasympsi}{\isacharparenright}\ {\isasymand}\ {\isasymnot}{\isacharparenleft}{\isasymphi}{\isasymand}{\isasympsi}{\isacharparenright}{\isachardoublequoteclose}\ \isanewline
\ \ \isacommand{abbreviation}\isamarkupfalse%
\ mxor\ \ \ {\isacharcolon}{\isacharcolon}\ {\isachardoublequoteopen}io{\isasymRightarrow}io{\isasymRightarrow}io{\isachardoublequoteclose}\ {\isacharparenleft}\isakeyword{infixr}{\isachardoublequoteopen}\isactrlbold {\isasymoplus}{\isachardoublequoteclose}{\isadigit{5}}{\isadigit{0}}{\isacharparenright}\isanewline
\ \ \ \ \isakeyword{where}\ {\isachardoublequoteopen}{\isasymphi}\isactrlbold {\isasymoplus}{\isasympsi}\ {\isasymequiv}\ {\isasymlambda}w{\isachardot}\ {\isacharparenleft}{\isasymphi}\ w{\isacharparenright}{\isasymoplus}{\isacharparenleft}{\isasympsi}\ w{\isacharparenright}{\isachardoublequoteclose}%
\isamarkupsubsection{Definition of Posibilist Quantifiers%
}
\isamarkuptrue%
\ \ \isacommand{abbreviation}\isamarkupfalse%
\ mforall\ \ \ {\isacharcolon}{\isacharcolon}\ {\isachardoublequoteopen}{\isacharparenleft}{\isacharprime}t{\isasymRightarrow}io{\isacharparenright}{\isasymRightarrow}io{\isachardoublequoteclose}\ {\isacharparenleft}{\isachardoublequoteopen}\isactrlbold {\isasymforall}{\isachardoublequoteclose}{\isacharparenright}\ \ \ \ \ \ \isanewline
\ \ \ \ \isakeyword{where}\ {\isachardoublequoteopen}\isactrlbold {\isasymforall}{\isasymPhi}\ {\isasymequiv}\ {\isasymlambda}w{\isachardot}{\isasymforall}x{\isachardot}\ {\isacharparenleft}{\isasymPhi}\ x\ w{\isacharparenright}{\isachardoublequoteclose}\isanewline
\ \ \isacommand{abbreviation}\isamarkupfalse%
\ mexists\ \ \ {\isacharcolon}{\isacharcolon}\ {\isachardoublequoteopen}{\isacharparenleft}{\isacharprime}t{\isasymRightarrow}io{\isacharparenright}{\isasymRightarrow}io{\isachardoublequoteclose}\ {\isacharparenleft}{\isachardoublequoteopen}\isactrlbold {\isasymexists}{\isachardoublequoteclose}{\isacharparenright}\ \isanewline
\ \ \ \ \isakeyword{where}\ {\isachardoublequoteopen}\isactrlbold {\isasymexists}{\isasymPhi}\ {\isasymequiv}\ {\isasymlambda}w{\isachardot}{\isasymexists}x{\isachardot}\ {\isacharparenleft}{\isasymPhi}\ x\ w{\isacharparenright}{\isachardoublequoteclose}\isanewline
\ \ \ \ \ \ \isanewline
\ \ \isacommand{abbreviation}\isamarkupfalse%
\ mforallB\ \ {\isacharcolon}{\isacharcolon}\ {\isachardoublequoteopen}{\isacharparenleft}{\isacharprime}t{\isasymRightarrow}io{\isacharparenright}{\isasymRightarrow}io{\isachardoublequoteclose}\ {\isacharparenleft}\isakeyword{binder}{\isachardoublequoteopen}\isactrlbold {\isasymforall}{\isachardoublequoteclose}{\isacharbrackleft}{\isadigit{8}}{\isacharbrackright}{\isadigit{9}}{\isacharparenright}\ %
\isamarkupcmt{Binder notation%
}
\isanewline
\ \ \ \ \isakeyword{where}\ {\isachardoublequoteopen}\isactrlbold {\isasymforall}x{\isachardot}\ {\isasymphi}{\isacharparenleft}x{\isacharparenright}\ {\isasymequiv}\ \isactrlbold {\isasymforall}{\isasymphi}{\isachardoublequoteclose}\ \ \isanewline
\ \ \isacommand{abbreviation}\isamarkupfalse%
\ mexistsB\ \ {\isacharcolon}{\isacharcolon}\ {\isachardoublequoteopen}{\isacharparenleft}{\isacharprime}t{\isasymRightarrow}io{\isacharparenright}{\isasymRightarrow}io{\isachardoublequoteclose}\ {\isacharparenleft}\isakeyword{binder}{\isachardoublequoteopen}\isactrlbold {\isasymexists}{\isachardoublequoteclose}{\isacharbrackleft}{\isadigit{8}}{\isacharbrackright}{\isadigit{9}}{\isacharparenright}\isanewline
\ \ \ \ \isakeyword{where}\ {\isachardoublequoteopen}\isactrlbold {\isasymexists}x{\isachardot}\ {\isasymphi}{\isacharparenleft}x{\isacharparenright}\ {\isasymequiv}\ \isactrlbold {\isasymexists}{\isasymphi}{\isachardoublequoteclose}%
\isamarkupsubsection{Definition of Actualist Quantifiers%
}
\isamarkuptrue%
%
\begin{isamarkuptext}%
The following predicate is used to model actualist quantifiers by restricting domains of quantification.
Note that since this is a meta-logical concept we never use it in our object language.%
\end{isamarkuptext}\isamarkuptrue%
\ \ \isacommand{consts}\isamarkupfalse%
\ Exists{\isacharcolon}{\isacharcolon}{\isachardoublequoteopen}{\isasymup}{\isasymlangle}{\isasymzero}{\isasymrangle}{\isachardoublequoteclose}\ {\isacharparenleft}{\isachardoublequoteopen}existsAt{\isachardoublequoteclose}{\isacharparenright}%
\begin{isamarkuptext}%
Note that no polymorphic types are needed in the definitions since actualist quantification only makes sense for individuals.%
\end{isamarkuptext}\isamarkuptrue%
\ \ \isacommand{abbreviation}\isamarkupfalse%
\ mforallAct\ \ \ {\isacharcolon}{\isacharcolon}\ {\isachardoublequoteopen}{\isasymup}{\isasymlangle}{\isasymup}{\isasymlangle}{\isasymzero}{\isasymrangle}{\isasymrangle}{\isachardoublequoteclose}\ {\isacharparenleft}{\isachardoublequoteopen}\isactrlbold {\isasymforall}\isactrlsup E{\isachardoublequoteclose}{\isacharparenright}\ \ \ \ \ \ \isanewline
\ \ \ \ \isakeyword{where}\ {\isachardoublequoteopen}\isactrlbold {\isasymforall}\isactrlsup E{\isasymPhi}\ {\isasymequiv}\ {\isasymlambda}w{\isachardot}{\isasymforall}x{\isachardot}\ {\isacharparenleft}existsAt\ x\ w{\isacharparenright}{\isasymlongrightarrow}{\isacharparenleft}{\isasymPhi}\ x\ w{\isacharparenright}{\isachardoublequoteclose}\isanewline
\ \ \isacommand{abbreviation}\isamarkupfalse%
\ mexistsAct\ \ \ {\isacharcolon}{\isacharcolon}\ {\isachardoublequoteopen}{\isasymup}{\isasymlangle}{\isasymup}{\isasymlangle}{\isasymzero}{\isasymrangle}{\isasymrangle}{\isachardoublequoteclose}\ {\isacharparenleft}{\isachardoublequoteopen}\isactrlbold {\isasymexists}\isactrlsup E{\isachardoublequoteclose}{\isacharparenright}\ \isanewline
\ \ \ \ \isakeyword{where}\ {\isachardoublequoteopen}\isactrlbold {\isasymexists}\isactrlsup E{\isasymPhi}\ {\isasymequiv}\ {\isasymlambda}w{\isachardot}{\isasymexists}x{\isachardot}\ {\isacharparenleft}existsAt\ x\ w{\isacharparenright}\ {\isasymand}\ {\isacharparenleft}{\isasymPhi}\ x\ w{\isacharparenright}{\isachardoublequoteclose}\isanewline
\isanewline
\ \ \isacommand{abbreviation}\isamarkupfalse%
\ mforallActB\ \ {\isacharcolon}{\isacharcolon}\ {\isachardoublequoteopen}{\isasymup}{\isasymlangle}{\isasymup}{\isasymlangle}{\isasymzero}{\isasymrangle}{\isasymrangle}{\isachardoublequoteclose}\ {\isacharparenleft}\isakeyword{binder}{\isachardoublequoteopen}\isactrlbold {\isasymforall}\isactrlsup E{\isachardoublequoteclose}{\isacharbrackleft}{\isadigit{8}}{\isacharbrackright}{\isadigit{9}}{\isacharparenright}\ %
\isamarkupcmt{Binder notation%
}
\isanewline
\ \ \ \ \isakeyword{where}\ {\isachardoublequoteopen}\isactrlbold {\isasymforall}\isactrlsup Ex{\isachardot}\ {\isasymphi}{\isacharparenleft}x{\isacharparenright}\ {\isasymequiv}\ \isactrlbold {\isasymforall}\isactrlsup E{\isasymphi}{\isachardoublequoteclose}\ \ \ \ \ \isanewline
\ \ \isacommand{abbreviation}\isamarkupfalse%
\ mexistsActB\ \ {\isacharcolon}{\isacharcolon}\ {\isachardoublequoteopen}{\isasymup}{\isasymlangle}{\isasymup}{\isasymlangle}{\isasymzero}{\isasymrangle}{\isasymrangle}{\isachardoublequoteclose}\ {\isacharparenleft}\isakeyword{binder}{\isachardoublequoteopen}\isactrlbold {\isasymexists}\isactrlsup E{\isachardoublequoteclose}{\isacharbrackleft}{\isadigit{8}}{\isacharbrackright}{\isadigit{9}}{\isacharparenright}\isanewline
\ \ \ \ \isakeyword{where}\ {\isachardoublequoteopen}\isactrlbold {\isasymexists}\isactrlsup Ex{\isachardot}\ {\isasymphi}{\isacharparenleft}x{\isacharparenright}\ {\isasymequiv}\ \isactrlbold {\isasymexists}\isactrlsup E{\isasymphi}{\isachardoublequoteclose}%
\isamarkupsubsection{Definition of Modal Operators%
}
\isamarkuptrue%
\ \ \isacommand{abbreviation}\isamarkupfalse%
\ mbox\ \ \ {\isacharcolon}{\isacharcolon}\ {\isachardoublequoteopen}io{\isasymRightarrow}io{\isachardoublequoteclose}\ {\isacharparenleft}{\isachardoublequoteopen}\isactrlbold {\isasymbox}{\isacharunderscore}{\isachardoublequoteclose}{\isacharbrackleft}{\isadigit{5}}{\isadigit{2}}{\isacharbrackright}{\isadigit{5}}{\isadigit{3}}{\isacharparenright}\isanewline
\ \ \ \ \isakeyword{where}\ {\isachardoublequoteopen}\isactrlbold {\isasymbox}{\isasymphi}\ {\isasymequiv}\ {\isasymlambda}w{\isachardot}{\isasymforall}v{\isachardot}\ {\isacharparenleft}w\ r\ v{\isacharparenright}{\isasymlongrightarrow}{\isacharparenleft}{\isasymphi}\ v{\isacharparenright}{\isachardoublequoteclose}\isanewline
\ \ \isacommand{abbreviation}\isamarkupfalse%
\ mdia\ \ \ {\isacharcolon}{\isacharcolon}\ {\isachardoublequoteopen}io{\isasymRightarrow}io{\isachardoublequoteclose}\ {\isacharparenleft}{\isachardoublequoteopen}\isactrlbold {\isasymdiamond}{\isacharunderscore}{\isachardoublequoteclose}{\isacharbrackleft}{\isadigit{5}}{\isadigit{2}}{\isacharbrackright}{\isadigit{5}}{\isadigit{3}}{\isacharparenright}\isanewline
\ \ \ \ \isakeyword{where}\ {\isachardoublequoteopen}\isactrlbold {\isasymdiamond}{\isasymphi}\ {\isasymequiv}\ {\isasymlambda}w{\isachardot}{\isasymexists}v{\isachardot}\ {\isacharparenleft}w\ r\ v{\isacharparenright}{\isasymand}{\isacharparenleft}{\isasymphi}\ v{\isacharparenright}{\isachardoublequoteclose}%
\isamarkupsubsection{Definition of the \isa{extension{\isacharunderscore}of} Operator%
}
\isamarkuptrue%
%
\begin{isamarkuptext}%
In contrast to the approach taken in Fitting's book (p. 88), the \isa{{\isasymdown}} operator is embedded as a binary operator
 applying to (world-dependent) atomic formulas whose first argument is a `relativized' term (preceded by \isa{{\isasymdown}}).
 Depending on the types involved we need to define this operator differently to ensure type correctness.%
\end{isamarkuptext}\isamarkuptrue%
%
\begin{isamarkuptext}%
(a) Predicate \isa{{\isasymphi}} takes an (intensional) individual concept as argument:%
\end{isamarkuptext}\isamarkuptrue%
\isacommand{abbreviation}\isamarkupfalse%
\ mextIndiv{\isacharcolon}{\isacharcolon}{\isachardoublequoteopen}{\isasymup}{\isasymlangle}{\isasymzero}{\isasymrangle}{\isasymRightarrow}{\isasymup}{\isasymzero}{\isasymRightarrow}io{\isachardoublequoteclose}\ {\isacharparenleft}\isakeyword{infix}\ {\isachardoublequoteopen}\isactrlbold {\isasymdownharpoonleft}{\isachardoublequoteclose}\ {\isadigit{6}}{\isadigit{0}}{\isacharparenright}\ \ \ \ \ \ \ \ \ \ \ \ \ \ \ \ \ \ \ \ \ \ \ \ \ \ \ \ \ \isanewline
\ \ \isakeyword{where}\ {\isachardoublequoteopen}{\isasymphi}\ \isactrlbold {\isasymdownharpoonleft}c\ {\isasymequiv}\ {\isasymlambda}w{\isachardot}\ {\isasymphi}\ {\isacharparenleft}c\ w{\isacharparenright}\ w{\isachardoublequoteclose}%
\begin{isamarkuptext}%
(b) Predicate \isa{{\isasymphi}} takes an intensional predicate as argument:%
\end{isamarkuptext}\isamarkuptrue%
\isacommand{abbreviation}\isamarkupfalse%
\ mextPredArg{\isacharcolon}{\isacharcolon}{\isachardoublequoteopen}{\isacharparenleft}{\isacharparenleft}{\isacharprime}t{\isasymRightarrow}io{\isacharparenright}{\isasymRightarrow}io{\isacharparenright}{\isasymRightarrow}{\isacharparenleft}{\isacharprime}t{\isasymRightarrow}io{\isacharparenright}{\isasymRightarrow}io{\isachardoublequoteclose}\ {\isacharparenleft}\isakeyword{infix}\ {\isachardoublequoteopen}\isactrlbold {\isasymdown}{\isachardoublequoteclose}\ {\isadigit{6}}{\isadigit{0}}{\isacharparenright}\isanewline
\ \ \isakeyword{where}\ {\isachardoublequoteopen}{\isasymphi}\ \isactrlbold {\isasymdown}P\ {\isasymequiv}\ {\isasymlambda}w{\isachardot}\ {\isasymphi}\ {\isacharparenleft}{\isasymlambda}x\ u{\isachardot}\ P\ x\ w{\isacharparenright}\ w{\isachardoublequoteclose}%
\begin{isamarkuptext}%
(c) Predicate \isa{{\isasymphi}} takes an extensional predicate as argument:%
\end{isamarkuptext}\isamarkuptrue%
\isacommand{abbreviation}\isamarkupfalse%
\ extPredArg{\isacharcolon}{\isacharcolon}{\isachardoublequoteopen}{\isacharparenleft}{\isacharparenleft}{\isacharprime}t{\isasymRightarrow}bool{\isacharparenright}{\isasymRightarrow}io{\isacharparenright}{\isasymRightarrow}{\isacharparenleft}{\isacharprime}t{\isasymRightarrow}io{\isacharparenright}{\isasymRightarrow}io{\isachardoublequoteclose}\ {\isacharparenleft}\isakeyword{infix}\ {\isachardoublequoteopen}{\isasymdown}{\isachardoublequoteclose}\ {\isadigit{6}}{\isadigit{0}}{\isacharparenright}\isanewline
\ \ \isakeyword{where}\ {\isachardoublequoteopen}{\isasymphi}\ {\isasymdown}P\ {\isasymequiv}\ {\isasymlambda}w{\isachardot}\ {\isasymphi}\ {\isacharparenleft}{\isasymlambda}x{\isachardot}\ P\ x\ w{\isacharparenright}\ w{\isachardoublequoteclose}%
\begin{isamarkuptext}%
(d) Predicate \isa{{\isasymphi}} takes an extensional predicate as first argument:%
\end{isamarkuptext}\isamarkuptrue%
\isacommand{abbreviation}\isamarkupfalse%
\ extPredArg{\isadigit{1}}{\isacharcolon}{\isacharcolon}{\isachardoublequoteopen}{\isacharparenleft}{\isacharparenleft}{\isacharprime}t{\isasymRightarrow}bool{\isacharparenright}{\isasymRightarrow}{\isacharprime}b{\isasymRightarrow}io{\isacharparenright}{\isasymRightarrow}{\isacharparenleft}{\isacharprime}t{\isasymRightarrow}io{\isacharparenright}{\isasymRightarrow}{\isacharprime}b{\isasymRightarrow}io{\isachardoublequoteclose}\ {\isacharparenleft}\isakeyword{infix}\ {\isachardoublequoteopen}{\isasymdown}\isactrlsub {\isadigit{1}}{\isachardoublequoteclose}\ {\isadigit{6}}{\isadigit{0}}{\isacharparenright}\isanewline
\ \ \isakeyword{where}\ {\isachardoublequoteopen}{\isasymphi}\ {\isasymdown}\isactrlsub {\isadigit{1}}P\ {\isasymequiv}\ {\isasymlambda}z{\isachardot}\ {\isasymlambda}w{\isachardot}\ {\isasymphi}\ {\isacharparenleft}{\isasymlambda}x{\isachardot}\ P\ x\ w{\isacharparenright}\ z\ w{\isachardoublequoteclose}%
\isamarkupsubsection{Definition of Equality%
}
\isamarkuptrue%
\ \ \isacommand{abbreviation}\isamarkupfalse%
\ meq\ \ \ \ {\isacharcolon}{\isacharcolon}\ {\isachardoublequoteopen}{\isacharprime}t{\isasymRightarrow}{\isacharprime}t{\isasymRightarrow}io{\isachardoublequoteclose}\ {\isacharparenleft}\isakeyword{infix}{\isachardoublequoteopen}\isactrlbold {\isasymapprox}{\isachardoublequoteclose}{\isadigit{6}}{\isadigit{0}}{\isacharparenright}\ %
\isamarkupcmt{normal equality (for all types)%
}
\isanewline
\ \ \ \ \isakeyword{where}\ {\isachardoublequoteopen}x\ \isactrlbold {\isasymapprox}\ y\ {\isasymequiv}\ {\isasymlambda}w{\isachardot}\ x\ {\isacharequal}\ y{\isachardoublequoteclose}\isanewline
\ \ \isacommand{abbreviation}\isamarkupfalse%
\ meqC\ \ \ {\isacharcolon}{\isacharcolon}\ {\isachardoublequoteopen}{\isasymup}{\isasymlangle}{\isasymup}{\isasymzero}{\isacharcomma}{\isasymup}{\isasymzero}{\isasymrangle}{\isachardoublequoteclose}\ {\isacharparenleft}\isakeyword{infixr}{\isachardoublequoteopen}\isactrlbold {\isasymapprox}\isactrlsup C{\isachardoublequoteclose}{\isadigit{5}}{\isadigit{2}}{\isacharparenright}\ %
\isamarkupcmt{eq. for individual concepts%
}
\isanewline
\ \ \ \ \isakeyword{where}\ {\isachardoublequoteopen}x\ \isactrlbold {\isasymapprox}\isactrlsup C\ y\ {\isasymequiv}\ {\isasymlambda}w{\isachardot}\ {\isasymforall}v{\isachardot}\ {\isacharparenleft}x\ v{\isacharparenright}\ {\isacharequal}\ {\isacharparenleft}y\ v{\isacharparenright}{\isachardoublequoteclose}\isanewline
\ \ \isacommand{abbreviation}\isamarkupfalse%
\ meqL\ \ \ {\isacharcolon}{\isacharcolon}\ {\isachardoublequoteopen}{\isasymup}{\isasymlangle}{\isasymzero}{\isacharcomma}{\isasymzero}{\isasymrangle}{\isachardoublequoteclose}\ {\isacharparenleft}\isakeyword{infixr}{\isachardoublequoteopen}\isactrlbold {\isasymapprox}\isactrlsup L{\isachardoublequoteclose}{\isadigit{5}}{\isadigit{2}}{\isacharparenright}\ %
\isamarkupcmt{Leibniz eq. for individuals%
}
\isanewline
\ \ \ \ \isakeyword{where}\ {\isachardoublequoteopen}x\ \isactrlbold {\isasymapprox}\isactrlsup L\ y\ {\isasymequiv}\ \isactrlbold {\isasymforall}{\isasymphi}{\isachardot}\ {\isasymphi}{\isacharparenleft}x{\isacharparenright}\isactrlbold {\isasymrightarrow}{\isasymphi}{\isacharparenleft}y{\isacharparenright}{\isachardoublequoteclose}%
\isamarkupsubsection{Miscellaneous%
}
\isamarkuptrue%
\ \ \isacommand{abbreviation}\isamarkupfalse%
\ negpred\ {\isacharcolon}{\isacharcolon}\ {\isachardoublequoteopen}{\isasymlangle}{\isasymzero}{\isasymrangle}{\isasymRightarrow}{\isasymlangle}{\isasymzero}{\isasymrangle}{\isachardoublequoteclose}\ {\isacharparenleft}{\isachardoublequoteopen}{\isasymrightharpoondown}{\isacharunderscore}{\isachardoublequoteclose}{\isacharbrackleft}{\isadigit{5}}{\isadigit{2}}{\isacharbrackright}{\isadigit{5}}{\isadigit{3}}{\isacharparenright}\ \isanewline
\ \ \ \ \isakeyword{where}\ {\isachardoublequoteopen}{\isasymrightharpoondown}{\isasymPhi}\ {\isasymequiv}\ {\isasymlambda}x{\isachardot}\ {\isasymnot}{\isacharparenleft}{\isasymPhi}\ x{\isacharparenright}{\isachardoublequoteclose}\ \isanewline
\ \ \isacommand{abbreviation}\isamarkupfalse%
\ mnegpred\ {\isacharcolon}{\isacharcolon}\ {\isachardoublequoteopen}{\isasymup}{\isasymlangle}{\isasymzero}{\isasymrangle}{\isasymRightarrow}{\isasymup}{\isasymlangle}{\isasymzero}{\isasymrangle}{\isachardoublequoteclose}\ {\isacharparenleft}{\isachardoublequoteopen}\isactrlbold {\isasymrightharpoondown}{\isacharunderscore}{\isachardoublequoteclose}{\isacharbrackleft}{\isadigit{5}}{\isadigit{2}}{\isacharbrackright}{\isadigit{5}}{\isadigit{3}}{\isacharparenright}\ \isanewline
\ \ \ \ \isakeyword{where}\ {\isachardoublequoteopen}\isactrlbold {\isasymrightharpoondown}{\isasymPhi}\ {\isasymequiv}\ {\isasymlambda}x{\isachardot}{\isasymlambda}w{\isachardot}\ {\isasymnot}{\isacharparenleft}{\isasymPhi}\ x\ w{\isacharparenright}{\isachardoublequoteclose}\ \isanewline
\ \ \isacommand{abbreviation}\isamarkupfalse%
\ mandpred\ {\isacharcolon}{\isacharcolon}\ {\isachardoublequoteopen}{\isasymup}{\isasymlangle}{\isasymzero}{\isasymrangle}{\isasymRightarrow}{\isasymup}{\isasymlangle}{\isasymzero}{\isasymrangle}{\isasymRightarrow}{\isasymup}{\isasymlangle}{\isasymzero}{\isasymrangle}{\isachardoublequoteclose}\ {\isacharparenleft}\isakeyword{infix}\ {\isachardoublequoteopen}\isactrlbold {\isacharampersand}{\isachardoublequoteclose}\ {\isadigit{5}}{\isadigit{3}}{\isacharparenright}\ \isanewline
\ \ \ \ \isakeyword{where}\ {\isachardoublequoteopen}{\isasymPhi}\ \isactrlbold {\isacharampersand}\ {\isasymphi}\ {\isasymequiv}\ {\isasymlambda}x{\isachardot}{\isasymlambda}w{\isachardot}\ {\isacharparenleft}{\isasymPhi}\ x\ w{\isacharparenright}\ {\isasymand}\ {\isacharparenleft}{\isasymphi}\ x\ w{\isacharparenright}{\isachardoublequoteclose}%
\isamarkupsubsection{Meta-logical Predicates%
}
\isamarkuptrue%
\ \isacommand{abbreviation}\isamarkupfalse%
\ valid\ {\isacharcolon}{\isacharcolon}\ {\isachardoublequoteopen}io{\isasymRightarrow}bool{\isachardoublequoteclose}\ {\isacharparenleft}{\isachardoublequoteopen}{\isasymlfloor}{\isacharunderscore}{\isasymrfloor}{\isachardoublequoteclose}\ {\isacharbrackleft}{\isadigit{8}}{\isacharbrackright}{\isacharparenright}\ \isakeyword{where}\ {\isachardoublequoteopen}{\isasymlfloor}{\isasympsi}{\isasymrfloor}\ {\isasymequiv}\ \ {\isasymforall}w{\isachardot}{\isacharparenleft}{\isasympsi}\ w{\isacharparenright}{\isachardoublequoteclose}\isanewline
\ \isacommand{abbreviation}\isamarkupfalse%
\ satisfiable\ {\isacharcolon}{\isacharcolon}\ {\isachardoublequoteopen}io{\isasymRightarrow}bool{\isachardoublequoteclose}\ {\isacharparenleft}{\isachardoublequoteopen}{\isasymlfloor}{\isacharunderscore}{\isasymrfloor}\isactrlsup s\isactrlsup a\isactrlsup t{\isachardoublequoteclose}\ {\isacharbrackleft}{\isadigit{8}}{\isacharbrackright}{\isacharparenright}\ \isakeyword{where}\ {\isachardoublequoteopen}{\isasymlfloor}{\isasympsi}{\isasymrfloor}\isactrlsup s\isactrlsup a\isactrlsup t\ {\isasymequiv}\ {\isasymexists}w{\isachardot}{\isacharparenleft}{\isasympsi}\ w{\isacharparenright}{\isachardoublequoteclose}\isanewline
\ \isacommand{abbreviation}\isamarkupfalse%
\ countersat\ {\isacharcolon}{\isacharcolon}\ {\isachardoublequoteopen}io{\isasymRightarrow}bool{\isachardoublequoteclose}\ {\isacharparenleft}{\isachardoublequoteopen}{\isasymlfloor}{\isacharunderscore}{\isasymrfloor}\isactrlsup c\isactrlsup s\isactrlsup a\isactrlsup t{\isachardoublequoteclose}\ {\isacharbrackleft}{\isadigit{8}}{\isacharbrackright}{\isacharparenright}\ \isakeyword{where}\ {\isachardoublequoteopen}{\isasymlfloor}{\isasympsi}{\isasymrfloor}\isactrlsup c\isactrlsup s\isactrlsup a\isactrlsup t\ {\isasymequiv}\ \ {\isasymexists}w{\isachardot}{\isasymnot}{\isacharparenleft}{\isasympsi}\ w{\isacharparenright}{\isachardoublequoteclose}\isanewline
\ \isacommand{abbreviation}\isamarkupfalse%
\ invalid\ {\isacharcolon}{\isacharcolon}\ {\isachardoublequoteopen}io{\isasymRightarrow}bool{\isachardoublequoteclose}\ {\isacharparenleft}{\isachardoublequoteopen}{\isasymlfloor}{\isacharunderscore}{\isasymrfloor}\isactrlsup i\isactrlsup n\isactrlsup v{\isachardoublequoteclose}\ {\isacharbrackleft}{\isadigit{8}}{\isacharbrackright}{\isacharparenright}\ \isakeyword{where}\ {\isachardoublequoteopen}{\isasymlfloor}{\isasympsi}{\isasymrfloor}\isactrlsup i\isactrlsup n\isactrlsup v\ {\isasymequiv}\ {\isasymforall}w{\isachardot}{\isasymnot}{\isacharparenleft}{\isasympsi}\ w{\isacharparenright}{\isachardoublequoteclose}%
\isamarkupsubsection{Verifying the Embedding%
}
\isamarkuptrue%
%
\begin{isamarkuptext}%
Verifying K Principle and Necessitation:%
\end{isamarkuptext}\isamarkuptrue%
\ \isacommand{lemma}\isamarkupfalse%
\ K{\isacharcolon}\ {\isachardoublequoteopen}{\isasymlfloor}{\isacharparenleft}\isactrlbold {\isasymbox}{\isacharparenleft}{\isasymphi}\ \isactrlbold {\isasymrightarrow}\ {\isasympsi}{\isacharparenright}{\isacharparenright}\ \isactrlbold {\isasymrightarrow}\ {\isacharparenleft}\isactrlbold {\isasymbox}{\isasymphi}\ \isactrlbold {\isasymrightarrow}\ \isactrlbold {\isasymbox}{\isasympsi}{\isacharparenright}{\isasymrfloor}{\isachardoublequoteclose}%
\isadelimproof
\ %
\endisadelimproof
%
\isatagproof
\isacommand{by}\isamarkupfalse%
\ simp\ \ \ \ %
\isamarkupcmt{K Schema%
}
%
\endisatagproof
{\isafoldproof}%
%
\isadelimproof
%
\endisadelimproof
\isanewline
\ \isacommand{lemma}\isamarkupfalse%
\ NEC{\isacharcolon}\ {\isachardoublequoteopen}{\isasymlfloor}{\isasymphi}{\isasymrfloor}\ {\isasymLongrightarrow}\ {\isasymlfloor}\isactrlbold {\isasymbox}{\isasymphi}{\isasymrfloor}{\isachardoublequoteclose}%
\isadelimproof
\ %
\endisadelimproof
%
\isatagproof
\isacommand{by}\isamarkupfalse%
\ simp\ \ \ \ %
\isamarkupcmt{Necessitation%
}
%
\endisatagproof
{\isafoldproof}%
%
\isadelimproof
%
\endisadelimproof
%
\begin{isamarkuptext}%
Barcan and Converse Barcan Formulas are satisfied for standard (possibilist) quantifiers:%
\end{isamarkuptext}\isamarkuptrue%
\ \isacommand{lemma}\isamarkupfalse%
\ {\isachardoublequoteopen}{\isasymlfloor}{\isacharparenleft}\isactrlbold {\isasymforall}x{\isachardot}\isactrlbold {\isasymbox}{\isacharparenleft}{\isasymphi}\ x{\isacharparenright}{\isacharparenright}\ \isactrlbold {\isasymrightarrow}\ \isactrlbold {\isasymbox}{\isacharparenleft}\isactrlbold {\isasymforall}x{\isachardot}{\isacharparenleft}{\isasymphi}\ x{\isacharparenright}{\isacharparenright}{\isasymrfloor}{\isachardoublequoteclose}%
\isadelimproof
\ %
\endisadelimproof
%
\isatagproof
\isacommand{by}\isamarkupfalse%
\ simp%
\endisatagproof
{\isafoldproof}%
%
\isadelimproof
%
\endisadelimproof
\isanewline
\ \isacommand{lemma}\isamarkupfalse%
\ {\isachardoublequoteopen}{\isasymlfloor}\isactrlbold {\isasymbox}{\isacharparenleft}\isactrlbold {\isasymforall}x{\isachardot}{\isacharparenleft}{\isasymphi}\ x{\isacharparenright}{\isacharparenright}\ \isactrlbold {\isasymrightarrow}\ {\isacharparenleft}\isactrlbold {\isasymforall}x{\isachardot}\isactrlbold {\isasymbox}{\isacharparenleft}{\isasymphi}\ x{\isacharparenright}{\isacharparenright}{\isasymrfloor}{\isachardoublequoteclose}%
\isadelimproof
\ %
\endisadelimproof
%
\isatagproof
\isacommand{by}\isamarkupfalse%
\ simp%
\endisatagproof
{\isafoldproof}%
%
\isadelimproof
%
\endisadelimproof
%
\begin{isamarkuptext}%
(Converse) Barcan Formulas not satisfied for actualist quantifiers:%
\end{isamarkuptext}\isamarkuptrue%
\ \isacommand{lemma}\isamarkupfalse%
\ {\isachardoublequoteopen}{\isasymlfloor}{\isacharparenleft}\isactrlbold {\isasymforall}\isactrlsup Ex{\isachardot}\isactrlbold {\isasymbox}{\isacharparenleft}{\isasymphi}\ x{\isacharparenright}{\isacharparenright}\ \isactrlbold {\isasymrightarrow}\ \isactrlbold {\isasymbox}{\isacharparenleft}\isactrlbold {\isasymforall}\isactrlsup Ex{\isachardot}{\isacharparenleft}{\isasymphi}\ x{\isacharparenright}{\isacharparenright}{\isasymrfloor}{\isachardoublequoteclose}\ \isacommand{nitpick}\isamarkupfalse%
%
\isadelimproof
\ %
\endisadelimproof
%
\isatagproof
\isacommand{oops}\isamarkupfalse%
\ %
\isamarkupcmt{countersatisfiable%
}
%
\endisatagproof
{\isafoldproof}%
%
\isadelimproof
%
\endisadelimproof
\isanewline
\ \isacommand{lemma}\isamarkupfalse%
\ {\isachardoublequoteopen}{\isasymlfloor}\isactrlbold {\isasymbox}{\isacharparenleft}\isactrlbold {\isasymforall}\isactrlsup Ex{\isachardot}{\isacharparenleft}{\isasymphi}\ x{\isacharparenright}{\isacharparenright}\ \isactrlbold {\isasymrightarrow}\ {\isacharparenleft}\isactrlbold {\isasymforall}\isactrlsup Ex{\isachardot}\isactrlbold {\isasymbox}{\isacharparenleft}{\isasymphi}\ x{\isacharparenright}{\isacharparenright}{\isasymrfloor}{\isachardoublequoteclose}\ \isacommand{nitpick}\isamarkupfalse%
%
\isadelimproof
\ %
\endisadelimproof
%
\isatagproof
\isacommand{oops}\isamarkupfalse%
\ %
\isamarkupcmt{countersatisfiable%
}
%
\endisatagproof
{\isafoldproof}%
%
\isadelimproof
%
\endisadelimproof
%
\begin{isamarkuptext}%
Well known relations between meta-logical notions:%
\end{isamarkuptext}\isamarkuptrue%
\ \isacommand{lemma}\isamarkupfalse%
\ \ {\isachardoublequoteopen}{\isasymlfloor}{\isasymphi}{\isasymrfloor}\ {\isasymlongleftrightarrow}\ {\isasymnot}{\isasymlfloor}{\isasymphi}{\isasymrfloor}\isactrlsup c\isactrlsup s\isactrlsup a\isactrlsup t{\isachardoublequoteclose}%
\isadelimproof
\ %
\endisadelimproof
%
\isatagproof
\isacommand{by}\isamarkupfalse%
\ simp%
\endisatagproof
{\isafoldproof}%
%
\isadelimproof
%
\endisadelimproof
\isanewline
\ \isacommand{lemma}\isamarkupfalse%
\ \ {\isachardoublequoteopen}{\isasymlfloor}{\isasymphi}{\isasymrfloor}\isactrlsup s\isactrlsup a\isactrlsup t\ {\isasymlongleftrightarrow}\ {\isasymnot}{\isasymlfloor}{\isasymphi}{\isasymrfloor}\isactrlsup i\isactrlsup n\isactrlsup v\ {\isachardoublequoteclose}%
\isadelimproof
\ %
\endisadelimproof
%
\isatagproof
\isacommand{by}\isamarkupfalse%
\ simp%
\endisatagproof
{\isafoldproof}%
%
\isadelimproof
%
\endisadelimproof
%
\begin{isamarkuptext}%
Contingent truth does not allow for necessitation:%
\end{isamarkuptext}\isamarkuptrue%
\ \isacommand{lemma}\isamarkupfalse%
\ {\isachardoublequoteopen}{\isasymlfloor}\isactrlbold {\isasymdiamond}{\isasymphi}{\isasymrfloor}\ \ {\isasymlongrightarrow}\ {\isasymlfloor}\isactrlbold {\isasymbox}{\isasymphi}{\isasymrfloor}{\isachardoublequoteclose}\ \isacommand{nitpick}\isamarkupfalse%
%
\isadelimproof
\ %
\endisadelimproof
%
\isatagproof
\isacommand{oops}\isamarkupfalse%
\ \ \ \ \ \ \ \ \ \ \ \ %
\isamarkupcmt{countersatisfiable%
}
%
\endisatagproof
{\isafoldproof}%
%
\isadelimproof
%
\endisadelimproof
\isanewline
\ \isacommand{lemma}\isamarkupfalse%
\ {\isachardoublequoteopen}{\isasymlfloor}\isactrlbold {\isasymbox}{\isasymphi}{\isasymrfloor}\isactrlsup s\isactrlsup a\isactrlsup t\ {\isasymlongrightarrow}\ {\isasymlfloor}\isactrlbold {\isasymbox}{\isasymphi}{\isasymrfloor}{\isachardoublequoteclose}\ \isacommand{nitpick}\isamarkupfalse%
%
\isadelimproof
\ %
\endisadelimproof
%
\isatagproof
\isacommand{oops}\isamarkupfalse%
\ \ \ \ \ \ \ \ \ \ \ %
\isamarkupcmt{countersatisfiable%
}
%
\endisatagproof
{\isafoldproof}%
%
\isadelimproof
%
\endisadelimproof
%
\begin{isamarkuptext}%
Modal Collapse is countersatisfiable:%
\end{isamarkuptext}\isamarkuptrue%
\ \isacommand{lemma}\isamarkupfalse%
\ {\isachardoublequoteopen}{\isasymlfloor}{\isasymphi}\ \isactrlbold {\isasymrightarrow}\ \isactrlbold {\isasymbox}{\isasymphi}{\isasymrfloor}{\isachardoublequoteclose}\ \isacommand{nitpick}\isamarkupfalse%
%
\isadelimproof
\ %
\endisadelimproof
%
\isatagproof
\isacommand{oops}\isamarkupfalse%
\ \ \ \ \ \ \ \ \ \ \ \ \ \ \ \ \ \ %
\isamarkupcmt{countersatisfiable%
}
%
\endisatagproof
{\isafoldproof}%
%
\isadelimproof
%
\endisadelimproof
%
\isamarkupsubsection{Useful Definitions for Axiomatization of Further Logics%
}
\isamarkuptrue%
%
\begin{isamarkuptext}%
The best known logics (\emph{K4, K5, KB, K45, KB5, D, D4, D5, D45, ...}) are obtained through
 axiomatization of combinations of the following:%
\end{isamarkuptext}\isamarkuptrue%
\ \ \isacommand{abbreviation}\isamarkupfalse%
\ M\ \isanewline
\ \ \ \ \isakeyword{where}\ {\isachardoublequoteopen}M\ {\isasymequiv}\ \isactrlbold {\isasymforall}{\isasymphi}{\isachardot}\ \isactrlbold {\isasymbox}{\isasymphi}\ \isactrlbold {\isasymrightarrow}\ {\isasymphi}{\isachardoublequoteclose}\isanewline
\ \ \isacommand{abbreviation}\isamarkupfalse%
\ B\ \isanewline
\ \ \ \ \isakeyword{where}\ {\isachardoublequoteopen}B\ {\isasymequiv}\ \isactrlbold {\isasymforall}{\isasymphi}{\isachardot}\ {\isasymphi}\ \isactrlbold {\isasymrightarrow}\ \ \isactrlbold {\isasymbox}\isactrlbold {\isasymdiamond}{\isasymphi}{\isachardoublequoteclose}\isanewline
\ \ \isacommand{abbreviation}\isamarkupfalse%
\ D\ \isanewline
\ \ \ \ \isakeyword{where}\ {\isachardoublequoteopen}D\ {\isasymequiv}\ \isactrlbold {\isasymforall}{\isasymphi}{\isachardot}\ \isactrlbold {\isasymbox}{\isasymphi}\ \isactrlbold {\isasymrightarrow}\ \isactrlbold {\isasymdiamond}{\isasymphi}{\isachardoublequoteclose}\isanewline
\ \ \isacommand{abbreviation}\isamarkupfalse%
\ IV\ \isanewline
\ \ \ \ \isakeyword{where}\ {\isachardoublequoteopen}IV\ {\isasymequiv}\ \isactrlbold {\isasymforall}{\isasymphi}{\isachardot}\ \isactrlbold {\isasymbox}{\isasymphi}\ \isactrlbold {\isasymrightarrow}\ \ \isactrlbold {\isasymbox}\isactrlbold {\isasymbox}{\isasymphi}{\isachardoublequoteclose}\isanewline
\ \ \isacommand{abbreviation}\isamarkupfalse%
\ V\ \isanewline
\ \ \ \ \isakeyword{where}\ {\isachardoublequoteopen}V\ {\isasymequiv}\ \isactrlbold {\isasymforall}{\isasymphi}{\isachardot}\ \isactrlbold {\isasymdiamond}{\isasymphi}\ \isactrlbold {\isasymrightarrow}\ \isactrlbold {\isasymbox}\isactrlbold {\isasymdiamond}{\isasymphi}{\isachardoublequoteclose}%
\begin{isamarkuptext}%
Because the embedding is of a semantic nature, it is more efficient to instead make use of 
  the well-known \emph{Sahlqvist correspondence}, which links axioms to constraints on a model's accessibility
  relation: axioms $M, B, D, IV, V$ impose reflexivity, symmetry, seriality, transitivity and euclideanness respectively.%
\end{isamarkuptext}\isamarkuptrue%
\ \ \isacommand{lemma}\isamarkupfalse%
\ {\isachardoublequoteopen}reflexive\ aRel\ \ {\isasymLongrightarrow}\ \ {\isasymlfloor}M{\isasymrfloor}{\isachardoublequoteclose}%
\isadelimproof
\ %
\endisadelimproof
%
\isatagproof
\isacommand{by}\isamarkupfalse%
\ blast\ %
\isamarkupcmt{aka T%
}
%
\endisatagproof
{\isafoldproof}%
%
\isadelimproof
%
\endisadelimproof
\isanewline
\ \ \isacommand{lemma}\isamarkupfalse%
\ {\isachardoublequoteopen}symmetric\ aRel\ {\isasymLongrightarrow}\ {\isasymlfloor}B{\isasymrfloor}{\isachardoublequoteclose}%
\isadelimproof
\ %
\endisadelimproof
%
\isatagproof
\isacommand{by}\isamarkupfalse%
\ blast%
\endisatagproof
{\isafoldproof}%
%
\isadelimproof
%
\endisadelimproof
\isanewline
\ \ \isacommand{lemma}\isamarkupfalse%
\ {\isachardoublequoteopen}serial\ aRel\ \ {\isasymLongrightarrow}\ {\isasymlfloor}D{\isasymrfloor}{\isachardoublequoteclose}%
\isadelimproof
\ %
\endisadelimproof
%
\isatagproof
\isacommand{by}\isamarkupfalse%
\ blast%
\endisatagproof
{\isafoldproof}%
%
\isadelimproof
%
\endisadelimproof
\ \ \ \ \ \ \ \ \ \isanewline
\ \ \isacommand{lemma}\isamarkupfalse%
\ {\isachardoublequoteopen}preorder\ aRel\ {\isasymLongrightarrow}\ \ {\isasymlfloor}M{\isasymrfloor}\ {\isasymand}\ {\isasymlfloor}IV{\isasymrfloor}{\isachardoublequoteclose}%
\isadelimproof
\ %
\endisadelimproof
%
\isatagproof
\isacommand{by}\isamarkupfalse%
\ blast\ %
\isamarkupcmt{S4 - reflexive + transitive%
}
%
\endisatagproof
{\isafoldproof}%
%
\isadelimproof
%
\endisadelimproof
\isanewline
\ \ \isacommand{lemma}\isamarkupfalse%
\ {\isachardoublequoteopen}equivalence\ aRel\ \ {\isasymLongrightarrow}\ \ {\isasymlfloor}M{\isasymrfloor}\ {\isasymand}\ {\isasymlfloor}V{\isasymrfloor}{\isachardoublequoteclose}%
\isadelimproof
\ %
\endisadelimproof
%
\isatagproof
\isacommand{by}\isamarkupfalse%
\ blast\ %
\isamarkupcmt{S5 - preorder + symmetric%
}
%
\endisatagproof
{\isafoldproof}%
%
\isadelimproof
%
\endisadelimproof
\isanewline
\ \ \isacommand{lemma}\isamarkupfalse%
\ {\isachardoublequoteopen}reflexive\ aRel\ {\isasymand}\ euclidean\ aRel\ \ {\isasymLongrightarrow}\ \ {\isasymlfloor}M{\isasymrfloor}\ {\isasymand}\ {\isasymlfloor}V{\isasymrfloor}{\isachardoublequoteclose}%
\isadelimproof
\ %
\endisadelimproof
%
\isatagproof
\isacommand{by}\isamarkupfalse%
\ blast\ %
\isamarkupcmt{S5%
}
%
\endisatagproof
{\isafoldproof}%
%
\isadelimproof
%
\endisadelimproof
%
\begin{isamarkuptext}%
Using these definitions, we can derive axioms for the most common modal logics. Thereby we 
  are free to use either the semantic constraints or the related \emph{Sahlqvist} axioms. Here we provide 
  both versions. In what follows we use the semantic constraints for improved performance.%
\end{isamarkuptext}\isamarkuptrue%
%
\isadelimtheory
%
\endisadelimtheory
%
\isatagtheory
%
\endisatagtheory
{\isafoldtheory}%
%
\isadelimtheory
%
\endisadelimtheory
%
\end{isabellebody}%
%%% Local Variables:
%%% mode: latex
%%% TeX-master: "root"
%%% End:


%
\begin{isabellebody}%
\setisabellecontext{ExamplesHOML}%
%
\isadelimtheory
%
\endisadelimtheory
%
\isatagtheory
%
\endisatagtheory
{\isafoldtheory}%
%
\isadelimtheory
%
\endisadelimtheory
%
\isamarkupsection{Examples in book%
}
\isamarkuptrue%
%
\isamarkupsubsection{Chapter 7 - Modal Logic - Syntax and Semantics%
}
\isamarkuptrue%
%
\isamarkupsubsubsection{beta/eta-redex Considerations (page 94)%
}
\isamarkuptrue%
%
\begin{isamarkuptext}%
beta/eta-redex is valid for non-relativized (intensional or extensional) terms (because they designate rigidly):%
\end{isamarkuptext}\isamarkuptrue%
\isacommand{lemma}\isamarkupfalse%
\ {\isachardoublequoteopen}{\isasymlfloor}{\isacharparenleft}{\isacharparenleft}{\isasymlambda}{\isasymalpha}{\isachardot}\ {\isasymphi}\ {\isasymalpha}{\isacharparenright}\ \ {\isacharparenleft}{\isasymtau}{\isacharcolon}{\isacharcolon}{\isasymup}O{\isacharparenright}{\isacharparenright}\ \isactrlbold {\isasymleftrightarrow}\ {\isacharparenleft}{\isasymphi}\ \ {\isasymtau}{\isacharparenright}{\isasymrfloor}{\isachardoublequoteclose}%
\isadelimproof
\ %
\endisadelimproof
%
\isatagproof
\isacommand{by}\isamarkupfalse%
\ simp%
\endisatagproof
{\isafoldproof}%
%
\isadelimproof
%
\endisadelimproof
\isanewline
\isacommand{lemma}\isamarkupfalse%
\ {\isachardoublequoteopen}{\isasymlfloor}{\isacharparenleft}{\isacharparenleft}{\isasymlambda}{\isasymalpha}{\isachardot}\ {\isasymphi}\ {\isasymalpha}{\isacharparenright}\ \ {\isacharparenleft}{\isasymtau}{\isacharcolon}{\isacharcolon}O{\isacharparenright}{\isacharparenright}\ \isactrlbold {\isasymleftrightarrow}\ {\isacharparenleft}{\isasymphi}\ \ {\isasymtau}{\isacharparenright}{\isasymrfloor}{\isachardoublequoteclose}%
\isadelimproof
\ %
\endisadelimproof
%
\isatagproof
\isacommand{by}\isamarkupfalse%
\ simp%
\endisatagproof
{\isafoldproof}%
%
\isadelimproof
%
\endisadelimproof
\isanewline
\isacommand{lemma}\isamarkupfalse%
\ {\isachardoublequoteopen}{\isasymlfloor}{\isacharparenleft}{\isacharparenleft}{\isasymlambda}{\isasymalpha}{\isachardot}\ \isactrlbold {\isasymbox}{\isasymphi}\ {\isasymalpha}{\isacharparenright}\ {\isacharparenleft}{\isasymtau}{\isacharcolon}{\isacharcolon}{\isasymup}O{\isacharparenright}{\isacharparenright}\ \isactrlbold {\isasymleftrightarrow}\ {\isacharparenleft}\isactrlbold {\isasymbox}{\isasymphi}\ {\isasymtau}{\isacharparenright}{\isasymrfloor}{\isachardoublequoteclose}%
\isadelimproof
\ %
\endisadelimproof
%
\isatagproof
\isacommand{by}\isamarkupfalse%
\ simp%
\endisatagproof
{\isafoldproof}%
%
\isadelimproof
%
\endisadelimproof
\isanewline
\isacommand{lemma}\isamarkupfalse%
\ {\isachardoublequoteopen}{\isasymlfloor}{\isacharparenleft}{\isacharparenleft}{\isasymlambda}{\isasymalpha}{\isachardot}\ \isactrlbold {\isasymbox}{\isasymphi}\ {\isasymalpha}{\isacharparenright}\ {\isacharparenleft}{\isasymtau}{\isacharcolon}{\isacharcolon}O{\isacharparenright}{\isacharparenright}\ \isactrlbold {\isasymleftrightarrow}\ {\isacharparenleft}\isactrlbold {\isasymbox}{\isasymphi}\ {\isasymtau}{\isacharparenright}{\isasymrfloor}{\isachardoublequoteclose}%
\isadelimproof
\ %
\endisadelimproof
%
\isatagproof
\isacommand{by}\isamarkupfalse%
\ simp%
\endisatagproof
{\isafoldproof}%
%
\isadelimproof
%
\endisadelimproof
%
\begin{isamarkuptext}%
beta/eta-redex is valid for relativized terms as long as no modal operators occur inside the predicate abstract:%
\end{isamarkuptext}\isamarkuptrue%
\isacommand{lemma}\isamarkupfalse%
\ {\isachardoublequoteopen}{\isasymlfloor}{\isacharparenleft}{\isacharparenleft}{\isasymlambda}{\isasymalpha}{\isachardot}\ {\isasymphi}\ {\isasymalpha}{\isacharparenright}\ \isactrlbold {\isasymdownharpoonleft}{\isacharparenleft}{\isasymtau}{\isacharcolon}{\isacharcolon}{\isasymup}O{\isacharparenright}{\isacharparenright}\ \isactrlbold {\isasymleftrightarrow}\ {\isacharparenleft}{\isasymphi}\ \isactrlbold {\isasymdownharpoonleft}{\isasymtau}{\isacharparenright}{\isasymrfloor}{\isachardoublequoteclose}%
\isadelimproof
\ %
\endisadelimproof
%
\isatagproof
\isacommand{by}\isamarkupfalse%
\ simp%
\endisatagproof
{\isafoldproof}%
%
\isadelimproof
%
\endisadelimproof
%
\begin{isamarkuptext}%
beta/eta-redex is non-valid for relativized terms when modal operators are present:%
\end{isamarkuptext}\isamarkuptrue%
\isacommand{lemma}\isamarkupfalse%
\ {\isachardoublequoteopen}{\isasymlfloor}{\isacharparenleft}{\isacharparenleft}{\isasymlambda}{\isasymalpha}{\isachardot}\ \isactrlbold {\isasymbox}{\isasymphi}\ {\isasymalpha}{\isacharparenright}\ \isactrlbold {\isasymdownharpoonleft}{\isacharparenleft}{\isasymtau}{\isacharcolon}{\isacharcolon}{\isasymup}O{\isacharparenright}{\isacharparenright}\ \isactrlbold {\isasymleftrightarrow}\ {\isacharparenleft}\isactrlbold {\isasymbox}{\isasymphi}\ \isactrlbold {\isasymdownharpoonleft}{\isasymtau}{\isacharparenright}{\isasymrfloor}{\isachardoublequoteclose}\ \isacommand{nitpick}\isamarkupfalse%
%
\isadelimproof
\ %
\endisadelimproof
%
\isatagproof
\isacommand{oops}\isamarkupfalse%
\ \ \ %
\isamarkupcmt{countersatisfiable%
}
%
\endisatagproof
{\isafoldproof}%
%
\isadelimproof
%
\endisadelimproof
\isanewline
\isacommand{lemma}\isamarkupfalse%
\ {\isachardoublequoteopen}{\isasymlfloor}{\isacharparenleft}{\isacharparenleft}{\isasymlambda}{\isasymalpha}{\isachardot}\ \isactrlbold {\isasymdiamond}{\isasymphi}\ {\isasymalpha}{\isacharparenright}\ \isactrlbold {\isasymdownharpoonleft}{\isacharparenleft}{\isasymtau}{\isacharcolon}{\isacharcolon}{\isasymup}O{\isacharparenright}{\isacharparenright}\ \isactrlbold {\isasymleftrightarrow}\ {\isacharparenleft}\isactrlbold {\isasymdiamond}{\isasymphi}\ \isactrlbold {\isasymdownharpoonleft}{\isasymtau}{\isacharparenright}{\isasymrfloor}{\isachardoublequoteclose}\ \isacommand{nitpick}\isamarkupfalse%
%
\isadelimproof
\ %
\endisadelimproof
%
\isatagproof
\isacommand{oops}\isamarkupfalse%
\ \ \ %
\isamarkupcmt{countersatisfiable%
}
%
\endisatagproof
{\isafoldproof}%
%
\isadelimproof
%
\endisadelimproof
%
\begin{isamarkuptext}%
Example 7.13 page 96:%
\end{isamarkuptext}\isamarkuptrue%
\isacommand{lemma}\isamarkupfalse%
\ {\isachardoublequoteopen}{\isasymlfloor}{\isacharparenleft}{\isasymlambda}X{\isachardot}\ \isactrlbold {\isasymdiamond}\isactrlbold {\isasymexists}X{\isacharparenright}\ \ {\isacharparenleft}P{\isacharcolon}{\isacharcolon}{\isasymup}{\isasymlangle}O{\isasymrangle}{\isacharparenright}\ \isactrlbold {\isasymrightarrow}\ \isactrlbold {\isasymdiamond}{\isacharparenleft}{\isacharparenleft}{\isasymlambda}X{\isachardot}\ \isactrlbold {\isasymexists}X{\isacharparenright}\ \ P{\isacharparenright}{\isasymrfloor}{\isachardoublequoteclose}%
\isadelimproof
\ \ %
\endisadelimproof
%
\isatagproof
\isacommand{by}\isamarkupfalse%
\ simp%
\endisatagproof
{\isafoldproof}%
%
\isadelimproof
%
\endisadelimproof
\isanewline
\isacommand{lemma}\isamarkupfalse%
\ {\isachardoublequoteopen}{\isasymlfloor}{\isacharparenleft}{\isasymlambda}X{\isachardot}\ \isactrlbold {\isasymdiamond}\isactrlbold {\isasymexists}X{\isacharparenright}\ \isactrlbold {\isasymdown}{\isacharparenleft}P{\isacharcolon}{\isacharcolon}{\isasymup}{\isasymlangle}O{\isasymrangle}{\isacharparenright}\ \isactrlbold {\isasymrightarrow}\ \isactrlbold {\isasymdiamond}{\isacharparenleft}{\isacharparenleft}{\isasymlambda}X{\isachardot}\ \isactrlbold {\isasymexists}X{\isacharparenright}\ \isactrlbold {\isasymdown}P{\isacharparenright}{\isasymrfloor}{\isachardoublequoteclose}\ \isacommand{nitpick}\isamarkupfalse%
{\isacharbrackleft}card\ {\isacharprime}t{\isacharequal}{\isadigit{1}}{\isacharcomma}\ card\ i{\isacharequal}{\isadigit{2}}{\isacharbrackright}%
\isadelimproof
\ %
\endisadelimproof
%
\isatagproof
\isacommand{oops}\isamarkupfalse%
\ %
\isamarkupcmt{nitpick finds same counterexample as book%
}
%
\endisatagproof
{\isafoldproof}%
%
\isadelimproof
%
\endisadelimproof
%
\begin{isamarkuptext}%
with other types for P:%
\end{isamarkuptext}\isamarkuptrue%
\isacommand{lemma}\isamarkupfalse%
\ {\isachardoublequoteopen}{\isasymlfloor}{\isacharparenleft}{\isasymlambda}X{\isachardot}\ \isactrlbold {\isasymdiamond}\isactrlbold {\isasymexists}X{\isacharparenright}\ \ {\isacharparenleft}P{\isacharcolon}{\isacharcolon}{\isasymup}{\isasymlangle}{\isasymup}O{\isasymrangle}{\isacharparenright}\ \isactrlbold {\isasymrightarrow}\ \isactrlbold {\isasymdiamond}{\isacharparenleft}{\isacharparenleft}{\isasymlambda}X{\isachardot}\ \isactrlbold {\isasymexists}X{\isacharparenright}\ \ P{\isacharparenright}{\isasymrfloor}{\isachardoublequoteclose}%
\isadelimproof
\ \ %
\endisadelimproof
%
\isatagproof
\isacommand{by}\isamarkupfalse%
\ simp%
\endisatagproof
{\isafoldproof}%
%
\isadelimproof
%
\endisadelimproof
\ \ \ \ \isanewline
\isacommand{lemma}\isamarkupfalse%
\ {\isachardoublequoteopen}{\isasymlfloor}{\isacharparenleft}{\isasymlambda}X{\isachardot}\ \isactrlbold {\isasymdiamond}\isactrlbold {\isasymexists}X{\isacharparenright}\ \isactrlbold {\isasymdown}{\isacharparenleft}P{\isacharcolon}{\isacharcolon}{\isasymup}{\isasymlangle}{\isasymup}O{\isasymrangle}{\isacharparenright}\ \isactrlbold {\isasymrightarrow}\ \isactrlbold {\isasymdiamond}{\isacharparenleft}{\isacharparenleft}{\isasymlambda}X{\isachardot}\ \isactrlbold {\isasymexists}X{\isacharparenright}\ \isactrlbold {\isasymdown}P{\isacharparenright}{\isasymrfloor}{\isachardoublequoteclose}\ \isacommand{nitpick}\isamarkupfalse%
{\isacharbrackleft}card\ {\isacharprime}t{\isacharequal}{\isadigit{1}}{\isacharcomma}\ card\ i{\isacharequal}{\isadigit{2}}{\isacharbrackright}%
\isadelimproof
\ %
\endisadelimproof
%
\isatagproof
\isacommand{oops}\isamarkupfalse%
\ %
\isamarkupcmt{countersatisfiable%
}
%
\endisatagproof
{\isafoldproof}%
%
\isadelimproof
%
\endisadelimproof
\isanewline
\isacommand{lemma}\isamarkupfalse%
\ {\isachardoublequoteopen}{\isasymlfloor}{\isacharparenleft}{\isasymlambda}X{\isachardot}\ \isactrlbold {\isasymdiamond}\isactrlbold {\isasymexists}X{\isacharparenright}\ \ {\isacharparenleft}P{\isacharcolon}{\isacharcolon}{\isasymup}{\isasymlangle}{\isasymlangle}O{\isasymrangle}{\isasymrangle}{\isacharparenright}\ \isactrlbold {\isasymrightarrow}\ \isactrlbold {\isasymdiamond}{\isacharparenleft}{\isacharparenleft}{\isasymlambda}X{\isachardot}\ \isactrlbold {\isasymexists}X{\isacharparenright}\ \ P{\isacharparenright}{\isasymrfloor}{\isachardoublequoteclose}%
\isadelimproof
\ \ %
\endisadelimproof
%
\isatagproof
\isacommand{by}\isamarkupfalse%
\ simp%
\endisatagproof
{\isafoldproof}%
%
\isadelimproof
%
\endisadelimproof
\ \ \ \ \isanewline
\isacommand{lemma}\isamarkupfalse%
\ {\isachardoublequoteopen}{\isasymlfloor}{\isacharparenleft}{\isasymlambda}X{\isachardot}\ \isactrlbold {\isasymdiamond}\isactrlbold {\isasymexists}X{\isacharparenright}\ \isactrlbold {\isasymdown}{\isacharparenleft}P{\isacharcolon}{\isacharcolon}{\isasymup}{\isasymlangle}{\isasymlangle}O{\isasymrangle}{\isasymrangle}{\isacharparenright}\ \isactrlbold {\isasymrightarrow}\ \isactrlbold {\isasymdiamond}{\isacharparenleft}{\isacharparenleft}{\isasymlambda}X{\isachardot}\ \isactrlbold {\isasymexists}X{\isacharparenright}\ \isactrlbold {\isasymdown}P{\isacharparenright}{\isasymrfloor}{\isachardoublequoteclose}\ \isacommand{nitpick}\isamarkupfalse%
{\isacharbrackleft}card\ {\isacharprime}t{\isacharequal}{\isadigit{1}}{\isacharcomma}\ card\ i{\isacharequal}{\isadigit{2}}{\isacharbrackright}%
\isadelimproof
\ %
\endisadelimproof
%
\isatagproof
\isacommand{oops}\isamarkupfalse%
\ %
\isamarkupcmt{countersatisfiable%
}
%
\endisatagproof
{\isafoldproof}%
%
\isadelimproof
%
\endisadelimproof
\isanewline
\isacommand{lemma}\isamarkupfalse%
\ {\isachardoublequoteopen}{\isasymlfloor}{\isacharparenleft}{\isasymlambda}X{\isachardot}\ \isactrlbold {\isasymdiamond}\isactrlbold {\isasymexists}X{\isacharparenright}\ \ {\isacharparenleft}P{\isacharcolon}{\isacharcolon}{\isasymup}{\isasymlangle}{\isasymup}{\isasymlangle}O{\isasymrangle}{\isasymrangle}{\isacharparenright}\isactrlbold {\isasymrightarrow}\ \isactrlbold {\isasymdiamond}{\isacharparenleft}{\isacharparenleft}{\isasymlambda}X{\isachardot}\ \isactrlbold {\isasymexists}X{\isacharparenright}\ \ P{\isacharparenright}{\isasymrfloor}{\isachardoublequoteclose}%
\isadelimproof
\ \ %
\endisadelimproof
%
\isatagproof
\isacommand{by}\isamarkupfalse%
\ simp%
\endisatagproof
{\isafoldproof}%
%
\isadelimproof
%
\endisadelimproof
\ \ \ \ \isanewline
\isacommand{lemma}\isamarkupfalse%
\ {\isachardoublequoteopen}{\isasymlfloor}{\isacharparenleft}{\isasymlambda}X{\isachardot}\ \isactrlbold {\isasymdiamond}\isactrlbold {\isasymexists}X{\isacharparenright}\ \isactrlbold {\isasymdown}{\isacharparenleft}P{\isacharcolon}{\isacharcolon}{\isasymup}{\isasymlangle}{\isasymup}{\isasymlangle}O{\isasymrangle}{\isasymrangle}{\isacharparenright}\isactrlbold {\isasymrightarrow}\ \isactrlbold {\isasymdiamond}{\isacharparenleft}{\isacharparenleft}{\isasymlambda}X{\isachardot}\ \isactrlbold {\isasymexists}X{\isacharparenright}\ \isactrlbold {\isasymdown}P{\isacharparenright}{\isasymrfloor}{\isachardoublequoteclose}\ \isacommand{nitpick}\isamarkupfalse%
{\isacharbrackleft}card\ {\isacharprime}t{\isacharequal}{\isadigit{1}}{\isacharcomma}\ card\ i{\isacharequal}{\isadigit{2}}{\isacharbrackright}%
\isadelimproof
\ %
\endisadelimproof
%
\isatagproof
\isacommand{oops}\isamarkupfalse%
\ %
\isamarkupcmt{countersatisfiable%
}
%
\endisatagproof
{\isafoldproof}%
%
\isadelimproof
%
\endisadelimproof
%
\begin{isamarkuptext}%
Example 7.14 page 98:%
\end{isamarkuptext}\isamarkuptrue%
\isacommand{lemma}\isamarkupfalse%
\ {\isachardoublequoteopen}{\isasymlfloor}{\isacharparenleft}{\isasymlambda}X{\isachardot}\ \isactrlbold {\isasymdiamond}\isactrlbold {\isasymexists}X{\isacharparenright}\ \isactrlbold {\isasymdown}{\isacharparenleft}P{\isacharcolon}{\isacharcolon}{\isasymup}{\isasymlangle}O{\isasymrangle}{\isacharparenright}\ \isactrlbold {\isasymrightarrow}\ {\isacharparenleft}{\isasymlambda}X{\isachardot}\ \isactrlbold {\isasymexists}X{\isacharparenright}\ \isactrlbold {\isasymdown}P{\isasymrfloor}{\isachardoublequoteclose}%
\isadelimproof
\ %
\endisadelimproof
%
\isatagproof
\isacommand{by}\isamarkupfalse%
\ simp%
\endisatagproof
{\isafoldproof}%
%
\isadelimproof
%
\endisadelimproof
\isanewline
\isacommand{lemma}\isamarkupfalse%
\ {\isachardoublequoteopen}{\isasymlfloor}{\isacharparenleft}{\isasymlambda}X{\isachardot}\ \isactrlbold {\isasymdiamond}\isactrlbold {\isasymexists}X{\isacharparenright}\ \ {\isacharparenleft}P{\isacharcolon}{\isacharcolon}{\isasymup}{\isasymlangle}O{\isasymrangle}{\isacharparenright}\ \isactrlbold {\isasymrightarrow}\ {\isacharparenleft}{\isasymlambda}X{\isachardot}\ \isactrlbold {\isasymexists}X{\isacharparenright}\ \ P{\isasymrfloor}{\isachardoublequoteclose}\ \isacommand{nitpick}\isamarkupfalse%
{\isacharbrackleft}card\ {\isacharprime}t{\isacharequal}{\isadigit{1}}{\isacharcomma}\ card\ i{\isacharequal}{\isadigit{2}}{\isacharbrackright}%
\isadelimproof
\ %
\endisadelimproof
%
\isatagproof
\isacommand{oops}\isamarkupfalse%
\ %
\isamarkupcmt{countersatisfiable%
}
%
\endisatagproof
{\isafoldproof}%
%
\isadelimproof
%
\endisadelimproof
%
\begin{isamarkuptext}%
with other types for P:%
\end{isamarkuptext}\isamarkuptrue%
\isacommand{lemma}\isamarkupfalse%
\ {\isachardoublequoteopen}{\isasymlfloor}{\isacharparenleft}{\isasymlambda}X{\isachardot}\ \isactrlbold {\isasymdiamond}\isactrlbold {\isasymexists}X{\isacharparenright}\ \isactrlbold {\isasymdown}{\isacharparenleft}P{\isacharcolon}{\isacharcolon}{\isasymup}{\isasymlangle}{\isasymup}O{\isasymrangle}{\isacharparenright}\ \isactrlbold {\isasymrightarrow}\ {\isacharparenleft}{\isasymlambda}X{\isachardot}\ \isactrlbold {\isasymexists}X{\isacharparenright}\ \isactrlbold {\isasymdown}P{\isasymrfloor}{\isachardoublequoteclose}%
\isadelimproof
\ %
\endisadelimproof
%
\isatagproof
\isacommand{by}\isamarkupfalse%
\ simp%
\endisatagproof
{\isafoldproof}%
%
\isadelimproof
%
\endisadelimproof
\isanewline
\isacommand{lemma}\isamarkupfalse%
\ {\isachardoublequoteopen}{\isasymlfloor}{\isacharparenleft}{\isasymlambda}X{\isachardot}\ \isactrlbold {\isasymdiamond}\isactrlbold {\isasymexists}X{\isacharparenright}\ \ {\isacharparenleft}P{\isacharcolon}{\isacharcolon}{\isasymup}{\isasymlangle}{\isasymup}O{\isasymrangle}{\isacharparenright}\ \isactrlbold {\isasymrightarrow}\ {\isacharparenleft}{\isasymlambda}X{\isachardot}\ \isactrlbold {\isasymexists}X{\isacharparenright}\ \ P{\isasymrfloor}{\isachardoublequoteclose}\ \isacommand{nitpick}\isamarkupfalse%
{\isacharbrackleft}card\ {\isacharprime}t{\isacharequal}{\isadigit{1}}{\isacharcomma}\ card\ i{\isacharequal}{\isadigit{2}}{\isacharbrackright}%
\isadelimproof
\ %
\endisadelimproof
%
\isatagproof
\isacommand{oops}\isamarkupfalse%
\ %
\isamarkupcmt{countersatisfiable%
}
%
\endisatagproof
{\isafoldproof}%
%
\isadelimproof
%
\endisadelimproof
\isanewline
\isacommand{lemma}\isamarkupfalse%
\ {\isachardoublequoteopen}{\isasymlfloor}{\isacharparenleft}{\isasymlambda}X{\isachardot}\ \isactrlbold {\isasymdiamond}\isactrlbold {\isasymexists}X{\isacharparenright}\ \isactrlbold {\isasymdown}{\isacharparenleft}P{\isacharcolon}{\isacharcolon}{\isasymup}{\isasymlangle}{\isasymlangle}O{\isasymrangle}{\isasymrangle}{\isacharparenright}\ \isactrlbold {\isasymrightarrow}\ {\isacharparenleft}{\isasymlambda}X{\isachardot}\ \isactrlbold {\isasymexists}X{\isacharparenright}\ \isactrlbold {\isasymdown}P{\isasymrfloor}{\isachardoublequoteclose}%
\isadelimproof
\ %
\endisadelimproof
%
\isatagproof
\isacommand{by}\isamarkupfalse%
\ simp%
\endisatagproof
{\isafoldproof}%
%
\isadelimproof
%
\endisadelimproof
\isanewline
\isacommand{lemma}\isamarkupfalse%
\ {\isachardoublequoteopen}{\isasymlfloor}{\isacharparenleft}{\isasymlambda}X{\isachardot}\ \isactrlbold {\isasymdiamond}\isactrlbold {\isasymexists}X{\isacharparenright}\ \ {\isacharparenleft}P{\isacharcolon}{\isacharcolon}{\isasymup}{\isasymlangle}{\isasymlangle}O{\isasymrangle}{\isasymrangle}{\isacharparenright}\ \isactrlbold {\isasymrightarrow}\ {\isacharparenleft}{\isasymlambda}X{\isachardot}\ \isactrlbold {\isasymexists}X{\isacharparenright}\ \ P{\isasymrfloor}{\isachardoublequoteclose}\ \isacommand{nitpick}\isamarkupfalse%
{\isacharbrackleft}card\ {\isacharprime}t{\isacharequal}{\isadigit{1}}{\isacharcomma}\ card\ i{\isacharequal}{\isadigit{2}}{\isacharbrackright}%
\isadelimproof
\ %
\endisadelimproof
%
\isatagproof
\isacommand{oops}\isamarkupfalse%
\ %
\isamarkupcmt{countersatisfiable%
}
%
\endisatagproof
{\isafoldproof}%
%
\isadelimproof
%
\endisadelimproof
\isanewline
\isacommand{lemma}\isamarkupfalse%
\ {\isachardoublequoteopen}{\isasymlfloor}{\isacharparenleft}{\isasymlambda}X{\isachardot}\ \isactrlbold {\isasymdiamond}\isactrlbold {\isasymexists}X{\isacharparenright}\ \isactrlbold {\isasymdown}{\isacharparenleft}P{\isacharcolon}{\isacharcolon}{\isasymup}{\isasymlangle}{\isasymup}{\isasymlangle}O{\isasymrangle}{\isasymrangle}{\isacharparenright}\isactrlbold {\isasymrightarrow}\ {\isacharparenleft}{\isasymlambda}X{\isachardot}\ \isactrlbold {\isasymexists}X{\isacharparenright}\ \isactrlbold {\isasymdown}P{\isasymrfloor}{\isachardoublequoteclose}%
\isadelimproof
\ %
\endisadelimproof
%
\isatagproof
\isacommand{by}\isamarkupfalse%
\ simp%
\endisatagproof
{\isafoldproof}%
%
\isadelimproof
%
\endisadelimproof
\isanewline
\isacommand{lemma}\isamarkupfalse%
\ {\isachardoublequoteopen}{\isasymlfloor}{\isacharparenleft}{\isasymlambda}X{\isachardot}\ \isactrlbold {\isasymdiamond}\isactrlbold {\isasymexists}X{\isacharparenright}\ \ {\isacharparenleft}P{\isacharcolon}{\isacharcolon}{\isasymup}{\isasymlangle}{\isasymup}{\isasymlangle}O{\isasymrangle}{\isasymrangle}{\isacharparenright}\isactrlbold {\isasymrightarrow}\ {\isacharparenleft}{\isasymlambda}X{\isachardot}\ \isactrlbold {\isasymexists}X{\isacharparenright}\ \ P{\isasymrfloor}{\isachardoublequoteclose}\ \isacommand{nitpick}\isamarkupfalse%
{\isacharbrackleft}card\ {\isacharprime}t{\isacharequal}{\isadigit{1}}{\isacharcomma}\ card\ i{\isacharequal}{\isadigit{2}}{\isacharbrackright}%
\isadelimproof
\ %
\endisadelimproof
%
\isatagproof
\isacommand{oops}\isamarkupfalse%
\ %
\isamarkupcmt{countersatisfiable%
}
%
\endisatagproof
{\isafoldproof}%
%
\isadelimproof
%
\endisadelimproof
%
\begin{isamarkuptext}%
Example 7.15 page 99:%
\end{isamarkuptext}\isamarkuptrue%
\isacommand{lemma}\isamarkupfalse%
\ {\isachardoublequoteopen}{\isasymlfloor}\isactrlbold {\isasymbox}{\isacharparenleft}P\ {\isacharparenleft}c{\isacharcolon}{\isacharcolon}{\isasymup}O{\isacharparenright}{\isacharparenright}\ \isactrlbold {\isasymrightarrow}\ {\isacharparenleft}\isactrlbold {\isasymexists}x{\isacharcolon}{\isacharcolon}{\isasymup}O{\isachardot}\ \isactrlbold {\isasymbox}{\isacharparenleft}P\ x{\isacharparenright}{\isacharparenright}{\isasymrfloor}{\isachardoublequoteclose}%
\isadelimproof
\ %
\endisadelimproof
%
\isatagproof
\isacommand{by}\isamarkupfalse%
\ auto%
\endisatagproof
{\isafoldproof}%
%
\isadelimproof
%
\endisadelimproof
%
\begin{isamarkuptext}%
for other types:%
\end{isamarkuptext}\isamarkuptrue%
\isacommand{lemma}\isamarkupfalse%
\ {\isachardoublequoteopen}{\isasymlfloor}\isactrlbold {\isasymbox}{\isacharparenleft}P\ {\isacharparenleft}c{\isacharcolon}{\isacharcolon}O{\isacharparenright}{\isacharparenright}\ \isactrlbold {\isasymrightarrow}\ {\isacharparenleft}\isactrlbold {\isasymexists}x{\isacharcolon}{\isacharcolon}O{\isachardot}\ \isactrlbold {\isasymbox}{\isacharparenleft}P\ x{\isacharparenright}{\isacharparenright}{\isasymrfloor}{\isachardoublequoteclose}%
\isadelimproof
\ %
\endisadelimproof
%
\isatagproof
\isacommand{by}\isamarkupfalse%
\ auto%
\endisatagproof
{\isafoldproof}%
%
\isadelimproof
%
\endisadelimproof
\isanewline
\isacommand{lemma}\isamarkupfalse%
\ {\isachardoublequoteopen}{\isasymlfloor}\isactrlbold {\isasymbox}{\isacharparenleft}P\ {\isacharparenleft}c{\isacharcolon}{\isacharcolon}{\isasymlangle}O{\isasymrangle}{\isacharparenright}{\isacharparenright}\ \isactrlbold {\isasymrightarrow}\ {\isacharparenleft}\isactrlbold {\isasymexists}x{\isacharcolon}{\isacharcolon}{\isasymlangle}O{\isasymrangle}{\isachardot}\ \isactrlbold {\isasymbox}{\isacharparenleft}P\ x{\isacharparenright}{\isacharparenright}{\isasymrfloor}{\isachardoublequoteclose}%
\isadelimproof
\ %
\endisadelimproof
%
\isatagproof
\isacommand{by}\isamarkupfalse%
\ auto%
\endisatagproof
{\isafoldproof}%
%
\isadelimproof
%
\endisadelimproof
%
\begin{isamarkuptext}%
Example 7.16 page 100:%
\end{isamarkuptext}\isamarkuptrue%
\isacommand{lemma}\isamarkupfalse%
\ {\isachardoublequoteopen}{\isasymlfloor}\isactrlbold {\isasymbox}{\isacharparenleft}P\ \isactrlbold {\isasymdownharpoonleft}{\isacharparenleft}c{\isacharcolon}{\isacharcolon}{\isasymup}O{\isacharparenright}{\isacharparenright}\ \isactrlbold {\isasymrightarrow}\ {\isacharparenleft}\isactrlbold {\isasymexists}x{\isacharcolon}{\isacharcolon}O{\isachardot}\ \isactrlbold {\isasymbox}{\isacharparenleft}P\ x{\isacharparenright}{\isacharparenright}{\isasymrfloor}{\isachardoublequoteclose}\ \isacommand{nitpick}\isamarkupfalse%
{\isacharbrackleft}card\ {\isacharprime}t{\isacharequal}{\isadigit{2}}{\isacharcomma}\ card\ i{\isacharequal}{\isadigit{2}}{\isacharbrackright}%
\isadelimproof
\ %
\endisadelimproof
%
\isatagproof
\isacommand{oops}\isamarkupfalse%
\ %
\isamarkupcmt{countersatisfiable (using only two worlds!)%
}
%
\endisatagproof
{\isafoldproof}%
%
\isadelimproof
%
\endisadelimproof
%
\begin{isamarkuptext}%
Example 7.17 page 101:%
\end{isamarkuptext}\isamarkuptrue%
\isacommand{lemma}\isamarkupfalse%
\ {\isachardoublequoteopen}{\isasymlfloor}\isactrlbold {\isasymforall}Z{\isacharcolon}{\isacharcolon}{\isasymup}O{\isachardot}\ {\isacharparenleft}{\isasymlambda}x{\isacharcolon}{\isacharcolon}O{\isachardot}\ \ \isactrlbold {\isasymbox}{\isacharparenleft}{\isacharparenleft}{\isasymlambda}y{\isacharcolon}{\isacharcolon}O{\isachardot}\ \ x\ \isactrlbold {\isasymapprox}\ y{\isacharparenright}\ \isactrlbold {\isasymdownharpoonleft}Z{\isacharparenright}{\isacharparenright}\ \isactrlbold {\isasymdownharpoonleft}Z{\isasymrfloor}{\isachardoublequoteclose}\ \isacommand{nitpick}\isamarkupfalse%
{\isacharbrackleft}card\ {\isacharprime}t{\isacharequal}{\isadigit{2}}{\isacharcomma}\ card\ i{\isacharequal}{\isadigit{2}}{\isacharbrackright}%
\isadelimproof
\ %
\endisadelimproof
%
\isatagproof
\isacommand{oops}\isamarkupfalse%
\ %
\isamarkupcmt{countersatisfiable%
}
%
\endisatagproof
{\isafoldproof}%
%
\isadelimproof
%
\endisadelimproof
\isanewline
\isacommand{lemma}\isamarkupfalse%
\ {\isachardoublequoteopen}{\isasymlfloor}\isactrlbold {\isasymforall}z{\isacharcolon}{\isacharcolon}O{\isachardot}\ \ {\isacharparenleft}{\isasymlambda}x{\isacharcolon}{\isacharcolon}O{\isachardot}\ \ \isactrlbold {\isasymbox}{\isacharparenleft}{\isacharparenleft}{\isasymlambda}y{\isacharcolon}{\isacharcolon}O{\isachardot}\ \ x\ \isactrlbold {\isasymapprox}\ y{\isacharparenright}\ \ z{\isacharparenright}{\isacharparenright}\ z{\isasymrfloor}{\isachardoublequoteclose}%
\isadelimproof
\ %
\endisadelimproof
%
\isatagproof
\isacommand{by}\isamarkupfalse%
\ simp%
\endisatagproof
{\isafoldproof}%
%
\isadelimproof
%
\endisadelimproof
\isanewline
\isacommand{lemma}\isamarkupfalse%
\ {\isachardoublequoteopen}{\isasymlfloor}\isactrlbold {\isasymforall}Z{\isacharcolon}{\isacharcolon}{\isasymup}O{\isachardot}\ {\isacharparenleft}{\isasymlambda}X{\isacharcolon}{\isacharcolon}{\isasymup}O{\isachardot}\ \isactrlbold {\isasymbox}{\isacharparenleft}{\isacharparenleft}{\isasymlambda}Y{\isacharcolon}{\isacharcolon}{\isasymup}O{\isachardot}\ X\ \isactrlbold {\isasymapprox}\ Y{\isacharparenright}\ \ Z{\isacharparenright}{\isacharparenright}\ Z{\isasymrfloor}{\isachardoublequoteclose}%
\isadelimproof
\ %
\endisadelimproof
%
\isatagproof
\isacommand{by}\isamarkupfalse%
\ simp%
\endisatagproof
{\isafoldproof}%
%
\isadelimproof
%
\endisadelimproof
%
\isamarkupsubsubsection{Exercises page 101%
}
\isamarkuptrue%
%
\begin{isamarkuptext}%
For Exercises 7.1 and 7.2 see variations on Examples 7.13 and 7.14 above.%
\end{isamarkuptext}\isamarkuptrue%
%
\begin{isamarkuptext}%
Exercise 7.3:%
\end{isamarkuptext}\isamarkuptrue%
\isacommand{lemma}\isamarkupfalse%
\ {\isachardoublequoteopen}{\isasymlfloor}\isactrlbold {\isasymdiamond}\isactrlbold {\isasymexists}{\isacharparenleft}P{\isacharcolon}{\isacharcolon}{\isasymup}{\isasymlangle}O{\isasymrangle}{\isacharparenright}\ \isactrlbold {\isasymrightarrow}\ {\isacharparenleft}\isactrlbold {\isasymexists}X{\isacharcolon}{\isacharcolon}{\isasymup}O{\isachardot}\ \isactrlbold {\isasymdiamond}{\isacharparenleft}P\ \isactrlbold {\isasymdownharpoonleft}X{\isacharparenright}{\isacharparenright}{\isasymrfloor}{\isachardoublequoteclose}%
\isadelimproof
\ %
\endisadelimproof
%
\isatagproof
\isacommand{by}\isamarkupfalse%
\ auto%
\endisatagproof
{\isafoldproof}%
%
\isadelimproof
%
\endisadelimproof
\isanewline
\isacommand{lemma}\isamarkupfalse%
\ {\isachardoublequoteopen}{\isasymlfloor}\isactrlbold {\isasymdiamond}\isactrlbold {\isasymexists}{\isacharparenleft}P{\isacharcolon}{\isacharcolon}{\isasymup}{\isasymlangle}{\isasymup}{\isasymlangle}O{\isasymrangle}{\isasymrangle}{\isacharparenright}\ \isactrlbold {\isasymrightarrow}\ {\isacharparenleft}\isactrlbold {\isasymexists}X{\isacharcolon}{\isacharcolon}{\isasymup}{\isasymlangle}O{\isasymrangle}{\isachardot}\ \isactrlbold {\isasymdiamond}{\isacharparenleft}P\ \isactrlbold {\isasymdown}X{\isacharparenright}{\isacharparenright}{\isasymrfloor}{\isachardoublequoteclose}\ \isacommand{nitpick}\isamarkupfalse%
{\isacharbrackleft}card\ {\isacharprime}t{\isacharequal}{\isadigit{1}}{\isacharcomma}\ card\ i{\isacharequal}{\isadigit{2}}{\isacharbrackright}%
\isadelimproof
\ %
\endisadelimproof
%
\isatagproof
\isacommand{oops}\isamarkupfalse%
%
\endisatagproof
{\isafoldproof}%
%
\isadelimproof
%
\endisadelimproof
%
\begin{isamarkuptext}%
Exercise 7.4:%
\end{isamarkuptext}\isamarkuptrue%
\isacommand{lemma}\isamarkupfalse%
\ {\isachardoublequoteopen}{\isasymlfloor}\isactrlbold {\isasymdiamond}{\isacharparenleft}\isactrlbold {\isasymexists}x{\isacharcolon}{\isacharcolon}O{\isachardot}\ {\isacharparenleft}{\isasymlambda}Y{\isachardot}\ Y\ x{\isacharparenright}\ \isactrlbold {\isasymdown}{\isacharparenleft}P{\isacharcolon}{\isacharcolon}{\isasymup}{\isasymlangle}O{\isasymrangle}{\isacharparenright}{\isacharparenright}\ \isactrlbold {\isasymrightarrow}\ {\isacharparenleft}\isactrlbold {\isasymexists}x{\isachardot}\ {\isacharparenleft}{\isasymlambda}Y{\isachardot}\ \isactrlbold {\isasymdiamond}{\isacharparenleft}Y\ x{\isacharparenright}{\isacharparenright}\ \isactrlbold {\isasymdown}P{\isacharparenright}{\isasymrfloor}{\isachardoublequoteclose}\ \isacommand{nitpick}\isamarkupfalse%
{\isacharbrackleft}card\ {\isacharprime}t{\isacharequal}{\isadigit{1}}{\isacharcomma}\ card\ i{\isacharequal}{\isadigit{2}}{\isacharbrackright}%
\isadelimproof
\ %
\endisadelimproof
%
\isatagproof
\isacommand{oops}\isamarkupfalse%
\ %
\isamarkupcmt{countersatisfiable%
}
%
\endisatagproof
{\isafoldproof}%
%
\isadelimproof
%
\endisadelimproof
%
\begin{isamarkuptext}%
For Exercise 7.5 see Example 7.17 above.%
\end{isamarkuptext}\isamarkuptrue%
%
\isamarkupsubsection{Chapter 9 - Miscellaneous Matters%
}
\isamarkuptrue%
%
\isamarkupsubsubsection{(1.1) Equality%
}
\isamarkuptrue%
%
\begin{isamarkuptext}%
Example 9.1:%
\end{isamarkuptext}\isamarkuptrue%
\isacommand{lemma}\isamarkupfalse%
\ {\isachardoublequoteopen}{\isasymlfloor}{\isacharparenleft}{\isacharparenleft}{\isasymlambda}X{\isachardot}\ \isactrlbold {\isasymbox}{\isacharparenleft}X\ \isactrlbold {\isasymdownharpoonleft}{\isacharparenleft}p{\isacharcolon}{\isacharcolon}{\isasymup}O{\isacharparenright}{\isacharparenright}{\isacharparenright}\ \isactrlbold {\isasymdown}{\isacharparenleft}{\isasymlambda}x{\isachardot}\ \isactrlbold {\isasymdiamond}{\isacharparenleft}{\isasymlambda}z{\isachardot}\ z\ \isactrlbold {\isasymapprox}\ x{\isacharparenright}\ \isactrlbold {\isasymdownharpoonleft}p{\isacharparenright}{\isacharparenright}{\isasymrfloor}{\isachardoublequoteclose}%
\isadelimproof
\ %
\endisadelimproof
%
\isatagproof
\isacommand{by}\isamarkupfalse%
\ auto\ %
\isamarkupcmt{using normal equality%
}
%
\endisatagproof
{\isafoldproof}%
%
\isadelimproof
%
\endisadelimproof
\isanewline
\isacommand{lemma}\isamarkupfalse%
\ {\isachardoublequoteopen}{\isasymlfloor}{\isacharparenleft}{\isacharparenleft}{\isasymlambda}X{\isachardot}\ \isactrlbold {\isasymbox}{\isacharparenleft}X\ \isactrlbold {\isasymdownharpoonleft}{\isacharparenleft}p{\isacharcolon}{\isacharcolon}{\isasymup}O{\isacharparenright}{\isacharparenright}{\isacharparenright}\ \isactrlbold {\isasymdown}{\isacharparenleft}{\isasymlambda}x{\isachardot}\ \isactrlbold {\isasymdiamond}{\isacharparenleft}{\isasymlambda}z{\isachardot}\ z\ \isactrlbold {\isasymapprox}\isactrlsup L\ x{\isacharparenright}\ \isactrlbold {\isasymdownharpoonleft}p{\isacharparenright}{\isacharparenright}{\isasymrfloor}{\isachardoublequoteclose}%
\isadelimproof
\ %
\endisadelimproof
%
\isatagproof
\isacommand{by}\isamarkupfalse%
\ auto\ %
\isamarkupcmt{using Leibniz equality%
}
%
\endisatagproof
{\isafoldproof}%
%
\isadelimproof
%
\endisadelimproof
\isanewline
\isacommand{lemma}\isamarkupfalse%
\ {\isachardoublequoteopen}{\isasymlfloor}{\isacharparenleft}{\isacharparenleft}{\isasymlambda}X{\isachardot}\ \isactrlbold {\isasymbox}{\isacharparenleft}X\ \ {\isacharparenleft}p{\isacharcolon}{\isacharcolon}{\isasymup}O{\isacharparenright}{\isacharparenright}{\isacharparenright}\ \isactrlbold {\isasymdown}{\isacharparenleft}{\isasymlambda}x{\isachardot}\ \isactrlbold {\isasymdiamond}{\isacharparenleft}{\isasymlambda}z{\isachardot}\ z\ \isactrlbold {\isasymapprox}\isactrlsup C\ x{\isacharparenright}\ p{\isacharparenright}{\isacharparenright}{\isasymrfloor}{\isachardoublequoteclose}%
\isadelimproof
\ %
\endisadelimproof
%
\isatagproof
\isacommand{by}\isamarkupfalse%
\ simp\ \ %
\isamarkupcmt{variation using equality for individual concepts%
}
%
\endisatagproof
{\isafoldproof}%
%
\isadelimproof
%
\endisadelimproof
%
\isamarkupsubsubsection{(1.2) Extensionality%
}
\isamarkuptrue%
%
\begin{isamarkuptext}%
In the book, extensionality is assumed (globally) for extensional terms. Extensionality is however
   already implicit in Isabelle/HOL:%
\end{isamarkuptext}\isamarkuptrue%
\isacommand{lemma}\isamarkupfalse%
\ EXT{\isacharcolon}\ {\isachardoublequoteopen}{\isasymforall}{\isasymalpha}{\isacharcolon}{\isacharcolon}{\isasymlangle}O{\isasymrangle}{\isachardot}\ {\isasymforall}{\isasymbeta}{\isacharcolon}{\isacharcolon}{\isasymlangle}O{\isasymrangle}{\isachardot}\ {\isacharparenleft}{\isasymforall}{\isasymgamma}{\isacharcolon}{\isacharcolon}O{\isachardot}\ {\isacharparenleft}{\isasymalpha}\ {\isasymgamma}\ {\isasymlongleftrightarrow}\ {\isasymbeta}\ {\isasymgamma}{\isacharparenright}{\isacharparenright}\ {\isasymlongrightarrow}\ {\isacharparenleft}{\isasymalpha}\ {\isacharequal}\ {\isasymbeta}{\isacharparenright}{\isachardoublequoteclose}%
\isadelimproof
\ %
\endisadelimproof
%
\isatagproof
\isacommand{by}\isamarkupfalse%
\ auto%
\endisatagproof
{\isafoldproof}%
%
\isadelimproof
%
\endisadelimproof
\isanewline
\isacommand{lemma}\isamarkupfalse%
\ EXT{\isacharunderscore}set{\isacharcolon}\ {\isachardoublequoteopen}{\isasymforall}{\isasymalpha}{\isacharcolon}{\isacharcolon}{\isasymlangle}{\isasymlangle}O{\isasymrangle}{\isasymrangle}{\isachardot}\ {\isasymforall}{\isasymbeta}{\isacharcolon}{\isacharcolon}{\isasymlangle}{\isasymlangle}O{\isasymrangle}{\isasymrangle}{\isachardot}\ {\isacharparenleft}{\isasymforall}{\isasymgamma}{\isacharcolon}{\isacharcolon}{\isasymlangle}O{\isasymrangle}{\isachardot}\ {\isacharparenleft}{\isasymalpha}\ {\isasymgamma}\ {\isasymlongleftrightarrow}\ {\isasymbeta}\ {\isasymgamma}{\isacharparenright}{\isacharparenright}\ {\isasymlongrightarrow}\ {\isacharparenleft}{\isasymalpha}\ {\isacharequal}\ {\isasymbeta}{\isacharparenright}{\isachardoublequoteclose}%
\isadelimproof
\ %
\endisadelimproof
%
\isatagproof
\isacommand{by}\isamarkupfalse%
\ auto%
\endisatagproof
{\isafoldproof}%
%
\isadelimproof
%
\endisadelimproof
%
\begin{isamarkuptext}%
Extensionality for intensional terms is also already implicit in the HOL embedding:%
\end{isamarkuptext}\isamarkuptrue%
\isacommand{lemma}\isamarkupfalse%
\ EXT{\isacharunderscore}intensional{\isacharcolon}\ \ \ \ \ \ {\isachardoublequoteopen}{\isasymlfloor}{\isacharparenleft}{\isasymlambda}x{\isachardot}\ {\isacharparenleft}{\isacharparenleft}{\isasymlambda}y{\isachardot}\ x\isactrlbold {\isasymapprox}y{\isacharparenright}\ \isactrlbold {\isasymdownharpoonleft}{\isacharparenleft}{\isasymalpha}{\isacharcolon}{\isacharcolon}{\isasymup}O\ {\isacharparenright}{\isacharparenright}{\isacharparenright}\ \ \isactrlbold {\isasymdownharpoonleft}{\isacharparenleft}{\isasymbeta}{\isacharcolon}{\isacharcolon}{\isasymup}O{\isacharparenright}\ {\isasymrfloor}\ {\isasymlongrightarrow}\ {\isasymalpha}\ {\isacharequal}\ {\isasymbeta}{\isachardoublequoteclose}%
\isadelimproof
\ %
\endisadelimproof
%
\isatagproof
\isacommand{by}\isamarkupfalse%
\ auto%
\endisatagproof
{\isafoldproof}%
%
\isadelimproof
%
\endisadelimproof
\isanewline
\isacommand{lemma}\isamarkupfalse%
\ EXT{\isacharunderscore}intensional{\isacharunderscore}pred{\isacharcolon}\ {\isachardoublequoteopen}{\isasymlfloor}{\isacharparenleft}{\isasymlambda}x{\isachardot}\ {\isacharparenleft}{\isacharparenleft}{\isasymlambda}y{\isachardot}\ x\isactrlbold {\isasymapprox}y{\isacharparenright}\ \isactrlbold {\isasymdown}{\isacharparenleft}{\isasymalpha}{\isacharcolon}{\isacharcolon}{\isasymup}{\isasymlangle}O{\isasymrangle}{\isacharparenright}{\isacharparenright}{\isacharparenright}\ \isactrlbold {\isasymdown}{\isacharparenleft}{\isasymbeta}{\isacharcolon}{\isacharcolon}{\isasymup}{\isasymlangle}O{\isasymrangle}{\isacharparenright}{\isasymrfloor}\ {\isasymlongrightarrow}\ {\isasymalpha}\ {\isacharequal}\ {\isasymbeta}{\isachardoublequoteclose}%
\isadelimproof
\ %
\endisadelimproof
%
\isatagproof
\isacommand{using}\isamarkupfalse%
\ ext\ \isacommand{by}\isamarkupfalse%
\ metis%
\endisatagproof
{\isafoldproof}%
%
\isadelimproof
%
\endisadelimproof
%
\isamarkupsubsubsection{(2) De re & de dicto%
}
\isamarkuptrue%
%
\begin{isamarkuptext}%
de re is equivalent to de dicto for non-relativized (extensional or intensional) terms:%
\end{isamarkuptext}\isamarkuptrue%
\isacommand{lemma}\isamarkupfalse%
\ {\isachardoublequoteopen}{\isasymlfloor}\isactrlbold {\isasymforall}{\isasymalpha}{\isachardot}\ {\isacharparenleft}{\isacharparenleft}{\isasymlambda}{\isasymbeta}{\isachardot}\ \isactrlbold {\isasymbox}{\isacharparenleft}{\isasymalpha}\ {\isasymbeta}{\isacharparenright}{\isacharparenright}\ {\isacharparenleft}{\isasymtau}{\isacharcolon}{\isacharcolon}O{\isacharparenright}{\isacharparenright}\ \ \ \isactrlbold {\isasymleftrightarrow}\ \isactrlbold {\isasymbox}{\isacharparenleft}{\isacharparenleft}{\isasymlambda}{\isasymbeta}{\isachardot}\ {\isacharparenleft}{\isasymalpha}\ {\isasymbeta}{\isacharparenright}{\isacharparenright}\ {\isasymtau}{\isacharparenright}{\isasymrfloor}{\isachardoublequoteclose}%
\isadelimproof
\ %
\endisadelimproof
%
\isatagproof
\isacommand{by}\isamarkupfalse%
\ simp%
\endisatagproof
{\isafoldproof}%
%
\isadelimproof
%
\endisadelimproof
\isanewline
\isacommand{lemma}\isamarkupfalse%
\ {\isachardoublequoteopen}{\isasymlfloor}\isactrlbold {\isasymforall}{\isasymalpha}{\isachardot}\ {\isacharparenleft}{\isacharparenleft}{\isasymlambda}{\isasymbeta}{\isachardot}\ \isactrlbold {\isasymbox}{\isacharparenleft}{\isasymalpha}\ {\isasymbeta}{\isacharparenright}{\isacharparenright}\ {\isacharparenleft}{\isasymtau}{\isacharcolon}{\isacharcolon}{\isasymup}O{\isacharparenright}{\isacharparenright}\ \ \isactrlbold {\isasymleftrightarrow}\ \isactrlbold {\isasymbox}{\isacharparenleft}{\isacharparenleft}{\isasymlambda}{\isasymbeta}{\isachardot}\ {\isacharparenleft}{\isasymalpha}\ {\isasymbeta}{\isacharparenright}{\isacharparenright}\ {\isasymtau}{\isacharparenright}{\isasymrfloor}{\isachardoublequoteclose}%
\isadelimproof
\ %
\endisadelimproof
%
\isatagproof
\isacommand{by}\isamarkupfalse%
\ simp%
\endisatagproof
{\isafoldproof}%
%
\isadelimproof
%
\endisadelimproof
\isanewline
\isacommand{lemma}\isamarkupfalse%
\ {\isachardoublequoteopen}{\isasymlfloor}\isactrlbold {\isasymforall}{\isasymalpha}{\isachardot}\ {\isacharparenleft}{\isacharparenleft}{\isasymlambda}{\isasymbeta}{\isachardot}\ \isactrlbold {\isasymbox}{\isacharparenleft}{\isasymalpha}\ {\isasymbeta}{\isacharparenright}{\isacharparenright}\ {\isacharparenleft}{\isasymtau}{\isacharcolon}{\isacharcolon}{\isasymlangle}O{\isasymrangle}{\isacharparenright}{\isacharparenright}\ \ \isactrlbold {\isasymleftrightarrow}\ \isactrlbold {\isasymbox}{\isacharparenleft}{\isacharparenleft}{\isasymlambda}{\isasymbeta}{\isachardot}\ {\isacharparenleft}{\isasymalpha}\ {\isasymbeta}{\isacharparenright}{\isacharparenright}\ {\isasymtau}{\isacharparenright}{\isasymrfloor}{\isachardoublequoteclose}%
\isadelimproof
\ %
\endisadelimproof
%
\isatagproof
\isacommand{by}\isamarkupfalse%
\ simp%
\endisatagproof
{\isafoldproof}%
%
\isadelimproof
%
\endisadelimproof
\isanewline
\isacommand{lemma}\isamarkupfalse%
\ {\isachardoublequoteopen}{\isasymlfloor}\isactrlbold {\isasymforall}{\isasymalpha}{\isachardot}\ {\isacharparenleft}{\isacharparenleft}{\isasymlambda}{\isasymbeta}{\isachardot}\ \isactrlbold {\isasymbox}{\isacharparenleft}{\isasymalpha}\ {\isasymbeta}{\isacharparenright}{\isacharparenright}\ {\isacharparenleft}{\isasymtau}{\isacharcolon}{\isacharcolon}{\isasymup}{\isasymlangle}O{\isasymrangle}{\isacharparenright}{\isacharparenright}\ \isactrlbold {\isasymleftrightarrow}\ \isactrlbold {\isasymbox}{\isacharparenleft}{\isacharparenleft}{\isasymlambda}{\isasymbeta}{\isachardot}\ {\isacharparenleft}{\isasymalpha}\ {\isasymbeta}{\isacharparenright}{\isacharparenright}\ {\isasymtau}{\isacharparenright}{\isasymrfloor}{\isachardoublequoteclose}%
\isadelimproof
\ %
\endisadelimproof
%
\isatagproof
\isacommand{by}\isamarkupfalse%
\ simp%
\endisatagproof
{\isafoldproof}%
%
\isadelimproof
%
\endisadelimproof
%
\begin{isamarkuptext}%
de re is not equivalent to de dicto for relativized (intensional) terms:%
\end{isamarkuptext}\isamarkuptrue%
\isacommand{lemma}\isamarkupfalse%
\ {\isachardoublequoteopen}{\isasymlfloor}\isactrlbold {\isasymforall}{\isasymalpha}{\isachardot}\ {\isacharparenleft}{\isacharparenleft}{\isasymlambda}{\isasymbeta}{\isachardot}\ \isactrlbold {\isasymbox}{\isacharparenleft}{\isasymalpha}\ {\isasymbeta}{\isacharparenright}{\isacharparenright}\ \isactrlbold {\isasymdownharpoonleft}{\isacharparenleft}{\isasymtau}{\isacharcolon}{\isacharcolon}{\isasymup}O{\isacharparenright}{\isacharparenright}\ \isactrlbold {\isasymleftrightarrow}\ \isactrlbold {\isasymbox}{\isacharparenleft}\ {\isacharparenleft}{\isasymlambda}{\isasymbeta}{\isachardot}\ {\isacharparenleft}{\isasymalpha}\ {\isasymbeta}{\isacharparenright}{\isacharparenright}\ \isactrlbold {\isasymdownharpoonleft}{\isasymtau}{\isacharparenright}{\isasymrfloor}{\isachardoublequoteclose}\ \isacommand{nitpick}\isamarkupfalse%
{\isacharbrackleft}card\ {\isacharprime}t{\isacharequal}{\isadigit{2}}{\isacharcomma}\ card\ i{\isacharequal}{\isadigit{2}}{\isacharbrackright}%
\isadelimproof
\ %
\endisadelimproof
%
\isatagproof
\isacommand{oops}\isamarkupfalse%
\ %
\isamarkupcmt{countersatisfiable%
}
%
\endisatagproof
{\isafoldproof}%
%
\isadelimproof
%
\endisadelimproof
\isanewline
\isacommand{lemma}\isamarkupfalse%
\ {\isachardoublequoteopen}{\isasymlfloor}\isactrlbold {\isasymforall}{\isasymalpha}{\isachardot}\ {\isacharparenleft}{\isacharparenleft}{\isasymlambda}{\isasymbeta}{\isachardot}\ \isactrlbold {\isasymbox}{\isacharparenleft}{\isasymalpha}\ {\isasymbeta}{\isacharparenright}{\isacharparenright}\ \isactrlbold {\isasymdown}{\isacharparenleft}{\isasymtau}{\isacharcolon}{\isacharcolon}{\isasymup}{\isasymlangle}O{\isasymrangle}{\isacharparenright}{\isacharparenright}\ \isactrlbold {\isasymleftrightarrow}\ \isactrlbold {\isasymbox}{\isacharparenleft}\ {\isacharparenleft}{\isasymlambda}{\isasymbeta}{\isachardot}\ {\isacharparenleft}{\isasymalpha}\ {\isasymbeta}{\isacharparenright}{\isacharparenright}\ \isactrlbold {\isasymdown}{\isasymtau}{\isacharparenright}{\isasymrfloor}{\isachardoublequoteclose}\ \isacommand{nitpick}\isamarkupfalse%
{\isacharbrackleft}card\ {\isacharprime}t{\isacharequal}{\isadigit{1}}{\isacharcomma}\ card\ i{\isacharequal}{\isadigit{2}}{\isacharbrackright}%
\isadelimproof
\ %
\endisadelimproof
%
\isatagproof
\isacommand{oops}\isamarkupfalse%
\ %
\isamarkupcmt{countersatisfiable%
}
%
\endisatagproof
{\isafoldproof}%
%
\isadelimproof
%
\endisadelimproof
%
\begin{isamarkuptext}%
Proposition 9.6 - Equivalences between de dicto and de re:%
\end{isamarkuptext}\isamarkuptrue%
\isacommand{abbreviation}\isamarkupfalse%
\ deDictoEquDeRe{\isacharcolon}{\isacharcolon}{\isachardoublequoteopen}{\isasymup}{\isasymlangle}{\isasymup}O{\isasymrangle}{\isachardoublequoteclose}\ \isakeyword{where}\ {\isachardoublequoteopen}deDictoEquDeRe\ {\isasymtau}\ {\isasymequiv}\ \isactrlbold {\isasymforall}{\isasymalpha}{\isachardot}\ {\isacharparenleft}{\isacharparenleft}{\isasymlambda}{\isasymbeta}{\isachardot}\ \isactrlbold {\isasymbox}{\isacharparenleft}{\isasymalpha}\ {\isasymbeta}{\isacharparenright}{\isacharparenright}\ \isactrlbold {\isasymdownharpoonleft}{\isasymtau}{\isacharparenright}\ \isactrlbold {\isasymleftrightarrow}\ \isactrlbold {\isasymbox}{\isacharparenleft}{\isacharparenleft}{\isasymlambda}{\isasymbeta}{\isachardot}\ {\isacharparenleft}{\isasymalpha}\ {\isasymbeta}{\isacharparenright}{\isacharparenright}\ \isactrlbold {\isasymdownharpoonleft}{\isasymtau}{\isacharparenright}{\isachardoublequoteclose}\isanewline
\isacommand{abbreviation}\isamarkupfalse%
\ deDictoImpliesDeRe{\isacharcolon}{\isacharcolon}{\isachardoublequoteopen}{\isasymup}{\isasymlangle}{\isasymup}O{\isasymrangle}{\isachardoublequoteclose}\ \isakeyword{where}\ {\isachardoublequoteopen}deDictoImpliesDeRe\ {\isasymtau}\ {\isasymequiv}\ \isactrlbold {\isasymforall}{\isasymalpha}{\isachardot}\ \isactrlbold {\isasymbox}{\isacharparenleft}{\isacharparenleft}{\isasymlambda}{\isasymbeta}{\isachardot}\ {\isacharparenleft}{\isasymalpha}\ {\isasymbeta}{\isacharparenright}{\isacharparenright}\ \isactrlbold {\isasymdownharpoonleft}{\isasymtau}{\isacharparenright}\ \isactrlbold {\isasymrightarrow}\ {\isacharparenleft}{\isacharparenleft}{\isasymlambda}{\isasymbeta}{\isachardot}\ \isactrlbold {\isasymbox}{\isacharparenleft}{\isasymalpha}\ {\isasymbeta}{\isacharparenright}{\isacharparenright}\ \isactrlbold {\isasymdownharpoonleft}{\isasymtau}{\isacharparenright}{\isachardoublequoteclose}\isanewline
\isacommand{abbreviation}\isamarkupfalse%
\ deReImpliesDeDicto{\isacharcolon}{\isacharcolon}{\isachardoublequoteopen}{\isasymup}{\isasymlangle}{\isasymup}O{\isasymrangle}{\isachardoublequoteclose}\ \isakeyword{where}\ {\isachardoublequoteopen}deReImpliesDeDicto\ {\isasymtau}\ {\isasymequiv}\ \isactrlbold {\isasymforall}{\isasymalpha}{\isachardot}\ {\isacharparenleft}{\isacharparenleft}{\isasymlambda}{\isasymbeta}{\isachardot}\ \isactrlbold {\isasymbox}{\isacharparenleft}{\isasymalpha}\ {\isasymbeta}{\isacharparenright}{\isacharparenright}\ \isactrlbold {\isasymdownharpoonleft}{\isasymtau}{\isacharparenright}\ \isactrlbold {\isasymrightarrow}\ \isactrlbold {\isasymbox}{\isacharparenleft}{\isacharparenleft}{\isasymlambda}{\isasymbeta}{\isachardot}\ {\isacharparenleft}{\isasymalpha}\ {\isasymbeta}{\isacharparenright}{\isacharparenright}\ \isactrlbold {\isasymdownharpoonleft}{\isasymtau}{\isacharparenright}{\isachardoublequoteclose}\isanewline
\ \ \isanewline
\isacommand{abbreviation}\isamarkupfalse%
\ deDictoEquDeRe{\isacharunderscore}pred{\isacharcolon}{\isacharcolon}{\isachardoublequoteopen}{\isacharparenleft}{\isacharprime}t{\isasymRightarrow}io{\isacharparenright}{\isasymRightarrow}io{\isachardoublequoteclose}\ \isakeyword{where}\ {\isachardoublequoteopen}deDictoEquDeRe{\isacharunderscore}pred\ {\isasymtau}\ {\isasymequiv}\ \isactrlbold {\isasymforall}{\isasymalpha}{\isachardot}\ {\isacharparenleft}{\isacharparenleft}{\isasymlambda}{\isasymbeta}{\isachardot}\ \isactrlbold {\isasymbox}{\isacharparenleft}{\isasymalpha}\ {\isasymbeta}{\isacharparenright}{\isacharparenright}\ \isactrlbold {\isasymdown}{\isasymtau}{\isacharparenright}\ \isactrlbold {\isasymleftrightarrow}\ \isactrlbold {\isasymbox}{\isacharparenleft}{\isacharparenleft}{\isasymlambda}{\isasymbeta}{\isachardot}\ {\isacharparenleft}{\isasymalpha}\ {\isasymbeta}{\isacharparenright}{\isacharparenright}\ \isactrlbold {\isasymdown}{\isasymtau}{\isacharparenright}{\isachardoublequoteclose}\isanewline
\isacommand{abbreviation}\isamarkupfalse%
\ deDictoImpliesDeRe{\isacharunderscore}pred{\isacharcolon}{\isacharcolon}{\isachardoublequoteopen}{\isacharparenleft}{\isacharprime}t{\isasymRightarrow}io{\isacharparenright}{\isasymRightarrow}io{\isachardoublequoteclose}\ \isakeyword{where}\ {\isachardoublequoteopen}deDictoImpliesDeRe{\isacharunderscore}pred\ {\isasymtau}\ {\isasymequiv}\ \isactrlbold {\isasymforall}{\isasymalpha}{\isachardot}\ \isactrlbold {\isasymbox}{\isacharparenleft}{\isacharparenleft}{\isasymlambda}{\isasymbeta}{\isachardot}\ {\isacharparenleft}{\isasymalpha}\ {\isasymbeta}{\isacharparenright}{\isacharparenright}\ \isactrlbold {\isasymdown}{\isasymtau}{\isacharparenright}\ \isactrlbold {\isasymrightarrow}\ {\isacharparenleft}{\isacharparenleft}{\isasymlambda}{\isasymbeta}{\isachardot}\ \isactrlbold {\isasymbox}{\isacharparenleft}{\isasymalpha}\ {\isasymbeta}{\isacharparenright}{\isacharparenright}\ \isactrlbold {\isasymdown}{\isasymtau}{\isacharparenright}{\isachardoublequoteclose}\isanewline
\isacommand{abbreviation}\isamarkupfalse%
\ deReImpliesDeDicto{\isacharunderscore}pred{\isacharcolon}{\isacharcolon}{\isachardoublequoteopen}{\isacharparenleft}{\isacharprime}t{\isasymRightarrow}io{\isacharparenright}{\isasymRightarrow}io{\isachardoublequoteclose}\ \isakeyword{where}\ {\isachardoublequoteopen}deReImpliesDeDicto{\isacharunderscore}pred\ {\isasymtau}\ {\isasymequiv}\ \isactrlbold {\isasymforall}{\isasymalpha}{\isachardot}\ {\isacharparenleft}{\isacharparenleft}{\isasymlambda}{\isasymbeta}{\isachardot}\ \isactrlbold {\isasymbox}{\isacharparenleft}{\isasymalpha}\ {\isasymbeta}{\isacharparenright}{\isacharparenright}\ \isactrlbold {\isasymdown}{\isasymtau}{\isacharparenright}\ \isactrlbold {\isasymrightarrow}\ \isactrlbold {\isasymbox}{\isacharparenleft}{\isacharparenleft}{\isasymlambda}{\isasymbeta}{\isachardot}\ {\isacharparenleft}{\isasymalpha}\ {\isasymbeta}{\isacharparenright}{\isacharparenright}\ \isactrlbold {\isasymdown}{\isasymtau}{\isacharparenright}{\isachardoublequoteclose}%
\begin{isamarkuptext}%
The following are valid only when using global consequence:%
\end{isamarkuptext}\isamarkuptrue%
%
\begin{isamarkuptext}%
(TODO: solvers need some help to find the proofs)%
\end{isamarkuptext}\isamarkuptrue%
\isacommand{lemma}\isamarkupfalse%
\ {\isachardoublequoteopen}{\isasymlfloor}deDictoImpliesDeRe\ {\isacharparenleft}{\isasymtau}{\isacharcolon}{\isacharcolon}{\isasymup}O{\isacharparenright}{\isasymrfloor}\ {\isasymlongrightarrow}\ {\isasymlfloor}deReImpliesDeDicto\ {\isasymtau}{\isasymrfloor}{\isachardoublequoteclose}%
\isadelimproof
\ %
\endisadelimproof
%
\isatagproof
\isacommand{oops}\isamarkupfalse%
%
\endisatagproof
{\isafoldproof}%
%
\isadelimproof
%
\endisadelimproof
\isanewline
\isacommand{lemma}\isamarkupfalse%
\ {\isachardoublequoteopen}{\isasymlfloor}deReImpliesDeDicto\ {\isacharparenleft}{\isasymtau}{\isacharcolon}{\isacharcolon}{\isasymup}O{\isacharparenright}{\isasymrfloor}\ {\isasymlongrightarrow}\ {\isasymlfloor}deDictoImpliesDeRe\ {\isasymtau}{\isasymrfloor}{\isachardoublequoteclose}%
\isadelimproof
\ %
\endisadelimproof
%
\isatagproof
\isacommand{oops}\isamarkupfalse%
%
\endisatagproof
{\isafoldproof}%
%
\isadelimproof
%
\endisadelimproof
\isanewline
\isacommand{lemma}\isamarkupfalse%
\ {\isachardoublequoteopen}{\isasymlfloor}deDictoImpliesDeRe{\isacharunderscore}pred\ {\isacharparenleft}{\isasymtau}{\isacharcolon}{\isacharcolon}{\isasymup}{\isasymlangle}O{\isasymrangle}{\isacharparenright}{\isasymrfloor}\ {\isasymlongrightarrow}\ {\isasymlfloor}deReImpliesDeDicto{\isacharunderscore}pred\ {\isasymtau}{\isasymrfloor}{\isachardoublequoteclose}%
\isadelimproof
\ %
\endisadelimproof
%
\isatagproof
\isacommand{oops}\isamarkupfalse%
%
\endisatagproof
{\isafoldproof}%
%
\isadelimproof
%
\endisadelimproof
\isanewline
\isacommand{lemma}\isamarkupfalse%
\ {\isachardoublequoteopen}{\isasymlfloor}deReImpliesDeDicto{\isacharunderscore}pred\ {\isacharparenleft}{\isasymtau}{\isacharcolon}{\isacharcolon}{\isasymup}{\isasymlangle}O{\isasymrangle}{\isacharparenright}{\isasymrfloor}\ {\isasymlongrightarrow}\ {\isasymlfloor}deDictoImpliesDeRe{\isacharunderscore}pred\ {\isasymtau}{\isasymrfloor}{\isachardoublequoteclose}%
\isadelimproof
\ %
\endisadelimproof
%
\isatagproof
\isacommand{oops}\isamarkupfalse%
%
\endisatagproof
{\isafoldproof}%
%
\isadelimproof
%
\endisadelimproof
%
\isamarkupsubsubsection{(3) Rigidity%
}
\isamarkuptrue%
%
\begin{isamarkuptext}%
Rigidity for intensional individuals:%
\end{isamarkuptext}\isamarkuptrue%
\isacommand{abbreviation}\isamarkupfalse%
\ rigidIndiv{\isacharcolon}{\isacharcolon}{\isachardoublequoteopen}{\isasymup}{\isasymlangle}{\isasymup}O{\isasymrangle}{\isachardoublequoteclose}\ \isakeyword{where}\isanewline
\ \ {\isachardoublequoteopen}rigidIndiv\ {\isasymtau}\ {\isasymequiv}\ {\isacharparenleft}{\isasymlambda}{\isasymbeta}{\isachardot}\ \isactrlbold {\isasymbox}{\isacharparenleft}{\isacharparenleft}{\isasymlambda}z{\isachardot}\ {\isasymbeta}\ \isactrlbold {\isasymapprox}\ z{\isacharparenright}\ \isactrlbold {\isasymdownharpoonleft}{\isasymtau}{\isacharparenright}{\isacharparenright}\ \isactrlbold {\isasymdownharpoonleft}{\isasymtau}{\isachardoublequoteclose}%
\begin{isamarkuptext}%
... and for intensional predicates:%
\end{isamarkuptext}\isamarkuptrue%
\isacommand{abbreviation}\isamarkupfalse%
\ rigidPred{\isacharcolon}{\isacharcolon}{\isachardoublequoteopen}{\isacharparenleft}{\isacharprime}t{\isasymRightarrow}io{\isacharparenright}{\isasymRightarrow}io{\isachardoublequoteclose}\ \isakeyword{where}\isanewline
\ \ {\isachardoublequoteopen}rigidPred\ {\isasymtau}\ {\isasymequiv}\ {\isacharparenleft}{\isasymlambda}{\isasymbeta}{\isachardot}\ \isactrlbold {\isasymbox}{\isacharparenleft}{\isacharparenleft}{\isasymlambda}z{\isachardot}\ {\isasymbeta}\ \isactrlbold {\isasymapprox}\ z{\isacharparenright}\ \isactrlbold {\isasymdown}{\isasymtau}{\isacharparenright}{\isacharparenright}\ \isactrlbold {\isasymdown}{\isasymtau}{\isachardoublequoteclose}%
\begin{isamarkuptext}%
Proposition 9.8 - We can prove it using local consequence (global consequence follows directly).%
\end{isamarkuptext}\isamarkuptrue%
\isacommand{lemma}\isamarkupfalse%
\ {\isachardoublequoteopen}{\isasymlfloor}rigidIndiv\ {\isacharparenleft}{\isasymtau}{\isacharcolon}{\isacharcolon}{\isasymup}O{\isacharparenright}\ \isactrlbold {\isasymrightarrow}\ deReImpliesDeDicto\ {\isasymtau}{\isasymrfloor}{\isachardoublequoteclose}%
\isadelimproof
\ %
\endisadelimproof
%
\isatagproof
\isacommand{by}\isamarkupfalse%
\ simp%
\endisatagproof
{\isafoldproof}%
%
\isadelimproof
%
\endisadelimproof
\isanewline
\isacommand{lemma}\isamarkupfalse%
\ {\isachardoublequoteopen}{\isasymlfloor}deReImpliesDeDicto\ {\isacharparenleft}{\isasymtau}{\isacharcolon}{\isacharcolon}{\isasymup}O{\isacharparenright}\ \isactrlbold {\isasymrightarrow}\ rigidIndiv\ {\isasymtau}{\isasymrfloor}{\isachardoublequoteclose}%
\isadelimproof
\ %
\endisadelimproof
%
\isatagproof
\isacommand{by}\isamarkupfalse%
\ auto%
\endisatagproof
{\isafoldproof}%
%
\isadelimproof
%
\endisadelimproof
\isanewline
\isacommand{lemma}\isamarkupfalse%
\ {\isachardoublequoteopen}{\isasymlfloor}rigidPred\ {\isacharparenleft}{\isasymtau}{\isacharcolon}{\isacharcolon}{\isasymup}{\isasymlangle}O{\isasymrangle}{\isacharparenright}\ \isactrlbold {\isasymrightarrow}\ deReImpliesDeDicto{\isacharunderscore}pred\ {\isasymtau}{\isasymrfloor}{\isachardoublequoteclose}%
\isadelimproof
\ %
\endisadelimproof
%
\isatagproof
\isacommand{by}\isamarkupfalse%
\ simp%
\endisatagproof
{\isafoldproof}%
%
\isadelimproof
%
\endisadelimproof
\isanewline
\isacommand{lemma}\isamarkupfalse%
\ {\isachardoublequoteopen}{\isasymlfloor}deReImpliesDeDicto{\isacharunderscore}pred\ {\isacharparenleft}{\isasymtau}{\isacharcolon}{\isacharcolon}{\isasymup}{\isasymlangle}O{\isasymrangle}{\isacharparenright}\ \isactrlbold {\isasymrightarrow}\ rigidPred\ {\isasymtau}{\isasymrfloor}{\isachardoublequoteclose}%
\isadelimproof
\ %
\endisadelimproof
%
\isatagproof
\isacommand{by}\isamarkupfalse%
\ auto%
\endisatagproof
{\isafoldproof}%
%
\isadelimproof
%
\endisadelimproof
%
\isamarkupsubsubsection{(4) Stability Conditions%
}
\isamarkuptrue%
\isacommand{axiomatization}\isamarkupfalse%
\ \isakeyword{where}\isanewline
S{\isadigit{5}}{\isacharcolon}\ {\isachardoublequoteopen}equivalence\ aRel{\isachardoublequoteclose}\ \ %
\isamarkupcmt{We use the Sahlqvist correspondence for improved performance%
}
%
\begin{isamarkuptext}%
Definition 9.10 - Stability:%
\end{isamarkuptext}\isamarkuptrue%
\isacommand{abbreviation}\isamarkupfalse%
\ stabilityA{\isacharcolon}{\isacharcolon}{\isachardoublequoteopen}{\isacharparenleft}{\isacharprime}t{\isasymRightarrow}io{\isacharparenright}{\isasymRightarrow}io{\isachardoublequoteclose}\ \isakeyword{where}\ {\isachardoublequoteopen}stabilityA\ {\isasymtau}\ {\isasymequiv}\ \isactrlbold {\isasymforall}{\isasymalpha}{\isachardot}\ {\isacharparenleft}{\isasymtau}\ {\isasymalpha}{\isacharparenright}\ \isactrlbold {\isasymrightarrow}\ \isactrlbold {\isasymbox}{\isacharparenleft}{\isasymtau}\ {\isasymalpha}{\isacharparenright}{\isachardoublequoteclose}\isanewline
\isacommand{abbreviation}\isamarkupfalse%
\ stabilityB{\isacharcolon}{\isacharcolon}{\isachardoublequoteopen}{\isacharparenleft}{\isacharprime}t{\isasymRightarrow}io{\isacharparenright}{\isasymRightarrow}io{\isachardoublequoteclose}\ \isakeyword{where}\ {\isachardoublequoteopen}stabilityB\ {\isasymtau}\ {\isasymequiv}\ \isactrlbold {\isasymforall}{\isasymalpha}{\isachardot}\ \isactrlbold {\isasymdiamond}{\isacharparenleft}{\isasymtau}\ {\isasymalpha}{\isacharparenright}\ \isactrlbold {\isasymrightarrow}\ {\isacharparenleft}{\isasymtau}\ {\isasymalpha}{\isacharparenright}{\isachardoublequoteclose}%
\begin{isamarkuptext}%
Proposition 9.10 - Note it is valid only for global consequence.%
\end{isamarkuptext}\isamarkuptrue%
\isacommand{lemma}\isamarkupfalse%
\ {\isachardoublequoteopen}{\isasymlfloor}stabilityA\ {\isacharparenleft}{\isasymtau}{\isacharcolon}{\isacharcolon}{\isasymup}{\isasymlangle}O{\isasymrangle}{\isacharparenright}{\isasymrfloor}\ {\isasymlongrightarrow}\ {\isasymlfloor}stabilityB\ {\isasymtau}{\isasymrfloor}{\isachardoublequoteclose}%
\isadelimproof
\ %
\endisadelimproof
%
\isatagproof
\isacommand{using}\isamarkupfalse%
\ S{\isadigit{5}}\ \isacommand{by}\isamarkupfalse%
\ blast%
\endisatagproof
{\isafoldproof}%
%
\isadelimproof
%
\endisadelimproof
\ \ \ \ \isanewline
\isacommand{lemma}\isamarkupfalse%
\ {\isachardoublequoteopen}{\isasymlfloor}stabilityA\ {\isacharparenleft}{\isasymtau}{\isacharcolon}{\isacharcolon}{\isasymup}{\isasymlangle}O{\isasymrangle}{\isacharparenright}\ \isactrlbold {\isasymrightarrow}\ stabilityB\ {\isasymtau}{\isasymrfloor}{\isachardoublequoteclose}\ \isacommand{nitpick}\isamarkupfalse%
{\isacharbrackleft}card\ {\isacharprime}t{\isacharequal}{\isadigit{1}}{\isacharcomma}\ card\ i{\isacharequal}{\isadigit{2}}{\isacharbrackright}%
\isadelimproof
\ %
\endisadelimproof
%
\isatagproof
\isacommand{oops}\isamarkupfalse%
\ %
\isamarkupcmt{countersatisfiable for local consequence%
}
%
\endisatagproof
{\isafoldproof}%
%
\isadelimproof
%
\endisadelimproof
\isanewline
\ \ \ \ \isanewline
\isacommand{lemma}\isamarkupfalse%
\ {\isachardoublequoteopen}{\isasymlfloor}stabilityB\ {\isacharparenleft}{\isasymtau}{\isacharcolon}{\isacharcolon}{\isasymup}{\isasymlangle}O{\isasymrangle}{\isacharparenright}{\isasymrfloor}\ {\isasymlongrightarrow}\ {\isasymlfloor}stabilityA\ {\isasymtau}{\isasymrfloor}{\isachardoublequoteclose}%
\isadelimproof
\ %
\endisadelimproof
%
\isatagproof
\isacommand{using}\isamarkupfalse%
\ S{\isadigit{5}}\ \isacommand{by}\isamarkupfalse%
\ blast%
\endisatagproof
{\isafoldproof}%
%
\isadelimproof
%
\endisadelimproof
\ \ \ \ \isanewline
\isacommand{lemma}\isamarkupfalse%
\ {\isachardoublequoteopen}{\isasymlfloor}stabilityB\ {\isacharparenleft}{\isasymtau}{\isacharcolon}{\isacharcolon}{\isasymup}{\isasymlangle}O{\isasymrangle}{\isacharparenright}\ \isactrlbold {\isasymrightarrow}\ stabilityA\ {\isasymtau}{\isasymrfloor}{\isachardoublequoteclose}\ \isacommand{nitpick}\isamarkupfalse%
{\isacharbrackleft}card\ {\isacharprime}t{\isacharequal}{\isadigit{1}}{\isacharcomma}\ card\ i{\isacharequal}{\isadigit{2}}{\isacharbrackright}%
\isadelimproof
\ %
\endisadelimproof
%
\isatagproof
\isacommand{oops}\isamarkupfalse%
\ %
\isamarkupcmt{countersatisfiable for local consequence%
}
%
\endisatagproof
{\isafoldproof}%
%
\isadelimproof
%
\endisadelimproof
%
\begin{isamarkuptext}%
Theorem 9.11 - Note that we can prove even local consequence!%
\end{isamarkuptext}\isamarkuptrue%
\isacommand{theorem}\isamarkupfalse%
\ {\isachardoublequoteopen}{\isasymlfloor}rigidPred\ {\isacharparenleft}{\isasymtau}{\isacharcolon}{\isacharcolon}{\isasymup}{\isasymlangle}O{\isasymrangle}{\isacharparenright}\ \isactrlbold {\isasymleftrightarrow}\ {\isacharparenleft}stabilityA\ {\isasymtau}\ \isactrlbold {\isasymand}\ stabilityB\ {\isasymtau}{\isacharparenright}{\isasymrfloor}{\isachardoublequoteclose}%
\isadelimproof
\ %
\endisadelimproof
%
\isatagproof
\isacommand{by}\isamarkupfalse%
\ meson%
\endisatagproof
{\isafoldproof}%
%
\isadelimproof
%
\endisadelimproof
\ \ \ \isanewline
\isacommand{theorem}\isamarkupfalse%
\ {\isachardoublequoteopen}{\isasymlfloor}rigidPred\ {\isacharparenleft}{\isasymtau}{\isacharcolon}{\isacharcolon}{\isasymup}{\isasymlangle}{\isasymup}O{\isasymrangle}{\isacharparenright}\ \isactrlbold {\isasymleftrightarrow}\ {\isacharparenleft}stabilityA\ {\isasymtau}\ \isactrlbold {\isasymand}\ stabilityB\ {\isasymtau}{\isacharparenright}{\isasymrfloor}{\isachardoublequoteclose}%
\isadelimproof
\ %
\endisadelimproof
%
\isatagproof
\isacommand{by}\isamarkupfalse%
\ meson%
\endisatagproof
{\isafoldproof}%
%
\isadelimproof
%
\endisadelimproof
\ \ \ \isanewline
\isacommand{theorem}\isamarkupfalse%
\ {\isachardoublequoteopen}{\isasymlfloor}rigidPred\ {\isacharparenleft}{\isasymtau}{\isacharcolon}{\isacharcolon}{\isasymup}{\isasymlangle}{\isasymup}{\isasymlangle}O{\isasymrangle}{\isasymrangle}{\isacharparenright}\ \isactrlbold {\isasymleftrightarrow}\ {\isacharparenleft}stabilityA\ {\isasymtau}\ \isactrlbold {\isasymand}\ stabilityB\ {\isasymtau}{\isacharparenright}{\isasymrfloor}{\isachardoublequoteclose}%
\isadelimproof
\ %
\endisadelimproof
%
\isatagproof
\isacommand{by}\isamarkupfalse%
\ meson\ \ \ \isanewline
%
\endisatagproof
{\isafoldproof}%
%
\isadelimproof
%
\endisadelimproof
%
\isadelimtheory
%
\endisadelimtheory
%
\isatagtheory
%
\endisatagtheory
{\isafoldtheory}%
%
\isadelimtheory
%
\endisadelimtheory
%
\end{isabellebody}%
%%% Local Variables:
%%% mode: latex
%%% TeX-master: "root"
%%% End:


%
\begin{isabellebody}%
\setisabellecontext{GoedelProof{\isacharunderscore}P{\isadigit{1}}}%
%
\isadelimtheory
%
\endisadelimtheory
%
\isatagtheory
%
\endisatagtheory
{\isafoldtheory}%
%
\isadelimtheory
%
\endisadelimtheory
%
\isamarkupsection{G\"odel's Argument, Formally%
}
\isamarkuptrue%
%
\begin{isamarkuptext}%
"G\"odel's particular version of the argument is a direct descendent of that of Leibniz, which in turn derives
  from one of Descartes. These arguments all have a two-part structure: prove God's existence is necessary,
  if possible; and prove God's existence is possible." \cite{Fitting} p. 138.%
\end{isamarkuptext}\isamarkuptrue%
%
\isamarkupsubsection{Part I - God's Existence is Possible%
}
\isamarkuptrue%
%
\begin{isamarkuptext}%
We separate G\"odel's Argument as presented in Fitting's textbook (ch. 11) in two parts. For the first one, while Leibniz provides
  some kind of proof for the compatibility of all perfections, G\"odel goes on to prove an analogous result:
 \emph{(T1) Every positive property is possibly instantiated}, which together with \emph{(T2) God is a positive property}
  directly implies the conclusion. In order to prove \emph{T1}, G\"odel assumes \emph{A2: Any property entailed by a positive property is positive}.%
\end{isamarkuptext}\isamarkuptrue%
%
\begin{isamarkuptext}%
We are currently contemplating a follow-up analysis of the philosophical implications of these axioms,
 which encompasses some criticism of the notion of \emph{property entailment} used by G\"odel throughout the argument.%
\end{isamarkuptext}\isamarkuptrue%
%
\isamarkupsubsubsection{General Definitions%
}
\isamarkuptrue%
\isacommand{abbreviation}\isamarkupfalse%
\ existencePredicate{\isacharcolon}{\isacharcolon}{\isachardoublequoteopen}{\isasymup}{\isasymlangle}{\isasymzero}{\isasymrangle}{\isachardoublequoteclose}\ {\isacharparenleft}{\isachardoublequoteopen}E{\isacharbang}{\isachardoublequoteclose}{\isacharparenright}\ \isanewline
\ \ \isakeyword{where}\ {\isachardoublequoteopen}E{\isacharbang}\ x\ \ {\isasymequiv}\ {\isasymlambda}w{\isachardot}\ {\isacharparenleft}\isactrlbold {\isasymexists}\isactrlsup Ey{\isachardot}\ y\isactrlbold {\isasymapprox}x{\isacharparenright}\ w{\isachardoublequoteclose}\ %
\isamarkupcmt{existence predicate in object language%
}
\isanewline
\isacommand{lemma}\isamarkupfalse%
\ {\isachardoublequoteopen}E{\isacharbang}\ x\ w\ {\isasymlongleftrightarrow}\ existsAt\ x\ w{\isachardoublequoteclose}\ \isanewline
%
\isadelimproof
\ \ %
\endisadelimproof
%
\isatagproof
\isacommand{by}\isamarkupfalse%
\ simp\ %
\isamarkupcmt{safety check: \isa{E{\isacharbang}} correctly matches its meta-logical counterpart%
}
%
\endisatagproof
{\isafoldproof}%
%
\isadelimproof
\isanewline
%
\endisadelimproof
\isanewline
\isacommand{consts}\isamarkupfalse%
\ positiveProperty{\isacharcolon}{\isacharcolon}{\isachardoublequoteopen}{\isasymup}{\isasymlangle}{\isasymup}{\isasymlangle}{\isasymzero}{\isasymrangle}{\isasymrangle}{\isachardoublequoteclose}\ {\isacharparenleft}{\isachardoublequoteopen}{\isasymP}{\isachardoublequoteclose}{\isacharparenright}\ %
\isamarkupcmt{positiveness/perfection%
}
%
\begin{isamarkuptext}%
Definitions of God (later shown to be equivalent under axiom \emph{A1b}):%
\end{isamarkuptext}\isamarkuptrue%
\isacommand{abbreviation}\isamarkupfalse%
\ God{\isacharcolon}{\isacharcolon}{\isachardoublequoteopen}{\isasymup}{\isasymlangle}{\isasymzero}{\isasymrangle}{\isachardoublequoteclose}\ {\isacharparenleft}{\isachardoublequoteopen}G{\isachardoublequoteclose}{\isacharparenright}\ \isakeyword{where}\ {\isachardoublequoteopen}G\ {\isasymequiv}\ {\isacharparenleft}{\isasymlambda}x{\isachardot}\ \isactrlbold {\isasymforall}Y{\isachardot}\ {\isasymP}\ Y\ \isactrlbold {\isasymrightarrow}\ Y\ x{\isacharparenright}{\isachardoublequoteclose}\isanewline
\isacommand{abbreviation}\isamarkupfalse%
\ God{\isacharunderscore}star{\isacharcolon}{\isacharcolon}{\isachardoublequoteopen}{\isasymup}{\isasymlangle}{\isasymzero}{\isasymrangle}{\isachardoublequoteclose}\ {\isacharparenleft}{\isachardoublequoteopen}G{\isacharasterisk}{\isachardoublequoteclose}{\isacharparenright}\ \isakeyword{where}\ {\isachardoublequoteopen}G{\isacharasterisk}\ {\isasymequiv}\ {\isacharparenleft}{\isasymlambda}x{\isachardot}\ \isactrlbold {\isasymforall}Y{\isachardot}\ {\isasymP}\ Y\ \isactrlbold {\isasymleftrightarrow}\ Y\ x{\isacharparenright}{\isachardoublequoteclose}%
\begin{isamarkuptext}%
Definitions needed to formalise \emph{A3}:%
\end{isamarkuptext}\isamarkuptrue%
\isacommand{abbreviation}\isamarkupfalse%
\ appliesToPositiveProps{\isacharcolon}{\isacharcolon}{\isachardoublequoteopen}{\isasymup}{\isasymlangle}{\isasymup}{\isasymlangle}{\isasymup}{\isasymlangle}{\isasymzero}{\isasymrangle}{\isasymrangle}{\isasymrangle}{\isachardoublequoteclose}\ {\isacharparenleft}{\isachardoublequoteopen}pos{\isachardoublequoteclose}{\isacharparenright}\ \isakeyword{where}\isanewline
\ \ {\isachardoublequoteopen}pos\ Z\ {\isasymequiv}\ \ \isactrlbold {\isasymforall}X{\isachardot}\ Z\ X\ \isactrlbold {\isasymrightarrow}\ {\isasymP}\ X{\isachardoublequoteclose}\isanewline
\isacommand{abbreviation}\isamarkupfalse%
\ intersectionOf{\isacharcolon}{\isacharcolon}{\isachardoublequoteopen}{\isasymup}{\isasymlangle}{\isasymup}{\isasymlangle}{\isasymzero}{\isasymrangle}{\isacharcomma}{\isasymup}{\isasymlangle}{\isasymup}{\isasymlangle}{\isasymzero}{\isasymrangle}{\isasymrangle}{\isasymrangle}{\isachardoublequoteclose}\ {\isacharparenleft}{\isachardoublequoteopen}intersec{\isachardoublequoteclose}{\isacharparenright}\ \isakeyword{where}\isanewline
\ \ {\isachardoublequoteopen}intersec\ X\ Z\ {\isasymequiv}\ \ \isactrlbold {\isasymbox}{\isacharparenleft}\isactrlbold {\isasymforall}x{\isachardot}{\isacharparenleft}X\ x\ \isactrlbold {\isasymleftrightarrow}\ {\isacharparenleft}\isactrlbold {\isasymforall}Y{\isachardot}\ {\isacharparenleft}Z\ Y{\isacharparenright}\ \isactrlbold {\isasymrightarrow}\ {\isacharparenleft}Y\ x{\isacharparenright}{\isacharparenright}{\isacharparenright}{\isacharparenright}{\isachardoublequoteclose}\ %
\isamarkupcmt{quantifier is possibilist%
}
\isanewline
\isacommand{abbreviation}\isamarkupfalse%
\ Entailment{\isacharcolon}{\isacharcolon}{\isachardoublequoteopen}{\isasymup}{\isasymlangle}{\isasymup}{\isasymlangle}{\isasymzero}{\isasymrangle}{\isacharcomma}{\isasymup}{\isasymlangle}{\isasymzero}{\isasymrangle}{\isasymrangle}{\isachardoublequoteclose}\ {\isacharparenleft}\isakeyword{infix}\ {\isachardoublequoteopen}{\isasymRrightarrow}{\isachardoublequoteclose}\ {\isadigit{6}}{\isadigit{0}}{\isacharparenright}\ \isakeyword{where}\isanewline
\ \ {\isachardoublequoteopen}X\ {\isasymRrightarrow}\ Y\ {\isasymequiv}\ \ \isactrlbold {\isasymbox}{\isacharparenleft}\isactrlbold {\isasymforall}\isactrlsup Ez{\isachardot}\ X\ z\ \isactrlbold {\isasymrightarrow}\ Y\ z{\isacharparenright}{\isachardoublequoteclose}%
\isamarkupsubsubsection{Axioms%
}
\isamarkuptrue%
\isacommand{axiomatization}\isamarkupfalse%
\ \isakeyword{where}\isanewline
\ \ A{\isadigit{1}}a{\isacharcolon}{\isachardoublequoteopen}{\isasymlfloor}\isactrlbold {\isasymforall}X{\isachardot}\ {\isasymP}\ {\isacharparenleft}\isactrlbold {\isasymrightharpoondown}X{\isacharparenright}\ \isactrlbold {\isasymrightarrow}\ \isactrlbold {\isasymnot}{\isacharparenleft}{\isasymP}\ X{\isacharparenright}\ {\isasymrfloor}{\isachardoublequoteclose}\ \isakeyword{and}\ \ \ \ \ \ %
\isamarkupcmt{axiom 11.3A%
}
\isanewline
\ \ A{\isadigit{1}}b{\isacharcolon}{\isachardoublequoteopen}{\isasymlfloor}\isactrlbold {\isasymforall}X{\isachardot}\ \isactrlbold {\isasymnot}{\isacharparenleft}{\isasymP}\ X{\isacharparenright}\ \isactrlbold {\isasymrightarrow}\ {\isasymP}\ {\isacharparenleft}\isactrlbold {\isasymrightharpoondown}X{\isacharparenright}{\isasymrfloor}{\isachardoublequoteclose}\ \isakeyword{and}\ \ \ \ \ \ \ %
\isamarkupcmt{axiom 11.3B%
}
\isanewline
\ \ A{\isadigit{2}}{\isacharcolon}\ {\isachardoublequoteopen}{\isasymlfloor}\isactrlbold {\isasymforall}X\ Y{\isachardot}\ {\isacharparenleft}{\isasymP}\ X\ \isactrlbold {\isasymand}\ {\isacharparenleft}X\ {\isasymRrightarrow}\ Y{\isacharparenright}{\isacharparenright}\ \isactrlbold {\isasymrightarrow}\ {\isasymP}\ Y{\isasymrfloor}{\isachardoublequoteclose}\ \isakeyword{and}\ \ \ %
\isamarkupcmt{axiom 11.5%
}
\isanewline
\ \ A{\isadigit{3}}{\isacharcolon}\ {\isachardoublequoteopen}{\isasymlfloor}\isactrlbold {\isasymforall}Z\ X{\isachardot}\ {\isacharparenleft}pos\ Z\ \isactrlbold {\isasymand}\ intersec\ X\ Z{\isacharparenright}\ \isactrlbold {\isasymrightarrow}\ {\isasymP}\ X{\isasymrfloor}{\isachardoublequoteclose}\ %
\isamarkupcmt{axiom 11.10%
}
\isanewline
\isanewline
\isacommand{lemma}\isamarkupfalse%
\ True\ \isacommand{nitpick}\isamarkupfalse%
{\isacharbrackleft}satisfy{\isacharbrackright}%
\isadelimproof
\ %
\endisadelimproof
%
\isatagproof
\isacommand{oops}\isamarkupfalse%
\ \ \ \ \ \ \ %
\isamarkupcmt{model found: axioms are consistent%
}
%
\endisatagproof
{\isafoldproof}%
%
\isadelimproof
%
\endisadelimproof
\isanewline
\ \ \ \ \isanewline
\isacommand{lemma}\isamarkupfalse%
\ {\isachardoublequoteopen}{\isasymlfloor}D{\isasymrfloor}{\isachardoublequoteclose}%
\isadelimproof
\ \ %
\endisadelimproof
%
\isatagproof
\isacommand{using}\isamarkupfalse%
\ A{\isadigit{1}}a\ A{\isadigit{1}}b\ A{\isadigit{2}}\ \isacommand{by}\isamarkupfalse%
\ blast\ %
\isamarkupcmt{axioms already imply \emph{D} axiom%
}
%
\endisatagproof
{\isafoldproof}%
%
\isadelimproof
%
\endisadelimproof
\isanewline
\isacommand{lemma}\isamarkupfalse%
\ {\isachardoublequoteopen}{\isasymlfloor}D{\isasymrfloor}{\isachardoublequoteclose}%
\isadelimproof
\ %
\endisadelimproof
%
\isatagproof
\isacommand{using}\isamarkupfalse%
\ A{\isadigit{1}}a\ A{\isadigit{3}}\ \isacommand{by}\isamarkupfalse%
\ metis%
\endisatagproof
{\isafoldproof}%
%
\isadelimproof
%
\endisadelimproof
%
\isamarkupsubsubsection{Theorems%
}
\isamarkuptrue%
\isacommand{lemma}\isamarkupfalse%
\ {\isachardoublequoteopen}{\isasymlfloor}\isactrlbold {\isasymexists}X{\isachardot}\ {\isasymP}\ X{\isasymrfloor}{\isachardoublequoteclose}%
\isadelimproof
\ %
\endisadelimproof
%
\isatagproof
\isacommand{using}\isamarkupfalse%
\ A{\isadigit{1}}b\ \isacommand{by}\isamarkupfalse%
\ auto%
\endisatagproof
{\isafoldproof}%
%
\isadelimproof
%
\endisadelimproof
\isanewline
\isacommand{lemma}\isamarkupfalse%
\ {\isachardoublequoteopen}{\isasymlfloor}\isactrlbold {\isasymexists}X{\isachardot}\ {\isasymP}\ X\ \isactrlbold {\isasymand}\ \ \isactrlbold {\isasymdiamond}\isactrlbold {\isasymexists}\isactrlsup E\ X{\isasymrfloor}{\isachardoublequoteclose}%
\isadelimproof
\ %
\endisadelimproof
%
\isatagproof
\isacommand{using}\isamarkupfalse%
\ A{\isadigit{1}}a\ A{\isadigit{1}}b\ A{\isadigit{2}}\ \isacommand{by}\isamarkupfalse%
\ metis%
\endisatagproof
{\isafoldproof}%
%
\isadelimproof
%
\endisadelimproof
%
\begin{isamarkuptext}%
Being self-identical is a positive property:%
\end{isamarkuptext}\isamarkuptrue%
\isacommand{lemma}\isamarkupfalse%
\ {\isachardoublequoteopen}{\isasymlfloor}{\isacharparenleft}\isactrlbold {\isasymexists}X{\isachardot}\ {\isasymP}\ X\ \isactrlbold {\isasymand}\ \ \isactrlbold {\isasymdiamond}\isactrlbold {\isasymexists}\isactrlsup E\ X{\isacharparenright}\ \isactrlbold {\isasymrightarrow}\ {\isasymP}\ {\isacharparenleft}{\isasymlambda}x\ w{\isachardot}\ x\ {\isacharequal}\ x{\isacharparenright}{\isasymrfloor}{\isachardoublequoteclose}%
\isadelimproof
\ %
\endisadelimproof
%
\isatagproof
\isacommand{using}\isamarkupfalse%
\ A{\isadigit{2}}\ \isacommand{by}\isamarkupfalse%
\ fastforce%
\endisatagproof
{\isafoldproof}%
%
\isadelimproof
%
\endisadelimproof
%
\begin{isamarkuptext}%
Proposition 11.6%
\end{isamarkuptext}\isamarkuptrue%
\isacommand{lemma}\isamarkupfalse%
\ {\isachardoublequoteopen}{\isasymlfloor}{\isacharparenleft}\isactrlbold {\isasymexists}X{\isachardot}\ {\isasymP}\ X{\isacharparenright}\ \isactrlbold {\isasymrightarrow}\ {\isasymP}\ {\isacharparenleft}{\isasymlambda}x\ w{\isachardot}\ x\ {\isacharequal}\ x{\isacharparenright}{\isasymrfloor}{\isachardoublequoteclose}%
\isadelimproof
\ %
\endisadelimproof
%
\isatagproof
\isacommand{using}\isamarkupfalse%
\ A{\isadigit{2}}\ \isacommand{by}\isamarkupfalse%
\ fastforce%
\endisatagproof
{\isafoldproof}%
%
\isadelimproof
%
\endisadelimproof
\isanewline
\ \ \ \ \isanewline
\isacommand{lemma}\isamarkupfalse%
\ {\isachardoublequoteopen}{\isasymlfloor}{\isasymP}\ {\isacharparenleft}{\isasymlambda}x\ w{\isachardot}\ x\ {\isacharequal}\ x{\isacharparenright}{\isasymrfloor}{\isachardoublequoteclose}%
\isadelimproof
\ %
\endisadelimproof
%
\isatagproof
\isacommand{using}\isamarkupfalse%
\ A{\isadigit{1}}b\ A{\isadigit{2}}\ \ \isacommand{by}\isamarkupfalse%
\ blast%
\endisatagproof
{\isafoldproof}%
%
\isadelimproof
%
\endisadelimproof
\isanewline
\isacommand{lemma}\isamarkupfalse%
\ {\isachardoublequoteopen}{\isasymlfloor}{\isasymP}\ {\isacharparenleft}{\isasymlambda}x\ w{\isachardot}\ x\ {\isacharequal}\ x{\isacharparenright}{\isasymrfloor}{\isachardoublequoteclose}%
\isadelimproof
\ %
\endisadelimproof
%
\isatagproof
\isacommand{using}\isamarkupfalse%
\ A{\isadigit{3}}\ \isacommand{by}\isamarkupfalse%
\ metis%
\endisatagproof
{\isafoldproof}%
%
\isadelimproof
%
\endisadelimproof
%
\begin{isamarkuptext}%
Being non-self-identical is a negative property:%
\end{isamarkuptext}\isamarkuptrue%
\isacommand{lemma}\isamarkupfalse%
\ {\isachardoublequoteopen}{\isasymlfloor}{\isacharparenleft}\isactrlbold {\isasymexists}X{\isachardot}\ {\isasymP}\ X\ \ \isactrlbold {\isasymand}\ \isactrlbold {\isasymdiamond}\isactrlbold {\isasymexists}\isactrlsup E\ X{\isacharparenright}\ \isactrlbold {\isasymrightarrow}\ \ {\isasymP}\ {\isacharparenleft}\isactrlbold {\isasymrightharpoondown}\ {\isacharparenleft}{\isasymlambda}x\ w{\isachardot}\ {\isasymnot}x\ {\isacharequal}\ x{\isacharparenright}{\isacharparenright}{\isasymrfloor}{\isachardoublequoteclose}\ \isanewline
%
\isadelimproof
\ \ %
\endisadelimproof
%
\isatagproof
\isacommand{using}\isamarkupfalse%
\ A{\isadigit{2}}\ \isacommand{by}\isamarkupfalse%
\ fastforce%
\endisatagproof
{\isafoldproof}%
%
\isadelimproof
\isanewline
%
\endisadelimproof
\ \ \ \ \isanewline
\isacommand{lemma}\isamarkupfalse%
\ {\isachardoublequoteopen}{\isasymlfloor}{\isacharparenleft}\isactrlbold {\isasymexists}X{\isachardot}\ {\isasymP}\ X{\isacharparenright}\ \isactrlbold {\isasymrightarrow}\ \ {\isasymP}\ {\isacharparenleft}\isactrlbold {\isasymrightharpoondown}\ {\isacharparenleft}{\isasymlambda}x\ w{\isachardot}\ {\isasymnot}x\ {\isacharequal}\ x{\isacharparenright}{\isacharparenright}{\isasymrfloor}{\isachardoublequoteclose}%
\isadelimproof
\ %
\endisadelimproof
%
\isatagproof
\isacommand{using}\isamarkupfalse%
\ A{\isadigit{2}}\ \isacommand{by}\isamarkupfalse%
\ fastforce%
\endisatagproof
{\isafoldproof}%
%
\isadelimproof
%
\endisadelimproof
\isanewline
\isacommand{lemma}\isamarkupfalse%
\ {\isachardoublequoteopen}{\isasymlfloor}{\isacharparenleft}\isactrlbold {\isasymexists}X{\isachardot}\ {\isasymP}\ X{\isacharparenright}\ \isactrlbold {\isasymrightarrow}\ \ {\isasymP}\ {\isacharparenleft}\isactrlbold {\isasymrightharpoondown}\ {\isacharparenleft}{\isasymlambda}x\ w{\isachardot}\ {\isasymnot}x\ {\isacharequal}\ x{\isacharparenright}{\isacharparenright}{\isasymrfloor}{\isachardoublequoteclose}%
\isadelimproof
\ %
\endisadelimproof
%
\isatagproof
\isacommand{using}\isamarkupfalse%
\ A{\isadigit{3}}\ \isacommand{by}\isamarkupfalse%
\ metis%
\endisatagproof
{\isafoldproof}%
%
\isadelimproof
%
\endisadelimproof
%
\begin{isamarkuptext}%
Proposition 11.7%
\end{isamarkuptext}\isamarkuptrue%
\isacommand{lemma}\isamarkupfalse%
\ {\isachardoublequoteopen}{\isasymlfloor}{\isacharparenleft}\isactrlbold {\isasymexists}X{\isachardot}\ {\isasymP}\ X{\isacharparenright}\ \isactrlbold {\isasymrightarrow}\ \isactrlbold {\isasymnot}{\isasymP}\ {\isacharparenleft}{\isacharparenleft}{\isasymlambda}x\ w{\isachardot}\ {\isasymnot}x\ {\isacharequal}\ x{\isacharparenright}{\isacharparenright}{\isasymrfloor}{\isachardoublequoteclose}%
\isadelimproof
\ \ %
\endisadelimproof
%
\isatagproof
\isacommand{using}\isamarkupfalse%
\ A{\isadigit{1}}a\ A{\isadigit{2}}\ \isacommand{by}\isamarkupfalse%
\ blast%
\endisatagproof
{\isafoldproof}%
%
\isadelimproof
%
\endisadelimproof
\isanewline
\isacommand{lemma}\isamarkupfalse%
\ {\isachardoublequoteopen}{\isasymlfloor}\isactrlbold {\isasymnot}{\isasymP}\ {\isacharparenleft}{\isasymlambda}x\ w{\isachardot}\ {\isasymnot}x\ {\isacharequal}\ x{\isacharparenright}{\isasymrfloor}{\isachardoublequoteclose}%
\isadelimproof
\ \ %
\endisadelimproof
%
\isatagproof
\isacommand{using}\isamarkupfalse%
\ A{\isadigit{1}}a\ A{\isadigit{2}}\ \isacommand{by}\isamarkupfalse%
\ blast%
\endisatagproof
{\isafoldproof}%
%
\isadelimproof
%
\endisadelimproof
%
\begin{isamarkuptext}%
Proposition 11.8 (Informal Proposition 1) - Positive properties are possibly instantiated:%
\end{isamarkuptext}\isamarkuptrue%
\isacommand{theorem}\isamarkupfalse%
\ T{\isadigit{1}}{\isacharcolon}\ {\isachardoublequoteopen}{\isasymlfloor}\isactrlbold {\isasymforall}X{\isachardot}\ {\isasymP}\ X\ \isactrlbold {\isasymrightarrow}\ \isactrlbold {\isasymdiamond}\isactrlbold {\isasymexists}\isactrlsup E\ X{\isasymrfloor}{\isachardoublequoteclose}%
\isadelimproof
\ %
\endisadelimproof
%
\isatagproof
\isacommand{using}\isamarkupfalse%
\ A{\isadigit{1}}a\ A{\isadigit{2}}\ \isacommand{by}\isamarkupfalse%
\ blast%
\endisatagproof
{\isafoldproof}%
%
\isadelimproof
%
\endisadelimproof
%
\begin{isamarkuptext}%
Proposition 11.14 - Both defs (\emph{God/God*}) are equivalent. For improved performance we may prefer to use one or the other:%
\end{isamarkuptext}\isamarkuptrue%
\isacommand{lemma}\isamarkupfalse%
\ GodDefsAreEquivalent{\isacharcolon}\ {\isachardoublequoteopen}{\isasymlfloor}\isactrlbold {\isasymforall}x{\isachardot}\ G\ x\ \isactrlbold {\isasymleftrightarrow}\ G{\isacharasterisk}\ x{\isasymrfloor}{\isachardoublequoteclose}%
\isadelimproof
\ %
\endisadelimproof
%
\isatagproof
\isacommand{using}\isamarkupfalse%
\ A{\isadigit{1}}b\ \isacommand{by}\isamarkupfalse%
\ force%
\endisatagproof
{\isafoldproof}%
%
\isadelimproof
%
\endisadelimproof
%
\begin{isamarkuptext}%
Proposition 11.15 - Possibilist existence of \emph{God} directly implies \emph{A1b}:%
\end{isamarkuptext}\isamarkuptrue%
\isacommand{lemma}\isamarkupfalse%
\ {\isachardoublequoteopen}{\isasymlfloor}\isactrlbold {\isasymexists}\ G{\isacharasterisk}\ \isactrlbold {\isasymrightarrow}\ {\isacharparenleft}\isactrlbold {\isasymforall}X{\isachardot}\ \isactrlbold {\isasymnot}{\isacharparenleft}{\isasymP}\ X{\isacharparenright}\ \isactrlbold {\isasymrightarrow}\ {\isasymP}\ {\isacharparenleft}\isactrlbold {\isasymrightharpoondown}X{\isacharparenright}{\isacharparenright}{\isasymrfloor}{\isachardoublequoteclose}%
\isadelimproof
\ %
\endisadelimproof
%
\isatagproof
\isacommand{by}\isamarkupfalse%
\ meson%
\endisatagproof
{\isafoldproof}%
%
\isadelimproof
%
\endisadelimproof
%
\begin{isamarkuptext}%
Proposition 11.16 - \emph{A3} implies \emph{P(G)} (local consequence):%
\end{isamarkuptext}\isamarkuptrue%
\isacommand{lemma}\isamarkupfalse%
\ A{\isadigit{3}}implT{\isadigit{2}}{\isacharunderscore}local{\isacharcolon}\ {\isachardoublequoteopen}{\isasymlfloor}{\isacharparenleft}\isactrlbold {\isasymforall}Z\ X{\isachardot}\ {\isacharparenleft}pos\ Z\ \isactrlbold {\isasymand}\ intersec\ X\ Z{\isacharparenright}\ \isactrlbold {\isasymrightarrow}\ {\isasymP}\ X{\isacharparenright}\ \isactrlbold {\isasymrightarrow}\ {\isasymP}\ G{\isasymrfloor}{\isachardoublequoteclose}\isanewline
%
\isadelimproof
%
\endisadelimproof
%
\isatagproof
\isacommand{proof}\isamarkupfalse%
\ {\isacharminus}\isanewline
\ \ \isacommand{{\isacharbraceleft}}\isamarkupfalse%
\isanewline
\ \ \isacommand{fix}\isamarkupfalse%
\ w\isanewline
\ \ \isacommand{have}\isamarkupfalse%
\ {\isadigit{1}}{\isacharcolon}\ {\isachardoublequoteopen}pos\ {\isasymP}\ w{\isachardoublequoteclose}\ \isacommand{by}\isamarkupfalse%
\ simp\isanewline
\ \ \isacommand{have}\isamarkupfalse%
\ {\isadigit{2}}{\isacharcolon}\ {\isachardoublequoteopen}intersec\ G\ {\isasymP}\ w{\isachardoublequoteclose}\ \isacommand{by}\isamarkupfalse%
\ simp\isanewline
\ \ \isacommand{{\isacharbraceleft}}\isamarkupfalse%
\ \ \ \ \isanewline
\ \ \ \ \isacommand{assume}\isamarkupfalse%
\ {\isachardoublequoteopen}{\isacharparenleft}\isactrlbold {\isasymforall}Z\ X{\isachardot}\ {\isacharparenleft}pos\ Z\ \isactrlbold {\isasymand}\ intersec\ X\ Z{\isacharparenright}\ \isactrlbold {\isasymrightarrow}\ {\isasymP}\ X{\isacharparenright}\ w{\isachardoublequoteclose}\isanewline
\ \ \ \ \isacommand{hence}\isamarkupfalse%
\ {\isachardoublequoteopen}{\isacharparenleft}\isactrlbold {\isasymforall}X{\isachardot}\ {\isacharparenleft}{\isacharparenleft}pos\ {\isasymP}{\isacharparenright}\ \isactrlbold {\isasymand}\ {\isacharparenleft}intersec\ X\ {\isasymP}{\isacharparenright}{\isacharparenright}\ \isactrlbold {\isasymrightarrow}\ {\isasymP}\ X{\isacharparenright}\ w{\isachardoublequoteclose}\ \ \isacommand{by}\isamarkupfalse%
\ {\isacharparenleft}rule\ allE{\isacharparenright}\ \ \ \isanewline
\ \ \ \ \isacommand{hence}\isamarkupfalse%
\ {\isachardoublequoteopen}{\isacharparenleft}{\isacharparenleft}{\isacharparenleft}pos\ {\isasymP}{\isacharparenright}\ \isactrlbold {\isasymand}\ {\isacharparenleft}intersec\ G\ {\isasymP}{\isacharparenright}{\isacharparenright}\ \isactrlbold {\isasymrightarrow}\ {\isasymP}\ G{\isacharparenright}\ w{\isachardoublequoteclose}\ \isacommand{by}\isamarkupfalse%
\ {\isacharparenleft}rule\ allE{\isacharparenright}\isanewline
\ \ \ \ \isacommand{hence}\isamarkupfalse%
\ {\isadigit{3}}{\isacharcolon}\ {\isachardoublequoteopen}{\isacharparenleft}{\isacharparenleft}pos\ {\isasymP}\ \isactrlbold {\isasymand}\ intersec\ G\ {\isasymP}{\isacharparenright}\ w{\isacharparenright}\ {\isasymlongrightarrow}\ {\isasymP}\ G\ w{\isachardoublequoteclose}\ \isacommand{by}\isamarkupfalse%
\ simp\isanewline
\ \ \ \ \isacommand{hence}\isamarkupfalse%
\ {\isadigit{4}}{\isacharcolon}\ {\isachardoublequoteopen}{\isacharparenleft}{\isacharparenleft}pos\ {\isasymP}{\isacharparenright}\ \isactrlbold {\isasymand}\ {\isacharparenleft}intersec\ G\ {\isasymP}{\isacharparenright}{\isacharparenright}\ w{\isachardoublequoteclose}\ \isacommand{using}\isamarkupfalse%
\ {\isadigit{1}}\ {\isadigit{2}}\ \isacommand{by}\isamarkupfalse%
\ simp\isanewline
\ \ \ \ \isacommand{from}\isamarkupfalse%
\ {\isadigit{3}}\ {\isadigit{4}}\ \isacommand{have}\isamarkupfalse%
\ {\isachardoublequoteopen}{\isasymP}\ G\ w{\isachardoublequoteclose}\ \isacommand{by}\isamarkupfalse%
\ {\isacharparenleft}rule\ mp{\isacharparenright}\isanewline
\ \ \isacommand{{\isacharbraceright}}\isamarkupfalse%
\isanewline
\ \ \isacommand{hence}\isamarkupfalse%
\ {\isachardoublequoteopen}{\isacharparenleft}\isactrlbold {\isasymforall}Z\ X{\isachardot}\ {\isacharparenleft}pos\ Z\ \isactrlbold {\isasymand}\ intersec\ X\ Z{\isacharparenright}\ \isactrlbold {\isasymrightarrow}\ {\isasymP}\ X{\isacharparenright}\ w\ \ {\isasymlongrightarrow}\ {\isasymP}\ G\ w{\isachardoublequoteclose}\ \isacommand{by}\isamarkupfalse%
\ {\isacharparenleft}rule\ impI{\isacharparenright}\isanewline
\ \ \isacommand{{\isacharbraceright}}\isamarkupfalse%
\ \isanewline
\ \ \isacommand{thus}\isamarkupfalse%
\ {\isacharquery}thesis\ \isacommand{by}\isamarkupfalse%
\ {\isacharparenleft}rule\ allI{\isacharparenright}\isanewline
\isacommand{qed}\isamarkupfalse%
%
\endisatagproof
{\isafoldproof}%
%
\isadelimproof
%
\endisadelimproof
%
\begin{isamarkuptext}%
\emph{A3} implies \isa{P{\isacharparenleft}G{\isacharparenright}} (as global consequence):%
\end{isamarkuptext}\isamarkuptrue%
\isacommand{lemma}\isamarkupfalse%
\ A{\isadigit{3}}implT{\isadigit{2}}{\isacharunderscore}global{\isacharcolon}\ {\isachardoublequoteopen}{\isasymlfloor}\isactrlbold {\isasymforall}Z\ X{\isachardot}\ {\isacharparenleft}pos\ Z\ \isactrlbold {\isasymand}\ intersec\ X\ Z{\isacharparenright}\ \isactrlbold {\isasymrightarrow}\ {\isasymP}\ X{\isasymrfloor}\ {\isasymlongrightarrow}\ {\isasymlfloor}{\isasymP}\ G{\isasymrfloor}{\isachardoublequoteclose}\ \isanewline
%
\isadelimproof
\ \ %
\endisadelimproof
%
\isatagproof
\isacommand{using}\isamarkupfalse%
\ A{\isadigit{3}}implT{\isadigit{2}}{\isacharunderscore}local\ \isacommand{by}\isamarkupfalse%
\ {\isacharparenleft}rule\ localImpGlobalCons{\isacharparenright}%
\endisatagproof
{\isafoldproof}%
%
\isadelimproof
%
\endisadelimproof
%
\begin{isamarkuptext}%
Being Godlike is a positive property. Note that this theorem can be axiomatized directly,
as noted by Dana Scott (see \cite{Fitting} p. 152). We will do so for the second part.%
\end{isamarkuptext}\isamarkuptrue%
\isacommand{theorem}\isamarkupfalse%
\ T{\isadigit{2}}{\isacharcolon}\ {\isachardoublequoteopen}{\isasymlfloor}{\isasymP}\ G{\isasymrfloor}{\isachardoublequoteclose}%
\isadelimproof
\ %
\endisadelimproof
%
\isatagproof
\isacommand{using}\isamarkupfalse%
\ A{\isadigit{3}}implT{\isadigit{2}}{\isacharunderscore}global\ A{\isadigit{3}}\ \isacommand{by}\isamarkupfalse%
\ simp%
\endisatagproof
{\isafoldproof}%
%
\isadelimproof
%
\endisadelimproof
%
\begin{isamarkuptext}%
Theorem 11.17 (Informal Proposition 3) - Possibly God exists:%
\end{isamarkuptext}\isamarkuptrue%
\isacommand{theorem}\isamarkupfalse%
\ T{\isadigit{3}}{\isacharcolon}\ {\isachardoublequoteopen}{\isasymlfloor}\isactrlbold {\isasymdiamond}\isactrlbold {\isasymexists}\isactrlsup E\ G{\isasymrfloor}{\isachardoublequoteclose}%
\isadelimproof
\ \ %
\endisadelimproof
%
\isatagproof
\isacommand{using}\isamarkupfalse%
\ T{\isadigit{1}}\ T{\isadigit{2}}\ \isacommand{by}\isamarkupfalse%
\ simp\isanewline
%
\endisatagproof
{\isafoldproof}%
%
\isadelimproof
%
\endisadelimproof
%
\isadelimtheory
%
\endisadelimtheory
%
\isatagtheory
%
\endisatagtheory
{\isafoldtheory}%
%
\isadelimtheory
%
\endisadelimtheory
%
\end{isabellebody}%
%%% Local Variables:
%%% mode: latex
%%% TeX-master: "root"
%%% End:


%
\begin{isabellebody}%
\setisabellecontext{GoedelProof{\isacharunderscore}P{\isadigit{2}}}%
%
%
%
%
%
%
%
\isamarkupsubsection{Part II - God's Existence is Necessary if Possible%
}
\isamarkuptrue%
%
\begin{isamarkuptext}%
We show here that God's necessary existence follows from its possible existence by adding some
 additional (potentially controversial) assumptions including an \emph{essentialist} premise 
 and the \emph{S5} axioms.
 Further results like monotheism and the rejection of free will (\emph{modal collapse}) are also proved.%
\end{isamarkuptext}\isamarkuptrue%
%
\isamarkupsubsubsection{General Definitions%
}
\isamarkuptrue%
\isacommand{abbreviation}\isamarkupfalse%
\ existencePredicate{\isacharcolon}{\isacharcolon}{\isachardoublequoteopen}{\isasymup}{\isasymlangle}{\isasymzero}{\isasymrangle}{\isachardoublequoteclose}\ {\isacharparenleft}{\isachardoublequoteopen}E{\isacharbang}{\isachardoublequoteclose}{\isacharparenright}\ \isakeyword{where}\isanewline
\ \ {\isachardoublequoteopen}E{\isacharbang}\ x\ \ {\isasymequiv}\ {\isacharparenleft}{\isasymlambda}w{\isachardot}\ {\isacharparenleft}\isactrlbold {\isasymexists}\isactrlsup Ey{\isachardot}\ y\isactrlbold {\isasymapprox}x{\isacharparenright}\ w{\isacharparenright}{\isachardoublequoteclose}\ \isanewline
\ \ \isanewline
\isacommand{consts}\isamarkupfalse%
\ positiveProperty{\isacharcolon}{\isacharcolon}{\isachardoublequoteopen}{\isasymup}{\isasymlangle}{\isasymup}{\isasymlangle}{\isasymzero}{\isasymrangle}{\isasymrangle}{\isachardoublequoteclose}\ {\isacharparenleft}{\isachardoublequoteopen}{\isasymP}{\isachardoublequoteclose}{\isacharparenright}\isanewline
\ \ \isanewline
\isacommand{abbreviation}\isamarkupfalse%
\ God{\isacharcolon}{\isacharcolon}{\isachardoublequoteopen}{\isasymup}{\isasymlangle}{\isasymzero}{\isasymrangle}{\isachardoublequoteclose}\ {\isacharparenleft}{\isachardoublequoteopen}G{\isachardoublequoteclose}{\isacharparenright}\ \isakeyword{where}\ {\isachardoublequoteopen}G\ {\isasymequiv}\ {\isacharparenleft}{\isasymlambda}x{\isachardot}\ \isactrlbold {\isasymforall}Y{\isachardot}\ {\isasymP}\ Y\ \isactrlbold {\isasymrightarrow}\ Y\ x{\isacharparenright}{\isachardoublequoteclose}\isanewline
\isacommand{abbreviation}\isamarkupfalse%
\ God{\isacharunderscore}star{\isacharcolon}{\isacharcolon}{\isachardoublequoteopen}{\isasymup}{\isasymlangle}{\isasymzero}{\isasymrangle}{\isachardoublequoteclose}\ {\isacharparenleft}{\isachardoublequoteopen}G{\isacharasterisk}{\isachardoublequoteclose}{\isacharparenright}\ \isakeyword{where}\isanewline
\ \ {\isachardoublequoteopen}G{\isacharasterisk}\ {\isasymequiv}\ {\isacharparenleft}{\isasymlambda}x{\isachardot}\ \isactrlbold {\isasymforall}Y{\isachardot}\ {\isasymP}\ Y\ \isactrlbold {\isasymleftrightarrow}\ Y\ x{\isacharparenright}{\isachardoublequoteclose}\isanewline
\ \ \isanewline
\isacommand{abbreviation}\isamarkupfalse%
\ Entailment{\isacharcolon}{\isacharcolon}{\isachardoublequoteopen}{\isasymup}{\isasymlangle}{\isasymup}{\isasymlangle}{\isasymzero}{\isasymrangle}{\isacharcomma}{\isasymup}{\isasymlangle}{\isasymzero}{\isasymrangle}{\isasymrangle}{\isachardoublequoteclose}\ {\isacharparenleft}\isakeyword{infix}\ {\isachardoublequoteopen}{\isasymRrightarrow}{\isachardoublequoteclose}\ {\isadigit{6}}{\isadigit{0}}{\isacharparenright}\ \isakeyword{where}\ \isanewline
\ \ {\isachardoublequoteopen}X\ {\isasymRrightarrow}\ Y\ {\isasymequiv}\ \ \isactrlbold {\isasymbox}{\isacharparenleft}\isactrlbold {\isasymforall}\isactrlsup Ez{\isachardot}\ X\ z\ \isactrlbold {\isasymrightarrow}\ Y\ z{\isacharparenright}{\isachardoublequoteclose}%
\isamarkupsubsubsection{Results from Part I%
}
\isamarkuptrue%
%
\begin{isamarkuptext}%
Note that the only use G\"odel makes of axiom A3 is to show that being Godlike is a positive property (\emph{T2}). 
 We follow therefore Scott's proposal and take (\emph{T2}) directly as an axiom:%
\end{isamarkuptext}\isamarkuptrue%
\isacommand{axiomatization}\isamarkupfalse%
\ \isakeyword{where}\isanewline
\ \ A{\isadigit{1}}a{\isacharcolon}{\isachardoublequoteopen}{\isasymlfloor}\isactrlbold {\isasymforall}X{\isachardot}\ {\isasymP}\ {\isacharparenleft}\isactrlbold {\isasymrightharpoondown}X{\isacharparenright}\ \isactrlbold {\isasymrightarrow}\ \isactrlbold {\isasymnot}{\isacharparenleft}{\isasymP}\ X{\isacharparenright}\ {\isasymrfloor}{\isachardoublequoteclose}\ \isakeyword{and}\ \ \ \ \ \ \ \ \ \ %
\isamarkupcmt{axiom 11.3A%
}
\isanewline
\ \ A{\isadigit{1}}b{\isacharcolon}{\isachardoublequoteopen}{\isasymlfloor}\isactrlbold {\isasymforall}X{\isachardot}\ \isactrlbold {\isasymnot}{\isacharparenleft}{\isasymP}\ X{\isacharparenright}\ \isactrlbold {\isasymrightarrow}\ {\isasymP}\ {\isacharparenleft}\isactrlbold {\isasymrightharpoondown}X{\isacharparenright}{\isasymrfloor}{\isachardoublequoteclose}\ \isakeyword{and}\ \ \ \ \ \ \ \ \ \ \ %
\isamarkupcmt{axiom 11.3B%
}
\isanewline
\ \ A{\isadigit{2}}{\isacharcolon}\ {\isachardoublequoteopen}{\isasymlfloor}\isactrlbold {\isasymforall}X\ Y{\isachardot}\ {\isacharparenleft}{\isasymP}\ X\ \isactrlbold {\isasymand}\ {\isacharparenleft}X\ {\isasymRrightarrow}\ Y{\isacharparenright}{\isacharparenright}\ \isactrlbold {\isasymrightarrow}\ {\isasymP}\ Y{\isasymrfloor}{\isachardoublequoteclose}\ \isakeyword{and}\ \ \ \ %
\isamarkupcmt{axiom 11.5%
}
\isanewline
\ \ T{\isadigit{2}}{\isacharcolon}\ {\isachardoublequoteopen}{\isasymlfloor}{\isasymP}\ G{\isasymrfloor}{\isachardoublequoteclose}\ \ \ \ \ \ \ \ \ \ \ \ \ \ \ \ \ \ \ \ \ \ \ \ \ \ \ \ \ \ \ \ \ \ %
\isamarkupcmt{proposition 11.16%
}
\isanewline
\ \ \ \ \ \ \ \ \isanewline
\isacommand{lemma}\isamarkupfalse%
\ True\ \isacommand{nitpick}\isamarkupfalse%
{\isacharbrackleft}satisfy{\isacharbrackright}%
\ %
%
\isacommand{oops}\isamarkupfalse%
\ %
\isamarkupcmt{model found: axioms are consistent%
}
%
%
%
\isanewline
\ \ \ \ \isanewline
\isacommand{lemma}\isamarkupfalse%
\ {\isachardoublequoteopen}{\isasymlfloor}D{\isasymrfloor}{\isachardoublequoteclose}%
\ \ %
%
\isacommand{using}\isamarkupfalse%
\ A{\isadigit{1}}a\ A{\isadigit{1}}b\ A{\isadigit{2}}\ \isacommand{by}\isamarkupfalse%
\ blast\ %
\isamarkupcmt{axioms already imply \emph{D} axiom%
}
%
%
%
\isanewline
\ \ \ \ \isanewline
\isacommand{lemma}\isamarkupfalse%
\ GodDefsAreEquivalent{\isacharcolon}\ {\isachardoublequoteopen}{\isasymlfloor}\isactrlbold {\isasymforall}x{\isachardot}\ G\ x\ \isactrlbold {\isasymleftrightarrow}\ G{\isacharasterisk}\ x{\isasymrfloor}{\isachardoublequoteclose}%
\ %
%
\isacommand{using}\isamarkupfalse%
\ A{\isadigit{1}}b\ \isacommand{by}\isamarkupfalse%
\ fastforce%
%
%
\ \isanewline
\ \ \ \ \isanewline
\isacommand{theorem}\isamarkupfalse%
\ T{\isadigit{1}}{\isacharcolon}\ {\isachardoublequoteopen}{\isasymlfloor}\isactrlbold {\isasymforall}X{\isachardot}\ {\isasymP}\ X\ \isactrlbold {\isasymrightarrow}\ \isactrlbold {\isasymdiamond}\isactrlbold {\isasymexists}\isactrlsup E\ X{\isasymrfloor}{\isachardoublequoteclose}\ \isanewline
%
\ \ %
%
\isacommand{using}\isamarkupfalse%
\ A{\isadigit{1}}a\ A{\isadigit{2}}\ \isacommand{by}\isamarkupfalse%
\ blast\ \ %
\isamarkupcmt{positive properties are possibly instantiated%
}
%
%
\ \ \isanewline
%
\isacommand{theorem}\isamarkupfalse%
\ T{\isadigit{3}}{\isacharcolon}\ {\isachardoublequoteopen}{\isasymlfloor}\isactrlbold {\isasymdiamond}\isactrlbold {\isasymexists}\isactrlsup E\ G{\isasymrfloor}{\isachardoublequoteclose}%
\ %
%
\isacommand{using}\isamarkupfalse%
\ T{\isadigit{1}}\ T{\isadigit{2}}\ \isacommand{by}\isamarkupfalse%
\ simp\ \ %
\isamarkupcmt{God exists possibly%
}
%
%
%
%
\isamarkupsubsubsection{Axioms%
}
\isamarkuptrue%
%
\begin{isamarkuptext}%
\isa{{\isasymP}} satisfies the so-called stability conditions (see \cite{Fitting}, p. 124), which means
 it designates rigidly (note that this makes for an \emph{essentialist} assumption).%
\end{isamarkuptext}\isamarkuptrue%
\isacommand{axiomatization}\isamarkupfalse%
\ \isakeyword{where}\isanewline
\ \ \ \ \ \ A{\isadigit{4}}a{\isacharcolon}\ {\isachardoublequoteopen}{\isasymlfloor}\isactrlbold {\isasymforall}X{\isachardot}\ {\isasymP}\ X\ \isactrlbold {\isasymrightarrow}\ \isactrlbold {\isasymbox}{\isacharparenleft}{\isasymP}\ X{\isacharparenright}{\isasymrfloor}{\isachardoublequoteclose}\ \ \ \ \ \ %
\isamarkupcmt{axiom 11.11%
}
\isanewline
\isacommand{lemma}\isamarkupfalse%
\ A{\isadigit{4}}b{\isacharcolon}\ {\isachardoublequoteopen}{\isasymlfloor}\isactrlbold {\isasymforall}X{\isachardot}\ \isactrlbold {\isasymnot}{\isacharparenleft}{\isasymP}\ X{\isacharparenright}\ \isactrlbold {\isasymrightarrow}\ \isactrlbold {\isasymbox}\isactrlbold {\isasymnot}{\isacharparenleft}{\isasymP}\ X{\isacharparenright}{\isasymrfloor}{\isachardoublequoteclose}%
\ %
%
\isacommand{using}\isamarkupfalse%
\ A{\isadigit{1}}a\ A{\isadigit{1}}b\ A{\isadigit{4}}a\ \isacommand{by}\isamarkupfalse%
\ blast%
%
%
\isanewline
\ \ \ \ \isanewline
\isacommand{abbreviation}\isamarkupfalse%
\ rigidPred{\isacharcolon}{\isacharcolon}{\isachardoublequoteopen}{\isacharparenleft}{\isacharprime}t{\isasymRightarrow}io{\isacharparenright}{\isasymRightarrow}io{\isachardoublequoteclose}\ \isakeyword{where}\isanewline
\ {\isachardoublequoteopen}rigidPred\ {\isasymtau}\ {\isasymequiv}\ {\isacharparenleft}{\isasymlambda}{\isasymbeta}{\isachardot}\ \isactrlbold {\isasymbox}{\isacharparenleft}{\isacharparenleft}{\isasymlambda}z{\isachardot}\ {\isasymbeta}\ \isactrlbold {\isasymapprox}\ z{\isacharparenright}\ \isactrlbold {\isasymdown}{\isasymtau}{\isacharparenright}{\isacharparenright}\ \isactrlbold {\isasymdown}{\isasymtau}{\isachardoublequoteclose}\isanewline
\ \isanewline
\isacommand{lemma}\isamarkupfalse%
\ {\isachardoublequoteopen}{\isasymlfloor}rigidPred\ {\isasymP}{\isasymrfloor}{\isachardoublequoteclose}\ \isanewline
%
\ \ %
%
\isacommand{using}\isamarkupfalse%
\ A{\isadigit{4}}a\ A{\isadigit{4}}b\ \isacommand{by}\isamarkupfalse%
\ blast\ %
\isamarkupcmt{\isa{{\isasymP}} is therefore rigid%
}
%
%
\isanewline
%
\ \ \ \ \isanewline
\isacommand{lemma}\isamarkupfalse%
\ True\ \isacommand{nitpick}\isamarkupfalse%
{\isacharbrackleft}satisfy{\isacharbrackright}%
\ %
%
\isacommand{oops}\isamarkupfalse%
\ %
\isamarkupcmt{model found: so far all axioms A1-4 consistent%
}
%
%
%
%
\isamarkupsubsubsection{Theorems%
}
\isamarkuptrue%
%
\begin{isamarkuptext}%
Remark: Essence is defined here (and in Fitting's variant) in the version of Scott; G\"odel's original version leads to the inconsistency
 reported in \cite{C55,C60}%
\end{isamarkuptext}\isamarkuptrue%
\isacommand{abbreviation}\isamarkupfalse%
\ essenceOf{\isacharcolon}{\isacharcolon}{\isachardoublequoteopen}{\isasymup}{\isasymlangle}{\isasymup}{\isasymlangle}{\isasymzero}{\isasymrangle}{\isacharcomma}{\isasymzero}{\isasymrangle}{\isachardoublequoteclose}\ {\isacharparenleft}{\isachardoublequoteopen}{\isasymE}{\isachardoublequoteclose}{\isacharparenright}\ \isakeyword{where}\isanewline
\ \ {\isachardoublequoteopen}{\isasymE}\ Y\ x\ {\isasymequiv}\ {\isacharparenleft}Y\ x{\isacharparenright}\ \isactrlbold {\isasymand}\ {\isacharparenleft}\isactrlbold {\isasymforall}Z{\isachardot}\ Z\ x\ \isactrlbold {\isasymrightarrow}\ Y\ {\isasymRrightarrow}\ Z{\isacharparenright}{\isachardoublequoteclose}\ \ \ \isanewline
\isacommand{abbreviation}\isamarkupfalse%
\ beingIdenticalTo{\isacharcolon}{\isacharcolon}{\isachardoublequoteopen}{\isasymzero}{\isasymRightarrow}{\isasymup}{\isasymlangle}{\isasymzero}{\isasymrangle}{\isachardoublequoteclose}\ {\isacharparenleft}{\isachardoublequoteopen}id{\isachardoublequoteclose}{\isacharparenright}\ \isakeyword{where}\isanewline
\ \ {\isachardoublequoteopen}id\ x\ \ {\isasymequiv}\ {\isacharparenleft}{\isasymlambda}y{\isachardot}\ y\isactrlbold {\isasymapprox}x{\isacharparenright}{\isachardoublequoteclose}\ \ %
\isamarkupcmt{note that \emph{id} is a rigid predicate%
}
%
\begin{isamarkuptext}%
Theorem 11.20 - Informal Proposition 5%
\end{isamarkuptext}\isamarkuptrue%
\isacommand{theorem}\isamarkupfalse%
\ GodIsEssential{\isacharcolon}\ {\isachardoublequoteopen}{\isasymlfloor}\isactrlbold {\isasymforall}x{\isachardot}\ G\ x\ \isactrlbold {\isasymrightarrow}\ {\isacharparenleft}{\isasymE}\ G\ x{\isacharparenright}{\isasymrfloor}{\isachardoublequoteclose}%
\ %
%
\isacommand{using}\isamarkupfalse%
\ A{\isadigit{1}}b\ A{\isadigit{4}}a\ \isacommand{by}\isamarkupfalse%
\ metis%
%
%
%
\begin{isamarkuptext}%
Theorem 11.21%
\end{isamarkuptext}\isamarkuptrue%
\isacommand{theorem}\isamarkupfalse%
\ {\isachardoublequoteopen}{\isasymlfloor}\isactrlbold {\isasymforall}x{\isachardot}\ G{\isacharasterisk}\ x\ \isactrlbold {\isasymrightarrow}\ {\isacharparenleft}{\isasymE}\ G{\isacharasterisk}\ x{\isacharparenright}{\isasymrfloor}{\isachardoublequoteclose}%
\ %
%
\isacommand{using}\isamarkupfalse%
\ A{\isadigit{4}}a\ \isacommand{by}\isamarkupfalse%
\ meson%
%
%
%
\begin{isamarkuptext}%
Theorem 11.22 - Something can have only one essence:%
\end{isamarkuptext}\isamarkuptrue%
\isacommand{theorem}\isamarkupfalse%
\ {\isachardoublequoteopen}{\isasymlfloor}\isactrlbold {\isasymforall}X\ Y\ z{\isachardot}\ {\isacharparenleft}{\isasymE}\ X\ z\ \isactrlbold {\isasymand}\ {\isasymE}\ Y\ z{\isacharparenright}\ \isactrlbold {\isasymrightarrow}\ {\isacharparenleft}X\ {\isasymRrightarrow}\ Y{\isacharparenright}{\isasymrfloor}{\isachardoublequoteclose}%
\ %
%
\isacommand{by}\isamarkupfalse%
\ meson%
%
%
%
\begin{isamarkuptext}%
Theorem 11.23 - An essence is a complete characterization of an individual:%
\end{isamarkuptext}\isamarkuptrue%
\isacommand{theorem}\isamarkupfalse%
\ EssencesCharacterizeCompletely{\isacharcolon}\ {\isachardoublequoteopen}{\isasymlfloor}\isactrlbold {\isasymforall}X\ y{\isachardot}\ {\isasymE}\ X\ y\ \isactrlbold {\isasymrightarrow}\ {\isacharparenleft}X\ {\isasymRrightarrow}\ {\isacharparenleft}id\ y{\isacharparenright}{\isacharparenright}{\isasymrfloor}{\isachardoublequoteclose}\isanewline
%
%
%
\isacommand{proof}\isamarkupfalse%
\ {\isacharparenleft}rule\ ccontr{\isacharparenright}\isanewline
\ \ \isacommand{assume}\isamarkupfalse%
\ {\isachardoublequoteopen}{\isasymnot}\ {\isasymlfloor}\isactrlbold {\isasymforall}X\ y{\isachardot}\ {\isasymE}\ X\ y\ \isactrlbold {\isasymrightarrow}\ {\isacharparenleft}X\ {\isasymRrightarrow}\ {\isacharparenleft}id\ y{\isacharparenright}{\isacharparenright}{\isasymrfloor}{\isachardoublequoteclose}\isanewline
\ \ \isacommand{hence}\isamarkupfalse%
\ {\isachardoublequoteopen}{\isasymexists}w{\isachardot}\ {\isasymnot}{\isacharparenleft}{\isacharparenleft}\ \isactrlbold {\isasymforall}X\ y{\isachardot}\ {\isasymE}\ X\ y\ \isactrlbold {\isasymrightarrow}\ X\ {\isasymRrightarrow}\ id\ y{\isacharparenright}\ w{\isacharparenright}{\isachardoublequoteclose}\ \isacommand{by}\isamarkupfalse%
\ simp\isanewline
\ \ \isacommand{then}\isamarkupfalse%
\ \isacommand{obtain}\isamarkupfalse%
\ w\ \isakeyword{where}\ {\isachardoublequoteopen}{\isasymnot}{\isacharparenleft}{\isacharparenleft}\ \isactrlbold {\isasymforall}X\ y{\isachardot}\ {\isasymE}\ X\ y\ \isactrlbold {\isasymrightarrow}\ X\ {\isasymRrightarrow}\ id\ y{\isacharparenright}\ w{\isacharparenright}{\isachardoublequoteclose}\ \isacommand{{\isachardot}{\isachardot}}\isamarkupfalse%
\isanewline
\ \ \isacommand{hence}\isamarkupfalse%
\ {\isachardoublequoteopen}{\isacharparenleft}\isactrlbold {\isasymexists}X\ y{\isachardot}\ {\isasymE}\ X\ y\ \isactrlbold {\isasymand}\ \isactrlbold {\isasymnot}{\isacharparenleft}X\ {\isasymRrightarrow}\ id\ y{\isacharparenright}{\isacharparenright}\ w{\isachardoublequoteclose}\ \isacommand{by}\isamarkupfalse%
\ simp\isanewline
\ \ \isacommand{hence}\isamarkupfalse%
\ {\isachardoublequoteopen}{\isasymexists}X\ y{\isachardot}\ {\isasymE}\ X\ y\ w\ {\isasymand}\ {\isacharparenleft}\isactrlbold {\isasymnot}{\isacharparenleft}X\ {\isasymRrightarrow}\ id\ y{\isacharparenright}{\isacharparenright}\ w{\isachardoublequoteclose}\ \isacommand{by}\isamarkupfalse%
\ simp\isanewline
\ \ \isacommand{then}\isamarkupfalse%
\ \isacommand{obtain}\isamarkupfalse%
\ P\ \isakeyword{where}\ {\isachardoublequoteopen}{\isasymexists}y{\isachardot}\ {\isasymE}\ P\ y\ w\ {\isasymand}\ {\isacharparenleft}\isactrlbold {\isasymnot}{\isacharparenleft}P\ {\isasymRrightarrow}\ id\ y{\isacharparenright}{\isacharparenright}\ w{\isachardoublequoteclose}\ \isacommand{{\isachardot}{\isachardot}}\isamarkupfalse%
\isanewline
\ \ \isacommand{then}\isamarkupfalse%
\ \isacommand{obtain}\isamarkupfalse%
\ a\ \isakeyword{where}\ {\isadigit{1}}{\isacharcolon}\ {\isachardoublequoteopen}{\isasymE}\ P\ a\ w\ {\isasymand}\ {\isacharparenleft}\isactrlbold {\isasymnot}{\isacharparenleft}P\ {\isasymRrightarrow}\ id\ a{\isacharparenright}{\isacharparenright}\ w{\isachardoublequoteclose}\ \isacommand{{\isachardot}{\isachardot}}\isamarkupfalse%
\isanewline
\ \ \isacommand{hence}\isamarkupfalse%
\ {\isadigit{2}}{\isacharcolon}\ {\isachardoublequoteopen}{\isasymE}\ P\ a\ w{\isachardoublequoteclose}\ \isacommand{by}\isamarkupfalse%
\ {\isacharparenleft}rule\ conjunct{\isadigit{1}}{\isacharparenright}\isanewline
\ \ \isacommand{from}\isamarkupfalse%
\ {\isadigit{1}}\ \isacommand{have}\isamarkupfalse%
\ {\isachardoublequoteopen}{\isacharparenleft}\isactrlbold {\isasymnot}{\isacharparenleft}P\ {\isasymRrightarrow}\ id\ a{\isacharparenright}{\isacharparenright}\ w{\isachardoublequoteclose}\ \isacommand{by}\isamarkupfalse%
\ {\isacharparenleft}rule\ conjunct{\isadigit{2}}{\isacharparenright}\isanewline
\ \ \isacommand{hence}\isamarkupfalse%
\ {\isachardoublequoteopen}{\isasymexists}x{\isachardot}\ {\isasymexists}z{\isachardot}\ w\ r\ x\ {\isasymand}\ \ existsAt\ z\ x\ {\isasymand}\ P\ z\ x\ {\isasymand}\ {\isasymnot}{\isacharparenleft}a\ {\isacharequal}\ z{\isacharparenright}{\isachardoublequoteclose}\ \isacommand{by}\isamarkupfalse%
\ blast\isanewline
\ \ \isacommand{then}\isamarkupfalse%
\ \isacommand{obtain}\isamarkupfalse%
\ w{\isadigit{1}}\ \isakeyword{where}\ {\isachardoublequoteopen}{\isasymexists}z{\isachardot}\ w\ r\ w{\isadigit{1}}\ {\isasymand}\ \ existsAt\ z\ w{\isadigit{1}}\ {\isasymand}\ P\ z\ w{\isadigit{1}}\ {\isasymand}\ {\isasymnot}{\isacharparenleft}a\ {\isacharequal}\ z{\isacharparenright}{\isachardoublequoteclose}\ \isacommand{{\isachardot}{\isachardot}}\isamarkupfalse%
\isanewline
\ \ \isacommand{then}\isamarkupfalse%
\ \isacommand{obtain}\isamarkupfalse%
\ b\ \isakeyword{where}\ {\isadigit{3}}{\isacharcolon}\ {\isachardoublequoteopen}w\ r\ w{\isadigit{1}}\ {\isasymand}\ \ existsAt\ b\ w{\isadigit{1}}\ {\isasymand}\ P\ b\ w{\isadigit{1}}\ {\isasymand}\ {\isasymnot}{\isacharparenleft}a\ {\isacharequal}\ b{\isacharparenright}{\isachardoublequoteclose}\ \isacommand{{\isachardot}{\isachardot}}\isamarkupfalse%
\isanewline
\ \ \isacommand{hence}\isamarkupfalse%
\ {\isachardoublequoteopen}w\ r\ w{\isadigit{1}}{\isachardoublequoteclose}\ \isacommand{by}\isamarkupfalse%
\ simp\isanewline
\ \ \isacommand{from}\isamarkupfalse%
\ {\isadigit{3}}\ \isacommand{have}\isamarkupfalse%
\ {\isachardoublequoteopen}existsAt\ b\ w{\isadigit{1}}{\isachardoublequoteclose}\ \isacommand{by}\isamarkupfalse%
\ simp\isanewline
\ \ \isacommand{from}\isamarkupfalse%
\ {\isadigit{3}}\ \isacommand{have}\isamarkupfalse%
\ {\isachardoublequoteopen}P\ b\ w{\isadigit{1}}{\isachardoublequoteclose}\ \isacommand{by}\isamarkupfalse%
\ simp\isanewline
\ \ \isacommand{from}\isamarkupfalse%
\ {\isadigit{3}}\ \isacommand{have}\isamarkupfalse%
\ {\isadigit{4}}{\isacharcolon}\ {\isachardoublequoteopen}\ {\isasymnot}{\isacharparenleft}a\ {\isacharequal}\ b{\isacharparenright}{\isachardoublequoteclose}\ \isacommand{by}\isamarkupfalse%
\ simp\isanewline
\ \ \isacommand{from}\isamarkupfalse%
\ {\isadigit{2}}\ \isacommand{have}\isamarkupfalse%
\ {\isachardoublequoteopen}P\ a\ w{\isachardoublequoteclose}\ \isacommand{by}\isamarkupfalse%
\ simp\isanewline
\ \ \isacommand{from}\isamarkupfalse%
\ {\isadigit{2}}\ \isacommand{have}\isamarkupfalse%
\ {\isachardoublequoteopen}{\isasymforall}Y{\isachardot}\ Y\ a\ w\ {\isasymlongrightarrow}\ {\isacharparenleft}{\isacharparenleft}P\ {\isasymRrightarrow}\ Y{\isacharparenright}\ w{\isacharparenright}{\isachardoublequoteclose}\ \isacommand{by}\isamarkupfalse%
\ auto\isanewline
\ \ \isacommand{hence}\isamarkupfalse%
\ {\isachardoublequoteopen}{\isacharparenleft}\isactrlbold {\isasymrightharpoondown}{\isacharparenleft}id\ b{\isacharparenright}{\isacharparenright}\ a\ w\ {\isasymlongrightarrow}\ {\isacharparenleft}P\ {\isasymRrightarrow}\ {\isacharparenleft}\isactrlbold {\isasymrightharpoondown}{\isacharparenleft}id\ b{\isacharparenright}{\isacharparenright}{\isacharparenright}\ w{\isachardoublequoteclose}\ \isacommand{by}\isamarkupfalse%
\ {\isacharparenleft}rule\ allE{\isacharparenright}\isanewline
\ \ \isacommand{hence}\isamarkupfalse%
\ {\isachardoublequoteopen}{\isasymnot}{\isacharparenleft}\isactrlbold {\isasymrightharpoondown}{\isacharparenleft}id\ b{\isacharparenright}{\isacharparenright}\ a\ w\ {\isasymor}\ {\isacharparenleft}{\isacharparenleft}P\ {\isasymRrightarrow}\ {\isacharparenleft}\isactrlbold {\isasymrightharpoondown}{\isacharparenleft}id\ b{\isacharparenright}{\isacharparenright}{\isacharparenright}\ w{\isacharparenright}{\isachardoublequoteclose}\ \isacommand{by}\isamarkupfalse%
\ blast\ \isanewline
\ \ \isacommand{then}\isamarkupfalse%
\ \isacommand{show}\isamarkupfalse%
\ False\ \isacommand{proof}\isamarkupfalse%
\isanewline
\ \ \ \ \isacommand{assume}\isamarkupfalse%
\ {\isachardoublequoteopen}{\isasymnot}{\isacharparenleft}\isactrlbold {\isasymrightharpoondown}{\isacharparenleft}id\ b{\isacharparenright}{\isacharparenright}\ a\ w{\isachardoublequoteclose}\isanewline
\ \ \ \ \isacommand{hence}\isamarkupfalse%
\ {\isachardoublequoteopen}a\ {\isacharequal}\ b{\isachardoublequoteclose}\ \isacommand{by}\isamarkupfalse%
\ simp\isanewline
\ \ \ \ \isacommand{thus}\isamarkupfalse%
\ False\ \isacommand{using}\isamarkupfalse%
\ {\isadigit{4}}\ \isacommand{by}\isamarkupfalse%
\ auto\isanewline
\ \ \ \ \isacommand{next}\isamarkupfalse%
\ \ \isanewline
\ \ \ \ \isacommand{assume}\isamarkupfalse%
\ {\isachardoublequoteopen}{\isacharparenleft}{\isacharparenleft}P\ {\isasymRrightarrow}\ {\isacharparenleft}\isactrlbold {\isasymrightharpoondown}{\isacharparenleft}id\ b{\isacharparenright}{\isacharparenright}{\isacharparenright}\ w{\isacharparenright}{\isachardoublequoteclose}\isanewline
\ \ \ \ \isacommand{hence}\isamarkupfalse%
\ {\isachardoublequoteopen}{\isasymforall}x{\isachardot}\ {\isasymforall}z{\isachardot}\ {\isacharparenleft}w\ r\ x\ {\isasymand}\ existsAt\ z\ x\ {\isasymand}\ P\ z\ x{\isacharparenright}\ {\isasymlongrightarrow}\ {\isacharparenleft}\isactrlbold {\isasymrightharpoondown}{\isacharparenleft}id\ b{\isacharparenright}{\isacharparenright}\ z\ x{\isachardoublequoteclose}\ \isacommand{by}\isamarkupfalse%
\ blast\isanewline
\ \ \ \ \isacommand{hence}\isamarkupfalse%
\ {\isachardoublequoteopen}{\isasymforall}z{\isachardot}\ {\isacharparenleft}w\ r\ w{\isadigit{1}}\ {\isasymand}\ existsAt\ z\ w{\isadigit{1}}\ {\isasymand}\ P\ z\ w{\isadigit{1}}{\isacharparenright}\ {\isasymlongrightarrow}\ {\isacharparenleft}\isactrlbold {\isasymrightharpoondown}{\isacharparenleft}id\ b{\isacharparenright}{\isacharparenright}\ z\ w{\isadigit{1}}{\isachardoublequoteclose}\ \isanewline
\ \ \ \ \ \ \ \ \isacommand{by}\isamarkupfalse%
\ {\isacharparenleft}rule\ allE{\isacharparenright}\isanewline
\ \ \ \ \isacommand{hence}\isamarkupfalse%
\ {\isachardoublequoteopen}{\isacharparenleft}w\ r\ w{\isadigit{1}}\ {\isasymand}\ existsAt\ b\ w{\isadigit{1}}\ {\isasymand}\ P\ b\ w{\isadigit{1}}{\isacharparenright}\ {\isasymlongrightarrow}\ {\isacharparenleft}\isactrlbold {\isasymrightharpoondown}{\isacharparenleft}id\ b{\isacharparenright}{\isacharparenright}\ b\ w{\isadigit{1}}{\isachardoublequoteclose}\ \isacommand{by}\isamarkupfalse%
\ {\isacharparenleft}rule\ allE{\isacharparenright}\isanewline
\ \ \ \ \isacommand{hence}\isamarkupfalse%
\ \ {\isachardoublequoteopen}{\isasymnot}{\isacharparenleft}w\ r\ w{\isadigit{1}}\ {\isasymand}\ existsAt\ b\ w{\isadigit{1}}\ {\isasymand}\ P\ b\ w{\isadigit{1}}{\isacharparenright}\ {\isasymor}\ {\isacharparenleft}\isactrlbold {\isasymrightharpoondown}{\isacharparenleft}id\ b{\isacharparenright}{\isacharparenright}\ b\ w{\isadigit{1}}{\isachardoublequoteclose}\ \isacommand{by}\isamarkupfalse%
\ simp\isanewline
\ \ \ \ \isacommand{hence}\isamarkupfalse%
\ {\isachardoublequoteopen}{\isacharparenleft}\isactrlbold {\isasymrightharpoondown}{\isacharparenleft}id\ b{\isacharparenright}{\isacharparenright}\ b\ w{\isachardoublequoteclose}\ \isacommand{using}\isamarkupfalse%
\ {\isadigit{3}}\ \isacommand{by}\isamarkupfalse%
\ simp\isanewline
\ \ \ \ \isacommand{hence}\isamarkupfalse%
\ {\isachardoublequoteopen}{\isasymnot}{\isacharparenleft}b{\isacharequal}b{\isacharparenright}{\isachardoublequoteclose}\ \isacommand{by}\isamarkupfalse%
\ simp\isanewline
\ \ \ \ \isacommand{thus}\isamarkupfalse%
\ False\ \isacommand{by}\isamarkupfalse%
\ simp\isanewline
\ \ \isacommand{qed}\isamarkupfalse%
\isanewline
\isacommand{qed}\isamarkupfalse%
%
%
%
%
\begin{isamarkuptext}%
Definition 11.24 - Necessary Existence (Informal Definition 6):%
\end{isamarkuptext}\isamarkuptrue%
\isacommand{abbreviation}\isamarkupfalse%
\ necessaryExistencePred{\isacharcolon}{\isacharcolon}{\isachardoublequoteopen}{\isasymup}{\isasymlangle}{\isasymzero}{\isasymrangle}{\isachardoublequoteclose}\ {\isacharparenleft}{\isachardoublequoteopen}NE{\isachardoublequoteclose}{\isacharparenright}\ \isanewline
\ \ \isakeyword{where}\ {\isachardoublequoteopen}NE\ x\ \ {\isasymequiv}\ {\isacharparenleft}{\isasymlambda}w{\isachardot}\ {\isacharparenleft}\isactrlbold {\isasymforall}Y{\isachardot}\ \ {\isasymE}\ Y\ x\ \isactrlbold {\isasymrightarrow}\ \isactrlbold {\isasymbox}\isactrlbold {\isasymexists}\isactrlsup E\ Y{\isacharparenright}\ w{\isacharparenright}{\isachardoublequoteclose}%
\begin{isamarkuptext}%
Axiom 11.25 (Informal Axiom 5)%
\end{isamarkuptext}\isamarkuptrue%
\isacommand{axiomatization}\isamarkupfalse%
\ \isakeyword{where}\ \isanewline
\ A{\isadigit{5}}{\isacharcolon}\ {\isachardoublequoteopen}{\isasymlfloor}{\isasymP}\ NE{\isasymrfloor}{\isachardoublequoteclose}\isanewline
\ \isanewline
\isacommand{lemma}\isamarkupfalse%
\ True\ \isacommand{nitpick}\isamarkupfalse%
{\isacharbrackleft}satisfy{\isacharbrackright}%
\ %
%
\isacommand{oops}\isamarkupfalse%
\ %
\isamarkupcmt{model found: so far all axioms consistent%
}
%
%
%
%
\begin{isamarkuptext}%
Theorem 11.26 (Informal Proposition 7) - Possibilist existence of God implies necessary actualist existence:%
\end{isamarkuptext}\isamarkuptrue%
\isacommand{theorem}\isamarkupfalse%
\ GodExistenceImpliesNecExistence{\isacharcolon}\ {\isachardoublequoteopen}{\isasymlfloor}\isactrlbold {\isasymexists}\ G\ \isactrlbold {\isasymrightarrow}\ \ \isactrlbold {\isasymbox}\isactrlbold {\isasymexists}\isactrlsup E\ G{\isasymrfloor}{\isachardoublequoteclose}\isanewline
%
%
%
\isacommand{proof}\isamarkupfalse%
\ {\isacharminus}\isanewline
\isacommand{{\isacharbraceleft}}\isamarkupfalse%
\isanewline
\ \ \isacommand{fix}\isamarkupfalse%
\ w\ \isanewline
\ \ \isacommand{{\isacharbraceleft}}\isamarkupfalse%
\isanewline
\ \ \ \ \isacommand{assume}\isamarkupfalse%
\ {\isachardoublequoteopen}{\isasymexists}x{\isachardot}\ G\ x\ w{\isachardoublequoteclose}\isanewline
\ \ \ \ \isacommand{then}\isamarkupfalse%
\ \isacommand{obtain}\isamarkupfalse%
\ g\ \isakeyword{where}\ {\isadigit{1}}{\isacharcolon}\ {\isachardoublequoteopen}G\ g\ w{\isachardoublequoteclose}\ \isacommand{{\isachardot}{\isachardot}}\isamarkupfalse%
\isanewline
\ \ \ \ \isacommand{hence}\isamarkupfalse%
\ {\isachardoublequoteopen}NE\ g\ w{\isachardoublequoteclose}\ \isacommand{using}\isamarkupfalse%
\ A{\isadigit{5}}\ \isacommand{by}\isamarkupfalse%
\ auto\ \ \ \ \ \ \ \ \ \ \ \ \ \ \ \ \ \ \ \ \ %
\isamarkupcmt{axiom 11.25%
}
\isanewline
\ \ \ \ \isacommand{hence}\isamarkupfalse%
\ {\isachardoublequoteopen}{\isasymforall}Y{\isachardot}\ {\isacharparenleft}{\isasymE}\ Y\ g\ w{\isacharparenright}\ {\isasymlongrightarrow}\ {\isacharparenleft}\isactrlbold {\isasymbox}\isactrlbold {\isasymexists}\isactrlsup E\ Y{\isacharparenright}\ w{\isachardoublequoteclose}\ \isacommand{by}\isamarkupfalse%
\ simp\isanewline
\ \ \ \ \isacommand{hence}\isamarkupfalse%
\ {\isadigit{2}}{\isacharcolon}\ {\isachardoublequoteopen}{\isacharparenleft}{\isasymE}\ G\ g\ w{\isacharparenright}\ {\isasymlongrightarrow}\ {\isacharparenleft}\isactrlbold {\isasymbox}\isactrlbold {\isasymexists}\isactrlsup E\ G{\isacharparenright}\ w{\isachardoublequoteclose}\ \isacommand{by}\isamarkupfalse%
\ {\isacharparenleft}rule\ allE{\isacharparenright}\isanewline
\ \ \ \ \isacommand{have}\isamarkupfalse%
\ \ {\isachardoublequoteopen}{\isacharparenleft}\isactrlbold {\isasymforall}x{\isachardot}\ G\ x\ \isactrlbold {\isasymrightarrow}\ {\isacharparenleft}{\isasymE}\ G\ x{\isacharparenright}{\isacharparenright}\ w{\isachardoublequoteclose}\ \isacommand{using}\isamarkupfalse%
\ GodIsEssential\isanewline
\ \ \ \ \ \ \isacommand{by}\isamarkupfalse%
\ {\isacharparenleft}rule\ allE{\isacharparenright}\ \ \ \ \ %
\isamarkupcmt{GodIsEssential follows from Axioms 11.11 and 11.3B%
}
\isanewline
\ \ \ \ \isacommand{hence}\isamarkupfalse%
\ \ {\isachardoublequoteopen}{\isacharparenleft}G\ g\ \isactrlbold {\isasymrightarrow}\ {\isacharparenleft}{\isasymE}\ G\ g{\isacharparenright}{\isacharparenright}\ w{\isachardoublequoteclose}\ \isacommand{by}\isamarkupfalse%
\ {\isacharparenleft}rule\ allE{\isacharparenright}\isanewline
\ \ \ \ \isacommand{hence}\isamarkupfalse%
\ \ {\isachardoublequoteopen}G\ g\ w\ {\isasymlongrightarrow}\ {\isasymE}\ G\ g\ w{\isachardoublequoteclose}\ \isacommand{by}\isamarkupfalse%
\ simp\isanewline
\ \ \ \ \isacommand{from}\isamarkupfalse%
\ this\ {\isadigit{1}}\ \isacommand{have}\isamarkupfalse%
\ {\isadigit{3}}{\isacharcolon}\ {\isachardoublequoteopen}{\isasymE}\ G\ g\ w{\isachardoublequoteclose}\ \isacommand{by}\isamarkupfalse%
\ {\isacharparenleft}rule\ mp{\isacharparenright}\isanewline
\ \ \ \ \isacommand{from}\isamarkupfalse%
\ {\isadigit{2}}\ {\isadigit{3}}\ \isacommand{have}\isamarkupfalse%
\ {\isachardoublequoteopen}{\isacharparenleft}\isactrlbold {\isasymbox}\isactrlbold {\isasymexists}\isactrlsup E\ G{\isacharparenright}\ w{\isachardoublequoteclose}\ \isacommand{by}\isamarkupfalse%
\ {\isacharparenleft}rule\ mp{\isacharparenright}\isanewline
\ \ \isacommand{{\isacharbraceright}}\isamarkupfalse%
\isanewline
\ \ \isacommand{hence}\isamarkupfalse%
\ {\isachardoublequoteopen}{\isacharparenleft}{\isasymexists}x{\isachardot}\ G\ x\ w{\isacharparenright}\ {\isasymlongrightarrow}\ {\isacharparenleft}\isactrlbold {\isasymbox}\isactrlbold {\isasymexists}\isactrlsup E\ G{\isacharparenright}\ w{\isachardoublequoteclose}\ \isacommand{by}\isamarkupfalse%
\ {\isacharparenleft}rule\ impI{\isacharparenright}\isanewline
\ \ \isacommand{hence}\isamarkupfalse%
\ {\isachardoublequoteopen}{\isacharparenleft}{\isacharparenleft}\isactrlbold {\isasymexists}x{\isachardot}\ G\ x{\isacharparenright}\ \isactrlbold {\isasymrightarrow}\ \ \isactrlbold {\isasymbox}\isactrlbold {\isasymexists}\isactrlsup E\ G{\isacharparenright}\ w{\isachardoublequoteclose}\ \isacommand{by}\isamarkupfalse%
\ simp\isanewline
\isacommand{{\isacharbraceright}}\isamarkupfalse%
\isanewline
\ \isacommand{thus}\isamarkupfalse%
\ {\isacharquery}thesis\ \isacommand{by}\isamarkupfalse%
\ {\isacharparenleft}rule\ allI{\isacharparenright}\ \isanewline
\isacommand{qed}\isamarkupfalse%
%
%
%
%
\begin{isamarkuptext}%
\emph{Modal collapse} is countersatisfiable (unless we introduce S5 axioms):%
\end{isamarkuptext}\isamarkuptrue%
\isacommand{lemma}\isamarkupfalse%
\ {\isachardoublequoteopen}{\isasymlfloor}\isactrlbold {\isasymforall}{\isasymPhi}{\isachardot}{\isacharparenleft}{\isasymPhi}\ \isactrlbold {\isasymrightarrow}\ {\isacharparenleft}\isactrlbold {\isasymbox}\ {\isasymPhi}{\isacharparenright}{\isacharparenright}{\isasymrfloor}{\isachardoublequoteclose}\ \isacommand{nitpick}\isamarkupfalse%
%
\ %
%
\isacommand{oops}\isamarkupfalse%
%
%
%
%
\begin{isamarkuptext}%
We postulate semantic frame conditions for some modal logics. Taken together, reflexivity, transitivity and symmetry
 make for an equivalence relation and therefore an \emph{S5} logic (via \emph{Sahlqvist correspondence}).
 We prefer to postulate them individually here in order to get more detailed information about their relevance 
 in the proofs presented below.%
\end{isamarkuptext}\isamarkuptrue%
\isacommand{axiomatization}\isamarkupfalse%
\ \isakeyword{where}\isanewline
\ refl{\isacharcolon}\ {\isachardoublequoteopen}reflexive\ aRel{\isachardoublequoteclose}\ \isakeyword{and}\isanewline
\ tran{\isacharcolon}\ {\isachardoublequoteopen}transitive\ aRel{\isachardoublequoteclose}\ \isakeyword{and}\isanewline
\ symm{\isacharcolon}\ {\isachardoublequoteopen}symmetric\ aRel{\isachardoublequoteclose}\isanewline
\ \isanewline
\isacommand{lemma}\isamarkupfalse%
\ True\ \isacommand{nitpick}\isamarkupfalse%
{\isacharbrackleft}satisfy{\isacharbrackright}%
\ %
%
\isacommand{oops}\isamarkupfalse%
\ %
\isamarkupcmt{model found: axioms still consistent%
}
%
%
%
%
\begin{isamarkuptext}%
Using an \emph{S5} logic, \emph{modal collapse} (\isa{{\isasymlfloor}\isactrlbold {\isasymforall}{\isasymPhi}{\isachardot}{\isacharparenleft}{\isasymPhi}\ \isactrlbold {\isasymrightarrow}\ {\isacharparenleft}\isactrlbold {\isasymbox}\ {\isasymPhi}{\isacharparenright}{\isacharparenright}{\isasymrfloor}}) is actually valid (see `More Objections' some pages below)%
\end{isamarkuptext}\isamarkuptrue%
%
\begin{isamarkuptext}%
We prove some useful inference rules:%
\end{isamarkuptext}\isamarkuptrue%
\isacommand{lemma}\isamarkupfalse%
\ modal{\isacharunderscore}distr{\isacharcolon}\ {\isachardoublequoteopen}{\isasymlfloor}\isactrlbold {\isasymbox}{\isacharparenleft}{\isasymphi}\ \isactrlbold {\isasymrightarrow}\ {\isasympsi}{\isacharparenright}{\isasymrfloor}\ {\isasymLongrightarrow}\ {\isasymlfloor}{\isacharparenleft}\isactrlbold {\isasymdiamond}{\isasymphi}\ \isactrlbold {\isasymrightarrow}\ \isactrlbold {\isasymdiamond}{\isasympsi}{\isacharparenright}{\isasymrfloor}{\isachardoublequoteclose}%
\ %
%
\isacommand{by}\isamarkupfalse%
\ blast%
%
%
\isanewline
\isacommand{lemma}\isamarkupfalse%
\ modal{\isacharunderscore}trans{\isacharcolon}\ {\isachardoublequoteopen}{\isacharparenleft}{\isasymlfloor}{\isasymphi}\ \isactrlbold {\isasymrightarrow}\ {\isasympsi}{\isasymrfloor}\ {\isasymand}\ {\isasymlfloor}{\isasympsi}\ \isactrlbold {\isasymrightarrow}\ {\isasymchi}{\isasymrfloor}{\isacharparenright}\ {\isasymLongrightarrow}\ {\isasymlfloor}{\isasymphi}\ \isactrlbold {\isasymrightarrow}\ {\isasymchi}{\isasymrfloor}{\isachardoublequoteclose}%
\ %
%
\isacommand{by}\isamarkupfalse%
\ simp%
%
%
%
\begin{isamarkuptext}%
Theorem 11.27 - Informal Proposition 8. Note that only symmetry and transitivity for the accessibility relation are used.%
\end{isamarkuptext}\isamarkuptrue%
\isacommand{theorem}\isamarkupfalse%
\ possExistenceImpliesNecEx{\isacharcolon}\ {\isachardoublequoteopen}{\isasymlfloor}\isactrlbold {\isasymdiamond}\isactrlbold {\isasymexists}\ G\ \isactrlbold {\isasymrightarrow}\ \isactrlbold {\isasymbox}\isactrlbold {\isasymexists}\isactrlsup E\ G{\isasymrfloor}{\isachardoublequoteclose}\ %
\isamarkupcmt{local consequence%
}
\isanewline
%
%
%
\isacommand{proof}\isamarkupfalse%
\ {\isacharminus}\isanewline
\ \ \isacommand{have}\isamarkupfalse%
\ {\isachardoublequoteopen}{\isasymlfloor}\isactrlbold {\isasymexists}\ G\ \isactrlbold {\isasymrightarrow}\ \isactrlbold {\isasymbox}\isactrlbold {\isasymexists}\isactrlsup E\ G{\isasymrfloor}{\isachardoublequoteclose}\ \isacommand{using}\isamarkupfalse%
\ GodExistenceImpliesNecExistence\ \isanewline
\ \ \ \ \isacommand{by}\isamarkupfalse%
\ simp\ %
\isamarkupcmt{follows from Axioms 11.11, 11.25 and 11.3B%
}
\isanewline
\ \ \isacommand{hence}\isamarkupfalse%
\ {\isachardoublequoteopen}{\isasymlfloor}\isactrlbold {\isasymbox}{\isacharparenleft}\isactrlbold {\isasymexists}\ G\ \isactrlbold {\isasymrightarrow}\ \isactrlbold {\isasymbox}\isactrlbold {\isasymexists}\isactrlsup E\ G{\isacharparenright}{\isasymrfloor}{\isachardoublequoteclose}\ \isacommand{using}\isamarkupfalse%
\ NEC\ \isacommand{by}\isamarkupfalse%
\ simp\isanewline
\ \ \isacommand{hence}\isamarkupfalse%
\ {\isadigit{1}}{\isacharcolon}\ {\isachardoublequoteopen}{\isasymlfloor}\isactrlbold {\isasymdiamond}\isactrlbold {\isasymexists}\ G\ \isactrlbold {\isasymrightarrow}\ \isactrlbold {\isasymdiamond}\isactrlbold {\isasymbox}\isactrlbold {\isasymexists}\isactrlsup E\ G{\isasymrfloor}{\isachardoublequoteclose}\ \isacommand{by}\isamarkupfalse%
\ {\isacharparenleft}rule\ modal{\isacharunderscore}distr{\isacharparenright}\isanewline
\ \ \isacommand{have}\isamarkupfalse%
\ {\isadigit{2}}{\isacharcolon}\ {\isachardoublequoteopen}{\isasymlfloor}\isactrlbold {\isasymdiamond}\isactrlbold {\isasymbox}\isactrlbold {\isasymexists}\isactrlsup E\ G\ \isactrlbold {\isasymrightarrow}\ \isactrlbold {\isasymbox}\isactrlbold {\isasymexists}\isactrlsup E\ G{\isasymrfloor}{\isachardoublequoteclose}\ \isacommand{using}\isamarkupfalse%
\ symm\ tran\ \isacommand{by}\isamarkupfalse%
\ metis\ %
\isamarkupcmt{frame conditions%
}
\isanewline
\ \ \isacommand{from}\isamarkupfalse%
\ {\isadigit{1}}\ {\isadigit{2}}\ \isacommand{have}\isamarkupfalse%
\ {\isachardoublequoteopen}{\isasymlfloor}\isactrlbold {\isasymdiamond}\isactrlbold {\isasymexists}\ G\ \isactrlbold {\isasymrightarrow}\ \isactrlbold {\isasymdiamond}\isactrlbold {\isasymbox}\isactrlbold {\isasymexists}\isactrlsup E\ G{\isasymrfloor}\ {\isasymand}\ {\isasymlfloor}\isactrlbold {\isasymdiamond}\isactrlbold {\isasymbox}\isactrlbold {\isasymexists}\isactrlsup E\ G\ \isactrlbold {\isasymrightarrow}\ \isactrlbold {\isasymbox}\isactrlbold {\isasymexists}\isactrlsup E\ G{\isasymrfloor}{\isachardoublequoteclose}\ \isacommand{by}\isamarkupfalse%
\ simp\isanewline
\ \ \isacommand{thus}\isamarkupfalse%
\ {\isacharquery}thesis\ \isacommand{by}\isamarkupfalse%
\ {\isacharparenleft}rule\ modal{\isacharunderscore}trans{\isacharparenright}\isanewline
\isacommand{qed}\isamarkupfalse%
%
%
\isanewline
%
\isanewline
\isacommand{lemma}\isamarkupfalse%
\ T{\isadigit{4}}{\isacharcolon}\ {\isachardoublequoteopen}{\isasymlfloor}\isactrlbold {\isasymdiamond}\isactrlbold {\isasymexists}\ G{\isasymrfloor}\ {\isasymlongrightarrow}\ {\isasymlfloor}\isactrlbold {\isasymbox}\isactrlbold {\isasymexists}\isactrlsup E\ G{\isasymrfloor}{\isachardoublequoteclose}%
\ %
%
\isacommand{using}\isamarkupfalse%
\ possExistenceImpliesNecEx\ \isanewline
\ \ \ \ \isacommand{by}\isamarkupfalse%
\ {\isacharparenleft}rule\ localImpGlobalCons{\isacharparenright}\ \ %
\isamarkupcmt{global consequence%
}
%
%
%
%
\begin{isamarkuptext}%
Corollary 11.28 - Necessary (actualist) existence of God (for both definitions); reflexivity is still not used:%
\end{isamarkuptext}\isamarkuptrue%
\isacommand{lemma}\isamarkupfalse%
\ GodNecExists{\isacharcolon}\ {\isachardoublequoteopen}{\isasymlfloor}\isactrlbold {\isasymbox}\isactrlbold {\isasymexists}\isactrlsup E\ G{\isasymrfloor}{\isachardoublequoteclose}%
\ %
%
\isacommand{using}\isamarkupfalse%
\ T{\isadigit{3}}\ T{\isadigit{4}}\ \isacommand{by}\isamarkupfalse%
\ metis%
%
%
\ \ \ \ \isanewline
\isacommand{lemma}\isamarkupfalse%
\ God{\isacharunderscore}starNecExists{\isacharcolon}\ {\isachardoublequoteopen}{\isasymlfloor}\isactrlbold {\isasymbox}\isactrlbold {\isasymexists}\isactrlsup E\ G{\isacharasterisk}{\isasymrfloor}{\isachardoublequoteclose}\ \isanewline
%
\ \ %
%
\isacommand{using}\isamarkupfalse%
\ GodNecExists\ GodDefsAreEquivalent\ \isacommand{by}\isamarkupfalse%
\ simp%
%
%
%
\isamarkupsubsubsection{Monotheism%
}
\isamarkuptrue%
%
\begin{isamarkuptext}%
Monotheism for non-normal models (with Leibniz equality) follows directly from God having all and only positive properties:%
\end{isamarkuptext}\isamarkuptrue%
\isacommand{theorem}\isamarkupfalse%
\ Monotheism{\isacharunderscore}LeibnizEq{\isacharcolon}\ {\isachardoublequoteopen}{\isasymlfloor}\isactrlbold {\isasymforall}x{\isachardot}\ G\ x\ \isactrlbold {\isasymrightarrow}\ {\isacharparenleft}\isactrlbold {\isasymforall}y{\isachardot}\ G\ y\ \isactrlbold {\isasymrightarrow}\ {\isacharparenleft}x\ \ \isactrlbold {\isasymapprox}\isactrlsup L\ y{\isacharparenright}{\isacharparenright}{\isasymrfloor}{\isachardoublequoteclose}\ \isanewline
%
\ \ %
%
\isacommand{using}\isamarkupfalse%
\ GodDefsAreEquivalent\ \isacommand{by}\isamarkupfalse%
\ simp%
%
%
%
\begin{isamarkuptext}%
Monotheism for normal models is trickier. We need to consider some previous results (p. 162):%
\end{isamarkuptext}\isamarkuptrue%
\isacommand{lemma}\isamarkupfalse%
\ GodExistenceIsValid{\isacharcolon}\ {\isachardoublequoteopen}{\isasymlfloor}\isactrlbold {\isasymexists}\isactrlsup E\ G{\isasymrfloor}{\isachardoublequoteclose}%
\ %
%
\isacommand{using}\isamarkupfalse%
\ GodNecExists\ refl\isanewline
\ \ \isacommand{by}\isamarkupfalse%
\ auto\ %
\isamarkupcmt{reflexivity is now required by the solver%
}
%
%
%
%
\begin{isamarkuptext}%
Proposition 11.29:%
\end{isamarkuptext}\isamarkuptrue%
\isacommand{theorem}\isamarkupfalse%
\ Monotheism{\isacharunderscore}normalModel{\isacharcolon}\ {\isachardoublequoteopen}{\isasymlfloor}\isactrlbold {\isasymexists}x{\isachardot}\isactrlbold {\isasymforall}y{\isachardot}\ G\ y\ \isactrlbold {\isasymleftrightarrow}\ x\ \isactrlbold {\isasymapprox}\ y{\isasymrfloor}{\isachardoublequoteclose}\isanewline
%
%
%
\isacommand{proof}\isamarkupfalse%
\ {\isacharminus}\isanewline
\isacommand{{\isacharbraceleft}}\isamarkupfalse%
\isanewline
\ \ \isacommand{fix}\isamarkupfalse%
\ w\ \isanewline
\ \ \isacommand{have}\isamarkupfalse%
\ {\isachardoublequoteopen}{\isasymlfloor}\isactrlbold {\isasymexists}\isactrlsup E\ G{\isasymrfloor}{\isachardoublequoteclose}\ \isacommand{using}\isamarkupfalse%
\ GodExistenceIsValid\ \isacommand{by}\isamarkupfalse%
\ simp\ %
\isamarkupcmt{follows from corollary 11.28%
}
\isanewline
\ \ \isacommand{hence}\isamarkupfalse%
\ {\isachardoublequoteopen}{\isacharparenleft}\isactrlbold {\isasymexists}\isactrlsup E\ G{\isacharparenright}\ w{\isachardoublequoteclose}\ \isacommand{by}\isamarkupfalse%
\ {\isacharparenleft}rule\ allE{\isacharparenright}\ \ \ \ \ \ \ \isanewline
\ \ \isacommand{then}\isamarkupfalse%
\ \isacommand{obtain}\isamarkupfalse%
\ g\ \isakeyword{where}\ {\isadigit{1}}{\isacharcolon}\ {\isachardoublequoteopen}existsAt\ g\ w\ {\isasymand}\ G\ g\ w{\isachardoublequoteclose}\ \isacommand{{\isachardot}{\isachardot}}\isamarkupfalse%
\isanewline
\ \ \isacommand{hence}\isamarkupfalse%
\ {\isadigit{2}}{\isacharcolon}\ {\isachardoublequoteopen}{\isasymE}\ G\ g\ w{\isachardoublequoteclose}\ \isacommand{using}\isamarkupfalse%
\ GodIsEssential\ \isacommand{by}\isamarkupfalse%
\ blast\ %
\isamarkupcmt{follows from ax. 11.11/11.3B%
}
\isanewline
\ \ \isacommand{{\isacharbraceleft}}\isamarkupfalse%
\isanewline
\ \ \ \ \isacommand{fix}\isamarkupfalse%
\ y\isanewline
\ \ \ \ \isacommand{have}\isamarkupfalse%
\ {\isachardoublequoteopen}G\ y\ w\ {\isasymlongleftrightarrow}\ {\isacharparenleft}g\ \isactrlbold {\isasymapprox}\ y{\isacharparenright}\ w{\isachardoublequoteclose}\ \isacommand{proof}\isamarkupfalse%
\ \isanewline
\ \ \ \ \ \ \isacommand{assume}\isamarkupfalse%
\ {\isachardoublequoteopen}G\ y\ w{\isachardoublequoteclose}\isanewline
\ \ \ \ \ \ \isacommand{hence}\isamarkupfalse%
\ {\isadigit{3}}{\isacharcolon}\ {\isachardoublequoteopen}{\isasymE}\ G\ y\ w{\isachardoublequoteclose}\ \isacommand{using}\isamarkupfalse%
\ GodIsEssential\ \isacommand{by}\isamarkupfalse%
\ blast\ \ \ \ \ \ \isanewline
\ \ \ \ \ \ \isacommand{have}\isamarkupfalse%
\ {\isachardoublequoteopen}{\isacharparenleft}{\isasymE}\ G\ y\ \isactrlbold {\isasymrightarrow}\ {\isacharparenleft}G\ {\isasymRrightarrow}\ id\ y{\isacharparenright}{\isacharparenright}\ w{\isachardoublequoteclose}\ \isacommand{using}\isamarkupfalse%
\ EssencesCharacterizeCompletely\isanewline
\ \ \ \ \ \ \ \ \isacommand{by}\isamarkupfalse%
\ simp\ %
\isamarkupcmt{follows from theorem 11.23%
}
\isanewline
\ \ \ \ \ \ \isacommand{hence}\isamarkupfalse%
\ {\isachardoublequoteopen}{\isasymE}\ G\ y\ w\ {\isasymlongrightarrow}\ {\isacharparenleft}{\isacharparenleft}G\ {\isasymRrightarrow}\ id\ y{\isacharparenright}\ w{\isacharparenright}{\isachardoublequoteclose}\ \isacommand{by}\isamarkupfalse%
\ simp\isanewline
\ \ \ \ \ \ \isacommand{from}\isamarkupfalse%
\ this\ {\isadigit{3}}\ \isacommand{have}\isamarkupfalse%
\ {\isachardoublequoteopen}{\isacharparenleft}G\ {\isasymRrightarrow}\ id\ y{\isacharparenright}\ w{\isachardoublequoteclose}\ \isacommand{by}\isamarkupfalse%
\ {\isacharparenleft}rule\ mp{\isacharparenright}\ \isanewline
\ \ \ \ \ \ \isacommand{hence}\isamarkupfalse%
\ {\isachardoublequoteopen}{\isacharparenleft}\isactrlbold {\isasymbox}{\isacharparenleft}\isactrlbold {\isasymforall}\isactrlsup Ez{\isachardot}\ G\ z\ \isactrlbold {\isasymrightarrow}\ z\ \isactrlbold {\isasymapprox}\ y{\isacharparenright}{\isacharparenright}\ w{\isachardoublequoteclose}\ \isacommand{by}\isamarkupfalse%
\ simp\isanewline
\ \ \ \ \ \ \isacommand{hence}\isamarkupfalse%
\ {\isachardoublequoteopen}{\isasymforall}x{\isachardot}\ w\ r\ x\ {\isasymlongrightarrow}\ {\isacharparenleft}{\isacharparenleft}{\isasymforall}z{\isachardot}\ {\isacharparenleft}existsAt\ z\ x\ {\isasymand}\ G\ z\ x{\isacharparenright}\ {\isasymlongrightarrow}\ z\ {\isacharequal}\ y{\isacharparenright}{\isacharparenright}{\isachardoublequoteclose}\ \isacommand{by}\isamarkupfalse%
\ auto\isanewline
\ \ \ \ \ \ \isacommand{hence}\isamarkupfalse%
\ {\isachardoublequoteopen}w\ r\ w\ {\isasymlongrightarrow}\ {\isacharparenleft}{\isacharparenleft}{\isasymforall}z{\isachardot}\ {\isacharparenleft}existsAt\ z\ w\ {\isasymand}\ G\ z\ w{\isacharparenright}\ {\isasymlongrightarrow}\ z\ {\isacharequal}\ y{\isacharparenright}{\isacharparenright}{\isachardoublequoteclose}\ \isacommand{by}\isamarkupfalse%
\ {\isacharparenleft}rule\ allE{\isacharparenright}\isanewline
\ \ \ \ \ \ \isacommand{hence}\isamarkupfalse%
\ {\isachardoublequoteopen}{\isasymforall}z{\isachardot}\ {\isacharparenleft}w\ r\ w\ {\isasymand}\ existsAt\ z\ w\ {\isasymand}\ G\ z\ w{\isacharparenright}\ {\isasymlongrightarrow}\ z\ {\isacharequal}\ y{\isachardoublequoteclose}\ \isacommand{by}\isamarkupfalse%
\ auto\isanewline
\ \ \ \ \ \ \isacommand{hence}\isamarkupfalse%
\ {\isadigit{4}}{\isacharcolon}\ {\isachardoublequoteopen}{\isacharparenleft}w\ r\ w\ {\isasymand}\ existsAt\ g\ w\ {\isasymand}\ G\ g\ w{\isacharparenright}\ {\isasymlongrightarrow}\ g\ {\isacharequal}\ y{\isachardoublequoteclose}\ \isacommand{by}\isamarkupfalse%
\ {\isacharparenleft}rule\ allE{\isacharparenright}\isanewline
\ \ \ \ \ \ \isacommand{have}\isamarkupfalse%
\ {\isachardoublequoteopen}w\ r\ w{\isachardoublequoteclose}\ \isacommand{using}\isamarkupfalse%
\ refl\ \isanewline
\ \ \ \ \ \ \ \ \isacommand{by}\isamarkupfalse%
\ simp\ %
\isamarkupcmt{using frame reflexivity (Axiom M)%
}
\isanewline
\ \ \ \ \ \ \isacommand{hence}\isamarkupfalse%
\ \ {\isachardoublequoteopen}w\ r\ w\ {\isasymand}\ {\isacharparenleft}existsAt\ g\ w\ {\isasymand}\ G\ g\ w{\isacharparenright}{\isachardoublequoteclose}\ \isacommand{using}\isamarkupfalse%
\ {\isadigit{1}}\ \isacommand{by}\isamarkupfalse%
\ {\isacharparenleft}rule\ conjI{\isacharparenright}\isanewline
\ \ \ \ \ \ \isacommand{from}\isamarkupfalse%
\ {\isadigit{4}}\ this\ \isacommand{have}\isamarkupfalse%
\ {\isachardoublequoteopen}g\ {\isacharequal}\ y{\isachardoublequoteclose}\ \isacommand{by}\isamarkupfalse%
\ {\isacharparenleft}rule\ mp{\isacharparenright}\isanewline
\ \ \ \ \ \ \isacommand{thus}\isamarkupfalse%
\ {\isachardoublequoteopen}{\isacharparenleft}g\ \isactrlbold {\isasymapprox}\ y{\isacharparenright}\ w{\isachardoublequoteclose}\ \isacommand{by}\isamarkupfalse%
\ simp\isanewline
\ \ \ \ \isacommand{next}\isamarkupfalse%
\isanewline
\ \ \ \ \ \ \isacommand{assume}\isamarkupfalse%
\ {\isachardoublequoteopen}{\isacharparenleft}g\ \isactrlbold {\isasymapprox}\ y{\isacharparenright}\ w{\isachardoublequoteclose}\isanewline
\ \ \ \ \ \ \isacommand{from}\isamarkupfalse%
\ this\ {\isadigit{2}}\ \isacommand{have}\isamarkupfalse%
\ {\isachardoublequoteopen}{\isasymE}\ G\ y\ w{\isachardoublequoteclose}\ \isacommand{by}\isamarkupfalse%
\ simp\isanewline
\ \ \ \ \ \ \isacommand{thus}\isamarkupfalse%
\ {\isachardoublequoteopen}G\ y\ w\ {\isachardoublequoteclose}\ \isacommand{by}\isamarkupfalse%
\ {\isacharparenleft}rule\ conjunct{\isadigit{1}}{\isacharparenright}\isanewline
\ \ \ \ \isacommand{qed}\isamarkupfalse%
\isanewline
\ \ \isacommand{{\isacharbraceright}}\isamarkupfalse%
\isanewline
\ \ \isacommand{hence}\isamarkupfalse%
\ {\isachardoublequoteopen}{\isasymforall}y{\isachardot}\ G\ y\ w\ {\isasymlongleftrightarrow}\ {\isacharparenleft}g\ \isactrlbold {\isasymapprox}\ y{\isacharparenright}\ w{\isachardoublequoteclose}\ \isacommand{by}\isamarkupfalse%
\ {\isacharparenleft}rule\ allI{\isacharparenright}\ \isanewline
\ \ \isacommand{hence}\isamarkupfalse%
\ {\isachardoublequoteopen}{\isasymexists}x{\isachardot}\ {\isacharparenleft}{\isasymforall}y{\isachardot}\ G\ y\ w\ {\isasymlongleftrightarrow}\ {\isacharparenleft}x\ \isactrlbold {\isasymapprox}\ y{\isacharparenright}\ w{\isacharparenright}{\isachardoublequoteclose}\ \isacommand{by}\isamarkupfalse%
\ {\isacharparenleft}rule\ exI{\isacharparenright}\ \isanewline
\ \ \isacommand{hence}\isamarkupfalse%
\ {\isachardoublequoteopen}{\isacharparenleft}\isactrlbold {\isasymexists}x{\isachardot}\ {\isacharparenleft}\isactrlbold {\isasymforall}y{\isachardot}\ G\ y\ \isactrlbold {\isasymleftrightarrow}\ {\isacharparenleft}x\ \isactrlbold {\isasymapprox}\ y{\isacharparenright}{\isacharparenright}{\isacharparenright}\ w{\isachardoublequoteclose}\ \isacommand{by}\isamarkupfalse%
\ simp\isanewline
\isacommand{{\isacharbraceright}}\isamarkupfalse%
\isanewline
\isacommand{thus}\isamarkupfalse%
\ {\isacharquery}thesis\ \isacommand{by}\isamarkupfalse%
\ {\isacharparenleft}rule\ allI{\isacharparenright}\ \isanewline
\isacommand{qed}\isamarkupfalse%
%
%
%
%
\begin{isamarkuptext}%
Corollary 11.30:%
\end{isamarkuptext}\isamarkuptrue%
\isacommand{lemma}\isamarkupfalse%
\ GodImpliesExistence{\isacharcolon}\ {\isachardoublequoteopen}{\isasymlfloor}\isactrlbold {\isasymforall}x{\isachardot}\ G\ x\ \isactrlbold {\isasymrightarrow}\ E{\isacharbang}\ x{\isasymrfloor}{\isachardoublequoteclose}\ \isanewline
%
\ \ %
%
\isacommand{using}\isamarkupfalse%
\ GodExistenceIsValid\ Monotheism{\isacharunderscore}normalModel\ \isacommand{by}\isamarkupfalse%
\ metis%
%
%
%
\isamarkupsubsubsection{Positive Properties are Necessarily Instantiated%
}
\isamarkuptrue%
\isacommand{lemma}\isamarkupfalse%
\ PosPropertiesNecExist{\isacharcolon}\ {\isachardoublequoteopen}{\isasymlfloor}\isactrlbold {\isasymforall}Y{\isachardot}\ {\isasymP}\ Y\ \isactrlbold {\isasymrightarrow}\ \isactrlbold {\isasymbox}\isactrlbold {\isasymexists}\isactrlsup E\ Y{\isasymrfloor}{\isachardoublequoteclose}%
\ %
%
\isacommand{using}\isamarkupfalse%
\ GodNecExists\ A{\isadigit{4}}a\isanewline
\ \ \isacommand{by}\isamarkupfalse%
\ meson\ %
\isamarkupcmt{proposition 11.31: follows from corollary 11.28 and axiom A4a%
}
%
%
%
%
\isamarkupsubsubsection{More Objections%
}
\isamarkuptrue%
%
\begin{isamarkuptext}%
Fitting discusses the objection raised by Sobel \cite{sobel2004logic}, who argues that G\"odel's axiom system
 is too strong: it implies that whatever is the case is so necessarily, i.e. the modal system collapses (\isa{{\isasymphi}\ {\isasymlongrightarrow}\ {\isasymbox}{\isasymphi}}).
 The \emph{modal collapse} has been philosophically interpreted as implying the absence of free will.%
\end{isamarkuptext}\isamarkuptrue%
%
\begin{isamarkuptext}%
We start by proving an useful FOL lemma:%
\end{isamarkuptext}\isamarkuptrue%
\isacommand{lemma}\isamarkupfalse%
\ useful{\isacharcolon}\ {\isachardoublequoteopen}{\isacharparenleft}{\isasymforall}x{\isachardot}\ {\isasymphi}\ x\ {\isasymlongrightarrow}\ {\isasympsi}{\isacharparenright}\ {\isasymLongrightarrow}\ {\isacharparenleft}{\isacharparenleft}{\isasymexists}x{\isachardot}\ {\isasymphi}\ x{\isacharparenright}\ {\isasymlongrightarrow}\ {\isasympsi}{\isacharparenright}{\isachardoublequoteclose}%
\ %
%
\isacommand{by}\isamarkupfalse%
\ simp%
%
%
%
\begin{isamarkuptext}%
In the context of our S5 axioms, the \emph{modal collapse} becomes valid (pp. 163-4):%
\end{isamarkuptext}\isamarkuptrue%
\isacommand{lemma}\isamarkupfalse%
\ ModalCollapse{\isacharcolon}\ {\isachardoublequoteopen}{\isasymlfloor}\isactrlbold {\isasymforall}{\isasymPhi}{\isachardot}{\isacharparenleft}{\isasymPhi}\ \isactrlbold {\isasymrightarrow}\ {\isacharparenleft}\isactrlbold {\isasymbox}\ {\isasymPhi}{\isacharparenright}{\isacharparenright}{\isasymrfloor}{\isachardoublequoteclose}\isanewline
%
%
%
\isacommand{proof}\isamarkupfalse%
\ {\isacharminus}\isanewline
\ \ \isacommand{{\isacharbraceleft}}\isamarkupfalse%
\isanewline
\ \ \isacommand{fix}\isamarkupfalse%
\ w\isanewline
\ \ \ \isacommand{{\isacharbraceleft}}\isamarkupfalse%
\isanewline
\ \ \ \ \isacommand{fix}\isamarkupfalse%
\ Q\isanewline
\ \ \ \ \isacommand{have}\isamarkupfalse%
\ {\isachardoublequoteopen}{\isacharparenleft}\isactrlbold {\isasymforall}x{\isachardot}\ G\ x\ \isactrlbold {\isasymrightarrow}\ {\isacharparenleft}{\isasymE}\ G\ x{\isacharparenright}{\isacharparenright}\ w{\isachardoublequoteclose}\ \isacommand{using}\isamarkupfalse%
\ GodIsEssential\ \isanewline
\ \ \ \ \ \ \isacommand{by}\isamarkupfalse%
\ {\isacharparenleft}rule\ allE{\isacharparenright}\ %
\isamarkupcmt{follows from Axioms 11.11 and 11.3B%
}
\isanewline
\ \ \ \ \isacommand{hence}\isamarkupfalse%
\ {\isachardoublequoteopen}{\isasymforall}x{\isachardot}\ G\ x\ w\ {\isasymlongrightarrow}\ {\isasymE}\ G\ x\ w{\isachardoublequoteclose}\ \isacommand{by}\isamarkupfalse%
\ simp\isanewline
\ \ \ \ \isacommand{hence}\isamarkupfalse%
\ {\isachardoublequoteopen}{\isasymforall}x{\isachardot}\ G\ x\ w\ {\isasymlongrightarrow}\ {\isacharparenleft}\isactrlbold {\isasymforall}Z{\isachardot}\ Z\ x\ \isactrlbold {\isasymrightarrow}\ \isactrlbold {\isasymbox}{\isacharparenleft}\isactrlbold {\isasymforall}\isactrlsup Ez{\isachardot}\ G\ z\ \isactrlbold {\isasymrightarrow}\ Z\ z{\isacharparenright}{\isacharparenright}\ w{\isachardoublequoteclose}\ \isacommand{by}\isamarkupfalse%
\ force\isanewline
\ \ \ \ \isacommand{hence}\isamarkupfalse%
\ {\isachardoublequoteopen}{\isasymforall}x{\isachardot}\ G\ x\ w\ {\isasymlongrightarrow}\ {\isacharparenleft}{\isacharparenleft}{\isasymlambda}y{\isachardot}\ Q{\isacharparenright}\ x\ \isactrlbold {\isasymrightarrow}\ \isactrlbold {\isasymbox}{\isacharparenleft}\isactrlbold {\isasymforall}\isactrlsup Ez{\isachardot}\ G\ z\ \isactrlbold {\isasymrightarrow}\ {\isacharparenleft}{\isasymlambda}y{\isachardot}\ Q{\isacharparenright}\ z{\isacharparenright}{\isacharparenright}\ w{\isachardoublequoteclose}\ \isacommand{by}\isamarkupfalse%
\ force\isanewline
\ \ \ \ \isacommand{hence}\isamarkupfalse%
\ {\isachardoublequoteopen}{\isasymforall}x{\isachardot}\ G\ x\ w\ {\isasymlongrightarrow}\ {\isacharparenleft}Q\ \isactrlbold {\isasymrightarrow}\ \isactrlbold {\isasymbox}{\isacharparenleft}\isactrlbold {\isasymforall}\isactrlsup Ez{\isachardot}\ G\ z\ \isactrlbold {\isasymrightarrow}\ Q{\isacharparenright}{\isacharparenright}\ w{\isachardoublequoteclose}\ \isacommand{by}\isamarkupfalse%
\ simp\isanewline
\ \ \ \ \isacommand{hence}\isamarkupfalse%
\ {\isadigit{1}}{\isacharcolon}\ {\isachardoublequoteopen}{\isacharparenleft}{\isasymexists}x{\isachardot}\ G\ x\ w{\isacharparenright}\ {\isasymlongrightarrow}\ {\isacharparenleft}{\isacharparenleft}Q\ \isactrlbold {\isasymrightarrow}\ \isactrlbold {\isasymbox}{\isacharparenleft}\isactrlbold {\isasymforall}\isactrlsup Ez{\isachardot}\ G\ z\ \isactrlbold {\isasymrightarrow}\ Q{\isacharparenright}{\isacharparenright}\ w{\isacharparenright}{\isachardoublequoteclose}\ \isacommand{by}\isamarkupfalse%
\ {\isacharparenleft}rule\ useful{\isacharparenright}\isanewline
\ \ \ \ \isacommand{have}\isamarkupfalse%
\ {\isachardoublequoteopen}{\isasymexists}x{\isachardot}\ G\ x\ w{\isachardoublequoteclose}\ \isacommand{using}\isamarkupfalse%
\ GodExistenceIsValid\ \isacommand{by}\isamarkupfalse%
\ auto\isanewline
\ \ \ \ \isacommand{from}\isamarkupfalse%
\ {\isadigit{1}}\ this\ \isacommand{have}\isamarkupfalse%
\ {\isachardoublequoteopen}{\isacharparenleft}Q\ \isactrlbold {\isasymrightarrow}\ \isactrlbold {\isasymbox}{\isacharparenleft}\isactrlbold {\isasymforall}\isactrlsup Ez{\isachardot}\ G\ z\ \isactrlbold {\isasymrightarrow}\ Q{\isacharparenright}{\isacharparenright}\ w{\isachardoublequoteclose}\ \isacommand{by}\isamarkupfalse%
\ {\isacharparenleft}rule\ mp{\isacharparenright}\isanewline
\ \ \ \ \isacommand{hence}\isamarkupfalse%
\ {\isachardoublequoteopen}{\isacharparenleft}Q\ \isactrlbold {\isasymrightarrow}\ \isactrlbold {\isasymbox}{\isacharparenleft}{\isacharparenleft}\isactrlbold {\isasymexists}\isactrlsup Ez{\isachardot}\ G\ z{\isacharparenright}\ \isactrlbold {\isasymrightarrow}\ Q{\isacharparenright}{\isacharparenright}\ w{\isachardoublequoteclose}\ \isacommand{using}\isamarkupfalse%
\ useful\ \isacommand{by}\isamarkupfalse%
\ blast\isanewline
\ \ \ \ \isacommand{hence}\isamarkupfalse%
\ {\isachardoublequoteopen}{\isacharparenleft}Q\ \isactrlbold {\isasymrightarrow}\ {\isacharparenleft}\isactrlbold {\isasymbox}{\isacharparenleft}\isactrlbold {\isasymexists}\isactrlsup Ez{\isachardot}\ G\ z{\isacharparenright}\ \isactrlbold {\isasymrightarrow}\ \isactrlbold {\isasymbox}Q{\isacharparenright}{\isacharparenright}\ w{\isachardoublequoteclose}\ \isacommand{by}\isamarkupfalse%
\ simp\isanewline
\ \ \ \ \isacommand{hence}\isamarkupfalse%
\ {\isachardoublequoteopen}{\isacharparenleft}Q\ \isactrlbold {\isasymrightarrow}\ \isactrlbold {\isasymbox}Q{\isacharparenright}\ w{\isachardoublequoteclose}\ \isacommand{using}\isamarkupfalse%
\ GodNecExists\ \isacommand{by}\isamarkupfalse%
\ simp\isanewline
\ \ \ \isacommand{{\isacharbraceright}}\isamarkupfalse%
\isanewline
\ \ \isacommand{hence}\isamarkupfalse%
\ {\isachardoublequoteopen}{\isacharparenleft}\isactrlbold {\isasymforall}{\isasymPhi}{\isachardot}\ {\isasymPhi}\ \isactrlbold {\isasymrightarrow}\ \isactrlbold {\isasymbox}\ {\isasymPhi}{\isacharparenright}\ w{\isachardoublequoteclose}\ \isacommand{by}\isamarkupfalse%
\ {\isacharparenleft}rule\ allI{\isacharparenright}\isanewline
\ \ \isacommand{{\isacharbraceright}}\isamarkupfalse%
\isanewline
\ \ \isacommand{thus}\isamarkupfalse%
\ {\isacharquery}thesis\ \isacommand{by}\isamarkupfalse%
\ {\isacharparenleft}rule\ allI{\isacharparenright}\isanewline
\isacommand{qed}\isamarkupfalse%
\isanewline
%
%
%
%
%
%
%
%
%
%
\end{isabellebody}%
%%% Local Variables:
%%% mode: latex
%%% TeX-master: "root"
%%% End:


%
\begin{isabellebody}%
\setisabellecontext{FittingProof}%
%
%
%
%
%
%
%
\isamarkupsection{Fitting's Variant%
}
\isamarkuptrue%
%
\begin{isamarkuptext}%
In this section we consider Fitting's solution to the objections raised in his discussion of G\"odel's Argument pp. 164-9, 
especially the problem of \emph{modal collapse}, which has been metaphysically interpreted as implying a rejection of free will.
Since we are generally commited to the existence of free will (in a pre-theoretical sense), such a result is
philosophically unappealing and rather seen as a problem in the argument's formalization.%
\end{isamarkuptext}\isamarkuptrue%
%
\begin{isamarkuptext}%
Remark: The `\isa{{\isasymlparr}{\isacharunderscore}{\isasymrparr}}' parentheses are used to convert an extensional object into its `rigid'
intensional counterpart (e.g. \isa{{\isasymlparr}{\isasymphi}{\isasymrparr}\ {\isasymequiv}\ {\isasymlambda}w{\isachardot}\ {\isasymphi}}).%
\end{isamarkuptext}\isamarkuptrue%
\isacommand{abbreviation}\isamarkupfalse%
\ Entailment{\isacharcolon}{\isacharcolon}{\isachardoublequoteopen}{\isasymup}{\isasymlangle}{\isasymlangle}{\isasymzero}{\isasymrangle}{\isacharcomma}{\isasymlangle}{\isasymzero}{\isasymrangle}{\isasymrangle}{\isachardoublequoteclose}\ {\isacharparenleft}\isakeyword{infix}{\isachardoublequoteopen}{\isasymRrightarrow}{\isachardoublequoteclose}{\isadigit{6}}{\isadigit{0}}{\isacharparenright}\isanewline
\ \ \isakeyword{where}\ {\isachardoublequoteopen}X\ {\isasymRrightarrow}\ Y\ {\isasymequiv}\ \isactrlbold {\isasymbox}{\isacharparenleft}\isactrlbold {\isasymforall}\isactrlsup Ez{\isachardot}\ {\isasymlparr}X\ z{\isasymrparr}\ \isactrlbold {\isasymrightarrow}\ {\isasymlparr}Y\ z{\isasymrparr}{\isacharparenright}{\isachardoublequoteclose}\ \ \isanewline
\isacommand{consts}\isamarkupfalse%
\ Positiveness{\isacharcolon}{\isacharcolon}{\isachardoublequoteopen}{\isasymup}{\isasymlangle}{\isasymlangle}{\isasymzero}{\isasymrangle}{\isasymrangle}{\isachardoublequoteclose}\ {\isacharparenleft}{\isachardoublequoteopen}{\isasymP}{\isachardoublequoteclose}{\isacharparenright}\isanewline
\isacommand{abbreviation}\isamarkupfalse%
\ Existence{\isacharcolon}{\isacharcolon}{\isachardoublequoteopen}{\isasymup}{\isasymlangle}{\isasymzero}{\isasymrangle}{\isachardoublequoteclose}\ {\isacharparenleft}{\isachardoublequoteopen}E{\isacharbang}{\isachardoublequoteclose}{\isacharparenright}\ \isakeyword{where}\ {\isachardoublequoteopen}E{\isacharbang}\ x\ {\isasymequiv}\ {\isasymlambda}w{\isachardot}\ {\isacharparenleft}\isactrlbold {\isasymexists}\isactrlsup Ey{\isachardot}\ y\isactrlbold {\isasymapprox}x{\isacharparenright}\ w{\isachardoublequoteclose}\isanewline
\isacommand{abbreviation}\isamarkupfalse%
\ God{\isacharcolon}{\isacharcolon}{\isachardoublequoteopen}{\isasymup}{\isasymlangle}{\isasymzero}{\isasymrangle}{\isachardoublequoteclose}\ {\isacharparenleft}{\isachardoublequoteopen}G{\isachardoublequoteclose}{\isacharparenright}\ \isakeyword{where}\ {\isachardoublequoteopen}G\ {\isasymequiv}\ {\isacharparenleft}{\isasymlambda}x{\isachardot}\ \isactrlbold {\isasymforall}Y{\isachardot}\ {\isasymP}\ Y\ \isactrlbold {\isasymrightarrow}\ {\isasymlparr}Y\ x{\isasymrparr}{\isacharparenright}{\isachardoublequoteclose}%
\isamarkupsubsection{Part I - God's Existence is Possible%
}
\isamarkuptrue%
\isacommand{axiomatization}\isamarkupfalse%
\ \isakeyword{where}\isanewline
\ \ A{\isadigit{1}}a{\isacharcolon}{\isachardoublequoteopen}{\isasymlfloor}\isactrlbold {\isasymforall}X{\isachardot}\ {\isasymP}\ {\isacharparenleft}{\isasymrightharpoondown}X{\isacharparenright}\ \isactrlbold {\isasymrightarrow}\ \isactrlbold {\isasymnot}{\isacharparenleft}{\isasymP}\ X{\isacharparenright}\ {\isasymrfloor}{\isachardoublequoteclose}\ \isakeyword{and}\ \ \ \ \ \ \ \ %
\isamarkupcmt{axiom 11.3A%
}
\isanewline
\ \ A{\isadigit{1}}b{\isacharcolon}{\isachardoublequoteopen}{\isasymlfloor}\isactrlbold {\isasymforall}X{\isachardot}\ \isactrlbold {\isasymnot}{\isacharparenleft}{\isasymP}\ X{\isacharparenright}\ \isactrlbold {\isasymrightarrow}\ {\isasymP}\ {\isacharparenleft}{\isasymrightharpoondown}X{\isacharparenright}{\isasymrfloor}{\isachardoublequoteclose}\ \isakeyword{and}\ \ \ \ \ \ \ \ \ %
\isamarkupcmt{axiom 11.3B%
}
\isanewline
\ \ A{\isadigit{2}}{\isacharcolon}\ {\isachardoublequoteopen}{\isasymlfloor}\isactrlbold {\isasymforall}X\ Y{\isachardot}\ {\isacharparenleft}{\isasymP}\ X\ \isactrlbold {\isasymand}\ {\isacharparenleft}X\ {\isasymRrightarrow}\ Y{\isacharparenright}{\isacharparenright}\ \isactrlbold {\isasymrightarrow}\ {\isasymP}\ Y{\isasymrfloor}{\isachardoublequoteclose}\ \isakeyword{and}\ \ %
\isamarkupcmt{axiom 11.5%
}
\isanewline
\ \ T{\isadigit{2}}{\isacharcolon}\ {\isachardoublequoteopen}{\isasymlfloor}{\isasymP}\ {\isasymdown}G{\isasymrfloor}{\isachardoublequoteclose}\ \ \ \ \ \ \ \ \ \ \ \ \ \ \ \ \ \ \ \ \ \ \ \ \ \ \ \ \ \ \ %
\isamarkupcmt{proposition 11.16 (modified)%
}
\isanewline
\isacommand{lemma}\isamarkupfalse%
\ True\ \isacommand{nitpick}\isamarkupfalse%
{\isacharbrackleft}satisfy{\isacharbrackright}%
\ %
%
\isacommand{oops}\isamarkupfalse%
\ %
\isamarkupcmt{model found: axioms are consistent%
}
%
%
%
%
\begin{isamarkuptext}%
\emph{T1} Positive properties are possibly instantiated%
\end{isamarkuptext}\isamarkuptrue%
\isacommand{theorem}\isamarkupfalse%
\ T{\isadigit{1}}{\isacharcolon}\ {\isachardoublequoteopen}{\isasymlfloor}\isactrlbold {\isasymforall}X{\isacharcolon}{\isacharcolon}{\isasymlangle}{\isasymzero}{\isasymrangle}{\isachardot}\ {\isasymP}\ X\ \isactrlbold {\isasymrightarrow}\ \isactrlbold {\isasymdiamond}{\isacharparenleft}\isactrlbold {\isasymexists}\isactrlsup Ez{\isachardot}\ {\isasymlparr}X\ z{\isasymrparr}{\isacharparenright}{\isasymrfloor}{\isachardoublequoteclose}%
\ %
%
\isacommand{using}\isamarkupfalse%
\ A{\isadigit{1}}a\ A{\isadigit{2}}\ \isacommand{by}\isamarkupfalse%
\ blast%
%
%
%
\begin{isamarkuptext}%
\emph{T3} (God exists possibly) can be formalized in two different ways, using a \emph{de re} or a \emph{de dicto} reading.%
\end{isamarkuptext}\isamarkuptrue%
\isacommand{theorem}\isamarkupfalse%
\ T{\isadigit{3}}{\isacharunderscore}deRe{\isacharcolon}\ {\isachardoublequoteopen}{\isasymlfloor}{\isacharparenleft}{\isasymlambda}X{\isachardot}\ \isactrlbold {\isasymdiamond}\isactrlbold {\isasymexists}\isactrlsup E\ X{\isacharparenright}\ \isactrlbold {\isasymdown}G{\isasymrfloor}{\isachardoublequoteclose}%
\ %
%
\isacommand{using}\isamarkupfalse%
\ T{\isadigit{1}}\ T{\isadigit{2}}\ \isacommand{by}\isamarkupfalse%
\ simp%
%
%
\ \isanewline
\isacommand{theorem}\isamarkupfalse%
\ T{\isadigit{3}}{\isacharunderscore}deDicto{\isacharcolon}\ {\isachardoublequoteopen}{\isasymlfloor}\isactrlbold {\isasymdiamond}\isactrlbold {\isasymexists}\isactrlsup E\ \isactrlbold {\isasymdown}G{\isasymrfloor}{\isachardoublequoteclose}\ \isacommand{nitpick}\isamarkupfalse%
%
\ %
%
\isacommand{oops}\isamarkupfalse%
\ %
\isamarkupcmt{countersatisfiable: not used%
}
%
%
%
%
\isamarkupsubsection{Part II - God's Existence is Necessary if Possible%
}
\isamarkuptrue%
\isacommand{axiomatization}\isamarkupfalse%
\ \isakeyword{where}\isanewline
\ \ \ \ \ \ A{\isadigit{4}}a{\isacharcolon}\ {\isachardoublequoteopen}{\isasymlfloor}\isactrlbold {\isasymforall}X{\isachardot}\ {\isasymP}\ X\ \isactrlbold {\isasymrightarrow}\ \isactrlbold {\isasymbox}{\isacharparenleft}{\isasymP}\ X{\isacharparenright}{\isasymrfloor}{\isachardoublequoteclose}\ \ \ \ \ \ %
\isamarkupcmt{axiom 11.11%
}
\isanewline
\isacommand{lemma}\isamarkupfalse%
\ A{\isadigit{4}}b{\isacharcolon}\ {\isachardoublequoteopen}{\isasymlfloor}\isactrlbold {\isasymforall}X{\isachardot}\ \isactrlbold {\isasymnot}{\isacharparenleft}{\isasymP}\ X{\isacharparenright}\ \isactrlbold {\isasymrightarrow}\ \isactrlbold {\isasymbox}\isactrlbold {\isasymnot}{\isacharparenleft}{\isasymP}\ X{\isacharparenright}{\isasymrfloor}{\isachardoublequoteclose}%
\ %
%
\isacommand{using}\isamarkupfalse%
\ A{\isadigit{1}}a\ A{\isadigit{1}}b\ A{\isadigit{4}}a\ \isacommand{by}\isamarkupfalse%
\ blast%
%
%
\isanewline
\ \ \ \ \isanewline
\isacommand{lemma}\isamarkupfalse%
\ True\ \isacommand{nitpick}\isamarkupfalse%
{\isacharbrackleft}satisfy{\isacharbrackright}%
\ %
%
\isacommand{oops}\isamarkupfalse%
\ %
\isamarkupcmt{model found: so far all axioms consistent%
}
%
%
%
\isanewline
\isacommand{lemma}\isamarkupfalse%
\ {\isachardoublequoteopen}{\isasymlfloor}rigidPred\ {\isasymP}{\isasymrfloor}{\isachardoublequoteclose}%
\ %
%
\isacommand{using}\isamarkupfalse%
\ A{\isadigit{4}}a\ A{\isadigit{4}}b\ \isacommand{by}\isamarkupfalse%
\ blast\ %
\isamarkupcmt{\isa{{\isasymP}} designates rigidly%
}
%
%
%
\isanewline
\ \ \ \ \isanewline
\isacommand{abbreviation}\isamarkupfalse%
\ essenceOf{\isacharcolon}{\isacharcolon}{\isachardoublequoteopen}{\isasymup}{\isasymlangle}{\isasymlangle}{\isasymzero}{\isasymrangle}{\isacharcomma}{\isasymzero}{\isasymrangle}{\isachardoublequoteclose}\ {\isacharparenleft}{\isachardoublequoteopen}{\isasymE}{\isachardoublequoteclose}{\isacharparenright}\ \isakeyword{where}\isanewline
\ \ {\isachardoublequoteopen}{\isasymE}\ Y\ x\ {\isasymequiv}\ {\isasymlparr}Y\ x{\isasymrparr}\ \isactrlbold {\isasymand}\ {\isacharparenleft}\isactrlbold {\isasymforall}Z{\isacharcolon}{\isacharcolon}{\isasymlangle}{\isasymzero}{\isasymrangle}{\isachardot}\ {\isasymlparr}Z\ x{\isasymrparr}\ \isactrlbold {\isasymrightarrow}\ Y\ {\isasymRrightarrow}\ Z{\isacharparenright}{\isachardoublequoteclose}\isanewline
\isacommand{theorem}\isamarkupfalse%
\ GodIsEssential{\isacharcolon}\ {\isachardoublequoteopen}{\isasymlfloor}\isactrlbold {\isasymforall}x{\isachardot}\ G\ x\ \isactrlbold {\isasymrightarrow}\ {\isacharparenleft}{\isacharparenleft}{\isasymE}\ {\isasymdown}\isactrlsub {\isadigit{1}}G{\isacharparenright}\ x{\isacharparenright}{\isasymrfloor}{\isachardoublequoteclose}%
\ %
%
\isacommand{using}\isamarkupfalse%
\ A{\isadigit{1}}b\ \isacommand{by}\isamarkupfalse%
\ metis%
%
%
\isanewline
\ \ \ \ \isanewline
\isacommand{abbreviation}\isamarkupfalse%
\ necessaryExistencePredicate\ {\isacharcolon}{\isacharcolon}\ {\isachardoublequoteopen}{\isasymup}{\isasymlangle}{\isasymzero}{\isasymrangle}{\isachardoublequoteclose}\ {\isacharparenleft}{\isachardoublequoteopen}NE{\isachardoublequoteclose}{\isacharparenright}\ \isakeyword{where}\isanewline
\ \ {\isachardoublequoteopen}NE\ x\ \ {\isasymequiv}\ {\isasymlambda}w{\isachardot}\ {\isacharparenleft}\isactrlbold {\isasymforall}Y{\isachardot}\ \ {\isasymE}\ Y\ x\ \isactrlbold {\isasymrightarrow}\ \isactrlbold {\isasymbox}{\isacharparenleft}\isactrlbold {\isasymexists}\isactrlsup Ez{\isachardot}\ {\isasymlparr}Y\ z{\isasymrparr}{\isacharparenright}{\isacharparenright}\ w{\isachardoublequoteclose}\isanewline
\ \ \isanewline
\isacommand{axiomatization}\isamarkupfalse%
\ \isakeyword{where}\ A{\isadigit{5}}{\isacharcolon}\ {\isachardoublequoteopen}{\isasymlfloor}{\isasymP}\ {\isasymdown}NE{\isasymrfloor}{\isachardoublequoteclose}\ \ \ \ \isanewline
\isacommand{lemma}\isamarkupfalse%
\ True\ \isacommand{nitpick}\isamarkupfalse%
{\isacharbrackleft}satisfy{\isacharbrackright}%
\ %
%
\isacommand{oops}\isamarkupfalse%
\ %
\isamarkupcmt{model found: so far all axioms consistent%
}
%
%
%
%
\begin{isamarkuptext}%
Theorem 11.26 (Informal Proposition 7) - (possibilist) existence of God implies necessary (actualist) existence.
This theorem can be formalized in two ways. Both of them are proven valid:%
\end{isamarkuptext}\isamarkuptrue%
\isacommand{theorem}\isamarkupfalse%
\ GodExImpNecEx{\isacharunderscore}v{\isadigit{1}}{\isacharcolon}\ {\isachardoublequoteopen}{\isasymlfloor}\isactrlbold {\isasymexists}\ \isactrlbold {\isasymdown}G\ \isactrlbold {\isasymrightarrow}\ \ \isactrlbold {\isasymbox}\isactrlbold {\isasymexists}\isactrlsup E\ \isactrlbold {\isasymdown}G{\isasymrfloor}{\isachardoublequoteclose}%
\ %
%
\isacommand{proof}\isamarkupfalse%
\ {\isacharminus}\ %
\isamarkupcmt{not shown here%
}
%
%
%
\ \ \isanewline
\isacommand{theorem}\isamarkupfalse%
\ GodExImpNecEx{\isacharunderscore}v{\isadigit{2}}{\isacharcolon}\ {\isachardoublequoteopen}{\isasymlfloor}\isactrlbold {\isasymexists}\ \isactrlbold {\isasymdown}G\ \isactrlbold {\isasymrightarrow}\ {\isacharparenleft}{\isacharparenleft}{\isasymlambda}X{\isachardot}\ \isactrlbold {\isasymbox}\isactrlbold {\isasymexists}\isactrlsup E\ X{\isacharparenright}\ \isactrlbold {\isasymdown}G{\isacharparenright}{\isasymrfloor}{\isachardoublequoteclose}\isanewline
%
\ \ %
%
\isacommand{using}\isamarkupfalse%
\ A{\isadigit{4}}a\ GodExImpNecEx{\isacharunderscore}v{\isadigit{1}}\ \isacommand{by}\isamarkupfalse%
\ metis\ %
\isamarkupcmt{can be proven by automated tools%
}
%
%
%
%
\begin{isamarkuptext}%
In contrast to G\"odel's argument (as presented by Fitting), the following theorems can be proven in logic \emph{K}
 (the \emph{S5} axioms are no longer needed):%
\end{isamarkuptext}\isamarkuptrue%
\isacommand{theorem}\isamarkupfalse%
\ possExImpNecEx{\isacharunderscore}v{\isadigit{1}}{\isacharcolon}\ {\isachardoublequoteopen}{\isasymlfloor}\isactrlbold {\isasymdiamond}\isactrlbold {\isasymexists}\ \isactrlbold {\isasymdown}G\ \isactrlbold {\isasymrightarrow}\ \isactrlbold {\isasymbox}\isactrlbold {\isasymexists}\isactrlsup E\ \isactrlbold {\isasymdown}G{\isasymrfloor}{\isachardoublequoteclose}\isanewline
%
\ \ %
%
\isacommand{using}\isamarkupfalse%
\ GodExImpNecEx{\isacharunderscore}v{\isadigit{1}}\ T{\isadigit{3}}{\isacharunderscore}deRe\ \isacommand{by}\isamarkupfalse%
\ metis%
%
\isanewline
%
\isacommand{theorem}\isamarkupfalse%
\ possExImpNecEx{\isacharunderscore}v{\isadigit{2}}{\isacharcolon}\ {\isachardoublequoteopen}{\isasymlfloor}{\isacharparenleft}{\isasymlambda}X{\isachardot}\isactrlbold {\isasymdiamond}\isactrlbold {\isasymexists}\isactrlsup E\ X{\isacharparenright}\ \isactrlbold {\isasymdown}G\ \isactrlbold {\isasymrightarrow}\ {\isacharparenleft}{\isasymlambda}X{\isachardot}\ \isactrlbold {\isasymbox}\isactrlbold {\isasymexists}\isactrlsup E\ X{\isacharparenright}\ \isactrlbold {\isasymdown}G{\isasymrfloor}{\isachardoublequoteclose}\isanewline
%
\ \ %
%
\isacommand{using}\isamarkupfalse%
\ GodExImpNecEx{\isacharunderscore}v{\isadigit{2}}\ \isacommand{by}\isamarkupfalse%
\ blast%
%
\isanewline
%
\isanewline
\isacommand{lemma}\isamarkupfalse%
\ T{\isadigit{4}}{\isacharunderscore}v{\isadigit{1}}{\isacharcolon}{\isachardoublequoteopen}{\isasymlfloor}\isactrlbold {\isasymdiamond}\isactrlbold {\isasymexists}\ \isactrlbold {\isasymdown}G{\isasymrfloor}\ {\isasymlongrightarrow}\ {\isasymlfloor}\isactrlbold {\isasymbox}\isactrlbold {\isasymexists}\isactrlsup E\ \isactrlbold {\isasymdown}G{\isasymrfloor}{\isachardoublequoteclose}%
\ %
%
\isacommand{using}\isamarkupfalse%
\ possExImpNecEx{\isacharunderscore}v{\isadigit{1}}\ \isacommand{by}\isamarkupfalse%
\ simp%
%
%
\isanewline
\isacommand{lemma}\isamarkupfalse%
\ T{\isadigit{4}}{\isacharunderscore}v{\isadigit{2}}{\isacharcolon}{\isachardoublequoteopen}{\isasymlfloor}{\isacharparenleft}{\isasymlambda}X{\isachardot}\ \isactrlbold {\isasymdiamond}\isactrlbold {\isasymexists}\isactrlsup E\ X{\isacharparenright}\ \isactrlbold {\isasymdown}G{\isasymrfloor}{\isasymlongrightarrow}{\isasymlfloor}{\isacharparenleft}{\isasymlambda}X{\isachardot}\ \isactrlbold {\isasymbox}\isactrlbold {\isasymexists}\isactrlsup E\ X{\isacharparenright}\ \isactrlbold {\isasymdown}G{\isasymrfloor}{\isachardoublequoteclose}%
\ %
%
\isacommand{using}\isamarkupfalse%
\ possExImpNecEx{\isacharunderscore}v{\isadigit{2}}\ \isacommand{by}\isamarkupfalse%
\ simp%
%
%
%
\isamarkupsubsection{Conclusion (\emph{De Re} and \emph{De Dicto} Reading)%
}
\isamarkuptrue%
%
\begin{isamarkuptext}%
Version I - Necessary Existence of God (\emph{de dicto}):%
\end{isamarkuptext}\isamarkuptrue%
\isacommand{lemma}\isamarkupfalse%
\ GodNecExists{\isacharunderscore}v{\isadigit{1}}{\isacharcolon}\ {\isachardoublequoteopen}{\isasymlfloor}\isactrlbold {\isasymbox}\isactrlbold {\isasymexists}\isactrlsup E\ \isactrlbold {\isasymdown}G{\isasymrfloor}{\isachardoublequoteclose}\isanewline
%
\ \ %
%
\isacommand{using}\isamarkupfalse%
\ GodExImpNecEx{\isacharunderscore}v{\isadigit{1}}\ T{\isadigit{3}}{\isacharunderscore}deRe\ \isacommand{by}\isamarkupfalse%
\ fastforce\ %
\isamarkupcmt{corollary 11.28%
}
%
%
\isanewline
%
\isacommand{lemma}\isamarkupfalse%
\ {\isachardoublequoteopen}{\isasymlfloor}\isactrlbold {\isasymbox}{\isacharparenleft}{\isasymlambda}X{\isachardot}\ \isactrlbold {\isasymexists}\isactrlsup E\ X{\isacharparenright}\ \isactrlbold {\isasymdown}G{\isasymrfloor}{\isachardoublequoteclose}\isanewline
%
\ \ %
%
\isacommand{using}\isamarkupfalse%
\ GodNecExists{\isacharunderscore}v{\isadigit{1}}\ \isacommand{by}\isamarkupfalse%
\ simp\ %
\isamarkupcmt{\emph{de dicto} shown here explicitly%
}
%
%
%
%
\begin{isamarkuptext}%
Version II - Necessary Existence of God (\emph{de re})%
\end{isamarkuptext}\isamarkuptrue%
\isacommand{lemma}\isamarkupfalse%
\ GodNecExists{\isacharunderscore}v{\isadigit{2}}{\isacharcolon}\ {\isachardoublequoteopen}{\isasymlfloor}{\isacharparenleft}{\isasymlambda}X{\isachardot}\ \isactrlbold {\isasymbox}\isactrlbold {\isasymexists}\isactrlsup E\ X{\isacharparenright}\ \isactrlbold {\isasymdown}G{\isasymrfloor}{\isachardoublequoteclose}\isanewline
%
\ \ %
%
\isacommand{using}\isamarkupfalse%
\ T{\isadigit{3}}{\isacharunderscore}deRe\ T{\isadigit{4}}{\isacharunderscore}v{\isadigit{2}}\ \isacommand{by}\isamarkupfalse%
\ blast%
%
%
%
\isamarkupsubsection{Modal Collapse%
}
\isamarkuptrue%
%
\begin{isamarkuptext}%
Modal collapse is countersatisfiable even in \emph{S5}. Note that countermodels with a cardinality of one 
for the domain of individuals are found by \emph{Nitpick} (the countermodel shown in the book has cardinality of two).%
\end{isamarkuptext}\isamarkuptrue%
\isacommand{axiomatization}\isamarkupfalse%
\ \isakeyword{where}\ S{\isadigit{5}}{\isacharcolon}\ {\isachardoublequoteopen}equivalence\ aRel{\isachardoublequoteclose}\ %
\isamarkupcmt{\emph{S5} axioms assumed%
}
\isanewline
\isacommand{lemma}\isamarkupfalse%
\ {\isachardoublequoteopen}{\isasymlfloor}\isactrlbold {\isasymforall}{\isasymPhi}{\isachardot}{\isacharparenleft}{\isasymPhi}\ \isactrlbold {\isasymrightarrow}\ {\isacharparenleft}\isactrlbold {\isasymbox}\ {\isasymPhi}{\isacharparenright}{\isacharparenright}{\isasymrfloor}{\isachardoublequoteclose}\ \isacommand{nitpick}\isamarkupfalse%
{\isacharbrackleft}card\ {\isacharprime}t{\isacharequal}{\isadigit{1}}{\isacharcomma}\ card\ i{\isacharequal}{\isadigit{2}}{\isacharbrackright}%
\ %
%
\isacommand{oops}\isamarkupfalse%
\ %
\isamarkupcmt{countermodel%
}
%
%
%
%
%
%
%
%
%
%
\end{isabellebody}%
%%% Local Variables:
%%% mode: latex
%%% TeX-master: "root"
%%% End:


%%% Local Variables:
%%% mode: latex
%%% TeX-master: "root"
%%% End:


% optional bibliography
\bibliographystyle{abbrv}
\bibliography{root}

\end{document}

%%% Local Variables:
%%% mode: latex
%%% TeX-master: t
%%% End:
